%% 博士、正常版本、强制使用 Windows 系统字体
\documentclass[lang=chs, degree=phd, blindreview=false, winfonts=true]{yanputhesis}
%%=============================================================================%
%% 导入宏包
%%-----------------------------------------------------------------------------%
\usepackage{geometry}                   %脚注
\usepackage{listings, listings-rust}    %rust高亮
\usepackage{listings-riscv}             %riscv汇编高亮
\usepackage{underscore}                 %使下划线不需要添加斜线转义
% \usepackage[english]{babel}   %解决label中不能添加下划线问题,但是带来副作用,会把文档语言设置为英文
\usepackage{placeins}                   %解决图片浮动问题
\usepackage{pifont}                     %可以使用带圈数字序号
\usepackage{tabularx}                   %制作更精细的表格
% \usepackage{algorithm}                  %伪代码算法包
% \usepackage{algorithmic}                %伪代码算法包
\usepackage{csquotes}                   %使用\enquote命令,引号
%%=============================================================================%
%% 基本信息录入
%%-----------------------------------------------------------------------------%
\header{NPUcore-IMPACT!!!设计文档}                                             %设置页眉
%%=============================================================================%
%% 文档开始
%%-----------------------------------------------------------------------------%
\begin{document}

\frontmatter                                                                   %前言部分
\makeBookCoverPage{NPUcore-IMPACT!!!\\设计文档}{figs/NPU_logo.png}          %封面
\setcounter{page}{1}                                                           %设置目录页编号从1开始
\tableofcontents                                                               %目录页

\begin{abstract}                                            % 中文摘要开始
   NPUcore-IMPACT是一个使用Rust编写的基于LoongArch架构的类Unix操作系统。本工作基于原先的NPUcore(RISCV)开发迭代并扩展,
   实现POSIX标准系统调用90个,支持信号机制及线程。
   目前NPUcore的OS家族支持国际开源RISC-V指令集、硬件开发板(K210、U740、星光二代)及Qemu模拟器,与自主龙芯LoongArch指令集、
   硬件开发板(2K500/1000/2000、3A5000)及龙芯虚拟机。由于本次龙芯赛道仅需支持2K1000开发板,因此我们后文的解释都会基于此开发板进行讲解。
   同时,我们也会在后文详细介绍我们工作的增量与OS特色。
   目前初赛的所有测试用例已经满分通过,并截至到7月29日,我们是决赛LA赛道唯一得分队伍。下图是我们队伍(NPUcore-IMPACT!!!)的初赛和决赛测试通过情况:
   \begin{figure}[htp]
    \centering
    \includegraphics[width=1\linewidth]{figs/abstract.png}
    \includegraphics[width=1\linewidth]{figs/决赛分数.png}
   \end{figure}

   我们的NPUcore大致由(如图0-1)四个模块构成:Syscall(系统调用)、Memory (内存管理)、Process(进程控制)、File System(文件系统)。
   \begin{figure}[htp]
    \centering
    \includegraphics[width=1\linewidth]{figs/NPUcore整体.png}
    \caption{NPUcore整体内核框架}
   \end{figure}
    \begin{keywords}                                        % 中文关键词开始
        NPUcore \sep OSKernel \sep    LoongArch                %
    \end{keywords}                                          % 中文关键词结束
\end{abstract}                                              % 中文摘要结束

\mainmatter                                                                    %正文部分
\sDefault
\chapter{NPUcore简介(Introduction)}

“NPUcore”是西北工业大学的操作系统内核构建实践型教学操作系统,曾获得2022年OSKernel大赛内核实现赛道一等奖。
NPUcore致力于使用Rust新型编程语言,帮助老师和学生自行研制一个操作系统微型内核,提升操作系统原理的实践体验并探索新型操作系统的设计与实现。
原始的2022版NPUcore具有内存管理、进程管理、文件系统核心系统调用功能,支持RISCV32/64指令集,可在对应的QEMU模拟器和SiFive-U740、K210等嵌入式开发板上运行。
该版本基于rCore-Tutorial迭代开发,重构90\%模块以支持Linux接口,共实现系统调用81个,是一个不错的baseline。
虽然该版本有着不错的性能,但却无法支持全部测例,以及国内自主研发的LoongArch龙芯架构。不仅如此,该版本不支持网络协议,EXT4文件系统,以及其它多种多样的外设,因此我们认为,这个版本仍然有很大的优化空间。
如此,针对初赛和决赛阶段,我们的贡献可以总结为以下四点,并在后文中详细展开:
\begin{enumerate}
    \item 独自实现了2022版本的NPUcore到2k1000平台(龙芯架构)的适配,并封装为一个arch包,方便后人持续开发。
    \item 基本完成了NPUcore在ext4文件系统的适配,但仍有少部分bug。
    \item 调研了几乎所有的开源轻量版ext4仓库,并针对此次适配做了一定总结。
    \item 其它小规模增量:
    \begin{itemize}
        \item 在NPUcore-重生之我是菜狗队伍的指导下,适配了网络模块,并在fat32文件系统上跑出分数。(由于这个增量更多属于另一组,所以我们不会在此次文档中进行大面积介绍)
        \item 独自在ext4文件系统上适配了PCI和SATA驱动,可以从镜像中读取到测例。
        \item 对在2k1000板子上烧录测例进行了初步探索,并总结出了对应的步骤。
    \end{itemize}
\end{enumerate}

% \section{Rust特性}

% Rust是一个“安全、并发、实用”,支持函数式、并发式、过程式以及面向对象的程序设计风格的新型语言。
% Rust在完全公开的情况下开发,并且相当欢迎社区的反馈。近些年,Rust语言在工业应用上的势头越来越猛。
% 基于Rust语言的种种特性,我们认为它更适合一些底层应用的开发,尤其是OSKernel。

% \textbf{1. 我们为什么选择Rust作为OS编程语言?}
% Rust 是一门内存安全的语言。对于 C/C++ 这样的手动管理内存的编程语言,我们在分配堆变量的时候需要调用 malloc/new函数,而当该变量使用完毕之后要手动调用
% free/delete 回收内存。这就要求程序员需要关注所有堆变量的生命周期并及时将其释放,否则就会造成内存泄漏的问题,而过早的释放堆变量又可能造成“use-after-
% free” 的问题。而 Rust 独特的所有权机制和借用检查,让编译器掌管变量的生
% 命周期,使得变量的回收变得可控,同时也杜绝了”use-after-free” 的问题,又不至于带来垃圾回收的开销。

% Rust还能够推断出类型的大小,然后分配正确的内存大小并将其设置为您要求的值。但这意味着无法分配未初始化的内存:Rust没有null的概念。 
% 此外,所有这些检查都是在编译时完成的,因此没有运行时开销,这也是为什么Rust被成为是安全的“C”。 
% 如果你编写了正确的C++代码,你将编写出与C++代码基本上相同的Rust代码。而且由于编译器的帮忙,编写错误的代码版本是不可能的。
% 所以,我们选择Rust语言的原因,不仅是因为他安全,还因为其享有和C一样的速度,和更丰富的库。


% \textbf{2.unsafe关键字}

% 几乎每个语言都有unsafe关键字,但Rust语言使用unsafe的原因可能与其它编程语言还有所不同。接下来我们展示一下unsafe的特性:
% \begin{lstlisting}[language={Rust}, label={code:unsafe},
% 	caption={unsafe展示(r1 是一个裸指针)}]
% fn main() {
%     let mut num = 5;

%     let r1 = &num as *const i32;

%     unsafe {
%         println!("r1 is: {}", *r1);
%     }
% }
% \end{lstlisting}
% 在代码块\ref{code:unsafe}中,r1 是一个裸指针(raw pointer),由于它具有破坏Rust内存安全的潜力,因此只能在unsafe代码块中使用,如果你去掉unsafe\{\},编译器会立刻报错。
% 在我们的NPUcore中,对于一个OS来说,安全是最大的保障,因此unsafe在初期NPUcore建设中给予了很大帮助。因为,即使做到小心谨慎,依然会有出错的可能性,但是 unsafe 语句块决定了:就算内存访问出错了,你也能立刻意识到,错误是在 unsafe 代码块中,而不花大量时间像无头苍蝇一样去寻找问题所在。
% unsafe不安全,但是该用的时候就要用,在一些时候,它能帮助我们大幅降低代码实现的成本。虽然在网上充斥着“千万不要使用 unsafe,因为它不安全”的言论。事实上,我们认为unsafe是一个有效且必要的手段,因此我们选择遵循如下规则去使用:
% \begin{enumerate}
%     \item 没必要用时,就不用;
%     \item 当有必要用时,就大胆用,但是要控制好边界;
%     \item 尽量保证unsafe的边界范围最小。
% \end{enumerate}

\section{NPUcore操作系统}

「NPUcore」是西北工业大学的操作系统内核构建实践型教学操作系统。致力于使用Rust新型编程语言,帮助老师和学生自行研制一个操作系统微型内核,提升操作系统原理的实践体验并探索新型操作系统的设计与实现。目前NPUcore具有内存管理、进程管理、文件系统核心功能,支持RISCV32/64指令集,可在QEMU模拟器和SiFive-U740、K210等嵌入式开发板上运行。

以下为NPUcore的所有的系统调用:

\section{预备知识及技能}
\subsection{RISC-V和LoongArch指令集介绍}
\textbf{1、RISC-V}

RISC-V(发音为“risk-five”)是一个基于精简指令集(RISC)原则的开源指令集架构(ISA),简易解释为开源软体运动相对应的一种“开源硬体”。该项目2010年始于加州大学柏克莱分校,但许多贡献者是该大学以外的志愿者和行业工作者。\\
与大多数指令集相比,RISC-V指令集可以自由地用于任何目的,允许任何人设计、制造和销售RISC-V芯片和软件而不必支付给任何公司专利费。虽然这不是第一个开源指令集,但它具有重要意义,因为其设计使其适用于现代计算设备(如仓库规模云计算机、高端移动电话和微小嵌入式系统)。设计者考虑到了这些用途中的性能与功率效率。该指令集还具有众多支持的软件,这解决了新指令集通常的弱点。\\
RISC-V指令集的设计考虑了小型、快速、低功耗的现实情况来实做,但并没有对特定的微架构做过度的设计。
新出现的RISC-V的核心目标是灵活适应未来的AIoT场景,保证基本功能,提供可配置的扩展功能。其开源特征使得学生都可以方便地设计一个RISC-V CPU。\\
写面向RISC-V的OS的代价仅仅是你了解RISC-V的Supevisor特权模式,知道OS在Supevisor特权模式下的控制能力。\\
\textbf{2、 LoongArch}

LoongArch是RISC中的一个具体实现。2020年,龙芯中科基于二十年的CPU研制和生态建设积累推出了龙架构(LoongArch™),包括基础架构部分和向量指令、虚拟化、二进制翻译等扩展部分,近2000条指令。

龙架构具有较好的自主性、先进性与兼容性。

龙架构从整个架构的顶层规划,到各部分的功能定义,再到细节上每条指令的编码、名称、含义,在架构上进行自主重新设计,具有充分的自主性。

龙架构摒弃了传统指令系统中部分不适应当前软硬件设计技术发展趋势的陈旧内容,吸纳了近年来指令系统设计领域诸多先进的技术发展成果。同原有兼容指令系统相比,不仅在硬件方面更易于高性能低功耗设计,而且在软件方面更易于编译优化和操作系统、虚拟机的开发。

龙架构在设计时充分考虑兼容生态需求,融合了各国际主流指令系统的主要功能特性,同时依托龙芯团队在二进制翻译方面十余年的技术积累创新,能够实现多种国际主流指令系统的高效二进制翻译。龙芯中科从 2020 年起新研的 CPU 均支持LoongArch™。

龙架构已得到国际开源软件界广泛认可与支持,正成为与X86/ARM并列的顶层开源生态系统。已向GNU组织申请到ELF Machine编号(258号),并获得Linux、Binutils、GDB、.NET、GCC、LLVM、Go、Chromium/V8、Mozilla / SpiderMonkey、Javascript、FFmpeg、libyuv、libvpx、OpenH264、SRS等音视频类软件社区、UEFI(UEFI规范、ACPI规范)以及国内龙蜥开源社区、欧拉openEuler开源社区的支持。

指令系统是软件生态的起点,只有从指令系统的根源上实现自主,才能打破软件生态发展受制于人的锁链。龙架构的推出,是龙芯中科长期坚持自主研发理念的重要成果体现,是全面转向生态建设历史关头的重大技术跨越。

\subsection{Rust语言及其主要特性}
Rust是由Mozilla主导开发的通用、编译型编程语言。设计准则为“安全、并发、实用”,支持函数式、并发式、过程式以及面向对象的程序设计风格。

Rust语言原本是Mozilla员工Graydon Hoare的个人项目,而Mozilla于2009年开始赞助这个项目,并且在2010年首次公开。也在同一年,其编译器原始码开始由原本的OCaml语言转移到用Rust语言,进行自我编译工作,称做“rustc”,并于2011年实际完成。这个可自我编译的编译器在架构上采用了LLVM做为它的后端。

第一个有版本号的Rust编译器于2012年1月发布。Rust1.0是第一个稳定版本,于2015年5月15日发布。

Rust在完全公开的情况下开发,并且相当欢迎社区的反馈。在1.0稳定版之前,语言设计也因为透过撰写Servo网页浏览器排版引擎和rustc编译器本身,而有进一步的改善。它虽然由Mozilla资助,但其实是一个共有项目,有很大部分的代碼是来自于社区的贡献者。

\textbf{1.所有权}

所有权是Rust的核心,也是其更有趣和独特的功能之一。“所有权”是指允许哪部分的代码修改内存。让我们从查看一些C++代码开始:
\begin{lstlisting}[language={Rust}, label={code:forktest},
	caption={forktest.rs}]
	int *dangling(void)
	{
		int i = 1234;
		return &i;
	}
	
	int add_one(void)
	{
		int *num = dangling();
		return *num + 1;
	}
\end{lstlisting}

dangling函数在栈上分配了一个整型,然后保存给一个变量i,最后返回了这个变量i的引用。这里有一个问题:当函数返回时栈内存变成失效。意味着在函数add\_one第二行,指针num指向了垃圾值,我们将无法得到想要的结果。虽然这个一个简单的例子,但是在C++的代码里会经常发生。当堆上的内存使用malloc(或new)分配,然后使用free(或delete)释放时,会出现类似的问题,但是您的代码会尝试使用指向该内存的指针执行某些操作。 更现代的C++使用RAII和构造函数/析构函数,但它们无法完全避免“悬空指针”。 这个问题被称为“悬空指针”,并且不可能编写出现“悬空指针”的Rust代码。 我们试试吧:
\begin{lstlisting}[language={Rust}, label={code:forktest},
	caption={forktest.rs}]
	fn dangling() -> &int {
		let i = 1234;
		return &i;
	}
	
	fn add_one() -> int {
		let num = dangling();
		return *num + 1;
	}
\end{lstlisting}

当你尝试编译这个程序时,你会得到一个有趣和非常长的错误信息:
\begin{lstlisting}[language={Rust}, label={code:forktest},
	caption={forktest.rs}]
	temp.rs:3:11: 3:13 error: borrowed value does not live long enough
	temp.rs:3     return &i;
	
	temp.rs:1:22: 4:1 note: borrowed pointer must be valid for the anonymous lifetime #1 defined on the block at 1:22...
	temp.rs:1 fn dangling() -> &int {
		temp.rs:2     let i = 1234;
		temp.rs:3     return &i;
		temp.rs:4 }
	
	temp.rs:1:22: 4:1 note: ...but borrowed value is only valid for the block at 1:22
	temp.rs:1 fn dangling() -> &int {      
		temp.rs:2     let i = 1234;            
		temp.rs:3     return &i;               
		temp.rs:4  }                            
	error: aborting due to previous error
\end{lstlisting}

为了完全理解这个错误信息,我们需要谈谈“拥有”某些东西意味着什么。 所以现在,让我们接受Rust不允许我们用悬空指针编写代码,一旦我们理解了所有权,我们就会回来看这块代码。

让我们先放下编程一会儿,先聊聊书籍。 我喜欢读实体书,有时候我真的很喜欢一本书,并告诉我的朋友他们应该阅读它。 当我读我的书时,我拥有它:这本书是我所拥有的。 当我把书借给别人一段时间,他们向我“借用”这本书。 当你借用一本书时,在特定的一段时间它是属于你的,然后你把它还给我,我又拥有它了。 对吗?

这个概念也直接应用于Rust代码:一些代码“拥有”一个指向内存的特定指针。 它是该指针的唯一所有者。 它还可以暂时将该内存借给其他代码:代码“借用”它。 借用它一段时间,称为“生命周期”。

这是关于所有权的所有。 那似乎并不那么难,对吧? 让我们回到那条错误信息:error: borrowed value does not live long enough。 我们试图使用Rust的借用指针&,借出一个特定的变量i。 但Rust知道函数返回后该变量无效,因此它告诉我们:
\begin{lstlisting}[language={Rust}, label={code:forktest},
	caption={forktest.rs}]
	borrowed pointer must be valid for the anonymous lifetime #1
	
	... but borrowed value is only valid for the block。
\end{lstlisting}

这是栈内存的一个很好的例子,但堆内存呢? Rust有第二种指针,一个'唯一'指针,你可以用$\sim$创建。 看看这个:
\begin{lstlisting}[language={Rust}, label={code:forktest},
	caption={forktest.rs}]
	fn dangling() -> ~int {
		let i = ~1234;
		return i;
	}
	
	fn add_one() -> int {
		let num = dangling();
		return *num + 1;
	}
\end{lstlisting}

此代码将成功编译。 请注意,我们使用指针指向该值而不是将1234分配给栈:$\sim$1234。 你可以大致比较这两行:
\begin{lstlisting}[language={Rust}, label={code:forktest},
	caption={forktest.rs}]
	// rust
	let i = ~1234;
	// C++
	int *i = new int;
	*i = 1234;
\end{lstlisting}

Rust能够推断出类型的大小,然后分配正确的内存大小并将其设置为您要求的值。 这意味着无法分配未初始化的内存:Rust没有null的概念。万岁! Rust和C++之间还有另外一个区别:Rust编译器还计算了i的生命周期,然后在它无效后插入相应的free调用,就像C++中的析构函数一样。 您可以获得手动分配堆内存的所有好处,而无需自己完成所有工作。 此外,所有这些检查都是在编译时完成的,因此没有运行时开销。 如果你编写了正确的C++代码,你将编写出与C++代码基本上相同的Rust代码。而且由于编译器的帮忙,编写错误的代码版本是不可能的。
你已经看到了一种情况,所有权和生命周期有利于防止在不太严格的语言中通常会出现的危险代码。现在让我们谈谈另一种情况:并发。

\textbf{2.并发:}

并发是当前软件世界中一个令人难以置信的热门话题。 对于计算机科学家来说,它一直是一个有趣的研究领域,但随着互联网的使用爆炸式增长,人们正在寻求改善给定的服务可以处理的用户数量。 并发是实现这一目标的一种方式。 但并发代码有一个很大的缺点:它很难推理,因为它是非确定性的。 编写好的并发代码有几种不同的方法,但让我们来谈谈Rust的所有权和生命周期的概念如何帮助实现正确并且并发的代码。

首先,让我们回顾一下Rust中的简单并发示例。 Rust允许你启动task,这是轻量级的“绿色”线程。 这些任务没有任何共享内存,因此,我们使用“通道”在task之间进行通信。 像这样:
\begin{lstlisting}[language={Rust}, label={code:forktest},
	caption={forktest.rs}]
	fn main() {
		let numbers = [1,2,3];
		
		let (port, chan)  = Chan::new();
		chan.send(numbers);
		
		do spawn {
			let numbers = port.recv();
			println!("{:d}", numbers[0]);
		}
	}
\end{lstlisting}

在这个例子中,我们创建了一个数字的vector。 然后我们创建一个新的Chan,这是Rust实现通道的包名。 这将返回通道的两个不同端:通道(channel)和端口(port)。 您将数据发送到通道端(channel),它从端口端(port)读出。 spawn函数可以启动一个task。 正如你在代码中看到的那样,我们在task中调用port.recv(),我们在外面调用chan.send(),传入vector。 然后打印vector的第一个元素。

这样做是因为Rust在通过channel发送时copy了vector。 这样,如果它是可变的,就不会有竞争条件。 但是,如果我们正在启动很多task,或者我们的数据非常庞大,那么为每个任务都copy副本会使我们的内存使用量膨胀而没有任何实际好处。

引入Arc。 Arc代表“原子引用计数”,它是一种在多个task之间共享不可变数据的方法。 这是一些代码:
\begin{lstlisting}[language={Rust}, label={code:forktest},
	caption={forktest.rs}]
	extern mod extra;
	use extra::arc::Arc;
	
	fn main() {
		let numbers = [1,2,3];
		
		let numbers_arc = Arc::new(numbers);
		
		for num in range(0, 3) {
			let (port, chan)  = Chan::new();
			chan.send(numbers_arc.clone());
			
			do spawn {
				let local_arc = port.recv();
				let task_numbers = local_arc.get();
				println!("{:d}", task_numbers[num]);
			}
		}
	}
\end{lstlisting}

这与我们之前的代码非常相似,除了现在我们循环三次,启动三个task,并在它们之间发送一个Arc。 Arc :: new创建一个新的Arc,.clone()返回Arc的新的引用,而.get()从Arc中获取该值。 因此,我们为每个task创建一个新的引用,将该引用发送到通道,然后使用引用打印出一个数字。 现在我们不copy vector。

Arcs非常适合不可变数据,但可变数据呢? 共享可变状态是并发程序的祸根。 您可以使用互斥锁(mutex)来保护共享的可变状态,但是如果您忘记获取互斥锁(mutex),则可能会发生错误。

Rust为共享可变状态提供了一个工具:RWArc。 Arc的这个变种允许Arc的内容发生变异。 看看这个:
\begin{lstlisting}[language={Rust}, label={code:forktest},
	caption={forktest.rs}]
	extern mod extra;
	use extra::arc::RWArc;
	
	fn main() {
		let numbers = [1,2,3];
		
		let numbers_arc = RWArc::new(numbers);
		
		for num in range(0, 3) {
			let (port, chan)  = Chan::new();
			chan.send(numbers_arc.clone());
			
			do spawn {
				let local_arc = port.recv();
				
				local_arc.write(|nums| {
					nums[num] += 1
				});
				
				local_arc.read(|nums| {
					println!("{:d}", nums[num]);
				})
			}
		}
	}
\end{lstlisting}

我们现在使用RWArc包来获取读/写Arc。 RWArc的API与Arc略有不同:读和写允许您读取和写入数据。 它们都将闭包作为参数,并且在写入的情况下,RWArc将获取互斥锁,然后将数据传递给此闭包。 闭包完成后,互斥锁被释放。

你可以看到在不记得获取锁的情况下是不可能改变状态的。 我们获得了共享可变状态的便利,同时保持不允许共享可变状态的安全性。

但是我们不能同时允许和禁止可变状态。 是什么赋予了这种能力的?

\textbf{3.unsafe:}

因此,Rust语言不允许共享可变状态,但我刚刚向您展示了一些允许共享可变状态的代码。 这怎么可能? 答案:unsafe。

你看,虽然Rust编译器非常聪明,并且可以避免你通常犯的错误,但它不是人工智能。 因为我们比编译器更聪明,有时候,我们需要克服这种安全行为。 为此,Rust有一个unsafe关键字。 在一个unsafe的代码块里,Rust关闭了许多安全检查。 如果您的程序出现问题,您只需要审核您在不安全范围内所做的事情,而不是整个程序。

如果Rust的主要目标之一是安全,为什么要关闭安全?
嗯,实际上只有三个主要原因:与外部代码连接,例如将FFI写入C库,性能(在某些情况下),以及围绕通常不安全的操作提供安全抽象。 我们的Arcs是最后一个目的的一个例子。 我们可以安全地分发对Arc的多个引用,因为我们确信数据是不可变的,因此可以安全地共享。 我们可以分发对RWArc的多个引用,因为我们知道我们已经将数据包装在互斥锁中,因此可以安全地共享。 但Rust编译器无法知道我们已经做出了这些选择,所以在Arcs的实现中,我们使用不安全的块来做(通常)危险的事情。 但是我们暴露了一个安全的接口,这意味着Arcs不可能被错误地使用。

这就是Rust的类型系统如何让你不会犯一些使并发编程变得困难的错误,同时也能获得像C++等语言一样的效率。

我希望这个对Rust的尝试能让您了解Rust是否适合您。 如果这是真的,我建议您查看完整的教程,以便对Rust的语法和概念进行全面,深入的探索。
\subsection{如何查资料}
\textbf{1、查手册}

\textbf{程序自带的文档}
(1) README 和 INSTALL

很多程序在编译或者安装过程中都会自带一个README和INSTALL文件, 不要漏掉, 否则可能会有重要的信息遗漏并导致某些严重问题。

其中, 如果INSTALL文件被单独呈现, 则其往往是解释软件的安装方式的, 有的软件有很特殊的安装要求, 如执行脚本的位置必须是在文件夹内或者文件夹外, 如果漏掉可能导致软件完全运行不起来。

README一般介绍软件的使用方式, 文档获取位置和帮助信息, 有时候也介绍安装方法。

例如, 你在NPUcore的源代码文件夹中可以找到README:
\begin{figure}[htb]
	\centering
	\includegraphics[width=\textwidth]{figures/02-01-readme.png}
	\caption{
		readme
	}
	\label{fig:readme}
\end{figure}
(2) help参数
绝大多数程序会自带一个help选项, 甚至不加任何参数。 例如, man命令的help参数:

\begin{lstlisting}[language={Rust}, label={code:forktest},
	caption={forktest.rs}]
	whatis --help
\end{lstlisting}

会打印出:
\begin{figure}[htb]
	\centering
	\includegraphics[width=\textwidth]{figures/02-01-help.png}
	\caption{
		help
	}
	\label{fig:help}
\end{figure}
那么, 如果你直接在终端中敲入help并回车, 会发生什么呢(假设你使用的是bash)?请自己试一下.

此外, 帮助文档的提供软件自身往往也会有自己的帮助文档, 显然自产自销是最合适的。

所以, 你不妨试一下man man, 或者进入man后有没有能看到的某些man自己的帮助文档(仔细找, 我这么说就一定有)。

之后的任何帮助套件也可以“自己帮助自己”, 所以文献中不再赘述。

(3) TAB补全

在Bash中,TAB补全是一种非常有用的功能,它可以让用户更快捷、更准确地输入命令和文件路径。在终端输入命令或文件路径时,如果按下TAB键,Bash会尝试自动补全输入的内容。

下面是关于Bash中TAB补全常见类型(如果没有特别说明, 则Bash会列出这些选项供选择):
1)命令补全:输入一个命令的前几个字母时,补全该命令的名称。如果有多个以该字符串开头的命令,
2)文件路径补全:输入一个文件或目录的路径时,补全路径中的文件或目录名称。如果有多个符合条件的文件或目录,
3)变量名补全:输入一个变量名时,补全该变量的名称。如果有多个符合条件的变量名,
4)命令参数补全:输入命令的参数时,补全该命令所支持的参数选项。如果有多个符合条件的参数选项,
5)目录补全:在输入路径时,如果您只知道路径中的某些部分,可以使用通配符进行补全。例如,输入"/u/lo*",按下TAB键可以自动补全为"/usr/local"。

总之,bash中的TAB补全是一种非常方便的功能,可以让用户更快速地输入命令和路径,并且减少输入错误的可能性。

此外, TAB补全需要程序自身和终端的支持, 有时候甚至需要单独配置, (例如rust的工具链就需要自行配置对应的shell补全选项)

\textbf{man}

在Linux操作系统中,man命令是一个非常重要的命令,它可以帮助用户查看Linux系统中各种命令的手册。

使用man只需要在终端中输入"man"加上要查看的命令名称,然后按下回车键即可。例如:

\begin{lstlisting}[language={Rust}, label={code:forktest},
	caption={forktest.rs}]
	man help
\end{lstlisting}

man命令将会显示出该命令的手册页,可以使用键盘上的箭头键进行滚动,并且可以使用“/”加上关键字进行查找。

在手册页中,可以查看该命令的使用方法、参数选项、示例以及其他相关信息。 man软件的本身的帮助信息可以在软件中按“h”查看。

当不再需要查看手册页时,可以按下“q”键退出man命令。

\textbf{完整手册}
多数成体系的大型软件系统会有自己对应的文档, 一般称为手册. 具体来说, 这种文档会出现在官方网站的Documentation环节, 且往往有在线或者线下PDF两种版本。

我们以GNU GCC为例, 在https://gcc.gnu.org/中, 浏览器搜索(一般快捷键是Ctrl-F)Documentation, 下方的Manual就是手册。点进去会有各种格式的手册。

有的成熟的软件或者语言会提供Tutorial 和 Reference Manual, 后者倾向于列举所有的性质, 前者则是为入门初学者提供的简单的教程。

一般而言, 绝大多数的软件是自身具有自己的手册的, 但部分软件的手册是集合型的, 或者本身就是其manpage的集合。

一个典型的例子是coreutils, 其中包括了cut, head, tail等简单工具;另一个是binutils, 包括各种GCC的常用工具。这时候需要自行查询其手册的所在之处。

\textbf{教材}

很多的软件都有自己的教材, 且层次从入门到精通都有, 如果你有需要, 可以找买一本合适的书从中学习。 一般教材会比官方手册更详细, 并提供作者自己的思考。

\textbf{TLDR}

TLDR是“Too long, don't read.”的缩写,
如果要最快获得某个命令的简单使用方法, tldr是一个不错的来源。例如我们输入

\begin{lstlisting}[language={Rust}, label={code:forktest},
	caption={forktest.rs}]
	$ tldr man
	
	Format and display manual pages.More information: https://www.man7.org/linux/man-pages/man1/man.1.html.
	
	- Display the man page for a command:
	man {{command}}
	
	- Display the man page for a command from section 7:
	man {{7}} {{command}}
	
	- List all available sections for a command:
	man -f {{command}}
	
	- Display the path searched for manpages:
	man --path
	
	- Display the location of a manpage rather than the manpage itself:
	man -w {{command}}
	
	- Display the man page using a specific locale:
	man {{command}} --locale={{locale}}
	
	- Search for manpages containing a search string:
	man -k "{{search_string}}"
\end{lstlisting}

\textbf{info}

注意, info是用某个目录作为中心数据库的, 所以完全可能存在在一个软件中可以阅读但在另一个软件中读不了的情况。

info中有大量长篇的完整文档, 一般就是上述完整手册。 一般各种IDE本身也会自带Info的阅读器。 只是info有自己的搜索, 历史记录等功能(有的功能是配合IDE使用的), 这里不再赘述。

此外, GNU套件几乎所有的工具都有info文档。所谓的GNU套件PDF文档就是用info相关的一个软件texinfo写的。

我们以gdb为例展示其内容。 终端输入“info gdb”可以得到:

\begin{figure}[htb]
	\centering
	\includegraphics[width=\textwidth]{figures/02-01-info.png}
	\caption{
		info
	}
	\label{fig:info}
\end{figure}

另外, info文档的安装方法如下:

对于软件自带文档,一般可以:

\begin{lstlisting}[language={Rust}, label={code:forktest},
	caption={info}]
	sudo apt-get install gdb-doc
	有时候需要去网站上搜索并下载, 就会需要:
	whereis info # 获得info的安装目录
	# 注意:有时候下载到的是一个info.tar.gz, 这时候需要自行解压, 
	# 但是如果是info.gz,则可以直接跳过这一步, 因为gz一般是自动解压的。
	# 另外,有时候文档是以texi后缀名出现的
	tar xf <infofilepath> <tmp_path>
	sudo install-info bison.info <one_of_the_info_paths>/dir
\end{lstlisting}

\textbf{自动补全与文档显示插件}

多数的IDE有自己的文档现实和自动补全插件, 可以在光标悬停在某个符号一段时间后自动显示特定的文档。

很多IDE还会集成之前的所说的这些文档查询方式, 从而在内部查询所有的文献。

请自行搜索自己的编辑器和IDE的文档寻找配置方式。

\textbf{搜索引擎}

如果你遇到什么工具你无法使用或者不会用, 可以尝试通过搜索引擎寻找替代品, 在线版或者其他能让你用上的方法(比如能够代理你请求的某些接口,软件和网站)。

\textbf{论坛}

论坛往往是最后一步, 一般来说很少出现别人没有发现过而自己发现的问题, 毕竟计算机工业确实过于发达了. 但是, 在stackoverflow和其他论坛上提问仍然有可能可以获得比较好的效果。

如果人家恰好碰到过或者有兴趣帮你解决问题的话是再幸运不过的事, 不过在此之前, 请先不要往下看, 尝试通过之前几步查找可能技术论坛以便解决问题。

然后这是一份作者根据自己回忆写的常见的论坛, 你可以试着在上面发帖. 此外, 一般使用量大的项目都有自己的论坛, 你也可以自行查找。








\chapter{相关工作(Related Work)}

\chapter{编写第一个系统调用}

什么是系统调用?在我们使用C语言编程时,使用过库函数提供的一些基本的函数,例如:控制台输出、文件读写。
我们使用库函数完成基本的操作,库函数是对操作系统提供的系统调用的进一步封装,并隐藏掉了一些操作。
系统调用工作在最底层,使用POSIX提供的接口,直接操作硬件,完成最基本的操作。

为什么需要借助于操作系统,用户不能直接操作硬件呢?
凡是与资源有关的操作、会影响到其它进程的操作,为了方便管理资源(防止恶意操作)、使进程间隔离,
操作系统必须介入,实现统一管理调度。操作系统为上层编程语言提供了一套接口,这套接口就是系统调用。
用户库封装系统调用为库函数还有以下优点:

\textbf{1. 简化用户程序的编写:}通过封装系统调用,用户程序可以使用更为简单和直观的接口来完成复杂的系统操作,
无需了解系统调用的底层细节,减少程序员的开发难度。

\textbf{2. 提高代码的可维护性:}通过库函数的封装,程序员可以对库函数进行多层封装,使程序的可读性、可维护性更高。
当需要进行修改时,只需修改库函数的实现,无需修改应用程序的代码,降低了代码的耦合度,减少了代码维护的成本。

\textbf{3. 提高程序的移植性:}不同的操作系统和硬件平台实现系统调用的方式可能略有不同。
使用库函数来封装系统调用可以提高程序的移植性。如果需要在不同操作系统下运行程序,只需更改库函数的实现即可。

\textbf{4. 方便进行错误处理:}库函数可以对系统调用返回值进行处理,根据返回值不同的情况进行错误处理。
在使用系统调用时,程序员需要手动进行错误处理,使用库函数可以减轻程序员的工作量。

总之,封装系统调用为库函数可以使得程序更加简单、稳定、易维护、易移植,并且可以提高程序员的开发效率。

为了区分一个操作是用户完成的,还是依赖于操作系统完成的,每种指令集体系结构都对此做出了区分。
以RISC-V架构为例,CPU的工作状态分为用户态、内核态等。执行用户程序指令时的状态为用户态,需要发起系统调用时,
库函数中ecall指令会使CPU发生陷入提高特权级,到达内核态。内核态完成操作后,使用ret指令降低特权级,回到用户态。

本章将首先以三个系统调用为例子,讲解在用户态程序中如何使用系统调用,之后使用调试工具跟踪观察系统调用的实现,
最后将对系统调用以及用户态、内核态等展开详细介绍。

\section{使用系统调用}

本节将以三个常见系统调用为例,简要介绍在用户态用户进程是如何使用系统调用的。

\subsection{fork}

fork是一种用于克隆进程的全部内存空间的系统调用,是Linux系统中创建一个新进程的重要方法,除了第一个进程,
所有的进程都是由fork创建的。

这是一个 Rust 语言编写的程序,主要目的是展示如何使用 fork 函数创建子进程,如\autoref{code:forktest}所示。

\begin{lstlisting}[language={Rust}, label={code:forktest},
    caption={forktest.rs}]
#![no_std]
#![no_main]

#[macro_use]
extern crate user_lib;

use user_lib::{exit, fork, wait};

const MAX_CHILD: usize = 30;

#[no_mangle]
pub fn main() -> i32 {
    for i in 0..MAX_CHILD {
        let pid = fork();
        if pid == 0 {
            println!("I am child {}", i);
            exit(0);
        } else {
            println!("forked child pid = {}", pid);
        }
        assert!(pid > 0);
    }
    let mut exit_code: i32 = 0;
    for _ in 0..MAX_CHILD {
        if wait(&mut exit_code) <= 0 {
            panic!("wait stopped early");
        }
    }
    if wait(&mut exit_code) > 0 {
        panic!("wait got too many");
    }
    println!("forktest pass.");
    0
}
\end{lstlisting}

\subsection{exec}
\subsection{sbrk}
\section{利用GDB跟踪getpid系统调用}

NPUcore的系统调用是基于中断来实现的,大致会经历以下步骤,如\autoref{table:系统调用通用过程}:
(用户态,内核态由CPU特定寄存器中的几位来表示)

\begin{table}[h]
    \centering
    \caption{系统调用通用过程}
    \label{table:系统调用通用过程}
    \begin{tabular}{|c|c|}
        \hline
        \textbf{用户态}  & \textbf{内核态}     \\\hline
                        & hello.c(执行ecall) \\\hline
        硬件断点保存     &                    \\\hline
        OS手动断点保存   &                    \\\hline
        中断处理         &                    \\\hline
        中断返回,OS手动断点恢复 &                \\\hline
        ret 硬件断点恢复 &                    \\\hline
                        & hello.c继续执行     \\\hline
    \end{tabular}
\end{table}

下面请你自己动手,使用调试软件跟踪一遍系统调用。
\chapter{NPUcore-IMPACT 增量}

在上述基础上,我们继续做了许多努力,让 NPUcore-IMPACT 通过了初赛的所有测试用例,以及实验性地初步支持了 EXT4 文件系统。

后文我们会分别详细地介绍这两部分的内容。

\section{初赛期间的增量}

\subsection{初赛测试用例}

我们针对性地对初赛的测试用例进行了调试,将问题归类定位到了如下两点。

然后我们分别对每个问题进行了细致的分析,最终逐个击破,通过了初赛的所有测试用例。

\textbf{1. statx 系统调用}

LoongArch 赛道的初赛测试用例中,mmap 与 munmap 这两个测例涉及到了一个新的系统调用 statx。

\begin{lstlisting}[label={man:statx}, caption={statx 手册}]
NAME
       statx - get file status (extended)

LIBRARY
       Standard C library (libc, -lc)

SYNOPSIS
       #define _GNU_SOURCE          /* See feature_test_macros(7) */
       #include <fcntl.h>           /* Definition of AT_* constants */
       #include <sys/stat.h>

       int statx(int dirfd, const char *restrict pathname, int flags,
                 unsigned int mask, struct statx *restrict statxbuf);

STANDARDS
       Linux.

HISTORY
       Linux 4.11, glibc 2.28.
\end{lstlisting}

如手册 \ref{man:statx} 中所示,这个系统调用涉及了文件信息的获取。

我们为 NPUcore-IMPACT 实现了这个新的系统调用,并与文件系统进行了整合。

\begin{lstlisting}[language={Rust}, caption={statx 系统调用入口}]
let ret = match syscall_id {
    // ...
    SYSCALL_STATX => sys_statx(
        args[0],
        args[1] as *const u8,
        args[2] as u32,
        args[3] as u32,
        args[4] as *mut u8,
    ),
    // ...
};
\end{lstlisting}

\begin{lstlisting}[language={Rust}, caption={statx 系统调用实现}]
pub trait File: DowncastSync {
    // ...
    fn get_statx(&self) -> Statx;
    // ...
}
\end{lstlisting}

随后我们进行了测试,成功通过了 mmap 与 munmap 测试用例。

\textbf{2. 文件描述符分配}

通过对 openat 测试用例进行调试,我们最终发现问题出在操作系统对文件描述符的分配上。

Unix 标准要求操作系统分配文件描述符时,总是分配该进程还未使用的最小的文件描述符;而 NPUcore 回收进程关闭的文件描述符时,使用了一个线性表;操作系统重新分配之前回收的文件描述符时,没有使用表中最小的文件描述符,最终导致出现了问题。

\begin{lstlisting}[language={Rust}, caption={回收文件描述符}]
match self.inner[fd].take() {
    Some(file_descriptor) => {
        self.recycled.push(fd as u8);
        // TODO: maybe replace this with balanced binary tree?
        self.recycled.sort_by(|a, b| b.cmp(a));
        Ok(file_descriptor)
    }
    None => Err(EBADF),
}
\end{lstlisting}

我们选择了在回收文件描述符后进行一次排序来解决这个问题。

这个方案不一定是性能最佳的方案,我们还有以下方案可选:

\begin{enumerate}
    \item 回收时不进行排序,重新分配时使用 $O(n)$ 时间寻找最小的文件描述符;
    \item 使用二叉平衡树替换线性表,从而在 $O(\log n)$ 时间进行回收与重新分配,但也许会带来内存分配的额外开销。
\end{enumerate}

未来此处成为性能瓶颈时,根据性能测试结果选用最优方案会是更好的选择。

\subsection{EXT4 文件系统}

EXT4(fourth extended filesystem)是 Linux 内核的一个日志文件系统,是 EXT3 文件系统的继任者。EXT4 文件系统具有许多改进和新特性,使其在性能、可靠性和可扩展性方面优于前代文件系统。

与 NPUcore 先前使用的 FAT32 文件系统相比,EXT4 文件系统不仅允许了更大的文件大小与卷大小,更有着显著的性能和效率提升。EXT4 文件系统使用了延迟分配和多块分配策略,显著减少了碎片并提高了写入性能;同时它支持 Extents 和更高效的分配策略,提高了文件操作的速度和效率。此外,EXT4 文件系统还支持日志记录,通过记录元数据变化确保系统崩溃时的数据一致性和完整性,检查速度快且更可靠。

ext4文件系统原理如下:
\begin{enumerate}
    \item 日志功能:ext4采用了日志功能(journaling),即在对文件系统进行操作时,会先将操作记录在日志中,然后再执行操作。这样可以在系统异常关机或崩溃时恢复文件系统的一致性。
    \item 内存缓存:ext4使用了内存缓存,将磁盘上的数据加载到内存中进行读写操作,以提高文件的访问速度。
    \item 数据块分配:ext4使用了多级索引结构来分配存储空间。它将文件系统的空间分为固定大小的块,每个块可以存储一定大小的数据。通过索引结构,可以快速定位文件的数据块。
    \item 空闲块管理:ext4使用了位图和B树来管理空闲块。位图记录着每个块的使用情况,而B树则用于索引和定位空闲块。
    \item 快照功能:ext4支持快照功能,可以在不影响原有数据的情况下对文件系统进行备份和恢复。
    \item 后日志预分配:ext4采用了一种称为"delayed allocation"的技术,即将数据块的分配推迟到真正需要写入数据时再进行。这样可以提高写入性能,减少磁盘的碎片化。
    \item 逐项更新:ext4在写入数据时,不会一次性更新整个文件,而是根据需要只更新部分数据。这种逐项更新的方式可以减少磁盘的I/O次数,提高性能。
\end{enumerate}

我们为 NPUcore-IMPACT 实验性地加入了 EXT4 文件系统支持,使得其可以从 EXT4 文件系统启动,并读写其中的文件。



\subsection{EXT4 文件系统的实现过程}

在我们提交的最终版中,我们使用了 lwext4 作为 EXT4 文件系统驱动,同时我们也会在后面介绍NPUcore对于其它EXT4-like文件系统适配的可能性。

lwext4 是一个针对嵌入式系统设计的轻量级 EXT4 文件系统实现,它旨在提供 EXT4 文件系统的关键特性,同时保持低资源消耗和高性能,以适应嵌入式系统的限制。
为了让 lwext4 能与 NPUcore-IMPACT 一起工作,我们对 NPUcore-IMPACT 的文件系统设计做出了一定调整。

我们借助 Rust 的 trait 语言特性,设计了一个 File trait,用于表示一个抽象的文件,或者说一个可以对其进行读写的对象。

\begin{lstlisting}[language={Rust}, caption={File trait}]
pub trait File: DowncastSync {
    fn deep_clone(&self) -> Arc<dyn File>;
    fn readable(&self) -> bool;
    fn writable(&self) -> bool;
    fn read(&self, offset: Option<&mut usize>, buf: &mut [u8]) -> usize;
    fn write(&self, offset: Option<&mut usize>, buf: &[u8]) -> usize;
    fn r_ready(&self) -> bool;
    fn w_ready(&self) -> bool;
    fn read_user(&self, offset: Option<usize>, buf: UserBuffer) -> usize;
    fn write_user(&self, offset: Option<usize>, buf: UserBuffer) -> usize;
    fn get_size(&self) -> usize;
    fn get_stat(&self) -> Stat;
    fn get_statx(&self) -> Statx;
    fn get_file_type(&self) -> DiskInodeType;
    fn is_dir(&self) -> bool {
        self.get_file_type().is_dir()
        // self.get_file_type() == DiskInodeType::Directory
    }
    fn is_file(&self) -> bool {
        self.get_file_type().is_file()
        // self.get_file_type() == DiskInodeType::File
    }
    fn info_dirtree_node(&self, dirnode_ptr: Weak<DirectoryTreeNode>);
    fn get_dirtree_node(&self) -> Option<Arc<DirectoryTreeNode>>;
    /// open
    fn open(&self, flags: OpenFlags, special_use: bool) -> Arc<dyn File>;
    fn open_subfile(&self) -> Result<Vec<(String, Arc<dyn File>)>, isize>;
    /// create
    fn create(&self, name: &str, file_type: DiskInodeType) -> Result<Arc<dyn File>, isize>;
    fn link_child(&self, name: &str, child: &Self) -> Result<(), isize>
    where
        Self: Sized;
    /// delete(unlink)
    fn unlink(&self, delete: bool) -> Result<(), isize>;
    /// dirent
    fn get_dirent(&self, count: usize) -> Vec<Dirent>;
    /// offset
    fn get_offset(&self) -> usize {
        self.lseek(0, SeekWhence::SEEK_CUR).unwrap()
    }
    fn lseek(&self, offset: isize, whence: SeekWhence) -> Result<usize, isize>;
    /// size
    fn modify_size(&self, diff: isize) -> Result<(), isize>;
    fn truncate_size(&self, new_size: usize) -> Result<(), isize>;
    // time
    fn set_timestamp(&self, ctime: Option<usize>, atime: Option<usize>, mtime: Option<usize>);
    /// cache
    fn get_single_cache(&self, offset: usize) -> Result<Arc<Mutex<PageCache>>, ()>;
    fn get_all_caches(&self) -> Result<Vec<Arc<Mutex<PageCache>>>, ()>;
    /// memory related
    fn oom(&self) -> usize;
    /// poll, select related
    fn hang_up(&self) -> bool;
    /// iotcl
    fn ioctl(&self, _cmd: u32, _argp: usize) -> isize {
        ENOTTY
    }
    /// fcntl
    fn fcntl(&self, cmd: u32, arg: u32) -> isize;
}
\end{lstlisting}

在此基础上,我们为 lwext4 提供的 ext4_file 类型实现我们的 File trait,让 NPUcore-IMPACT 可以对其进行读写,从而实现 EXT4 文件系统的支持。

由于 lwext4 依赖 libc 进行内存分配,为了让它能工作在没有 libc 的环境下,我们还需要对其做出一定修改。

\begin{lstlisting}[language={C}, caption={管理 lwext4 内存}]
#if CONFIG_USE_USER_MALLOC

#define ext4_malloc  ext4_user_malloc
#define ext4_calloc  ext4_user_calloc
#define ext4_realloc ext4_user_realloc
#define ext4_free    ext4_user_free

#else

#define ext4_malloc  malloc
#define ext4_calloc  calloc
#define ext4_realloc realloc
#define ext4_free    free

#endif
\end{lstlisting}

我们希望让 NPUcore-IMPACT 为 lwext4 管理内存,为此我们实现 ext4_user_malloc、ext4_user_calloc、ext4_user_realloc、ext4_user_free 这四个内存管理函数,并将其与 lwext4 链接,从而让 lwext4 可以使用我们为它分配的内存,并在合适的时候回收这些内存。

\begin{lstlisting}[language={Rust}, caption={NPUcore-IMPACT 为 lwext4 分配内存}]
#[no_mangle]
pub extern "C" fn ext4_user_malloc(size: ::core::ffi::c_size_t) -> *mut ::core::ffi::c_void {
    HEAP_ALLOCATOR
        .lock()
        .alloc(Layout::array::<u8>(size).unwrap())
        .unwrap()
        .as_ptr() as *mut ::core::ffi::c_void
}
\end{lstlisting}

为了便于调试,我们需要在 lwext4 执行时打印日志,得益于 Rust 与 C 跨语言互操作十分方便,我们直接在 Rust 侧编写了打印日志的工具函数。

\begin{lstlisting}[language={Rust}, caption={在 lwext4 的 C 语言代码中打印日志}]
#[no_mangle]
pub extern "C" fn os_log(str: *const ::core::ffi::c_char) {
    let str = unsafe { CStr::from_ptr(str) };
    log::info!("{str:?}");
}

#[no_mangle]
pub extern "C" fn os_var_log(name: *const ::core::ffi::c_char, value: ::core::ffi::c_int) {
    let name = unsafe { CStr::from_ptr(name) };
    log::info!("{name:?}: {value}");
}
\end{lstlisting}

使用 \#[no_mangle] 可以让编译器不对函数名字进行混淆,使得我们可以在 C 语言侧直接调用 os_log 与 os_var_log 日志函数。


\subsubsection{LA 体系下 FAT32 与 EXT4 的区别}

\begin{enumerate}
    \item \textit{与指令集相关}
    % INPROCESS
    \item \textit{与 NPUcore 相关}
    % INPROCESS
\end{enumerate}

\subsubsection{敲定实现方式}

我们参考了历年不同赛道的优秀作品,最后给出了如下的适配方式:

\textit{我们采用第三方包将稳定 C 库作为外部库调入 NPUcore 中,如\autoref{ext4-complexe}所示:}

\begin{table}[htbp]
    \centering
    \begin{tabular}{|c|c|}
        \hline
        选用技术栈 & 作用 \\
        \hline
        lwext4 & 稳定的 ext4 文件系统外部库 \\
        bindgen & rust-lang 官方开发的FFI生成工具 \\
        \hline
    \end{tabular}
    \caption{选用技术栈}
\end{table}


\begin{figure}
    \centering
    \includegraphics[width=0.6\linewidth]{figs/plan-ext.png}
    \caption{ext4 实现结构图}
    \label{ext4-complexe}
\end{figure}

\begin{enumerate}
    \item \textit{根据 lwext4 或者类似的库理清楚他的函数调用,必要的话给出一个 .h 文件用于包装函数入口:} \\ \textit{The wrapper.h file will include all the various headers containing declarations of structs and functions we would like bindings for. In the particular case of bzip2, this is pretty easy since the entire public API is contained in a single header. For a project like SpiderMonkey, where the public API is split across multiple header files and grouped by functionality, we'd want to include all those headers we want to bind to in this single wrapper.h entry point for bindgen.}\footnote{参考 bindgen 手册https://rust-lang.github.io/rust-bindgen/tutorial-2.html},这意味着,\textbf{对于一个比较复杂而分散的项目,我们最好给出一个包装文件}.
    \item \textit{对于转换完成的rs库,视情况给出rust调用}
    \item \textit{转换我们的fs适配新的rs库} \\ 这部分很简单,我们相当于已经拿来一个ext文件系统了,剩下的就是直接使用调用就行了。在makefile里和rust代码里加入feature就可以做到针对不同文件系统的编译与运行
\end{enumerate}

\vspace{1em}

对于其中可能出现的问题,可见如下列表:

\begin{enumerate}
    \item \textbf{移植的时候会不会出现不适配龙芯情况:}99\%不会,目前查出来 Bindgen 使用 Clang 对 C 文件进行编译,之后反编译(\textit{仅使用 Clang ,不使用 LLVM 编译为机器码})回 Rust ,所以生成的代码最后编译时间还是走的 make 中的 loongarch-gcc .具体编译环节的参考如下:https://blog.csdn.net/xhhjin/article/details/81164076
    \item \textbf{lwext4 的水平如何,是否会存在包本身的问题:} C 语言库,方便阅读;稳定性比较强,多平台测试过,支持小端序,测试过的架构有 x86/AMD64 , ARM 系列以及其的各种嵌入式架构
\end{enumerate}

\subsubsection{第一次适配(LWEXT4-C + Bindgen)}

在第一次适配中,我们试图通过上述方式完成 EXT4 文件系统对于 LA 的适配,然而,我们遇到了许多问题
\begin{enumerate} 
    \item \textit{no_std 环境问题:}我们发现,离开了标准 C 环境的 lwext4 的适配情况并没有我们想象的顺利。在一步步 debug 的过程中,我们经历了如下问题:
    \begin{figure}[htbp] 
        \centering 
        \includegraphics[width=0.5\linewidth]{figs/ext4c.png} 
        \caption{Debug 流程图} 
        \label{debug-ext4c} 
    \end{figure} 
    %\item \textit{工作量问题:}我们试图向内核中添加 C_std 环境,但是由于较大的工作量失败了
\end{enumerate}

\subsubsection{第二次适配(LWEXT4-RUST)}

经过一定时间的查找资料,我们发现了下一个 lwext4 库,其 Supported Features 具体如下:\footnote{github网址:https://github.com/elliott10/lwext4_rust},然而这个包的适配过程仍然十分艰难:

\begin{itemize}
    \item lwext4_rust for x86_64, riscv64 and aarch64 on Rust OS is supported
    \item File system mount and unmount operations
    \item Filetypes: regular, directories, softlinks
    \item Journal recovery \& transactions
    \item memory as Block Cache
\end{itemize}

由于在其 Dependences 中发现了如下信息:

\begin{center}
    \textbf{C musl-based cross compile toolchains}
    \begin{itemize}
        \centering
        \item x86_64-linux-musl-gcc
        \item riscv64-linux-musl-gcc
        \item aarch64-linux-musl-gcc
    \end{itemize}
\end{center}

\textit{我们认为,其在我们拥有 LA 相关工具链的情况下是可以适配至我们的 LA 指令集操作系统上的}

经过一段时间的分析,我们认为 lwext4 系列的库\textbf{由于一定原因与 LA 指令集并不适配}

\subsubsection{第三次适配(EXT4-View)}

由于前两次适配都设计到lwext4相关,并且其存在于mkfs不相干的特性(但这并不是使用lwext4-mkfs没有成功的根本原因)。于是我们在github上自行检索并找到了这个EXT4-View\footnote{https://github.com/nicholasbishop/ext4-view-rs}仓库。我们试图将这个版本的EXT4与我们的NPUcore进行适配。

EXT4-View由一个谷歌研究员开发并持续维护中。该库提供了一个 Rust crate,允许对 ext4 文件系统进行\textbf{只读访问}。该 crate是no_std,因此可在嵌入式上下文中使用。不过,它需要 alloc。
这个仓库的基本属性可以总结为如下的部分:
\begin{enumerate}
    \item 所有有效的 ext4 文件系统都应该是可读的。
    \item 无效数据绝不会导致崩溃、panic或无限循环。
    \item 主软件包中没有不安全代码(允许在依赖包中出现)。
\end{enumerate}
使用方法为:
\begin{lstlisting}[language={Rust}, caption={ext4-view在kernel中的基本使用方法示例}]
use ext4_view::{Ext4, Metadata};

let fs_data: Vec<u8> = get_fs_data_from_somewhere();
let fs = Ext4::load(Box::new(data_source))?;

// If the std feature is enabled, you can load a filesystem by path:
let fs = Ext4::load_from_path(std::path::Path::new("some-fs.bin"))?;

// The Ext4 type has methods very similar to std::fs:
let path = "/some/file/path";
let file_data: Vec<u8> = fs.read(path)?;
let file_str: String = fs.read_to_string(path)?;
let exists: bool = fs.exists(path)?;
let metadata: Metadata = fs.metadata(path)?;
for entry in fs.read_dir("/some/dir")? {
    let entry = entry?;
    println!("{}", entry.path().display());
}
    \end{lstlisting}

而载入这个文件系统的方法只有两步,首先需要将img转换为.bin文件,并将.bin引入到kernel中,用一个指针指向它作为基本目录。实现代码如下:
\begin{lstlisting}[language={Rust}, caption={将测例加载进入kernel}]
    fn load_test_disk1() -> Ext4 {
        const DATA: &[u8] = include_bytes!("../../test_data/test_disk1.bin");
        Ext4::load(Box::new(DATA.to_vec())).unwrap()
    }

\end{lstlisting}

在适配中我们发现了两个明显的缺点:第一,由于加载kernel的地址为0x9000000090000000,计算后发现,只有64M的空间。所以我们的uImage大小不能超过64M,而本次全部测例有120M左右,因此没有办法将全部测例封装并测试。
第二,也是最致命的缺点。经过三天的适配后,我们发现内核中存在了很多奇怪的bug,包括但不限于找不到根目录,块设备加载出错等。刚开始我们认为是我们的kernel适配没有完全成功,而经过检查后发现是仓库本身存在问题,目前的版本不是完全完善的ext4版本。
而该仓库也仅有一个只读文件系统,没有办法完成针对本次比赛“完整的”EXT4适配,因此我们最终也放弃了这个仓库。

这个仓库值得后续的高度关注,因为它代码风格统一,接口完善,适配简单,应该能成为后续适配者的一个优质选择。

\subsubsection{第四次适配(Alien-rust)}

最后抱着试一试的态度,我们找到了往年的特等奖得主Alien\footnote{https://gitlab.eduxiji.net/202310007101563/Alien}仓库(该仓库仍然在持续开源并推进中。它一个用 rust 实现的简单操作系统。目的是探索如何使用模块来构建一个完整的操作系统,因此系统由一系列独立的模块组成。

他们的仓库中有提到对于lwext4的修复与推进工作:有了c实现的支持,我们只需要在rust中生成相关的头文件以及静态库。在做这部分之前,我们首先查看了一下crates.io中是否已有相关的实现,幸运的是,2年前已经有一个实现lwext4, 在简单阅读了其实现之后,我们打算参考其实现重新编写,因为其已经缺乏维护,并且不包含对no_std环境的支持。这个已有的实现给予我们很好的想法。
最终我们根据Alien的提示,适配了一个仍然针对LA2K1000开发板存在bug的NPUcore版本。该版本仅支持执行部分测例,并没有实现完整的“文件系统”应有的部分。但是我们仍将我们针对这个仓库的适配过程做一个小总结。

首先我们需要将文件系统从原先的FAT32转换为lwext4中的对应的Inode和FileSystem。这里的EasyFileSystem是一层针对文件系统的抽象接口。
\begin{lstlisting}[language={Rust}, caption={FILE_SYSTEM修改}]
pub type EasyFileSystem = lwext4_rs::FileSystem<crate::arch::BlockDeviceImpl>;
type DiskInodeType = lwext4_rs::FileType;
    lazy_static! {
        pub static ref FILE_SYSTEM: EasyFileSystem = EasyFileSystem::new(
            MountHandle::mount(
                RegisterHandle::register(BlockDevice::new(BlockDeviceImpl::new()), "shit".to_string())
                    .unwrap(),
                "/".to_string(),
                false,
                false,
            )
            .unwrap()
        )
        .unwrap();
    }
\end{lstlisting}

我们针对这层抽象继续修改对应的根目录:
\begin{lstlisting}[language={Rust}, caption={ROOT修改}]
    lazy_static! {
        pub static ref ROOT: Arc<DirectoryTreeNode> = {
            FILE_SYSTEM.readdir("/").unwrap();
    
            let inode = DirectoryTreeNode::new(
                "".to_string(),
                Arc::new(FileSystem::new(FS::Fat32)),
                Arc::new(OpenOptions::new().read(true).write(true).open("/").unwrap()),
                // OSInode::new(Arc::new()),
                Weak::new(),
            );
            inode.add_special_use();
            inode
        };
        static ref DIRECTORY_VEC: Mutex<(Vec<Weak<DirectoryTreeNode>>, usize)> =
            Mutex::new((Vec::new(), 0));
        static ref PATH_CACHE: Mutex<(String, Weak<DirectoryTreeNode>)> =
            Mutex::new(("".to_string(), Weak::new()));
    }
    \end{lstlisting}

针对这层块设备,我们也适配了对应的PCI和SATA块的读写部分,可以识别到测例。这里的lock和unlock方法为开发中,因为诸多测例都不需要这个方法,close则为默认关闭成功。
\begin{lstlisting}[language={Rust}, caption={ROOT修改}]
    impl lwext4_rs::BlockDeviceInterface for SataBlock{
        fn read_block(&mut self, buf: &mut [u8], mut block_id: u64, block_count: u32) -> lwext4_rs::Result<usize> {
            // kernel BLOCK_SZ=2048, SATA BLOCK_SIZE=512,four times
            block_id = block_id * (BLOCK_SZ as u64 / BLOCK_SIZE as u64);
            for buf in buf.chunks_mut(BLOCK_SIZE) {
                self.0
                    .lock()
                    .read_block(block_id, buf);
                block_id += 1;
            }
            Ok(0)
        }
    
        fn write_block(&mut self, buf: &[u8], mut block_id: u64, block_count: u32) -> lwext4_rs::Result<usize> {
            block_id = block_id * (BLOCK_SZ as u64 / BLOCK_SIZE as u64);
            for buf in buf.chunks(BLOCK_SIZE) {
                self.0
                    .lock()
                    .write_block(block_id, buf);
                block_id += 1;
            }
            Ok(0)
        }
        
        fn close(&mut self) -> lwext4_rs::Result<()> {
            Ok(())
        }
    
        fn open(&mut self) -> lwext4_rs::Result<lwext4_rs::BlockDeviceConfig> {
            Ok(lwext4_rs::BlockDeviceConfig{
                block_size: BLOCK_SIZE as u32,
                block_count: 999,
                part_size: BLOCK_SIZE as u64 * 2,
                part_offset: 0
            })
        }
    
        fn lock(&mut self) -> lwext4_rs::Result<()> {
            Ok(())
        }
    
        fn unlock(&mut self) -> lwext4_rs::Result<()> {
            Ok(())
        }
    }
    \end{lstlisting}
    
我们最终在决赛提交的也是这个版本,虽然它仍然有各种问题,但是我们将测例直接放入kernel中是可以跑出对应分数的。这个文件系统适配仍然非常不完善,甚至在文件系统初始化时都会报panic(我们跳过文件系统这一层,直接执行测例跑出的分数),因此我们后续仍然会持续推进并开发。
% \chapter{进程管理}
\section{进程生命周期和资源复用}
\subsection{进程生命周期}
进程指的是在系统中运行的一个程序的实例。而进程的生命周期包括从创建,就绪,阻塞,运行中,退出。
\begin{figure}[htb]
    \centering
    \includegraphics[width=\textwidth]{figures/05-01-进程生命周期示意图.png}
    \caption{
        进程生命周期示意图
    }
    \label{fig:user virtual process}
\end{figure}
在一个进程被创建之后,他会进入npucore中的就绪队列,在被操作系统调度之后将会进入到运行状态。
在运行状态下时间片耗尽或者是主动让出CPU的时候,进程会进入到就绪队列中,等待下一次被调度。
在运行的时候调用诸如wait等系统调用,进程会进入到阻塞队列中,等待被唤醒。
当进程执行结束,他会退出,释放系统资源。

npucore团队在进行性能调优之时,发现在操作系统运行示例程序时,IO操作导致CPU挂起的性能损失非常之大,因此我们将调度器进行了大改,使其完全支持了阻塞式的进程调度模式。
阻塞和非阻塞IO是访问设备的两种模式,驱动程序可以灵活的支持者两种用户空间对设备的访问方式。
阻塞操作是指在执行设备操作时,若不能获得资源,则挂起进程直到满足可操作的条件后再进行操
作。被挂起的进程进入睡眠状态,被调度器的运行队列移走,直到等待的条件被满足。
非阻塞是指在不能进行设备操作时,并不挂起,他要么放弃,要么不停地查询,直到可以进行操作为止。
在阻塞访问时,不能获取资源的进程将进入休眠,它将会让出CPU,因为阻塞的进程会进入休眠状态,所
以必须要有一个动作能唤醒该进程,唤醒进程的地方最大的可能发生在中断里面,因为在硬件资源获得的
同时往往伴随着一个中断。而非阻塞的进程则不断的尝试,直到可以进行IO。
与 linux 操作系统的设计类似, npucore采用等待队列的方式实现阻塞式调度器,将在后续章节介绍。
为了可以在npucore上同时运行多个进程,npucore实现了进程的创建,在内核中加载进程到内存,同时为其分配系统资源。
npucore为每一个进程分配系统资源,包括内存、文件描述符、CPU等,而实现系统资源的分配的方式是通过系统调用fork。
npucore中,fork系统调用用于创建一个新的进程,新的进程称为子进程,原来的进程称为父进程。
而所有的其他进程都是initproc的子进程,他们通过fork得到。initproc是需要在内核启动过程中创建的第一个进程。
对应的代码如下:
\begin{lstlisting}[language={Rust}]
    lazy_static! {
    pub static ref INITPROC: Arc<TaskControlBlock> = Arc::new({
        let elf = ROOT_FD.open("initproc", OpenFlags::O_RDONLY, true).unwrap();
        TaskControlBlock::new(elf)
    });
}
\end{lstlisting}
当创建一个新的进程时,用户进程通过fork得到一个原本进程的副本,为其分配系统资源,然后调用execve来讲elf文件加载到内存以创建一个新的进程。
每个进程都有自己的内存空间、代码和数据,它们是系统中资源的分配单位。他们的创建是由elf文件指定的。

为了保证所有的进程都能够被调度,从而避免进程饥饿的发生,npucore实现了进程的调度机制,从而实现阻塞和唤醒。
当一个进程主动放弃CPU或者被动的被剥夺CPU的使用权时,它会让出CPU,变成等待状态,这个过程称为阻塞。
在进程的视角看来,他会有一个自己独占CPU的“幻觉”,因为每一个阻塞和唤醒的时候进程的状态总是保持不变的。这样可以保证进程执行的正确性。
而npucore让一个被阻塞的进程重新开始执行的行为叫做唤醒。唤醒的同时会恢复进程的现场,包括阻塞时的寄存器状态。

而当一个进程执行结束,它就会退出,将它所占有的系统资源释放。进程的退出保证了npucore避免出现资源的永久占用的情况。
上述过程就是一个进程从“生”到“死”,保证了npucore可以正确且高效的执行对应的程序。

\subsection{资源复用}
为了实现多进程同时运行,操作系统需要对CPU,内存,外设等资源进行复用。
复用在资源有限的情况下是一个常用且实际的思想。围绕着复用的思想,我们可以提出以下几个问题:

如何实现上下文切换?

虽然保存现场思想是简单的,但是实际的实现却不是那么显然。在npucore中我们使用了一段所有进程共享的跳板代码和一个进程的私有的保存现场的帧来实现。

如何让进程如何实现透明调度,也就是用户进程不知道自己被调度了?

npucore实现了内置的计时器,当计时器中断发生时,内核会调用schedule函数,从而实现进程的调度。

进程的资源回收不能由进程自己来完成,否则进程退出时会出现资源泄露的情况,如何实现进程的资源回收?

npucore在exit之后,会释放一部分资源,但是不会释放所有的资源,从而进入僵尸状态,父进程来完成剩下的进程资源的回收。

如何在并发的情况下不会错过对进程的唤醒?

npucore中是一个单核的操作系统,在进入关键代码的时候会保证CPU不会调度其他进程,从而保证了进程的唤醒不会被错过。

此外,在内存方面,npucore采用了虚拟内存的方式,将物理内存映射到虚拟内存,从而实现了内存的复用。
对于用户的elf程序,程序的入口总是相同的,从某种程度上讲,程序将会“共享”同一个地址。
但是实际上,物理地址并不能被共享,所以虚拟内存就派上了用场。在上一章我们已经介绍了虚拟内存的实现,这里不再赘述。
I/O设备的复用将在I/O章节详细介绍。在这一章中,我们将介绍进程所拥有的各种内核的资源,诸如文件描述符等。

管理进程时,npucore使用了TCB的数据结构,不同于传统的PCB数据结构,npucore将线程视为共享栈的进程。
TCB的数据结构如下:
\begin{lstlisting}[language={Rust}]
pub struct TaskControlBlock {
    // immutable
    pub pid: PidHandle,
    pub tid: usize,
    pub tgid: usize,
    pub kstack: KernelStack,
    pub ustack_base: usize,
    pub exit_signal: Signals,
    // mutable
    inner: Mutex<TaskControlBlockInner>,
    // shareable and mutable
    pub exe: Arc<Mutex<FileDescriptor>>,
    pub tid_allocator: Arc<Mutex<RecycleAllocator>>,
    pub files: Arc<Mutex<FdTable>>,
    pub fs: Arc<Mutex<FsStatus>>,
    pub vm: Arc<Mutex<MemorySet>>,
    pub sighand: Arc<Mutex<Vec<Option<Box<SigAction>>>>>,
    pub futex: Arc<Mutex<Futex>>,
}
\end{lstlisting}
在上面的定义中,我们可以看到TCB中包含了进程的进程号,线程号,线程组号,内核栈,用户栈,退出信号,以及一些共享的资源。
后面将具体讲解如何管理这些资源以及进程之间的调度。
\chapter{目前仍存在的bug与解决思路}

在硬件方面,由于今年的开发板从 2k500 换成了 2k1000 且其\textbf{标称型号与实际型号相同性存疑},我们在上板期间遇到了许多问题,这里不再详细列举,下面给出网上找到的 2k1000 开发板与实物存在的区别

\begin{figure}[htbp]
	\centering
	\begin{minipage}{0.49\linewidth}
		\centering
		\includegraphics[width=0.9\linewidth]{figs/2k1000pi.jpg}
		\caption{网上找到的 2k1000 开发板}
		\label{Board-Internet}
	\end{minipage}
	\begin{minipage}{0.49\linewidth}
		\centering
		\includegraphics[width=0.9\linewidth]{figs/2k10dp1v11.jpg}
		\caption{实物}
		\label{Board-RealEstate}
	\end{minipage}
\end{figure}

下表中给出我们在 Debug 期间找到的不同之处

\begin{table}[htbp]
	\begin{center}
		\begin{tabular}{|c|c|c|c|c|c|}
			\hline
			& NAND & SATA接口 & 串口 & uboot系统\\
			\hline
			\hline
			网传 & 有 & 有 & 有 & \textbf{存在差异} \\
			\hline
			实物 & 无 & 无 & 无 & \textbf{存在差异} \\
			\hline
		\end{tabular}
		\caption{2k1000 实际不同之处}
	\end{center}
\end{table}

即便如此,我们得到的 2k1000 开发板的 u-boot 系统仍然不尽人意,以下是我们找到的几个问题

\begin{enumerate}
	\item \textit{NAND 芯片缺失问题:}我们发现,开发板在子系统中搭载了 NAND-Subsystem ,但利用指令扫描设备时并未发现 NAND 芯片,这令我们困扰,在仔细检查过开发板后,我们发现开发板其实并未搭载 NAND 芯片.
		  \begin{figure}[htbp]
			\centering
			\includegraphics[width=0.5\linewidth]{figs/nand设备情况.png}
			\caption{nand设备情况}
			\label{nand设备情况}
		  \end{figure}
	\item \textit{MMC 子系统缺失问题:}不仅如此,我们在后续进行烧录操作时,发现本开发板搭载的 u-boot 系统甚至没有搭载 MMC 子系统,这表示这我们无法对上面的 SSD 进行操作, \textbf{这直接导致了下面存在的问题}.
\end{enumerate}

\begin{center}
	\textbf{上板期间遇到的问题 —— 无法烧录}
\end{center}

由于复赛期间我们强制要求使用 mkfs_ext4 进行镜像文件系统的制作,并强制将测例封装进文件系统,我们在上板测试时发现了如下问题

\textbf{其表现为:}我们在正常烧录内核时,会在初始化  报Panicked,具体现象与代码如下:

\begin{figure}[htbp]
	\centering
	\includegraphics[width=0.5\linewidth]{figs/bug.png}
	\caption{报错信息}
	\label{information of bug}
\end{figure}

经过分析,我们认为,在这里 Panicked 是因为内核在初始化时没有找到文件系统,\textbf{所以产生了 Block 相关问题},我们的文件系统在 QEMU 中是提前使用指令挂载至虚拟硬盘之中的

\begin{figure}[htbp]
	\centering
	\includegraphics[width=0.5\linewidth]{figs/hda.png}
	\caption{QEMU 中挂载源码}
	\label{information of bug}
\end{figure}

于是我们试图通过某种方式绕过这些问题

\begin{enumerate}
	\item \textit{通过 USB 启动:}我们在发现无法针对板上存储设备进行操作之后,试图利用 USB 启动方式绕过 u-boot 的缺陷
	\begin{enumerate}
		\item \textbf{挂载方式:}通过将 USB 设备作为硬盘\footnote{参考例子:https://www.cnblogs.com/nhdlb/p/14726431.html},将引导区域与内核区域放在一起以绕过这个缺陷
		\item \textbf{结果:}内核无法识别两个分区的( A 分区挂载引导与内核; B 分区挂载文件系统镜像) USB 启动硬盘
	\end{enumerate}
	\item \textit{刷入 PMON 系统:}
	\begin{enumerate}
		\item \textbf{PMON 系统简介:}
		\begin{enumerate}
			\item 龙芯平台计算机目前多采用 PMON 作为基本的输入输出系统
			\item PMON具有强大而丰富的功能,包括硬件初始化、操作系统引导和硬件测试、程序调式等功能
			\item 它提供多种加载操作系统的方式,可以从优盘、光盘、 tftp 服务器和硬盘等媒介加载;它提供对内存、串口、显示、网络、硬盘等的基础测试工具;此外,它还支持软件升级
		\end{enumerate}
	\end{enumerate}
\end{enumerate}

\end{document}
