\chapter{编写第一个系统调用}

什么是系统调用?在我们使用C语言编程时,使用过库函数提供的一些基本的函数,例如:控制台输出、文件读写。
我们使用库函数完成基本的操作,库函数是对操作系统提供的系统调用的进一步封装,并隐藏掉了一些操作。
系统调用工作在最底层,使用POSIX提供的接口,直接操作硬件,完成最基本的操作。

为什么需要借助于操作系统,用户不能直接操作硬件呢?
凡是与资源有关的操作、会影响到其它进程的操作,为了方便管理资源(防止恶意操作)、使进程间隔离,
操作系统必须介入,实现统一管理调度。操作系统为上层编程语言提供了一套接口,这套接口就是系统调用。
用户库封装系统调用为库函数还有以下优点:

\textbf{1. 简化用户程序的编写:}通过封装系统调用,用户程序可以使用更为简单和直观的接口来完成复杂的系统操作,
无需了解系统调用的底层细节,减少程序员的开发难度。

\textbf{2. 提高代码的可维护性:}通过库函数的封装,程序员可以对库函数进行多层封装,使程序的可读性、可维护性更高。
当需要进行修改时,只需修改库函数的实现,无需修改应用程序的代码,降低了代码的耦合度,减少了代码维护的成本。

\textbf{3. 提高程序的移植性:}不同的操作系统和硬件平台实现系统调用的方式可能略有不同。
使用库函数来封装系统调用可以提高程序的移植性。如果需要在不同操作系统下运行程序,只需更改库函数的实现即可。

\textbf{4. 方便进行错误处理:}库函数可以对系统调用返回值进行处理,根据返回值不同的情况进行错误处理。
在使用系统调用时,程序员需要手动进行错误处理,使用库函数可以减轻程序员的工作量。

总之,封装系统调用为库函数可以使得程序更加简单、稳定、易维护、易移植,并且可以提高程序员的开发效率。

为了区分一个操作是用户完成的,还是依赖于操作系统完成的,每种指令集体系结构都对此做出了区分。
以RISC-V架构为例,CPU的工作状态分为用户态、内核态等。执行用户程序指令时的状态为用户态,需要发起系统调用时,
库函数中ecall指令会使CPU发生陷入提高特权级,到达内核态。内核态完成操作后,使用ret指令降低特权级,回到用户态。

本章将首先以三个系统调用为例子,讲解在用户态程序中如何使用系统调用,之后使用调试工具跟踪观察系统调用的实现,
最后将对系统调用以及用户态、内核态等展开详细介绍。

\section{使用系统调用}

本节将以三个常见系统调用为例,简要介绍在用户态用户进程是如何使用系统调用的。

\subsection{fork}

fork是一种用于克隆进程的全部内存空间的系统调用,是Linux系统中创建一个新进程的重要方法,除了第一个进程,
所有的进程都是由fork创建的。

这是一个 Rust 语言编写的程序,主要目的是展示如何使用 fork 函数创建子进程,如\autoref{code:forktest}所示。

\begin{lstlisting}[language={Rust}, label={code:forktest},
    caption={forktest.rs}]
#![no_std]
#![no_main]

#[macro_use]
extern crate user_lib;

use user_lib::{exit, fork, wait};

const MAX_CHILD: usize = 30;

#[no_mangle]
pub fn main() -> i32 {
    for i in 0..MAX_CHILD {
        let pid = fork();
        if pid == 0 {
            println!("I am child {}", i);
            exit(0);
        } else {
            println!("forked child pid = {}", pid);
        }
        assert!(pid > 0);
    }
    let mut exit_code: i32 = 0;
    for _ in 0..MAX_CHILD {
        if wait(&mut exit_code) <= 0 {
            panic!("wait stopped early");
        }
    }
    if wait(&mut exit_code) > 0 {
        panic!("wait got too many");
    }
    println!("forktest pass.");
    0
}
\end{lstlisting}

\subsection{exec}
\subsection{sbrk}