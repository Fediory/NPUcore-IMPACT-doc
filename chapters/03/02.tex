\section{LoongArch的启动步骤}
首先是 NPUcore 的主函数:
\begin{lstlisting}[language={rust}, label={code:refill},
	caption={os/src/main.rs}]
#[no_mangle]
pub fn rust_main() -> ! {
    println!("[kernel] NPUcore-IMAPCT!!! ENTER!");
    bootstrap_init();
    mem_clear();
    console::log_init();
    move_to_high_address(); \\ img move in kernel
    println!("[kernel] Console initialized.");
    mm::init();

    machine_init();
    println!("[kernel] Hello, world!");

    //machine independent initialization
    fs::directory_tree::init_fs();
    task::add_initproc();

    // note that in run_tasks(), there is yet *another* pre_start_init(),
    // which is used to turn on interrupts in some archs like LoongArch.
    task::run_tasks();
    panic!("Unreachable in rust_main!");
}
\end{lstlisting}
我们可以看到, 除了 bootstrap_init() 和 machine_init(), 其他都是平台无关
的, 因此这里主要介绍平台相关的启动流程。其中, machine_init() 的发生在 mm::init() 后。
\begin{lstlisting}[language={rust}, label={code:refill},
	caption={os/src/arch/la64/mod.rs - bootstrap\_init}]
    pub fn bootstrap_init() {
        /* if CPUId::read().get_core_id() != 0 {
         *     loop {}
         * } */
        ECfg::empty()
            .set_line_based_interrupt_vector(LineBasedInterrupt::TIMER)
            .write();
        EUEn::read().set_float_point_stat(true).write();
        // Timer & other Interrupts
        TIClr::read().clear_timer().write();
        TCfg::read().set_enable(false).write();
        CrMd::read()
            .set_watchpoint_enabled(false)
            .set_paging(true)
            .set_ie(false)
            .write();
    
        // Trap/Exception Hanlder initialization.
        set_kernel_trap_entry();
        set_machine_err_trap_ent();
        TLBREntry::read().set_addr(srfill as usize).write();
    
        // MMU Setup
        DMW2::read()
            .set_plv0(true)
            .set_plv1(false)
            .set_plv2(false)
            .set_plv3(false)
            .set_vesg(SUC_DMW_VESG)
            .set_mat(MemoryAccessType::StronglyOrderedUnCached)
            .write();
        DMW3::empty().write();
        //DMW1::empty().write();
    
        STLBPS::read().set_ps(PTE_WIDTH_BITS).write();
        TLBREHi::read().set_page_size(PTE_WIDTH_BITS).write();
        PWCL::read()
            .set_ptbase(PAGE_SIZE_BITS)
            .set_ptwidth(DIR_WIDTH)
            .set_dir1_base(PAGE_SIZE_BITS + DIR_WIDTH)
            .set_dir1_width(DIR_WIDTH) // 512*512*4096 should be enough for 256MiB of 2k1000.
            .set_dir2_base(0)
            .set_dir2_width(0)
            .set_pte_width(PTE_WIDTH)
            .write();
        PWCH::read()
            .set_dir3_base(PAGE_SIZE_BITS + DIR_WIDTH * 2)
            .set_dir3_width(DIR_WIDTH)
            .set_dir4_base(0)
            .set_dir4_width(0)
            .write();
    
        println!("[kernel] UART address: {:#x}", UART_BASE);
        println!("[bootstrap_init] {:?}", PRCfg1::read());
    }    
\end{lstlisting}
我们可以看到, 在上面的启动过程中, 龙芯实际上中断要触发有几个使
能层:
\begin{enumerate}
    \item 中断本身的使能, 来自 ECfg。
    \item 中断源的使能 (如果存在), e.g. 时钟中断来自 TCfg。
    \item CrMd: 当前模式寄存器的中断使能。
    \item 最后是 Trap Handler 相关的设置和内存相关的初始化。
\end{enumerate}
对于机器相关初始化, 主要是初始化时钟相关的内容。
\begin{lstlisting}[language={rust}, label={code:refill},
	caption={os/src/arch/la64/mod.rs - machine\_init}]
    pub fn machine_init() {
    // remap_test not supported for lack of DMW read only privilege support
    trap::init();
    get_timer_freq_first_time();
    /* println!(
     *     "[machine_init] VALEN: {}, PALEN: {}",
     *     cfg0.get_valen(),
     *     cfg0.get_palen()
     * ); */
    for i in 0..=6 {
        let j: usize;
        unsafe { core::arch::asm!("cpucfg {0},{1}",out(reg) j,in(reg) i) };
        println!("[CPUCFG {:#x}] {}", i, j);
    }
    for i in 0x10..=0x14 {
        let j: usize;
        unsafe { core::arch::asm!("cpucfg {0},{1}",out(reg) j,in(reg) i) };
        println!("[CPUCFG {:#x}] {}", i, j);
    }
    println!("{:?}", Misc::read());
    println!("{:?}", RVACfg::read());
    println!("[machine_init] MMAP_BASE: {:#x}", MMAP_BASE);
    trap::enable_timer_interrupt();
}
\end{lstlisting}
之所以将这些内容放到这里, 是因为 LoongArch 下的恒等时钟频率是可以通过指令获取的, 
如果因此我们获取后存入静态数据区域, 所以需要在
bss 段被清 0 后才开始处理时钟相关中断。