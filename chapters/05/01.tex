\chapter{进程管理}
\section{进程生命周期和资源复用}
\subsection{进程生命周期}
进程指的是在系统中运行的一个程序的实例。而进程的生命周期包括从创建,阻塞,唤醒,退出等。
为了可以在npucore上同时运行多个进程,npucore实现了进程的创建,在内核中加载进程到内存,同时为其分配系统资源。
npucore为每一个进程分配系统资源,包括内存、文件描述符、CPU等,而实现系统资源的分配的方式是通过系统调用fork。
npucore中,fork系统调用用于创建一个新的进程,新的进程称为子进程,原来的进程称为父进程。
而所有的其他进程都是initproc的子进程,他们通过fork得到。initproc是需要在内核启动过程中创建的第一个进程。
对应的代码如下:
\begin{lstlisting}[language=rust]
    lazy_static! {
    pub static ref INITPROC: Arc<TaskControlBlock> = Arc::new({
        let elf = ROOT_FD.open("initproc", OpenFlags::O_RDONLY, true).unwrap();
        TaskControlBlock::new(elf)
    });
}
\end{lstlisting}
当创建一个新的进程时,用户进程通过fork得到一个原本进程的副本,为其分配系统资源,然后调用execve来讲elf文件加载到内存以创建一个新的进程。
每个进程都有自己的内存空间、代码和数据,它们是系统中资源的分配单位。他们的创建是由elf文件指定的。

为了保证所有的进程都能够被调度,从而避免进程饥饿的发生,npucore实现了进程的调度机制,从而实现阻塞和唤醒。
当一个进程主动放弃CPU或者被动的被剥夺CPU的使用权时,它会让出CPU,变成等待状态,这个过程称为阻塞。
在进程的视角看来,他会有一个自己独占CPU的“幻觉”,因为每一个阻塞和唤醒的时候进程的状态总是保持不变的。这样可以保证进程执行的正确性。
而npucore让一个被阻塞的进程重新开始执行的行为叫做唤醒。唤醒的同时会恢复进程的现场,包括阻塞时的寄存器状态。

而当一个进程执行结束,它就会退出,将它所占有的系统资源释放。进程的退出保证了npucore避免出现资源的永久占用的情况。
上述过程就是一个进程从“生”到“死”,保证了npucore可以正确且高效的执行对应的程序。

\subsection{资源复用}
为了实现多进程同时运行,操作系统需要对CPU,内存,外设等资源进行复用。
复用在资源有限的情况下是一个常用且实际的思想。围绕着复用的思想,我们可以提出以下几个问题:

如何实现上下文切换?

虽然保存现场思想是简单的,但是实际的实现却不是那么显然。在npucore中我们使用了一段所有进程共享的跳板代码和一个进程的私有的保存现场的帧来实现。

如何让进程如何实现透明调度,也就是用户进程不知道自己被调度了?

npucore实现了内置的计时器,当计时器中断发生时,内核会调用schedule函数,从而实现进程的调度。

进程的资源回收不能由进程自己来完成,否则进程退出时会出现资源泄露的情况,如何实现进程的资源回收?

npucore在exit之后,会释放一部分资源,但是不会释放所有的资源,从而进入僵尸状态,父进程来完成剩下的进程资源的回收。

如何在并发的情况下不会错过对进程的唤醒?

npucore中是一个单核的操作系统,在进入关键代码的时候会保证CPU不会调度其他进程,从而保证了进程的唤醒不会被错过。