\chapter{嵌入式硬件平台简介及内核运行}
\section{基于嵌入式硬件的操作系统开发流程}

引言:

随着计算机处理器制造技术的日新月异和集成电路的蓬勃发展,小小的芯片上就可以搭载一个强大的处理器,这也就为嵌入式系统营造了一片良好的土壤,为嵌入式开发创造了一片蓝海。日常生活中,嵌入式系统无处不在,我们使用的手机,平板电脑,空调,冰箱等各种各样的物件中,都存在嵌入式系统的影子。

嵌入式系统的开发与传统的PC程序开发是不用的。其设计了硬件和软件的开发,是一个协同工作的统一体。

那么什么是嵌入式系统?

嵌入式系统的英文名称是Embedded System。embedded意为嵌入...之中的,system意为系统。嵌入式系统有一些显著的特点,它的形式多样、体积比较小,可以灵活地适应各种设备的需求,能够根据用户需求灵活地裁剪软硬件模块。

嵌入式操作系统是计算机发展的一个方向,是一个体积小型化,功能多样化的方向,而另一个方向则是体积大型化,处理能力超强的大型计算机。计算机大型化适用于那些需要进行大量数据存储和大量数据处理的环境中,例如银行需要的是计算机的处理能力和稳定性,计算机的功耗和占用空间反而是其次需要考虑的。计算机小型化适用于响应时间和数据吞吐要求不高,但对功耗和空间占用有要求的场景,像是手机,PC电脑,电冰箱,空调等设备。

嵌入式操作系统种类繁多,按照系统硬件的核心处理器来区分,可以分为嵌入式微控制器和嵌入式微处理器。嵌入式微控制器就是传统意义上的单片机。单片机就是把一个计算机的主要功能集成到了一个芯片上,其通常包含了运算处理单元、ARM、flash存储器,以及一些IO接口。嵌入式微处理器与单片机相比,其处理器的处理能力更强,目前主流的嵌入式未处理为32位,单片机多是8位和16位。嵌入式微处理器在一个芯片上集成了复杂的功能,其有运算单元,指令存储器,数据存储器,缓冲区,总线,一些IO接口。在这样一个功能完善的处理器上,就可以搭载操作系统,帮助嵌入式开发。下图为嵌入式微处理器的基本结构。

\begin{figure}[H]
	\centering
	\includegraphics{figures/08-01-嵌入式微处理器基本结构.png}
	\caption{嵌入式微处理器的基本结构}
\end{figure}

嵌入式系统按层次来分,与传统pc一样,仍然分为硬件和软件层,硬件层包括嵌入式微处理器和各种外部设备,根据应用需求的不同,各种外部设备又各不相同;软件层又可以分为操作系统和应用软件两层,操作系统作为软硬件接口,负责管理系统资源,这与传统pc是一样的,用户仍然与应用软件打交道,看不到下层。

一般情况下,我们在一台计算机上编写了代码,便直接用该计算机上的编译器,编译后的到可执行代码,程序便可以执行,像这样的情况,我们就称之为“本地编译”。然而,有的时候我们并不能本地编译,我们需要在一种平台上编译出能运行在体系结构不同的另一种平台上能够运行的程序,也就是需要“交叉编译”。嵌入式开发过程中,目的平台常常还没有操作系统,谈不上运行编译器,平台运行至少需要两个东西,bootloader(启动引导代码)以及操作系统核心。这种情况下,我们只能求助于交叉编译。

在一个平台上编译成功后,我们便可以将操作系统烧写入开发板(目的平台)中,但在烧写之前,我们还涉及到了一个计算机与开发板的通信问题,这也就涉及到了串口这个概念,我们通过串口,实现计算机与外部设备的通信,来解决把操作系统烧写入开发板的问题。

\subsection{串口简介}

\subsubsection{什么是串口}

串口是串行接口(Serial Port)的简称,是一种常用的计算机接口。由于连线少、通信控制简单而得到广泛的使用。串口有几种标准,常见的一种称为 RS232 接口的标准是在1970年由美国电子工业协会(EIA)和几家计算机厂商共同制定的。RS232标准应用广泛,其全称是“数据终端设备(DTE)和数据通讯设备(DCE)串行二进制数据交换接口”,该标准定义了串口的电气接口特性和各种信号电平等。


标准串口协议支持的最高数据传输率是115Kbps。一些改进的串口控制器支持更高甚至460Kbps的数据传输率,如增强型串口ESP(Enhanced Serial Port)和超级增强型串口Super ESP。


RS232 串口使用D型数据接口,最初有9针和25针两种连接方式。随着计算机技术的不断进步,25针的串口连接方式已经被淘汰,目前所有的RS232串口都使用9针连接方式。

\subsubsection{串口工作原理}

串口通过直接连接在两台设备之间的线发送和接收数据,两台设备通信最少需要三根线(发送数据、接收数据和接地)才可以通信。以最常见的RS232串口为例,通信距离较近时(<12m),可以用电缆线直接连接标准RS232端口。如果传输距离远,可以通过调制解调器(MODEM)传输。因为串口设备工作频率低且容易受到干扰,远距离传输会造成数据丢失。

\begin{table}[!ht]
	\centering
	\begin{tabular}{|c|c|c|}
		\hline
		\textbf{针号} & \textbf{功能说明} & \textbf{缩写} \\
		\hline
		1 & 数据载波检测 & DCD  \\
		2 & 接收数据 & RXD  \\
		3 & 发送数据 & TXD \\
		4 & 数据终端准备 & DTR \\
		5 & 信号地 & GND \\
		6 & 数据设备准备好 & DSR \\
		7 & 请求发送 & RTS \\
		8 & 清除发送 & CTS \\
		9 & 振铃指示 & BELL \\
		\hline
	\end{tabular}
	\caption{DB9接口的RS232 串口数据线定义}
	\label{DB9接口的RS232 串口数据线定义}
\end{table}

表 \ref{DB9接口的RS232 串口数据线定义}是常见的9针接口串口各条线定义,RS232标准的串口不仅提供了数据发送和接收的功能,同时可以进行数据流控制。对于普通应用来说,连接好两个数据线和地线就可以通信。

\subsubsection{Windows系统下的串口工具}

MobaXterm 又名 MobaXVT,是一款增强型终端、X 服务器和 Unix 命令集(GNU/ Cygwin)工具箱。可以开启多个终端视窗,以最新的 X 服务器为基础的 X.Org,可以轻松地来试用 Unix/Linux 上的 GNU Unix 命令。这样一来,我们可以不用安装虚拟机来试用虚拟环境,然后只要通过 MobaXterm 就可以使用大多数的 linux 命令。MobaXterm 还有很强的扩展能力,可以集成插件来运行 Emacs、Fontforge、Gcc, G++ and development tools、MPlayer、Perl、Curl、Corkscrew、 Tcl / Tk / Expect、 Screen、 Png2Ico 、 NEdit  Midnight Commander 等程序。


MobaXterm的下载较为简单,进入官网直接下载即可。需要注意的是MobaXterm 分免费家庭版和收费专业版:
\begin{itemize}
	\item 家庭版(Home):家庭版又分便捷版和安装版。便捷版不需要安装,下载压缩包后解压即可使用。安装版则需一步步安装后才能使用。
	\item 专业版(Professional):专业版会在 Session数、SSH tunnels 数和其他一些定制化配置进行限制。
\end{itemize}

MobaXterm的界面结构为:
\begin{itemize}
	\item 主页:打开MobaXterm后,绝大部分内容会被主页占据,主页有快捷按钮“start local terminal”以及最近的Session记录,可以方便打开终端,进行命令行操作。
	\item 菜单栏:MobaXterm的菜单栏如下,分为menu bar和buttons bar两行。用户可通过menu bar设置MobaXterm。buttons bar则为用户提供了一系列功能,如用户可通过Session启动远程会话,选择创建 SSH、Telnet、Rlogin、RDP、VNC、XDMCP、FTP、SFTP 或串行会话。开始的每个会话都会自动保存并显示在左侧边栏中。
	\item 侧边栏:侧边栏分为三个视图,分别为Sessions,Tools,Macros。Sessions负责管理使用过的Session配置,可以在任意Session选项上右击进行编辑。Tools分为四部分Terminal games、System、Office、NetWork的工具集合。使用非常的方便。Macros可以录制操作过程,下次使用时直接回放即可。
\end{itemize}

\begin{figure}[h]
	\centering
	\includegraphics[width=0.6\textwidth]{figures/08-01-mobaXterm界面.png}
	\caption{mobaXterm界面}
	\label{mobaXterm界面}
\end{figure}


MobaXterm是一款功能强大的综合性工具,支持SSH、RDP、FTP、Serial等多种通信协议。在本节中,我们将重点介绍其支持的Serial协议功能。MobaXterm可用作串口终端,与串口调试助手等工具相比,它提供了更强大的功能。在使用串口调试助手等工具时,虽然可以用于打印调试信息,但无法实现终端使用,即不能输入命令。


在MobaXterm中,通过点击"Session",选择"Serial",打开串口设置窗口。首先,选择要配置的串口号,并使用串口线将开发板连接到计算机上。然后,设置波特率,MobaXterm软件能够自动识别串口,用户只需从下拉菜单中选择相应的波特率。此外,还需进行其他串口功能的设置。点击"Advanced Serial settings"选项卡,可配置串口的其他功能,包括串口号、数据位、停止位、奇偶校验和硬件流控等。若需配置与终端相关的功能,则可点击"Terminal settings",进行相关设置,如终端字体以及字体大小等。完成设置后,点击窗口下方的"OK"按钮即可。

\subsubsection{Linux系统下的串口工具}

串口是嵌入式开发使用最多的通信方式。Linux系统提供了一个串口工具minicom,可以完成复杂的串口通信工作。若将minicom移植到板卡上,我们就可以借助minicom对串口进行读写操作。


本节介绍 minicom的使用。首先是安装 minicom,在 Ubuntu Linux 系统 shell 下输入“sudo apt-get install minicom”,按回车键后即可安装minicom 软件。软件安装好后,第一次使用之前需要配置minicom。

\begin{enumerate}
	\item 在shell中输入 sudo minicom-s,出现 minicom配置界面,下图 \ref{minicom配置界面}所示。minicom配置菜单在屏幕中央,每个菜单项都包括了一组配置。
	\begin{figure}[h]
		\centering
		\includegraphics[width=0.4\textwidth]{figures/08-01-minicom配置界面.png}
		\caption{minicom配置界面}
		\label{minicom配置界面}
	\end{figure}

	\item 用光标键移动高亮条到 Serial Port setup菜单项,按回车键后进入串口参数配置界面,如下图 \ref{minicom配置端口界面}所示。串口配置界面列出了串口的配置,每个配置前都有一个英文字母,代表进入配置项的快捷键。首先配置端口,输入小写字母a,光标移动到了/dev/tty8字符串最后,并且进入
	到编辑模式。以笔者机器为例,修改为/dev/tty0,代表连接到系统的第一个串口。
	\begin{figure}[h]
		\centering
		\includegraphics[width=0.6\textwidth]{figures/08-01-minicom配置端口界面.png}
		\caption{minicom配置端口界面}
		\label{minicom配置端口界面}
	\end{figure}

	\item 配置好串口设备后按回车键,保存参数并且回到提示界面。输入小写字母e,进入串口参数配置界面,如下图 \ref{minicom配置串口参数}所示。串口参数界面可以配置串口波特率、数据位、停止位等信息。一般只需要配置波特率,如在笔者机器上需要配置波特率是38400,输入小写字母d,屏幕上方current字符串后的波特率改变为38400。
	\begin{figure}[h]
		\centering
		\includegraphics[width=0.6\textwidth]{figures/08-01-minicom配置串口参数.png}
		\caption{minicom配置串口参数}
		\label{minicom配置串口参数}
	\end{figure}

	\item 设置好波特率后按回车键,保存退出,回到串口配置界面,如下图 \ref{minicom配置端口结束}所示。串口被设置为tty0,波特率是38400,其他配置使用默认设置。如果保存配置,直接按回车键退出。选择Save setup as dfl选项后按回车键,配置信息被保存为默认配置文件,下次启动的时候会自动加载。保存默认配置后,选择Exit选项后按回车键,退出配置界面,minicom自动进入终端界面。在终端界面会自动连接到串口,如果串口没有连接任何设备,屏幕右下角的状态提示为Offline。
	\begin{figure}[h]
		\centering
		\includegraphics[width=0.6\textwidth]{figures/08-01-minicom配置端口结束.png}
		\caption{minicom配置端口结束}
		\label{minicom配置端口结束}
	\end{figure}

	\item 退出 minicom,使用Ctrl+a键,然后输入字母z,出现minicom的命令菜单,如下图 \ref{minicom命令界面}所示。
	\begin{figure}[h]
		\centering
		\includegraphics[width=0.6\textwidth]{figures/08-01-minicom命令界面.png}
		\caption{minicom命令界面}
		\label{minicom命令界面}
	\end{figure}
\end{enumerate}



Minicom 提供了丰富的功能,是 Linux 中串口通信和远程管理的重要工具:

\begin{itemize}
	\item 调试和配置串口设备: Minicom 可以连接和调试各种串口设备,如调制解调器、路由器、交换机等。用户可以通过 Minicom 查看设备的输出信息,发送指令进行配置和调试。

	\item 远程终端连接: Minicom 可以作为一个终端仿真器,用于远程连接到其他计算机或设备。通过串口连接,用户可以在本地计算机上操作远程设备,进行远程管理和维护。

	\item 数据传输和文件传输: Minicom 支持通过串口进行数据传输,可用于传输文件、备份数据等。用户可以通过 Minicom 将文件发送到远程设备或从远程设备接收文件。

	\item 系统调试和故障排查: Minicom 可用于调试和排查系统故障。通过串口连接到系统控制台,用户可以查看系统的启动信息、错误日志等,帮助定位和解决问题。

	\item 嵌入式开发和调试: 对于嵌入式系统开发者来说,Minicom 是一个重要的工具。它可用于与嵌入式设备进行通信,进行程序下载、调试和测试。
\end{itemize}
