\chapter{NPUcore简介(Introduction)}

“NPUcore”是西北工业大学的操作系统内核构建实践型教学操作系统,曾获得2022年OSKernel大赛内核实现赛道一等奖。
NPUcore致力于使用Rust新型编程语言,帮助老师和学生自行研制一个操作系统微型内核,提升操作系统原理的实践体验并探索新型操作系统的设计与实现。
原始的2022版NPUcore具有内存管理、进程管理、文件系统核心系统调用功能,支持RISCV32/64指令集,可在对应的QEMU模拟器和SiFive-U740、K210等嵌入式开发板上运行。
该版本基于rCore-Tutorial迭代开发,重构90\%模块以支持Linux接口,共实现系统调用81个,且满分通过libc-test,是一个不错的baseline。
虽然该版本有着不错的性能,但却无法支持全部测例,以及国内自主研发的LoongArch龙芯架构。不仅如此,该版本不支持网络协议,EXT4文件系统,以及其它多种多样的外设,因此我们认为,这个版本仍然有很大的优化空间。
接下来,我们将分三部分介绍初版NPUcore,分别是:Rust特性、目标平台、系统调用。在这些章节中,我们也会提到我们的修改内容。最后我们会给出我们的文件目录树,搭建出新的整体NPUcore-IMPACT架构。

\section{Rust特性}

Rust是一个“安全、并发、实用”,支持函数式、并发式、过程式以及面向对象的程序设计风格的新型语言。
Rust在完全公开的情况下开发,并且相当欢迎社区的反馈。近些年,Rust语言在工业应用上的势头越来越猛。
基于Rust语言的种种特性,我们认为它更适合一些底层应用的开发,尤其是OSKernel。

\textbf{1. 我们为什么选择Rust作为OS编程语言?}
Rust 是一门内存安全的语言。对于 C/C++ 这样的手动管理内存的编程语言,我们在分配堆变量的时候需要调用 malloc/new函数,而当该变量使用完毕之后要手动调用
free/delete 回收内存。这就要求程序员需要关注所有堆变量的生命周期并及时将其释放,否则就会造成内存泄漏的问题,而过早的释放堆变量又可能造成“use-after-
free” 的问题。而 Rust 独特的所有权机制和借用检查,让编译器掌管变量的生
命周期,使得变量的回收变得可控,同时也杜绝了”use-after-free” 的问题,又不至于带来垃圾回收的开销。

Rust还能够推断出类型的大小,然后分配正确的内存大小并将其设置为您要求的值。但这意味着无法分配未初始化的内存:Rust没有null的概念。 
此外,所有这些检查都是在编译时完成的,因此没有运行时开销,这也是为什么Rust被成为是安全的“C”。 
如果你编写了正确的C++代码,你将编写出与C++代码基本上相同的Rust代码。而且由于编译器的帮忙,编写错误的代码版本是不可能的。
所以,我们选择Rust语言的原因,不仅是因为他安全,还因为其享有和C一样的速度,和更丰富的库。


\textbf{2.unsafe关键字}

几乎每个语言都有unsafe关键字,但Rust语言使用unsafe的原因可能与其它编程语言还有所不同。接下来我们展示一下unsafe的特性:
\begin{lstlisting}[language={Rust}, label={code:unsafe},
	caption={unsafe展示(r1 是一个裸指针)}]
	fn main() {
    let mut num = 5;

    let r1 = &num as *const i32;

    unsafe {
        println!("r1 is: {}", *r1);
    }
}
\end{lstlisting}
在代码块\ref{code:unsafe}中,r1 是一个裸指针(raw pointer),由于它具有破坏Rust内存安全的潜力,因此只能在unsafe代码块中使用,如果你去掉unsafe\{\},编译器会立刻报错。
在我们的NPUcore中,对于一个OS来说,安全是最大的保障,因此unsafe在初期NPUcore建设中给予了很大帮助。因为,即使做到小心谨慎,依然会有出错的可能性,但是 unsafe 语句块决定了:就算内存访问出错了,你也能立刻意识到,错误是在 unsafe 代码块中,而不花大量时间像无头苍蝇一样去寻找问题所在。
unsafe不安全,但是该用的时候就要用,在一些时候,它能帮助我们大幅降低代码实现的成本。虽然在网上充斥着“千万不要使用 unsafe,因为它不安全”的言论。事实上,我们认为unsafe是一个有效且必要的手段,因此我们选择遵循如下规则去使用:
\begin{enumerate}
    \item 没必要用时,就不用;
    \item 当有必要用时,就大胆用,但是要控制好边界;
    \item 尽量保证unsafe的边界范围最小。
\end{enumerate}





