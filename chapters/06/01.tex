\chapter{初识块设备}
\section{驱动程序}

驱动程序是一种软件组件,是操作系统与机外设之间的接口,可让操作系统和设备彼此通信。从操作系统架构上看,驱动程序与I/O设备靠的更近,离应用程序更远,这使得驱动程序需要站在协助所有进程的全局角度来处理各种I/O操作。这也就意味着在驱动程序的设计实现中,尽量不要与单个进程建立直接的联系,而是在全局角度对I/O设备进行统一处理。
	
上面只是介绍了CPU和I/O设备之间的交互手段。如果从操作系统角度来看,我们还需要对特定设备编写驱动程序。它一般需包括如下一些操作:
	
\begin{itemize}
	\item 定义设备相关的数据结构,包括设备信息、设备状态、设备操作标识等
	\item 设备初始化,即完成对设备的初始配置,分配I/O操作所需的内存,设置好中断处理例程
	\item 如果设备会产生中断,需要有处理这个设备中断的中断处理例程(Interrupt Handler)
	\item 根据操作系统上层模块(如文件系统)的要求(如读磁盘数据),给I/O设备发出命令,检测和处理设备出现的错误
	\item 与操作系统上层模块或应用进行交互,完成上层模块或应用的要求(如上传读出的磁盘数据)
\end{itemize}
	
从驱动程序I/O操作的执行模式上看,主要有两种模式的I/O操作:异步和同步。同步模式下的处理逻辑类似函数调用,从应用程序发出I/O请求,通过同步的系统调用传递到操作系统内核中,操作系统内核的各个层级进行相应处理,并最终把相关的I/O操作命令转给了驱动程序。一般情况下,驱动程序完成相应的I/O操作会比较慢(相对于CPU而言),所以操作系统会让代表应用程序的进程进入等待状态,进行进程切换。但相应的I/O操作执行完毕后(操作系统通过轮询或中断方式感知),操作系统会在合适的时机唤醒等待的进程,从而进程能够继续执行。
		
异步I/O操作是一个效率更高的执行模式,即应用程序发出I/O请求后,并不会等待此I/O操作完成,而是继续处理应用程序的其它任务(这个任务切换会通过运行时库或操作系统来完成)。调用异步I/O操作的应用程序需要通过某种方式(比如某种异步通知机制)来确定I/O操作何时完成。注:这部分可以通过协程技术来实现,但目前我们不会就此展开讨论。
	
编写驱动程序代码其实需要的知识储备还是比较多的,需要注意如下的一些内容:
	
\begin{itemize}
	\item 了解硬件规范:从而能够正确地与硬件交互,并能处理访问硬件出错的情况;
	\item 了解操作系统,由于驱动程序与它所管理的设备会同时执行,也可能与操作系统其他模块并行/并发访问相关共享资源,所以需要考虑同步互斥的问题(后续会深入讲解操作系统同步互斥机制),并考虑到申请资源失败后的处理;
	\item 理解驱动程序执行中所在的可能的上下文环境:如果是在进行中断处理(如在执行 trap\_handler 函数),那是在中断上下文中执行;如果是在代表进程的内核线程中执行后续的I/O操作(如收发TCP包),那是在内核线程上下文执行。这样才能写出正确的驱动程序。
\end{itemize}

