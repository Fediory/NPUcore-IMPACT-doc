\subsection{read}
read函数的实现逻辑基本与write函数相同,只在对块缓存的处理上有不同之处,所以在这里对重复的内容不在赘述,只进行不同之处的讲解。

在read函数的块级操作中,与write的区别主要体现在read_at_block_cache_rlock方法的循环处理中。
write中循环处理是将用户缓存区写入块缓存中,但是在read中调用的是块级的.get_cache().lock.().read()方法,
是将块缓存的内容(src)copy到用户到的buf(dst)中,最后返回给用户以达到读文件的效果。除此之外,read与write的另一不同之处在于read并没有追加模式。(write有对于追加写的判断)

\begin{lstlisting}[
    language={Rust},
    caption={read与write实现的不同之处}
]
.get_cache(
    start_cache,
    || -> Vec<usize> { self.get_neighboring_sec(&lock.clus_list, start_cache) },
    &self.fs.block_device,
)
.lock()
// I know hardcoding 4096 in is bad, but I can't get around Rust's syntax checking...
.read(0, |data_block: &[u8; 4096]| {
    let dst = &mut buf[read_size..read_size + block_read_size];
    let src = &data_block[start % PageCacheManager::CACHE_SZ
        ..start % PageCacheManager::CACHE_SZ + block_read_size];
    dst.copy_from_slice(src);
});
\end{lstlisting}

