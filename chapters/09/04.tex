\section{重要系统调用实现}
\subsection{mmap}
\subsubsection{内存映射}
内存映射,简而言之就是将用户空间的一段内存区域映射到内核空间,映射成功后,用户对这段内存区域的修改可以直接反映到内核空间,同样,内核空间对这段区域的修改也直接反映用户空间。
以下是一个把普遍文件映射到用户空间的内存区域的示意图。

\begin{figure}[H]
    \centering
    \caption[short]{内存映射}
    \includegraphics[width=0.8\linewidth]{figures/09-04-mmap-内存映射.png}
\end{figure}

\subsubsection{mmap主要用途}
\textbf{传统的读写文件}

一般来说,修改一个文件的内容需要如下3个步骤:把文件内容读入到内存中。修改内存中的内容。把内存的数据写入到文件中。

从传统读写文件的过程中,我们可以发现有个地方可以优化:如果可以直接在用户空间读写页缓存,那么就可以免去将页缓存的数据复制到用户空间缓冲区的过程。
那么,有没有这样的技术能实现上面所说的方式呢?答案是肯定的,就是mmap。

\textbf{使用mmap读写文件}

mmap用于把文件映射到用户空间中,简单说mmap就是把一个文件的内容在内存里面做一个映像。那么对于内核空间与用户空间两者之间需要大量数据传输等操作的效率是非常高的。进程可以像读写内存一样对普通文件的操作。
mmap系统调用也可以使得进程之间通过映射同一个普通文件实现共享内存。

UNIX网络编程第二卷进程间通信对mmap函数进行了说明。该函数主要用途有三个:
\begin{enumerate}
    \item 将一个普通文件映射到内存中,通常在需要对文件进行频繁读写时使用,这样用内存读写取代I/O读写,以获得较高的性能;
    \item 将特殊文件进行匿名内存映射,可以为关联进程提供共享内存空间;
    \item 为无关联的进程提供共享内存空间,一般也是将一个普通文件映射到内存中。
\end{enumerate}

\subsubsection{mmap实现}
下面将结合npucore的代码来介绍mmap的实现

从上文可知,要实现mmap,就要实现用户空间到内核空间的内存映射。所以mmap会分别确定要映射的用户空间和内存空间。

\textbf{用户空间}

mmap会申请一块适合的虚拟内存作为待映射的用户空间
\begin{lstlisting}[language={Rust}, label={code:new_area},
	caption={os/src/mm/memory_set.rs}]
let area: &mut MapArea = &mut self.areas[idx];
let start_va = area.inner.vpn_range.get_end()
let end_va = start_va + len;

let mut new_area: MapArea = MapArea::new(
    start_va,
    end_va,
    MapType::Framed,
    map_perm,
    map_file,
);
\end{lstlisting}
MapArea是一个描述虚拟地址的结构体,它指向某块用户空间的首尾,mmap在获取该结构体后会将其链入其进程管理的所有虚拟地址中。

\begin{figure}[H]
    \centering
    \caption[short]{虚拟地址结构体}
    \includegraphics[width=0.8\linewidth]{figures/09-04-mmap-虚拟地址结构体.png}
\end{figure}

\textbf{内核空间}

mmap获取待映射的内核空间是通过进程打开的文件描述符获取的
\begin{lstlisting}[language={Rust}, label={code:new_area},
	caption={os/src/mm/memory_set.rs}]
let fd_table = task.files.lock();
match fd_table.get_ref(fd) {
    Ok(file_descriptor) => {
        if !file_descriptor.readable() {
            return EACCES;
        }
        let file = file_descriptor.file.deep_clone();
        file.lseek(offset as isize, SeekWhence::SEEK_SET).unwrap();
        new_area.map_file = Some(file);
    }
    Err(errno) => return errno,
}
\end{lstlisting}
npucore中将文件对应的内核空间file存入了new\_area.map\_file中
这样既确定了要映射的用户空间和内存空间,也完成了它们之间的映射。

\textbf{其他}

mmap支持文件映射和匿名映射,如果是文件映射,则会通过上述的确定内核空间的过程,如果是匿名映射,那么该进程只是申请了一块空间,这片空间不与任何文件关联。

\subsection{munmap}

\subsubsection{munmap用途与实现}
munmap执行与mmap相反的操作,也就是删除用户空间与内核空间的映射。

munmap会通过用户传入的地址信息删除页表中的有关映射,并将在mmap中新插入的Maparea删除。

因大部分内容已在mmap介绍过,这里不做赘述。

\subsection{execve}
\subsubsection{execve函数整体介绍}
execve 用一个新的程序来代替当前进程的内存和寄存器,但是其文件描述符、进程 id 和父进程都是不变的。

它根据文件系统中保存的某个文件来初始化用户部分。execve通过 open()打开二进制文件。然后,它读取 ELF 头。应用程序以通行的 ELF 格式来描述。一个 ELF 二进制文件包括了一个 ELF 头,然后是连续几个程序段的头。

exec 会检查文件是否包含 ELF 二进制代码。一个 ELF 二进制文件是以4个“魔法数字”开头的,即 0x7F,“E”,“L”,“F”。如果 ELF 头中包含正确的魔法数字, exec  就会认为该二进制文件的结构是正确的。
\begin{lstlisting}[language={Rust}, 
    caption={execve输入参数与返回值}]
pub fn sys_execve(
    pathname: *const u8,
    mut argv: *const *const u8,
    mut envp: *const *const u8,
) -> isize
\end{lstlisting}
\begin{itemize}
    \item pathname 要执行的新程序的文件路径。
    \item argc 参数数组,用于传递给新程序的命令行参数。
    \item envp 环境变量数组,用于设置新程序的环境变量。
    \item 返回值 成功,返回SUCCESS,否则,返回对应的错误码
\end{itemize}
\subsubsection{整体流程}
整个execve函数的整体流程图展示如下:
\begin{figure}[H]
    \centering
    \scalebox{0.5}{\includegraphics{figures/09-04-execve-流程图.png}}
    \caption{execve流程图}
\end{figure}

\subsubsection{代码详析}
下面将以NPUcore为例,讲述execve的实现。

1、函数的输入参数与返回值情况,已经在前面讲过,在此不过多赘述。
\begin{lstlisting}[language={Rust}, 
    caption={execve输入参数与返回值}]
pub fn sys_execve(
    pathname: *const u8,
    mut argv: *const *const u8,
    mut envp: *const *const u8,
) -> isize {
    //load_elf 成功
    SUCCESS
    //失败
    Err
}
\end{lstlisting}
2、将argv和envp变量字符串向量,通过循环,将*const *const u8转为Vec<String>数据类型。为后面的执行做准备。
\begin{lstlisting}[language={Rust}, 
    caption={execve参数转换}]
let mut argv_vec: Vec<String> = Vec::with_capacity(16);
    let mut envp_vec: Vec<String> = Vec::with_capacity(16);
    if !argv.is_null() {
        loop {
            let arg_ptr = match translated_ref(token, argv) {
                Ok(argv) => *argv,
                Err(errno) => return errno,
            };
            if arg_ptr.is_null() {
                break;
            }
            argv_vec.push(match translated_str(token, arg_ptr) {
                Ok(arg) => arg,
                Err(errno) => return errno,
            });
            unsafe {
                argv = argv.add(1);
            }
        }
    }
    if !envp.is_null() {
        loop {
            let env_ptr = match translated_ref(token, envp) {
                Ok(envp) => *envp,
                Err(errno) => return errno,
            };
            if env_ptr.is_null() {
                break;
            }
            envp_vec.push(match translated_str(token, env_ptr) {
                Ok(env) => env,
                Err(errno) => return errno,
            });
            unsafe {
                envp = envp.add(1);
            }
        }
    }
\end{lstlisting}
3、debug 层信息输出
\begin{lstlisting}[language={Rust}, 
    caption={execve的debug信息}]
debug!(
    "[exec] argv: {:?} /* {} vars */, envp: {:?} /* {} vars */",
    argv_vec,
    argv_vec.len(),
    envp_vec,
    envp_vec.len()
);
\end{lstlisting}
4、准备将 ELF 文件的内容载入当前进程,要对读取的文件做以下检查:
\begin{itemize}
    \item 文件大小大于等于 4
    \item 同时,ELF文件以 4 位魔数"7fELF"开头,因此应检查文件首是否有此魔数。 
\end{itemize}

\begin{lstlisting}[language={Rust}, 
    caption={对elf文件检查}]
match working_inode.open(&path, OpenFlags::O_RDONLY, false) {
    Ok(file) => {
        if file.get_size() < 4 {
            return ENOEXEC;
        }
        let mut magic_number = Box::<[u8; 4]>::new([0; 4]);
        // this operation may be expensive... I'm not sure
        file.read(Some(&mut 0usize), magic_number.as_mut_slice());
        let elf = match magic_number.as_slice() {
            b"\x7fELF" => file,
            b"#!" => {
                let shell_file = working_inode
                    .open(DEFAULT_SHELL, OpenFlags::O_RDONLY, false)
                    .unwrap();
                argv_vec.insert(0, DEFAULT_SHELL.to_string());
                shell_file
            }
            _ => return ENOEXEC,
        };

\end{lstlisting}
5、真正的载入过程。 loaf_elf 将当前的elf文件内容覆盖当前的进程。show_frame_consumption! 宏输出对应的信息。
\begin{lstlisting}[language={Rust}, 
    caption={execve载入可执行文件}]
let task = current_task().unwrap();
        show_frame_consumption! {
            "load_elf";
            if let Err(errno) = task.load_elf(elf, &argv_vec, &envp_vec) {
                return errno;
            };
        }
        // should return 0 in success
        SUCCESS
    }
    Err(errno) => errno,
}

\end{lstlisting}
6、深入load_elf()函数,了解execve的核心实现

可以很直观的发现,execve调用的elf载入函数load_elf()与
os::task::task::TaskControlBlock pub fn new(elf: FileDescriptor) -> Self 
第一个进程初始化函数的整体内容是比较相似的,都包括
\begin{itemize}
    \item 将elf文件映射到内存地址空间中
    \item 分配用户态资源
    \item 设置tcb信息
\end{itemize}
这些核心步骤。读者学习的时候不妨比照来阅读学习。

与new函数不同的是,load_elf会在原有的函数基础之上,进行部分信息的覆写。
首先,load_elf会覆写新进程的trap_cx对应的内容,并将其地址写入新进程的trap_cx_ppn中。

\begin{lstlisting}[language={Rust}, 
    caption={load_elf信息覆写-1}]
 // initialize trap_cx
    let trap_cx = TrapContext::app_init_context(
        if let Some(interp_entry) = elf_info.interp_entry {
            interp_entry
        } else {
            elf_info.entry
        },
        user_sp,
        KERNEL_SPACE.lock().token(),
        self.kstack.get_top(),
        trap_handler as usize,
    );
    // **** hold current PCB lock
    let mut inner = self.acquire_inner_lock();
    // update trap_cx ppn
    inner.trap_cx_ppn = (&memory_set)
        .translate(VirtAddr::from(self.trap_cx_user_va()).into())
        .unwrap()
        .ppn();
    *inner.get_trap_cx() = trap_cx;
\end{lstlisting}

第二,刷新子进程tid,堆指针等信息

\begin{lstlisting}[language={Rust}, 
    caption={load_elf信息覆写-2}]
// clear clear_child_tid
inner.clear_child_tid = 0;
// clear robust_list
inner.robust_list = RobustList::default();
// update heap pointers
inner.heap_bottom = program_break;
inner.heap_pt = program_break;
\end{lstlisting}

最后,刷新tcb的elf、fd表、memory_set、futex等信息,并返回载入成功的标识。

\begin{lstlisting}[language={Rust}, 
    caption={load_elf信息覆写-3}]
// track the change of ELF file
    *self.exe.lock() = elf;
    // flush cloexec fd
    self.files.lock().iter_mut().for_each(|fd| match fd {
        Some(file) => {
            if file.get_cloexec() {
                *fd = None;
            }
        }
        None => (),
    });
    // substitute memory_set
    *self.vm.lock() = memory_set;
    // flush signal handler
    for sigact in self.sighand.lock().iter_mut() {
        *sigact = None;
    }
    // flush futex
    self.futex.lock().clear();
    if self.tid_allocator.lock().get_allocated() > 1 {
        let mut manager = TASK_MANAGER.lock();
        // destory all other threads
        manager
            .ready_queue
            .retain(|task| (*task).tgid != (*self).tgid);
        manager
            .interruptible_queue
            .retain(|task| (*task).tgid != (*self).tgid);
    };
    Ok(())
\end{lstlisting}