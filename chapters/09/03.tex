\section{支持BusyBox所需系统调用}
\subsection{获取系统调用相关信息}
Posix规范规定了一系列的系统调用,linux提供了查看系统调用相关信息的命令,如下所示:
\begin{lstlisting}[language=bash]
    man 2 open 
\end{lstlisting}
可以查看open系统调用的相关信息,如下所示:
\begin{figure}[H]
  \centering
  \includegraphics[width=0.8\textwidth]{figures/09-03-open系统调用示意图.jpg}
  \caption{open系统调用相关信息}
  \label{fig:open_syscall}
\end{figure}
每个man手册页面都有类似的结构,由以下部分组成
\begin{itemize}
    \item NAME:命令的名称和简短描述
    \item SYNOPSIS:命令的语法和选项 初步描述了命令的基本用法
    \item DESCRIPTION:命令的详细描述和用法
    \item OPTIONS:命令的选项和参数
    \item EXAMPLES:使用命令的示例
    \item FILES:命令可能使用的文件
    \item SEE ALSO:相关命令和手册页面
    \item BUGS:已知的命令错误和限制
    \item AUTHOR:命令的作者和贡献者
\end{itemize}
通过在文档中找到我们想要的规范信息,我们可以知道实现系统调用的细节规范,如报错类型,返回类型,接口参数等。

接下来,我们以open系统调用为例,来看看如何实现系统调用。
第一步,需要从linux man page中找到open系统调用的系统调用号和函数签名,并将其如下所示:

第二步,分析该系统调用设计的模块,具体而言是需要实现的功能与内核的哪一个部分关联,例如内存映射、进程管理、进程通信、线程相关等。
之后,分析具体实现的框架,也就是做那些操作,我们需要的函数接口是哪些,从而可以得出一个实现的伪代码。

最后,在代码实现上,关注手册中的细节,例如可能出现的错误信息,对于特殊情况的处理。

如此,我们就完整的实现了一个系统调用。类似的对于其他的系统调用,我们都可以类似的实现。