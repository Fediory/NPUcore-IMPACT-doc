\section{文件描述符层}
一个进程可以访问的多个文件,所以在操作系统中需要有一个管理
进程访问的多个文件的结构,这就是 \textbf{文件描述符表} (File
Descriptor Table) ,其中的每个 \textbf{文件描述符} (File Descriptor) 代
表了一个特定读写属性的I/O资源。

为简化操作系统设计实现,可以让每个进程都带有一个线性的 \textbf{文
件描述符表} ,记录该进程请求内核打开并读写的那些文件集合。
而 \textbf{文件描述符} (File Descriptor) 则是一个非负整数,表示\textbf{文件描述符表}中一个打开的\textbf{ 文件描述符} 所处的位置(可理解为数组下
标)。进程通过\textbf{文件描述符},可以在自身的\textbf{文件描述符表}中找到对
应的文件记录信息,从而也就找到了对应的文件,并对文件进行读
写。当打开( open )或创建( create ) 一个文件的时
候,一般情况下内核会返回给应用刚刚打开或创建的文件对应的文
件描述符;而当应用想关闭( close )一个文件的时候,也需
要向内核提供对应的\textbf{文件描述符},以完成对应文件相关资源的回收
操作。

因为 OSInode 也是一种要放到进程\textbf{文件描述符表}中文件,并可
通过 sys\_read/write 系统调用进行读写操作,因此我们也需
要为它实现 File Trait :
\begin{lstlisting}[language=Rust]
fn readable(&self) -> bool {
	self.readable
}
fn writable(&self) -> bool {
	self.writable
}
fn read(\&self, offset: Option<\&mut usize>, buffer: \&mut [u8]) -> usize {
	match offset {
		Some(offset) => {
			let len =
			self.inner.read_at_block_cache(*offset,
			buffer);
			*offset += len;
			len
		}
		None => {
			let mut offset =
			self.offset.lock();
			let len =
			self.inner.read_at_block_cache(*offset,
			buffer);
			*offset += len;
			len
		}
	}
}

fn write(\&self, offset: Option<\&mut
usize>, buffer: \&[u8]) -> usize {
	match offset {
		Some(offset) => {
			let len =
			self.inner.write_at_block_cache(*offset,
			buffer);
			*offset += len;
			len
		}
		None => {
			let mut offset =
			self.offset.lock();
			let inode_lock =
			self.inner.write();
			if self.append {
				*offset =
				self.inner.get_file_size_wlock(\&inode_lock)
				as usize;
			}
			let len = self
			.inner
   .write_at_block_cache_lock(\&inode_lock,
   *offset, buffer);
   *offset += len;
   len
}
}
}
\end{lstlisting}

read 将数据从文件读取到提供的缓冲区中。 offset 参数是
一个可选的 usize 的可变引用,允许指定开始读取的偏移量。如
果提供了 Some(offset) ,则从指定的偏移量开始读取,并相
应地更新偏移量。如果提供了 None ,该方法锁定偏移量,使用
当前偏移量从文件中读取数据,并相应地更新偏移量。

write 将提供的缓冲区中的数据写入文件。与 read 方法类
似,它接受一个可选的 usize 的可变引用作为偏移量参数。如果
提供了 Some(offset) ,则从指定的偏移量开始写入,并相应
地更新偏移量。如果提供了 None ,该方法锁定偏移量,获取文
件inode的写锁,然后将数据写入文件。如果设置了 append 标
志,它在写入之前将偏移量更新为文件的末尾。

同时,在NPUcore中,使用文件描述符表 Fdtable 来进行对文
件描述符的管理:

\begin{lstlisting}[language=Rust]
pub struct FdTable {
	inner: Vec<Option<FileDescriptor>>,
	recycled: Vec<u8>,
	soft_limit: usize,
	hard_limit: usize,
}
\end{lstlisting}

可以理解为文件描述符表内部存在一个向量组,我们通过open等
操作得到的文件描述符(非负整数),对应的就是该数组的下标。

\begin{lstlisting}[language=Rust]
pub struct FileDescriptor {
	cloexec: bool,
	nonblock: bool,
	pub file: Arc<dyn File>,
}
\end{lstlisting}

所以在一个进程中,活跃的文件被存放在文件描述符表中,我们通
过文件描述符表获得我们需要的文件描述符,从而找到我们需要操
作的文件,进行处理。