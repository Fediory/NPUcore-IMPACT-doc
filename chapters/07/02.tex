\section{块设备接口层}

块设备接口层为文件系统提供了对块设备进行读写的操作接口,其代码在easy-fs/src/block\_dev.rs中。作为easy-fs库的最底层,其声明了一个块设备的抽象接口BlockDevice,该trait定义了设备驱动所需要实现的读写接口,如下:

\begin{lstlisting}[language={Rust}, label={code:blockdevice},
	caption={easy-fs/src/block\_dev.rs}]
pub trait BlockDevice: Send + Sync + Any {
	/// Read block from BlockDevice
	/// The function panics when the size of 'buf' is not a multiple of BLOCK_SZ
	fn read_block(\&self, block_id: usize, buf: \&mut [u8]);
	
	/// Write block into the file system.
	/// The function panics when the size of 'buf' is not a multiple of BLOCK_SZ
	fn write_block(\&self, block_id: usize, buf: \&[u8]);
	
	/// We should rewrite the API for K210 since it supports NATIVE multi-block clearing
	fn clear_block(\&self, block_id: usize, num: u8) {
		self.write_block(block_id, \&[num; BLOCK_SZ]);
	}
	
	/// We should rewrite the API for K210 if it supports NATIVE multi-block clearing
	fn clear_mult_block(\&self, block_id: usize, cnt: usize, num: u8) {
		for i in block_id..block_id + cnt {
			self.write_block(i, \&[num; BLOCK_SZ]);
		}
	}
}
\end{lstlisting}

该trait主要实现两个抽象方法:

\begin{itemize}
	\item [$\bullet$]
	read\_block:将编号为block\_id的块从磁盘读入内存中的缓冲区buf;
	\item [$\bullet$]
	write\_block:将内存中的缓冲区buf中的数据写入磁盘编号为block\_id的块。
\end{itemize}

在easy-fs中并没有一个实现了BlockDevice trait的具体类型,因为块设备仅支持以块为单位进行随机读写,而这两个方法需要由具体的块设备驱动来实现。实际上,这是需要由文件系统的使用者(比如操作系统内核或直接测试easy-fs文件系统的easy-fs-fuse应用程序)提供并接入到easy-fs库的。

块设备接口层及其以下的设备驱动层实现后,向上为接下来要介绍的easy-fs库的块缓存层提供这两个方法调用,进行块缓存的管理。换句话说,easy-fs可以访问实现了BlockDevice trait的块设备驱动程序。

同时,Npucore为了实现K210提供的关于用给定字符清空单个或若干个连续磁盘块的API,还对该trait的write\_block方法进行了进一步封装,增加了clear\_block与clear\_mult\_block方法,实现了按照fat32文件系统的磁盘块大小,写入给定字符的功能。至于更复杂的功能,则需要由上层层次进行实现。

Npucore目前实现了对SD卡的驱动实现。在os/src/drivers/block/sdcard.rs中对上述trait进行了实现,通过K210依赖中提供的GPIO端口SPI串口底层支持,可以对SD卡直接进行操作。具体代码如下:

\begin{lstlisting}[language={Rust}, label={code:blockdevice},
	caption={os/src/drivers/block/sdcard.rs}]
impl BlockDevice for SDCardWrapper {
	fn read_block(&self, block_id: usize, buf: &mut [u8]) {
		let lock = self.0.lock();
		let mut result = lock.read_sector(buf, block_id as u32);
		let mut cont_cnt = 0;
		while result.is_err() {
			if cont_cnt >= 0 {
				log::error!("[sdcard] read_sector(buf, {}) error. Retrying...", block_id);
				result = lock.read_sector(buf, block_id as u32);
			}
			cont_cnt += 1;
			if cont_cnt >= 5 {
				log::error!(
				"[sdcard] read_sector(buf[{}], {}) error exceeded contineous retry count, waiting...",
				buf.len(),
				block_id
				);
				Self::wait_for_one_sec();
				if lock.read_sector(buf, block_id as u32).is_err() {
					lock.init();
					Self::wait_for_one_sec();
				} else {
					break;
				}
				cont_cnt = 0;
			}
		}
	}
	fn write_block(&self, block_id: usize, buf: &[u8]) {
		let lock = self.0.lock();
		let mut result = lock.write_sector(buf, block_id as u32);
		let mut cont_cnt = 0;
		while result.is_err() {
			if cont_cnt >= 0 {
				log::error!(
				"[sdcard] write_sector(buf, {}) error. Retrying...",
				block_id
				);
				result = lock.write_sector(buf, block_id as u32);
			}
			cont_cnt += 1;
			if cont_cnt >= 5 {
				log::error!(
				"[sdcard] write_sector(buf[{}], {}) error exceeded contineous retry count, waiting...",
				buf.len(),
				block_id
				);
				Self::wait_for_one_sec();
				if lock.write_sector(buf, block_id as u32).is_err() {
					lock.init();
					Self::wait_for_one_sec();
				} else {
					break;
				}
				cont_cnt = 0;
			}
		}
	}
}
\end{lstlisting}

这两个方法实现了对块号为block\_id的SD存储区进行直接读写。注意,还设置了在I/O故障的情况的读写最大重试次数为5。