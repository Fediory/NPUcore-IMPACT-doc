\chapter{内核性能评估}

\section{Lmbench简介}

Lmbench是一个轻量级、可移植且符合ANSI/C标准的微型性能评估工具,专为UNIX/POSIX系统设计,旨在帮助系统开发者深入了解关键操作的基础成本。
lmbench的主要目标是衡量系统性能的两个关键特征:反应时间和带宽。

\textbf{带宽方面}的测试主要涵盖了以下几个方面:

\begin{itemize}
	
	\item 读取缓存文件(bw\_file\_rd)
	
	这项测试用于评估从硬盘到内存的文件读取速度,从而衡量文件读取时的性能。通过这个测试可以了解系统在读取缓存文件时的表现。
	
	\item 拷贝内存、读/写内存(bw\_mem\_*, bw\_mmap\_rd)
	
	bw\_mem\_cp (memory copy):评估进程CPU与内存之间数据传输的速度,主要衡量内存之间数据拷贝的效率。
	
	bw\_mem\_rd (memory reading and summing): 测试从内存读取数据的速度,通过读取数据并进行求和操作来评估内存读取的性能。
	
	bw\_mem\_wr (memory writing):评估向内存写入数据的速度,了解系统的内存写入性能。
	
	bw\_mmap\_rd (memory mapping reading):通过从硬盘到内存的映射并测试其读取速度,衡量内存映射文件读取的效率。
	
	\item 管道、TCP(bw\_pipe, bw\_tcp)
	
	bw\_pipe:测试管道数据传输的速度,通常指定了传输数据的量和管道的大小。这个测试可以衡量管道通信的效率。
	
	bw\_tcp:通过TCP/IP双向读取数据来评估系统的TCP通信性能。该测试可以提供关于TCP网络通信速度的信息。
	
\end{itemize}

\textbf{延时方面}的测试主要涵盖了以下几个方面:

\begin{itemize}
	
	\item  上下文切换
	
	上下文切换指的是在多任务系统中,从一个任务(或进程)切换到另一个任务时,保存和恢复进程状态所需的时间。Lmbench能够测量不同条件下的上下文切换时间,比如2p/0k、2p/16k 等参数。 这些测试条件表示在不同的进程大小和数量下进行的上下文切换。通过这些测试,可以了解系统在不同负载下切换上下文所需的时间,即便无任务执行(0k),也能评估出纯粹的上下文切换成本。
	
	\item 网络
	
	Lmbench涵盖了不同类型的网络通信方式的评估,包括TCP、UDP、RPC/TCP、RPC/UDP 等。 这些测试项目评估了通过不同协议进行网络通信时所需的时间。TCP和UDP代表了常见的传输层协议,而RPC则是远程过程调用。这些测试提供了在不同网络通信方式下的延迟情况,有助于评估网络性能。
	
	\item 文件系统建立和删除
	
	文件系统操作评估了文件操作的延迟,包括0K/10K File Create/Delete。 这些测试项目评估了在创建或删除不同大小的文件时所需的时间。通过这些测试,可以了解系统在文件系统操作上的性能表现,特别是针对文件的创建和删除操作。
	
	\item 进程创建
	
	Lmbench评估了创建新进程所需的时间,包括fork proc、exec proc、sh proc 等测试项目。 这些测试项目评估了创建新进程、执行新命令、使用系统shell来运行新程序等操作所需的时间。这些评估可帮助了解系统在处理进程创建时的效率。
	
	\item 信号处理
	
	对信号的处理效率也进行了评估,比如sig inst、sig hndl。 这些测试评估了信号的安装和处理所需的时间。这对于了解系统处理信号及其响应的效率非常重要,尤其是在面对频繁信号触发的情况下。
	
	\item 上层系统调用
	
	这部分测试涉及系统调用和相关操作的性能评估,比如null call、null I/O、stat、open clos 等测试项目。 这些测试项目评估了对系统调用和IO操作的响应速度。了解系统在这些操作上的性能可以帮助优化IO密集型任务或高频率系统调用的应用。
	
	\item 内存读入反应时间
	
	这一部分评估了内存操作的延迟和性能,包括lat\_pagefault、lat\_mem\_rd 等测试项目。 这些测试评估了硬盘与内存之间的分页错误、CPU与内存之间的读取延迟等。通过这些测试,可以了解系统在不同内存操作方面的性能表现。
	
\end{itemize}

lmbench常见命令及其作用如下表所示:

\begin{table}[!ht]
	\centering
	\begin{tabular}{|c|c|c|}
		\hline
		\textbf{针号} & \textbf{命令} & \textbf{作用} \\
		\hline
		1 & bw_file_rd & 测试从硬盘到内存的文件读取速度 \\
		2 & bw_mem_* & 测试进程CPU与内存之间的传输速度(包括内存拷贝、读写、映射等)\\
		3 & lat_pagefault & 测试硬盘与内存之间的分页错误延迟 \\
		4 & lat_ctx & 测试上下文切换所需的时间 \\
		5 & lat_proc & 测试进程创建的延迟 \\
		6 & lat_unix & 测试本地通信时延 \\
		7 & lat_tcp/lat_udp/lat_unix_connect & 测试网络连接时延 \\
		8 & lat_fs/lat_select & 测试文件系统操作的延迟 \\
		9 & lat_syscall &  测试系统调用的延迟\\
		10 & lat_mem_rd/lat_mmap & 测试内存读取延迟和内存到硬盘的映射延迟 \\
		\hline
	\end{tabular}
	\caption{lmbench常见命令及其作用}
	\label{lmbench常见命令及其作用}
\end{table}