\section{多种性能指标测试评估}

\subsection{NPUcore系统调用性能分析}

NPUcore性能得分如下:

\begin{table}
    \centering
    \begin{tabular}{|c|c|}
        \hline
        测例 & 耗时 \\
        \hline
        b\_malloc\_tiny2 (0) & 0.005085920 \\
        \hline
        b\_malloc\_big2 (0) & 0.028311200 \\
        \hline
        b\_malloc\_thread\_stress (0) & 0.066339520 \\
        \hline
        b\_string\_strstr ("abcdefghijklmnopqrstuvwxyz") & 0.011712000 \\
        \hline
        b\_string\_strstr ("azbycxdwevfugthsirjqkplomn") & 0.017838240 \\
        \hline
        b\_string\_strstr ("aaaaaaaaaaaaaacccccccccccc") & 0.011262640 \\
        \hline
        b\_string\_strstr ("aaaaaaaaaaaaaaaaaaaaaaaaac") & 0.010942000 \\
        \hline
        b\_string\_strstr ("aaaaaaaaaaaaaaaaaaaaaaaaaaaaaaaac") & 0.013523120 \\
        \hline
        b\_string\_memset (0) & 0.009740400 \\
        \hline
        b\_string\_strchr (0) & 0.011392640 \\
        \hline
        b\_string\_strlen (0) & 0.009787360 \\
        \hline
        b\_pthread\_createjoin\_serial1 (0) & 0.439336960 \\
        \hline
        b\_pthread\_createjoin\_serial2 (0) & 0.426977760 \\
        \hline
        b\_pthread\_create\_serial1 (0) & 0.476990480 \\
        \hline
        b\_pthread\_uselesslock (0) & 0.060514480 \\
        \hline
        b\_utf8\_bigbuf (0) & 0.032511520 \\
        \hline
        b\_utf8\_onebyone (0) & 0.091743280 \\
        \hline
        b\_stdio\_putcgetc (0) & 0.439399360 \\
        \hline
        b\_stdio\_putcgetc\_unlocked (0) & 0.426097680 \\
        \hline
        b\_regex\_compile ("(a|b|c)*d*b") & 0.057043200 \\
        \hline
    \end{tabular}
    \caption{Npucore的libc-bench}
\end{table}

\begin{table}
    \centering
    \begin{tabular}{|l|r|}
        \hline
        \textbf{测试名称} & \textbf{数值} \\
        \hline
        Unixbench DHRY2 test(lps) & 61405230 \\
        \hline
        Unixbench WHETSTONE test(MFLOPS) & 1269.670 \\
        \hline
        Unixbench SYSCALL test(lps) & 222615 \\
        \hline
        Unixbench CONTEXT test(lps) & 168113 \\
        \hline
        Unixbench PIPE test(lps) & 413979 \\
        \hline
        Unixbench SPAWN test(lps) & 64394 \\
        \hline
        Unixbench EXECL test(lps) & 73597 \\
        \hline
        Unixbench ARITHOH test(lps) & 7032261867 \\
        \hline
        Unixbench SHORT test(lps) & 179342297 \\
        \hline
        Unixbench INT test(lps) & 179660281 \\
        \hline
        Unixbench LONG test(lps) & 179644882 \\
        \hline
        Unixbench FLOAT test(lps) & 178725171 \\
        \hline
        Unixbench DOUBLE test(lps) & 179034748 \\
        \hline
        Unixbench HANOI test(lps) & 677685 \\
        \hline
        Unixbench EXEC test(lps) & 39493 \\
        \hline
    \end{tabular}
    \caption{Npucore的Unixbench测试结果}
    \label{Npucore的Unixbench测试结果}
\end{table}


\subsection{其他内核系统调用性能分析}


\subsubsection{Titanix}


\subsubsection{PLNTRY}

首先,从https://gitlab.eduxiji.net/PLNTRY/OSKernel2023-umi/-/blob/comp3-coverage地址处克隆仓库,
切换到master分支。将比赛提供的镜像文件复制到OSKernel2023-umi/third-party/img文件目录下,
命名为sdcard-comp2.img,在主文件目录下make all,make run即可运行该kernel。
注意,所有的测试结果不直接在终端显示,而在debug/ qemu.log 中显示。

下面将展示PLNTRY的性能得分情况,并与npucore进行简单的对比。

\begin{table}
    \centering
    \begin{tabular}{|c|c|}
        \hline
        测例 & 耗时 \\
        \hline
        b\_malloc\_sparse (0) & 0.749882000 \\
        \hline
        b\_malloc\_bubble (0) & 0.795564000 \\
        \hline
        b\_malloc\_tiny1 (0) & 0.020421000 \\
        \hline
        b\_malloc\_tiny2 (0) & 0.013895000 \\
        \hline
        b\_malloc\_big1 (0) & 0.205780000 \\
        \hline
        b\_malloc\_big2 (0) & 0.159222000 \\
        \hline
        b\_malloc\_thread\_stress (0) & 0.065488000 \\
        \hline
        b\_malloc\_thread\_local (0) & 0.057541000 \\
        \hline
        b\_string\_strstr ("abcdefghijklmnopqrstuvwxyz") & 0.014606000 \\
        \hline
        b\_string\_strstr ("azbycxdwevfugthsirjqkplomn") & 0.022971000 \\
        \hline
        b\_string\_strstr ("aaaaaaaaaaaaaacccccccccccc") & 0.014078000 \\
        \hline
        b\_string\_strstr ("aaaaaaaaaaaaaaaaaaaaaaaaac") & 0.013603000 \\
        \hline
        b\_string\_strstr ("aaaaaaaaaaaaaaaaaaaaaaaaaaaaaaaac") & 0.017628000 \\
        \hline
        b\_string\_memset (0) & 0.012005000 \\
        \hline
        b\_string\_strchr (0) & 0.014630000 \\
        \hline
        b\_string\_strlen (0) & 0.012690000 \\
        \hline
        b\_pthread\_createjoin\_serial1 (0) & 1.048143000 \\
        \hline
        b\_pthread\_createjoin\_serial2 (0) & 1.020008000 \\
        \hline
        b\_pthread\_create\_serial1 (0) & 2.689056000 \\
        \hline
        b\_pthread\_uselesslock (0) & 0.078151000 \\
        \hline
        b\_utf8\_bigbuf (0) & 0.037977000 \\
        \hline
        b\_utf8\_onebyone (0) & 0.117029000 \\
        \hline
        b\_stdio\_putcgetc (0) & 0.361184000 \\
        \hline
        b\_stdio\_putcgetc\_unlocked (0) & 0.339361000 \\
        \hline
        b\_regex\_compile ("(a|b|c)*d*b") & 0.075878000 \\
        \hline
        b\_regex\_search ("(a|b|c)*d*b") & 0.086110000 \\
        \hline
        b\_regex\_search ("a\{25\}b") & 0.256301000 \\
        \hline
    \end{tabular}
    \caption{libc-bench}
\end{table}

性能优秀的原因:
\begin{itemize}
    \item 内存管理:在页帧管理实现写时复制策略和通用 IO 缓
存,减少 IO 设备访问次数。
在地址空间管理实现懒分配策略,提高内存.

在设计UMI的页帧管理模块的部分,借鉴 Fuchsia 设计了一套 RAII 的基于二叉树形的数据结构。
每个节点逻辑上是其父节点的一个切片,有标志指示是否拥有写时复制(CoW)特性。
页帧通过引用计数和缓存状态的更新在树形结构中复制和流动。例如在提交页帧的时候,
依次从自身的哈希表、父节点的哈希表、I/O 后端读写、新清零页的顺序来依次访问并提
交页帧。

通过如上说明可以看到,Phys 结构体可以作为任意读写+寻址的后端的页缓存,包
括普通文件、块设备等。这样,虽然缺乏了一些特定场景的优化,可以避免重复的页缓
存代码。并且由于写时复制和懒分配两个特性,任何涉及到内存分配的场景都会获得对应的
性能提升。同时,每个节点可以实现同时读写页帧,在不浪费页帧的情况下提升页缓存的并
发性。
    \item 线程管理与调度:使用细粒度锁和 Rust 所有权系统管理线程
的本地状态和信息
支持软抢占和任务窃取的 SMP 多核调度器
基于有栈协程模式的特权级切换

传统的操作系统往往采用有栈协程将任务的调用栈和上下文分开保存,通过汇编代码手动
切换函数调用栈来进行任务切换。每个任务的调用栈都会有一定的内存浪费(空闲),并都
会有栈溢出风险。

而无栈协程则将任务的信息同一保存成状态机,统一存放在堆上,由执行器通过更改指
针来切换执行的任务。从图中我们可以发现,调用栈仅与每个执行器一一绑定,一定程度上
减少了内存浪费、降低栈溢出风险。

    \item 文件系统:统一的虚拟文件系统接口
支持 debugfs、FAT32、procfs、devfs 等多
种存储和内存文件系统。本身并没有很多创新之处。
    \item 网络协议栈:基于 smoltcp 的多设备接口网络协议栈
TCP 独立的 ACCEPT 队列
\end{itemize}

其结构图如下(引用于其文档):
\begin{figure}[H]
    \centering
    \includegraphics{figures/10-03-plntry.png}
    \caption{SD卡与K210接口}
\end{figure}

