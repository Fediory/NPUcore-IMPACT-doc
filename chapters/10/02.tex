\section{支持运行Lmbench}

\subsection{支持运行Lmbench}
要从lmbench二进制文件中提取出相关的系统调用,过程基本与提取busybox的系统调用类似。
仍然可以用一行命令完成。

\begin{lstlisting}[language=bash]
    # 从二进制文件中获取系统调用号,将objfile改为提取的目标二进制文件
    objdump -d objfile | grep -B 9 ecall | grep "li.a7" | tee syscall.txt

    # 如果报错:objdump: can't disassemble for architecture UNKNOWN! ,是由于当前的objdump并非RISC-V架构,尝试
    riscv64-linux-gnu-objdump -d busybox | grep -B 9 ecall | grep "li.a7" | tee syscall.txt

\end{lstlisting}

上述命令的详细解释,请参考第九章第三节,在此不再赘述。

在本节中,我们通过比较得到Lmbench中存在\textbf{17}条完全不同于Busybox中的系统调用。

\begin{table}[h]
\centering
\begin{tabular}{|c|c|c|}
\hline
\textbf{指令} & \textbf{系统调用} & \textbf{系统调用号} \\
\hline
li a7,101 & \_\_NR\_nanosleep & 101 \\
\hline
li a7,114 & \_\_NR\_clock\_getres & 114 \\
\hline
li a7,118 & \_\_NR\_sched\_setparam & 118 \\
\hline
li a7,119 & \_\_NR\_sched\_setscheduler & 119 \\
\hline
li a7,120 & \_\_NR\_sched\_getscheduler & 120 \\
\hline
li a7,121 & \_\_NR\_sched\_getparam & 121 \\
\hline
li a7,131 & \_\_NR\_tgkill & 131 \\
\hline
li a7,165 & \_\_NR\_getrusage & 165 \\
\hline
li a7,202 & \_\_NR\_accept & 202 \\
\hline
li a7,203 & \_\_NR\_connect & 203 \\
\hline
li a7,206 & \_\_NR\_sendto & 206 \\
\hline
li a7,207 & \_\_NR\_recvfrom & 207 \\
\hline
li a7,211 & \_\_NR\_sendmsg & 211 \\
\hline
li a7,212 & \_\_NR\_recvmsg & 212 \\
\hline
li a7,227 & \_\_NR\_msync & 227 \\
\hline
li a7,72 & \_\_NR\_pselect6 & 72 \\
\hline
li a7,82 & \_\_NR\_fsync & 82 \\
\hline
\end{tabular}
\caption{Lmbench系统调用}
\label{table:syscalls_compact}
\end{table}

针对这些系统调用,我们将其分为了以下\textbf{7}种类型:

\begin{itemize}
    \item \textbf{时间和定时器相关系统调用}

    \item \textbf{进程调度和管理系统调用}

    \item \textbf{信号和线程管理系统调用}

    \item \textbf{资源和性能监测系统调用}

    \item \textbf{网络编程相关系统调用}

    \item \textbf{文件操作和数据同步系统调用}

    \item \textbf{事件等待和选择系统调用}
\end{itemize}

接下来,我们将分类为大家讲解各个类别的系统调用功能。

\begin{itemize}
    \item \textbf{时间和定时器相关系统调用}
    \begin{itemize}
        \item \texttt{\_\_NR\_nanosleep (101)}: 用于导致当前进程休眠指定的时间,通常用于实现定时器功能。
        \item \texttt{\_\_NR\_clock\_getres (114)}: 用于获取指定时钟的分辨率(精度),可以用于测量时间间隔。
    \end{itemize}

    \item \textbf{进程调度和管理系统调用}
    \begin{itemize}
        \item \texttt{\_\_NR\_sched\_setparam (118)}: 用于设置指定进程的调度参数,例如设置进程的优先级等。
        \item \texttt{\_\_NR\_sched\_setscheduler (119)}: 用于设置指定进程的调度策略,例如设置为实时调度或普通调度。
        \item \texttt{\_\_NR\_sched\_getscheduler (120)}: 用于获取指定进程的当前调度策略。
        \item \texttt{\_\_NR\_sched\_getparam (121)}: 用于获取指定进程的调度参数,例如获取进程的优先级等。
    \end{itemize}

    \item \textbf{信号和线程管理系统调用}
    \begin{itemize}
        \item \texttt{\_\_NR\_tgkill (131)}: 用于向指定线程组(thread group)中的线程发送信号。
    \end{itemize}

    \item \textbf{资源和性能监测系统调用}
    \begin{itemize}
        \item \texttt{\_\_NR\_getrusage (165)}: 用于获取进程或子进程的资源使用情况,如CPU时间、内存使用等信息。
    \end{itemize}

    \item \textbf{网络编程相关系统调用}
    \begin{itemize}
        \item \texttt{\_\_NR\_accept (202)}: 用于在网络编程中接受来自客户端的连接请求。
        \item \texttt{\_\_NR\_connect (203)}: 用于在网络编程中建立与远程服务器的连接。
        \item \texttt{\_\_NR\_sendto (206)}: 用于在网络编程中向指定目标发送数据。
        \item \texttt{\_\_NR\_recvfrom (207)}: 用于在网络编程中从指定源接收数据。
        \item \texttt{\_\_NR\_sendmsg (211)}: 用于发送消息,通常用于进程间通信(IPC)。
        \item \texttt{\_\_NR\_recvmsg (212)}: 用于接收消息,通常用于进程间通信(IPC)。
    \end{itemize}

    \item \textbf{文件操作和数据同步系统调用}
    \begin{itemize}
        \item \texttt{\_\_NR\_msync (227)}: 用于将内存中的数据同步到文件中,通常用于文件映射。
        \item \texttt{\_\_NR\_fsync (82)}: 用于将指定文件的数据和元数据刷新到磁盘上,以确保数据持久化。
    \end{itemize}

    \item \textbf{事件等待和选择系统调用}
    \begin{itemize}
        \item \texttt{\_\_NR\_pselect6 (72)}: 用于在多个文件描述符上等待事件,并在事件发生时返回。
    \end{itemize}
\end{itemize}

\begin{table}[h]
\centering
\begin{tabular}{|c|c|c|}
\hline
\textbf{系统调用类别} & \textbf{系统调用} & \textbf{系统调用号} \\
\hline
时间和定时器相关 & \_\_NR\_nanosleep & 101 \\
& \_\_NR\_clock\_getres & 114 \\
\hline
进程调度和管理 & \_\_NR\_sched\_setparam & 118 \\
& \_\_NR\_sched\_setscheduler & 119 \\
& \_\_NR\_sched\_getscheduler & 120 \\
& \_\_NR\_sched\_getparam & 121 \\
\hline
信号和线程管理 & \_\_NR\_tgkill & 131 \\
\hline
资源和性能监测 & \_\_NR\_getrusage & 165 \\
\hline
网络编程相关 & \_\_NR\_accept & 202 \\
& \_\_NR\_connect & 203 \\
& \_\_NR\_sendto & 206 \\
& \_\_NR\_recvfrom & 207 \\
& \_\_NR\_sendmsg & 211 \\
& \_\_NR\_recvmsg & 212 \\
\hline
文件操作和数据同步 & \_\_NR\_msync & 227 \\
& \_\_NR\_fsync & 82 \\
\hline
事件等待和选择 & \_\_NR\_pselect6 & 72 \\
\hline
\end{tabular}
\caption{Lmbench分类系统调用}
\label{table:syscalls_compact}
\end{table}