\documentclass[12pt, a4paper]{ctexart}
\usepackage{graphicx} % Required for inserting images
\graphicspath{{Figures/}{logo/}} 
\usepackage{listings}
\usepackage{verbatim}
\usepackage{ragged2e}
\title{物理内存分配}
\author{B3-边程曦-化运涛-张田田-杨浩森}
%\date{November 2023}

 
\clearpage
\begin{document}
\justifying
	\clearpage
	\maketitle  
	\section{引言}
	我们已经实现了多级页表的分页管理机制,大大简化了物理内存分配的复杂度,每次分配和回收都以一个页面为单位,新建进程时地址空间为空,没有对应的物理页面的。随着进程的不断运行,逐渐申请物理页面,所占的物理内存不断增加。这就要求内核要给用户进程提供物理内存分配的功能。因此,我们需要通过以下几个方面来学习如何实现物理内存页面的分配管理。
	
	1. 划分内核在不同平台的可动态分配的物理地址空间范围。
	
	2. 实现物理页面的RAIIt特性,即生命周期随着页面的申请而分配,随着进程结束而释放。
	
	3. 实现栈式的物理内存分配管理,通过栈的维护来管理空闲的物理页面,实现分配时从栈顶弹出,回收时压栈的效果。
	
	4. 向用户进程提供申请和释放物理页面的接口方法。
	
	接下来我们先来看本节涉及到的数据结构关系:
\clearpage
\begin{figure}[h]
		\centering
		\includegraphics[width=14cm,height=11cm]{2.png}
		\caption{本节涉及的数据结构关系}
\end{figure}    
\justifying  
1.FrameAllocator trait:这是一个特性,定义了管理物理内存帧的标准行为。它规定了创建、分配、释放物理内存帧的方法。

2.FrameTracker 结构体:用于表示单个物理内存帧的状态,包括物理页号(PhysPageNum)以及一些操作方法。这些方法可能涉及物理内存的初始化、输出调试信息和回收等,用于创建和管理单个物理内存帧的状态。

3.FRAME\_ALLOCATOR 全局静态变量: 这是用于多线程环境下访问物理内存帧分配器的锁。它提供了对 FrameAllocator 实例的安全并发访问。

4.StackFrameAllocator 结构体:实现了 FrameAllocator 这个 trait,用于管理物理内存帧。它采用基于栈的策略来管理物理内存,实现了创建全局变量FRAME\_ALLOCATOR 的功能。除此之外,这个结构体还涉及到调用 FrameTracker 中的方法,这些方法用于对物理页面进行操作,如初始化、调试输出和回收等。

综上所述,StackFrameAllocator 结构体实现了 FrameAllocator 这个特性,它在管理物理内存帧的同时,通过调用 FrameTracker 中的方法对物理页面进行操作。同时,它创建了一个全局变量 FRAME\_ALLOCATOR,并提供了线程安全的访问物理内存分配器的机制。

 
	\section{物理内存分配器}
	物理内存分配器在计算机操作系统中扮演着非常重要的角色 ,它负责动态地分配和释放物理内存资源 ,以确保程序在运行时能够获得足够的物理内存资源 ,并尽可能地提高系统的稳定性和性能。
	\begin{figure}[h]
		\centering
		\includegraphics[width=10cm,height=10cm]{3.png}
	\end{figure}   
 
	物理内存分配器通常包括一个页面置换算法和一个内存映射表。页面置换算法用于在内存池中选择要 替换的页面 ,以腾出空间来分配新的页面; 内存映射表用于记录页面在内存中的位置和大小 ,以及对应的 虚拟地址。在分配物理内存时 ,物理内存分配器会根据页面大小和页面置换算法来决定哪些页面需要被替 换 ,并将新的页面分配到内存池中。在程序运 行过程中,如果程序需要的物理内存超过了当前可用的物理内存 ,物理内存分配器会根据一定的算法进行 内存释放 ,以腾出物理内存资源,然后从内存池中提取所需的物理内存 ,并将其分配给程序运行所需的页面。

	
	\subsection{物理内存空间}
	
	要想有效的管理物理内存 ,首先要清楚我们管理的内存空间范围。在 NPUcore 中 ,不同平台上的物理内存空间范围如下 :
	
	\begin{lstlisting}[language=R]
  // os\src\config .rs
  pub const MEMORY\_START : usize = 0x8000\_0000;
  #[cfg(all(not(feature = "board\_k210"), not(feature = "board\_fu740")))]
  pub const MEMORY\_END : usize = 0x809e\_0000;
  #[cfg(feature = "board\_k210")]
  pub const MEMORY\_END : usize = 0x8080\_0000;
  #[cfg(feature = "board\_fu740")]
  pub const MEMORY\_END : usize = 0x9000\_0000;
  pub const PAGE\_SIZE : usize = 0x1000;
    \end{lstlisting}

	在三个平台上 ,物理内存的起始物理地址 MEMORY\_START 均为  0x80000000 ,单个页面大小 PAGE\_SIZE 均为 0x1000 ,即 4096 字节。
	
	在 k210 上 ,我们硬编码整块物理内存的终止物理地址 MEMORY\_END 为 0x80800000 ,这意味着可用 内存大小为  8MiB 。
	
	在 fu740 上 ,   MEMORY\_END 为  0x9000\_0000 ,可用内存大小为 256MiB 。
	
	如果没有 在这两个平台上  (也就是在 qemu 模拟器上)  ,   MEMORY\_END 被设置为  0x809e\_0000 ,可用内存大小将  近 10MiB 。	
	\subsection{物理内容分配器初始化}
	
	在编写程序时 ,我们需要将内存分配给自己编写的代码和各种外部库 ,如果在此之前未进行正确的内存初始化 ,程序可能会出现各种问题 ,例如内存泄漏、程序崩溃等。内存初始化是计算机编程中非常重要的一个步骤 ,用以确保程序在运行时能够正确地分配和释放内存,避免不必要的内存浪费和错误它关系到程序能否正确运行以及运行的稳定性和可靠性。
	
	内存的初始化中包含物理内存分配器的初始化 ,在这一步骤中 ,使用上文提到的物理内存空间来初始 化内存分配器 ,让内存分配器清楚待分配的物理空间范围。
	
	\begin{lstlisting}[language=R]
  // os\src\mm\frame_allocator .rs
  pub fn init_frame_allocator() {
    	extern "C" {
	   fn ekernel();
    }
    FRAME_ALLOCATOR .write() .init(
    PhysAddr : :from(ekernel as usize) .ceil(),
    PhysAddr : :from(MEMORY_END) .floor(),
    	);
  }
	\end{lstlisting}
	
	ekernel 即为内核空间的代码和数据存放的末尾 ,此地址即为 MEMORY\_START ,从 MEMORY\_START 到 MEMORY\_END ,剩下的空间都将被 frame\_allocator 分配使用。
	
	\subsection{物理内存分配器接口}
	在了解了所分配的物理空间范围并完成物理内存分配器的初始化后,我们可以详细了解物理内存分配器提供的接口,以及基本的物理内存分配器需要实现哪些功能。
	
	接口描述:物理内存分配器的接口包括一个自身的 new() 方法,以及实现物理页面的分配和回收。需要注意的是,这里有一个未初始化的页面分配方法alloc uninit(),省去初始化操作将会缩短分配时间。
\begin{lstlisting}[language=R]
 //os\src\mm\frame_allocator.rs
   trait FrameAllocator {
   fn new()->Self;
   fn alloc(&mut self) -> Option<FrameTracker>;
   unsafe fn alloc_uninit(&mut self) -> Option<FrameTracker>;
   fn dealloc(&mut self , ppn : PhysPageNum);
  }
\end{lstlisting}
    
    RAII与FrameTracker在前文介绍地址空间时提到,FrameTracker绑定了每个物理页面作为物理页面追踪器,将物理页面追踪器映射到虚拟页面有利于RAII和页面查找;RAIl指的是FrameTracker与物理页面具有相同的生命周期,在获取物理页面构建FrameTracker时会将物理页面初始化,在drop FrameTracker时也会自动回收物理页面。
    
    \begin{lstlisting}[language=R]
 //os\src\mm\frame_allocator.rs
   pub struct FrameTracker {
     pub ppn:PhysPageNum,
   }
 ///RAII phantom for physical pages
    \end{lstlisting}

    new方法具体实现在new()方法中,获取物理页面号后,通过get\_dwords\_array()获取64字节的数组,逐个元素赋零以实现页面的初始化(清零)。
\begin{lstlisting}[language=R]
 pub fn new(ppn : PhysPageNum) -> Self {
 // page cleaning
  let dwords_array = ppn .get_dwords_array();
  for i in dwords_array {
	  *i = 0;
	}
  Self { ppn }
}
\end{lstlisting}
    
    new\_uninit()提高性能原因:循环赋零会产生时间开销。为了应对一些情况下不需要对页面进行清零操作的情况(比如上一节介绍的COW处理,在申请到新页面时直接进行完全拷贝),提供了new\_uninit()方法,它不执行清零操作,直接返回由物理页面号构造的FrameTracker实例。
    \begin{lstlisting}[language=R]
 pub unsafe fn new_uninit(ppn:PhysPageNum)->Self {
    Self {ppn }
  }
    \end{lstlisting}

正是在FrameTracker提供了new\_uninit的方法,才在上层的FrameAllocator中支持了alloc\_uninit。

	\begin{comment}
	清楚了所分配的物理空间范围 ,并对物理内存分配器进行初始化后 ,我们来具体看⼀下物理内存分配 器所提供的接口 ,了解⼀下⼀个基本的物理内存分配器需要实现什么功能。
	
	物理内存分配器的接口包含⼀个自身的new方法 ,和实现物理页面的分配和回收。注意这里有⼀个不 初始化的页面分配方法 ,省去初始化操作将会缩短分配时间。
   \begin{lstlisting}[language=R]
  // os\src\mm\frame_allocator .rs
  trait FrameAllocator {
    fn new() -> Self;
    fn alloc(&mut self) -> Option<FrameTracker>;
    unsafe fn alloc_uninit(&mut self) -> Option<FrameTracker>;
    fn dealloc(&mut self, ppn : PhysPageNum);
	\end{lstlisting}
	之前介绍地址空间的时候是 FrameTracker 绑定每个物理页面 ,作为物理页面追踪器。将物理页面追 踪器映射到虚拟页面有利于 RAII 和进行页面查找。  复习⼀下 RAII :在这里指 FrameTracker 与物理页面 具有相同的生命周期 ,在获取物理页面构建 FrameTracker 时就将物理页面初始化 ,在 drop FrameTracker 时也将自动回收物理页面。
	\begin{lstlisting}[language=R]
  // os\src\mm\frame_allocator .rs
  pub struct FrameTracker {
  	pub ppn : PhysPageNum,
  }
  /// RAII phantom for physical pages
  impl FrameTracker {
  	pub fn new(ppn : PhysPageNum) -> Self {
	  	// page cleaning
	  	let dwords_array = ppn .get_dwords_array();
	  	for i in dwords_array {
	  		*i = 0;
	  	}
	  	Self { ppn }
  	}
  	pub unsafe fn new_uninit(ppn : PhysPageNum) -> Self {
	  	Self { ppn }
  	}
  }
   \end{lstlisting}
	在 new 方法中 ,获取物理页号后通过 get\_dwords\_array 取得 64 字节的数组 ,逐个赋零来实现页面 的初始化  (清零)  。
	
	然而循环赋零是有时间开销的 ,为了应对有些情况不需要对页面进行清零操作 ,比如上⼀小节介绍到 COW 的处理 ,申请到新页面会直接对其进行完全拷贝 ,那么这种情况下清零就是不必要的开销。为了提 高性能 ,提供了 new\_uninit 的方法 ,不清零 ,直接返回由物理页号构造的 FrameTracker 。
	
	正是在 FrameTracker 提供了 new\_uninit 的方法 ,才在上层的 FrameAllocator 中支持 了 alloc\_uninit 。
    \end{comment}
	\subsection{全局物理内存分配器}
	
	全局物理内存分配器是一种用于分配和释放物理内存的机制 ,用于解决程序在运行时的内存分配 和释放问题。在计算机程序中 ,内存分配和释放通常是由不同的进程或线程进行的 ,这可能会导致内存泄 漏和其他问题。
	
	如果没有全局物理内存分配器 ,每个进程或线程都需要自己管理内存分配和释放 ,那么可能会出现以 下问题:
	
	1. 内存泄漏 :当程序在分配内存时 ,如果未能正确释放内存 ,则可能会导致内存泄漏。这会导致程序占 用的内存不断增加 ,最终导致程序崩溃或拒绝服务。
	
	2. 内存分配异常 :如果进程或线程需要分配的内存大小超过了系统可用的内存大小 ,则可能会导致内存 分配异常。这可能会导致程序崩溃或拒绝服务。
	
	3. 内存分配和释放的不稳定性 :每个进程或线程都需要自己管理内存分配和释放,这可能会导致内存分 配和释放的不稳定性。例如 ,某个进程或线程可能会在释放内存后不久就重新分配内存 ,从而导致内存泄漏或其他问题。
	
	为了解决这些问题 ,我们需要一个全局的物理内存分配器来负责管理程序运行时的物理内存分 配和释放。全局物理内存分配器可以确保内存分配和释放的稳定性和可靠性 ,减少内存泄漏和其他问题的 发生。此外 ,全局物理内存分配器还可以提高程序的性能和吞吐量 ,因为内存分配和释放不再由不同的进 程或线程单独管理 ,而是由一个统一的内存管理机制来负责。
	
	接下来我们深入 NPUcore 内部 ,探究一下 NPUcore 是如何实现全局物理内存分配器的。
	
	首先创建一下  StackFrameAllocator 的全局实例  FRAME\_ALLOCATOR  :
\begin{lstlisting}[language=R]
  // os\src\mm\frame_allocator .rs
    type FrameAllocatorImpl = StackFrameAllocator;
    lazy_static ! {
      pub static ref FRAME_ALLOCATOR : RwLock<FrameAllocatorImpl> =
    RwLock : :new(FrameAllocatorImpl : :new());
    }
\end{lstlisting}

	这里我们使用  RwLock\textless T\textgreater 来包裹栈式物理页帧分配器 ,加上一层读写锁。每次对该分配器进行操作之前 ,我们都需要先通过  FRAME\_ALLOCATOR .write() 拿到分配器的写权限 ,以保证排他性。
	
	公开给其他内核模块调用的分配/回收物理页帧的接口为 :
\begin{lstlisting}[language=R]
  // os\src\mm\frame_allocator .rs
  pub fn frame_alloc() -> Option<Arc<FrameTracker>> {
    FRAME_ALLOCATOR
   .write() .alloc() .map( |frame_tracker| Arc : :new(frame_tracker))
  }
  pub unsafe fn frame_alloc_uninit() -> Option<Arc<FrameTracker>> {
    FRAME_ALLOCATOR
   .write() .alloc_uninit() .map( |frame_tracker| Arc : :new(frame_tracker))
  }
  pub fn frame_dealloc(ppn : PhysPageNum) {
    FRAME_ALLOCATOR .write() .dealloc(ppn);
  }
\end{lstlisting}

	alloc 申请来的 frame\_tracker 用 Arc 包裹实现自动引用计数再返回。在 FrameTracker 被Drop的时候 ,会调用 frame\_dealloc 方法。
\begin{lstlisting}[language=R]
  // os\src\mm\frame_allocator .rs
  impl Drop for FrameTracker {
    // Automatically recycle the physical frame when
    fn drop(&mut self) {
      // println !("do drop at {}", self .ppn .0);
      frame_dealloc(self .ppn);
    }
   }
\end{lstlisting}

	这样就做到了当一个  FrameTracker 实例被回收的时候 ,它的  drop 方法会自动被编译器调用 ,通 过之前实现的  frame\_dealloc ,进一步调用 StackFrameAllocator 的 dealloc ,即可将它控制的物理 页帧回收以供后续使用。
	
	通过全局物理内存分配器的包装 ,在内核其他模块的视角下 ,  申请一个物理页面会返回 FrameTracker ,当 FrameTracker 生命期结束,物理页面也就被自动回收 ,体现了 RAII 的思想。
	\section{内存分配办法}
	全局物理内存分配器提供给内核其他模块申请物理内存的统一方法 ,而物理内存分配方法的具体实现 在内存分配器中。接下来让我们深入物理内存分配器内部,来探究一下在物理内存分配器内部是如何实现 物理内存分配的。
	
	在物理内存分配方法中 ,一种常用并且简单有效的管理策略为栈式内存管理。  它不但存取速度快 ,而 且数据存取操作十分简单 ,表示操作也比较容易 ,存储空间可用性较大 ,  占用存储空间也小,有很强的 数据抽象能力 ,程序代码表示简洁,工作方便。
	
	栈式管理 ,数据的读写只能从一端进行 ,遵循后进先出的原则 ,与堆放木柴一样 ,最后放进去的木柴 在上面 ,是下次最先取出的一块木柴。
	
	NPUcore 使用的正是栈式的内存管理策略 ,下面我们将以 NPUcore 的具体实现为例 ,来介绍物理内 存分配方法。
	\subsection{栈式内存管理}
	首先用一个结构体来记录空闲的物理页面 :
	
	pub struct StackFrameAllocator {
		
        current: usize,     
		
        end: usize,         
		
        recycled: Vec<usize>,}
	
	为这个 StackFrameAllocator 实现 FrameAllocator 接口 ,   new 方法将空闲内存区间两端均设为 0,然后创建一个新的向量。
\begin{lstlisting}[language=R]
  // os\src\mm\frame_allocator .rs
  impl FrameAllocator for StackFrameAllocator {
    fn new() -> Self {
	  Self {
		  current : 0,
		  end : 0,
		  recycled : Vec : :new(),
	    }
	 }
    }
\end{lstlisting}
	
	在它真正被使用起来之前 ,需要调用 StackFrameAllocator 的  init 方法将自身的  [current,end) 初始化为可用物理页号区间。
	
	用结束物理页号减去起始物理页号得到剩余有多少物理页面。在内核运行时输出的 last {} Physical Frames, 就是起源于这里 ,输出还有多少空闲页面。这里还用了向量的 reserve 方 法来预留出足够的空间 ,避免在运行中不断的重新分配。该方法的方法签名为“pub fn reserve(\&mut self, additional: usize)”,用于为 Vec \textless T\textgreater 预留至少 additional 个元素的容量。调用reserve 后, Vec 的容量将大于或等于 self.len() + additional。如果当前容量已经足够,该方法则不执行任何操作。
	
\begin{lstlisting}[language=R]
  // os\src\mm\frame_allocator .rs
  impl StackFrameAllocator {
  pub fn init(&mut self, l : PhysPageNum, r : PhysPageNum) {
    self .current = l .0;
    self .end = r .0;
    let last_frames = self .end - self .current;
    self .recycled .reserve(last_frames);
    println !("last {} Physical Frames .", last_frames);
  }
  pub fn unallocated_frames(&self) -> usize {
    self .recycled .len() + self .end - self .current
  }
  pub fn free_space_size(&self) -> usize {
    self .unallocated_frames() * PAGE_SIZE
  }
  }
\end{lstlisting}
	
	unallocated\_frames 用于计算未被分配的页面数量 ,为 recycled 中的空闲页面和整块的空闲内存 页面数量之和。   free\_space\_size 计算空闲空间的大小 ,只需空闲页面的数量乘页面大小。
	\subsection{物理页帧分配}
	核心为物理页帧的分配和回收 ,先看物理页帧的分配 :
	
	在分配  alloc 的时候 ,首先会检查栈  recycled 内有没有之前回收的物理页号 ,如果有的话弹出栈顶 ,使用  into 方法将  usize 转换成了物理页号  PhysPageNum ,构造 frame\_tracker 返回。否则检 查 current ==  end ,若相等则表示内存耗尽 ,没有空闲页面 ,返回 None 。
\begin{lstlisting}[language=R]
  // os\src\mm\frame_allocator .rs
  impl FrameAllocator for StackFrameAllocator {
    fn alloc(&mut self) -> Option<FrameTracker> {
	if let Some(ppn) = self .recycled .pop() {
	  let frame_tracker = FrameTracker : :new(ppn .into());
	  log : :trace !("[frame_alloc] {:?}", frame_tracker);
	  Some(frame_tracker)
	} else if self .current == self .end {
	  None
	} else {
	  self .current += 1;
	  let frame_tracker = FrameTracker : :new((self .current - 1) .into());
	  log : :trace !("[frame_alloc] {:?}", frame_tracker);
	  Some(frame_tracker)
	}
   }
  }
\end{lstlisting}
	
	若不等说明之前从未分配过的物理页号区间还有剩余 ,可以在 [  current ,  end ) 上进行分配 ,我们 分配它的左端点  current  (即从低地址向高地址分配)  ,同时将管理器内部维护的  current 加  1 代表current 已被分配了。  同样将  usize 转换成了物理页号 ,构造 frame\_tracker 返回。
	
	之前提到过 FrameAllocator 有⼀个不初始化分配方法 alloc\_uninit ,我们看一下二者的不同 :
	\begin{figure}[h]
	  \centering
	  \includegraphics[width=14cm,height=3cm]{1.png}
	  \caption{ }
	\end{figure}
	
	其实也就是在 FrameTracker  new 的时候不同 ,在 alloc 中使用的是 FrameTracker: :new 这个有初 始化的方法 ,在 alloc\_uninit 中使用的是 FrameTracker: :new\_uninit 这个不初始化的方法。
	\subsection{物理页帧回收}
	物理页帧的回收是整个内存管理中一个重要的环节,它负责释放不再被使用的页面以供后续分配使用。让我们来看一下物理页面的回收机制以及其实现的过程。
	
\begin{lstlisting}[language=R]
  // os\src\mm\frame_allocator .rs
  // Deallocate a physical page
  fn dealloc(&mut self, ppn : PhysPageNum) {
    log : :trace !("[frame_dealloc] {:?}", ppn);
    let ppn = ppn .0;
    // validity check, note that this should be unnecessary for RELEASE build and it
    if option_env !("MODE") == Some("debug") && ppn >= self .current
    self .recycled .iter() .find( |&v| *v == ppn) .is_some()
    {
    	panic !("Frame ppn={:#x} has not been allocated !", ppn);
    }
    // recycle
    self .recycled .push(ppn);
}
\end{lstlisting}
  
  在这个回收的过程中,首先记录了将要回收的物理页面ppn,然后在调试模式下进行了有效性检查,以确保被释放的物理页面确实已经被分配过,防止程序在运行时因未分配的页面进行回收而出现问题。
  
  这个有效性检查使用了recycled.iter().find()方法,耗时较多,这也是为何在构建RELEASE版本时不必执行该检查的原因。对于发布版本来说,已经经过充分测试,假设不会出现未分配页面回收的情况。
  
  实际的页面回收机制非常简单,它仅将要回收的页面号ppn压入回收页面的recycled向量中。这个向量用于存储可供重新分配的物理页面。物理页面的回收是内存管理中重要的一环,通过这个机制,系统可以及时释放不再使用的页面,并通过有效性检查和回收流程来确保内存的正确性和可靠性。
\end{document}
