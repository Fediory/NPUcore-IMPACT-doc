\chapter{NPUcore-IMPACT 增量}

在上述基础上,我们继续做了许多努力,让 NPUcore-IMPACT 通过了初赛的所有测试用例,以及实验性地初步支持了 EXT4 文件系统。

后文我们会分别详细地介绍这两部分的内容。

\section{初赛期间的增量}

\subsection{初赛测试用例}

我们针对性地对初赛的测试用例进行了调试,将问题归类定位到了如下两点。

然后我们分别对每个问题进行了细致的分析,最终逐个击破,通过了初赛的所有测试用例。

\textbf{1. statx 系统调用}

LoongArch 赛道的初赛测试用例中,mmap 与 munmap 这两个测例涉及到了一个新的系统调用 statx。

\begin{lstlisting}[label={man:statx}, caption={statx 手册}]
NAME
       statx - get file status (extended)

LIBRARY
       Standard C library (libc, -lc)

SYNOPSIS
       #define _GNU_SOURCE          /* See feature_test_macros(7) */
       #include <fcntl.h>           /* Definition of AT_* constants */
       #include <sys/stat.h>

       int statx(int dirfd, const char *restrict pathname, int flags,
                 unsigned int mask, struct statx *restrict statxbuf);

STANDARDS
       Linux.

HISTORY
       Linux 4.11, glibc 2.28.
\end{lstlisting}

如手册 \ref{man:statx} 中所示,这个系统调用涉及了文件信息的获取。

我们为 NPUcore-IMPACT 实现了这个新的系统调用,并与文件系统进行了整合。

\begin{lstlisting}[language={Rust}, caption={statx 系统调用入口}]
let ret = match syscall_id {
    // ...
    SYSCALL_STATX => sys_statx(
        args[0],
        args[1] as *const u8,
        args[2] as u32,
        args[3] as u32,
        args[4] as *mut u8,
    ),
    // ...
};
\end{lstlisting}

\begin{lstlisting}[language={Rust}, caption={statx 系统调用实现}]
pub trait File: DowncastSync {
    // ...
    fn get_statx(&self) -> Statx;
    // ...
}
\end{lstlisting}

随后我们进行了测试,成功通过了 mmap 与 munmap 测试用例。

\textbf{2. 文件描述符分配}

通过对 openat 测试用例进行调试,我们最终发现问题出在操作系统对文件描述符的分配上。

Unix 标准要求操作系统分配文件描述符时,总是分配该进程还未使用的最小的文件描述符;而 NPUcore 回收进程关闭的文件描述符时,使用了一个线性表;操作系统重新分配之前回收的文件描述符时,没有使用表中最小的文件描述符,最终导致出现了问题。

\begin{lstlisting}[language={Rust}, caption={回收文件描述符}]
match self.inner[fd].take() {
    Some(file_descriptor) => {
        self.recycled.push(fd as u8);
        // TODO: maybe replace this with balanced binary tree?
        self.recycled.sort_by(|a, b| b.cmp(a));
        Ok(file_descriptor)
    }
    None => Err(EBADF),
}
\end{lstlisting}

我们选择了在回收文件描述符后进行一次排序来解决这个问题。

这个方案不一定是性能最佳的方案,我们还有以下方案可选:

\begin{enumerate}
    \item 回收时不进行排序,重新分配时使用 $O(n)$ 时间寻找最小的文件描述符;
    \item 使用二叉平衡树替换线性表,从而在 $O(\log n)$ 时间进行回收与重新分配,但也许会带来内存分配的额外开销。
\end{enumerate}

未来此处成为性能瓶颈时,根据性能测试结果选用最优方案会是更好的选择。
