\subsection{EXT4 文件系统}

EXT4(fourth extended filesystem)是 Linux 内核的一个日志文件系统,是 EXT3 文件系统的继任者。EXT4 文件系统具有许多改进和新特性,使其在性能、可靠性和可扩展性方面优于前代文件系统。

与 NPUcore 先前使用的 FAT32 文件系统相比,EXT4 文件系统不仅允许了更大的文件大小与卷大小,更有着显著的性能和效率提升。EXT4 文件系统使用了延迟分配和多块分配策略,显著减少了碎片并提高了写入性能;同时它支持 Extents 和更高效的分配策略,提高了文件操作的速度和效率。此外,EXT4 文件系统还支持日志记录,通过记录元数据变化确保系统崩溃时的数据一致性和完整性,检查速度快且更可靠。

ext4文件系统原理如下:
\begin{enumerate}
    \item 日志功能:ext4采用了日志功能(journaling),即在对文件系统进行操作时,会先将操作记录在日志中,然后再执行操作。这样可以在系统异常关机或崩溃时恢复文件系统的一致性。
    \item 内存缓存:ext4使用了内存缓存,将磁盘上的数据加载到内存中进行读写操作,以提高文件的访问速度。
    \item 数据块分配:ext4使用了多级索引结构来分配存储空间。它将文件系统的空间分为固定大小的块,每个块可以存储一定大小的数据。通过索引结构,可以快速定位文件的数据块。
    \item 空闲块管理:ext4使用了位图和B树来管理空闲块。位图记录着每个块的使用情况,而B树则用于索引和定位空闲块。
    \item 快照功能:ext4支持快照功能,可以在不影响原有数据的情况下对文件系统进行备份和恢复。
    \item 后日志预分配:ext4采用了一种称为"delayed allocation"的技术,即将数据块的分配推迟到真正需要写入数据时再进行。这样可以提高写入性能,减少磁盘的碎片化。
    \item 逐项更新:ext4在写入数据时,不会一次性更新整个文件,而是根据需要只更新部分数据。这种逐项更新的方式可以减少磁盘的I/O次数,提高性能。
\end{enumerate}

我们为 NPUcore-IMPACT 实验性地加入了 EXT4 文件系统支持,使得其可以从 EXT4 文件系统启动,并读写其中的文件。

