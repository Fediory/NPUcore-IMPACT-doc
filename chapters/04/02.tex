\subsection{EXT4 文件系统}

EXT4(fourth extended filesystem)是 Linux 内核的一个日志文件系统,是 EXT3 文件系统的继任者。EXT4 文件系统具有许多改进和新特性,使其在性能、可靠性和可扩展性方面优于前代文件系统。

与 NPUcore 先前使用的 FAT32 文件系统相比,EXT4 文件系统不仅允许了更大的文件大小与卷大小,更有着显著的性能和效率提升。EXT4 文件系统使用了延迟分配和多块分配策略,显著减少了碎片并提高了写入性能;同时它支持 Extents 和更高效的分配策略,提高了文件操作的速度和效率。此外,EXT4 文件系统还支持日志记录,通过记录元数据变化确保系统崩溃时的数据一致性和完整性,检查速度快且更可靠。

我们为 NPUcore-IMPACT 实验性地加入了 EXT4 文件系统支持,使得其可以从 EXT4 文件系统启动,并读写其中的文件。

我们使用了 lwext4 作为 EXT4 文件系统驱动。lwext4 是一个针对嵌入式系统设计的轻量级 EXT4 文件系统实现,它旨在提供 EXT4 文件系统的关键特性,同时保持低资源消耗和高性能,以适应嵌入式系统的限制。

为了让 lwext4 能与 NPUcore-IMPACT 一起工作,我们对 NPUcore-IMPACT 的文件系统设计做出了一定调整。

我们借助 Rust 的 trait 语言特性,设计了一个 File trait,用于表示一个抽象的文件,或者说一个可以对其进行读写的对象。

\begin{lstlisting}[language={Rust}, caption={File trait}]
pub trait File: DowncastSync {
    fn deep_clone(&self) -> Arc<dyn File>;
    fn readable(&self) -> bool;
    fn writable(&self) -> bool;
    fn read(&self, offset: Option<&mut usize>, buf: &mut [u8]) -> usize;
    fn write(&self, offset: Option<&mut usize>, buf: &[u8]) -> usize;
    fn r_ready(&self) -> bool;
    fn w_ready(&self) -> bool;
    fn read_user(&self, offset: Option<usize>, buf: UserBuffer) -> usize;
    fn write_user(&self, offset: Option<usize>, buf: UserBuffer) -> usize;
    fn get_size(&self) -> usize;
    fn get_stat(&self) -> Stat;
    fn get_statx(&self) -> Statx;
    fn get_file_type(&self) -> DiskInodeType;
    fn is_dir(&self) -> bool {
        self.get_file_type().is_dir()
        // self.get_file_type() == DiskInodeType::Directory
    }
    fn is_file(&self) -> bool {
        self.get_file_type().is_file()
        // self.get_file_type() == DiskInodeType::File
    }
    fn info_dirtree_node(&self, dirnode_ptr: Weak<DirectoryTreeNode>);
    fn get_dirtree_node(&self) -> Option<Arc<DirectoryTreeNode>>;
    /// open
    fn open(&self, flags: OpenFlags, special_use: bool) -> Arc<dyn File>;
    fn open_subfile(&self) -> Result<Vec<(String, Arc<dyn File>)>, isize>;
    /// create
    fn create(&self, name: &str, file_type: DiskInodeType) -> Result<Arc<dyn File>, isize>;
    fn link_child(&self, name: &str, child: &Self) -> Result<(), isize>
    where
        Self: Sized;
    /// delete(unlink)
    fn unlink(&self, delete: bool) -> Result<(), isize>;
    /// dirent
    fn get_dirent(&self, count: usize) -> Vec<Dirent>;
    /// offset
    fn get_offset(&self) -> usize {
        self.lseek(0, SeekWhence::SEEK_CUR).unwrap()
    }
    fn lseek(&self, offset: isize, whence: SeekWhence) -> Result<usize, isize>;
    /// size
    fn modify_size(&self, diff: isize) -> Result<(), isize>;
    fn truncate_size(&self, new_size: usize) -> Result<(), isize>;
    // time
    fn set_timestamp(&self, ctime: Option<usize>, atime: Option<usize>, mtime: Option<usize>);
    /// cache
    fn get_single_cache(&self, offset: usize) -> Result<Arc<Mutex<PageCache>>, ()>;
    fn get_all_caches(&self) -> Result<Vec<Arc<Mutex<PageCache>>>, ()>;
    /// memory related
    fn oom(&self) -> usize;
    /// poll, select related
    fn hang_up(&self) -> bool;
    /// iotcl
    fn ioctl(&self, _cmd: u32, _argp: usize) -> isize {
        ENOTTY
    }
    /// fcntl
    fn fcntl(&self, cmd: u32, arg: u32) -> isize;
}
\end{lstlisting}

在此基础上,我们为 lwext4 提供的 ext4_file 类型实现我们的 File trait,让 NPUcore-IMPACT 可以对其进行读写,从而实现 EXT4 文件系统的支持。

由于 lwext4 依赖 libc 进行内存分配,为了让它能工作在没有 libc 的环境下,我们还需要对其做出一定修改。

\begin{lstlisting}[language={C}, caption={管理 lwext4 内存}]
#if CONFIG_USE_USER_MALLOC

#define ext4_malloc  ext4_user_malloc
#define ext4_calloc  ext4_user_calloc
#define ext4_realloc ext4_user_realloc
#define ext4_free    ext4_user_free

#else

#define ext4_malloc  malloc
#define ext4_calloc  calloc
#define ext4_realloc realloc
#define ext4_free    free

#endif
\end{lstlisting}

我们希望让 NPUcore-IMPACT 为 lwext4 管理内存,为此我们实现 ext4_user_malloc、ext4_user_calloc、ext4_user_realloc、ext4_user_free 这四个内存管理函数,并将其与 lwext4 链接,从而让 lwext4 可以使用我们为它分配的内存,并在合适的时候回收这些内存。

\begin{lstlisting}[language={Rust}, caption={NPUcore-IMPACT 为 lwext4 分配内存}]
#[no_mangle]
pub extern "C" fn ext4_user_malloc(size: ::core::ffi::c_size_t) -> *mut ::core::ffi::c_void {
    HEAP_ALLOCATOR
        .lock()
        .alloc(Layout::array::<u8>(size).unwrap())
        .unwrap()
        .as_ptr() as *mut ::core::ffi::c_void
}
\end{lstlisting}

为了便于调试,我们需要在 lwext4 执行时打印日志,得益于 Rust 与 C 跨语言互操作十分方便,我们直接在 Rust 侧编写了打印日志的工具函数。

\begin{lstlisting}[language={Rust}, caption={在 lwext4 的 C 语言代码中打印日志}]
#[no_mangle]
pub extern "C" fn os_log(str: *const ::core::ffi::c_char) {
    let str = unsafe { CStr::from_ptr(str) };
    log::info!("{str:?}");
}

#[no_mangle]
pub extern "C" fn os_var_log(name: *const ::core::ffi::c_char, value: ::core::ffi::c_int) {
    let name = unsafe { CStr::from_ptr(name) };
    log::info!("{name:?}: {value}");
}
\end{lstlisting}

使用 \#[no_mangle] 可以让编译器不对函数名字进行混淆,使得我们可以在 C 语言侧直接调用 os_log 与 os_var_log 日志函数。
