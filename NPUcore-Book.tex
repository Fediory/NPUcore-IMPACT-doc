%% 博士、正常版本、强制使用 Windows 系统字体
\documentclass[lang=chs, degree=phd, blindreview=false, winfonts=true]{yanputhesis}
%%=============================================================================%
%% 导入宏包
%%-----------------------------------------------------------------------------%
\usepackage{listings, listings-rust}    %rust高亮
\usepackage{listings-riscv}             %riscv汇编高亮
\usepackage{underscore}                 %使下划线不需要添加斜线转义
% \usepackage[english]{babel}   %解决label中不能添加下划线问题,但是带来副作用,会把文档语言设置为英文
\usepackage{placeins}                   %解决图片浮动问题
\usepackage{pifont}                     %可以使用带圈数字序号
\usepackage{tabularx}                   %制作更精细的表格
%%=============================================================================%
%% 基本信息录入
%%-----------------------------------------------------------------------------%
\header{NPUcore操作系统内核构建实践}                                             %设置页眉
%%=============================================================================%
%% 文档开始
%%-----------------------------------------------------------------------------%
\begin{document}

\frontmatter                                                                   %前言部分
\makeBookCoverPage{NPUcore操作系统内核构建实践 \\ 基础篇}{NPU_logo.png}          %封面
\setcounter{page}{1}                                                           %设置目录页编号从1开始
\tableofcontents                                                               %目录页

\mainmatter                                                                    %正文部分
\sDefault
\chapter{NPUcore简介(Introduction)}

“NPUcore”是西北工业大学的操作系统内核构建实践型教学操作系统,曾获得2022年OSKernel大赛内核实现赛道一等奖。
NPUcore致力于使用Rust新型编程语言,帮助老师和学生自行研制一个操作系统微型内核,提升操作系统原理的实践体验并探索新型操作系统的设计与实现。
原始的2022版NPUcore具有内存管理、进程管理、文件系统核心系统调用功能,支持RISCV32/64指令集,可在对应的QEMU模拟器和SiFive-U740、K210等嵌入式开发板上运行。
该版本基于rCore-Tutorial迭代开发,重构90\%模块以支持Linux接口,共实现系统调用81个,是一个不错的baseline。
虽然该版本有着不错的性能,但却无法支持全部测例,以及国内自主研发的LoongArch龙芯架构。不仅如此,该版本不支持网络协议,EXT4文件系统,以及其它多种多样的外设,因此我们认为,这个版本仍然有很大的优化空间。
如此,针对初赛和决赛阶段,我们的贡献可以总结为以下四点,并在后文中详细展开:
\begin{enumerate}
    \item 独自实现了2022版本的NPUcore到2k1000平台(龙芯架构)的适配,并封装为一个arch包,方便后人持续开发。
    \item 基本完成了NPUcore在ext4文件系统的适配,但仍有少部分bug。
    \item 调研了几乎所有的开源轻量版ext4仓库,并针对此次适配做了一定总结。
    \item 其它小规模增量:
    \begin{itemize}
        \item 在NPUcore-重生之我是菜狗队伍的指导下,适配了网络模块,并在fat32文件系统上跑出分数。(由于这个增量更多属于另一组,所以我们不会在此次文档中进行大面积介绍)
        \item 独自在ext4文件系统上适配了PCI和SATA驱动,可以从镜像中读取到测例。
        \item 对在2k1000板子上烧录测例进行了初步探索,并总结出了对应的步骤。
    \end{itemize}
\end{enumerate}

% \section{Rust特性}

% Rust是一个“安全、并发、实用”,支持函数式、并发式、过程式以及面向对象的程序设计风格的新型语言。
% Rust在完全公开的情况下开发,并且相当欢迎社区的反馈。近些年,Rust语言在工业应用上的势头越来越猛。
% 基于Rust语言的种种特性,我们认为它更适合一些底层应用的开发,尤其是OSKernel。

% \textbf{1. 我们为什么选择Rust作为OS编程语言?}
% Rust 是一门内存安全的语言。对于 C/C++ 这样的手动管理内存的编程语言,我们在分配堆变量的时候需要调用 malloc/new函数,而当该变量使用完毕之后要手动调用
% free/delete 回收内存。这就要求程序员需要关注所有堆变量的生命周期并及时将其释放,否则就会造成内存泄漏的问题,而过早的释放堆变量又可能造成“use-after-
% free” 的问题。而 Rust 独特的所有权机制和借用检查,让编译器掌管变量的生
% 命周期,使得变量的回收变得可控,同时也杜绝了”use-after-free” 的问题,又不至于带来垃圾回收的开销。

% Rust还能够推断出类型的大小,然后分配正确的内存大小并将其设置为您要求的值。但这意味着无法分配未初始化的内存:Rust没有null的概念。 
% 此外,所有这些检查都是在编译时完成的,因此没有运行时开销,这也是为什么Rust被成为是安全的“C”。 
% 如果你编写了正确的C++代码,你将编写出与C++代码基本上相同的Rust代码。而且由于编译器的帮忙,编写错误的代码版本是不可能的。
% 所以,我们选择Rust语言的原因,不仅是因为他安全,还因为其享有和C一样的速度,和更丰富的库。


% \textbf{2.unsafe关键字}

% 几乎每个语言都有unsafe关键字,但Rust语言使用unsafe的原因可能与其它编程语言还有所不同。接下来我们展示一下unsafe的特性:
% \begin{lstlisting}[language={Rust}, label={code:unsafe},
% 	caption={unsafe展示(r1 是一个裸指针)}]
% fn main() {
%     let mut num = 5;

%     let r1 = &num as *const i32;

%     unsafe {
%         println!("r1 is: {}", *r1);
%     }
% }
% \end{lstlisting}
% 在代码块\ref{code:unsafe}中,r1 是一个裸指针(raw pointer),由于它具有破坏Rust内存安全的潜力,因此只能在unsafe代码块中使用,如果你去掉unsafe\{\},编译器会立刻报错。
% 在我们的NPUcore中,对于一个OS来说,安全是最大的保障,因此unsafe在初期NPUcore建设中给予了很大帮助。因为,即使做到小心谨慎,依然会有出错的可能性,但是 unsafe 语句块决定了:就算内存访问出错了,你也能立刻意识到,错误是在 unsafe 代码块中,而不花大量时间像无头苍蝇一样去寻找问题所在。
% unsafe不安全,但是该用的时候就要用,在一些时候,它能帮助我们大幅降低代码实现的成本。虽然在网上充斥着“千万不要使用 unsafe,因为它不安全”的言论。事实上,我们认为unsafe是一个有效且必要的手段,因此我们选择遵循如下规则去使用:
% \begin{enumerate}
%     \item 没必要用时,就不用;
%     \item 当有必要用时,就大胆用,但是要控制好边界;
%     \item 尽量保证unsafe的边界范围最小。
% \end{enumerate}

\section{NPUcore操作系统}

「NPUcore」是西北工业大学的操作系统内核构建实践型教学操作系统。致力于使用Rust新型编程语言,帮助老师和学生自行研制一个操作系统微型内核,提升操作系统原理的实践体验并探索新型操作系统的设计与实现。目前NPUcore具有内存管理、进程管理、文件系统核心功能,支持RISCV32/64指令集,可在QEMU模拟器和SiFive-U740、K210等嵌入式开发板上运行。

以下为NPUcore的所有的系统调用:

\section{预备知识及技能}
\subsection{RISC-V和LoongArch指令集介绍}
\textbf{1、RISC-V}

RISC-V(发音为“risk-five”)是一个基于精简指令集(RISC)原则的开源指令集架构(ISA),简易解释为开源软体运动相对应的一种“开源硬体”。该项目2010年始于加州大学柏克莱分校,但许多贡献者是该大学以外的志愿者和行业工作者。\\
与大多数指令集相比,RISC-V指令集可以自由地用于任何目的,允许任何人设计、制造和销售RISC-V芯片和软件而不必支付给任何公司专利费。虽然这不是第一个开源指令集,但它具有重要意义,因为其设计使其适用于现代计算设备(如仓库规模云计算机、高端移动电话和微小嵌入式系统)。设计者考虑到了这些用途中的性能与功率效率。该指令集还具有众多支持的软件,这解决了新指令集通常的弱点。\\
RISC-V指令集的设计考虑了小型、快速、低功耗的现实情况来实做,但并没有对特定的微架构做过度的设计。
新出现的RISC-V的核心目标是灵活适应未来的AIoT场景,保证基本功能,提供可配置的扩展功能。其开源特征使得学生都可以方便地设计一个RISC-V CPU。\\
写面向RISC-V的OS的代价仅仅是你了解RISC-V的Supevisor特权模式,知道OS在Supevisor特权模式下的控制能力。\\
\textbf{2、 LoongArch}

LoongArch是RISC中的一个具体实现。2020年,龙芯中科基于二十年的CPU研制和生态建设积累推出了龙架构(LoongArch™),包括基础架构部分和向量指令、虚拟化、二进制翻译等扩展部分,近2000条指令。

龙架构具有较好的自主性、先进性与兼容性。

龙架构从整个架构的顶层规划,到各部分的功能定义,再到细节上每条指令的编码、名称、含义,在架构上进行自主重新设计,具有充分的自主性。

龙架构摒弃了传统指令系统中部分不适应当前软硬件设计技术发展趋势的陈旧内容,吸纳了近年来指令系统设计领域诸多先进的技术发展成果。同原有兼容指令系统相比,不仅在硬件方面更易于高性能低功耗设计,而且在软件方面更易于编译优化和操作系统、虚拟机的开发。

龙架构在设计时充分考虑兼容生态需求,融合了各国际主流指令系统的主要功能特性,同时依托龙芯团队在二进制翻译方面十余年的技术积累创新,能够实现多种国际主流指令系统的高效二进制翻译。龙芯中科从 2020 年起新研的 CPU 均支持LoongArch™。

龙架构已得到国际开源软件界广泛认可与支持,正成为与X86/ARM并列的顶层开源生态系统。已向GNU组织申请到ELF Machine编号(258号),并获得Linux、Binutils、GDB、.NET、GCC、LLVM、Go、Chromium/V8、Mozilla / SpiderMonkey、Javascript、FFmpeg、libyuv、libvpx、OpenH264、SRS等音视频类软件社区、UEFI(UEFI规范、ACPI规范)以及国内龙蜥开源社区、欧拉openEuler开源社区的支持。

指令系统是软件生态的起点,只有从指令系统的根源上实现自主,才能打破软件生态发展受制于人的锁链。龙架构的推出,是龙芯中科长期坚持自主研发理念的重要成果体现,是全面转向生态建设历史关头的重大技术跨越。

\subsection{Rust语言及其主要特性}
Rust是由Mozilla主导开发的通用、编译型编程语言。设计准则为“安全、并发、实用”,支持函数式、并发式、过程式以及面向对象的程序设计风格。

Rust语言原本是Mozilla员工Graydon Hoare的个人项目,而Mozilla于2009年开始赞助这个项目,并且在2010年首次公开。也在同一年,其编译器原始码开始由原本的OCaml语言转移到用Rust语言,进行自我编译工作,称做“rustc”,并于2011年实际完成。这个可自我编译的编译器在架构上采用了LLVM做为它的后端。

第一个有版本号的Rust编译器于2012年1月发布。Rust1.0是第一个稳定版本,于2015年5月15日发布。

Rust在完全公开的情况下开发,并且相当欢迎社区的反馈。在1.0稳定版之前,语言设计也因为透过撰写Servo网页浏览器排版引擎和rustc编译器本身,而有进一步的改善。它虽然由Mozilla资助,但其实是一个共有项目,有很大部分的代碼是来自于社区的贡献者。

\textbf{1.所有权}

所有权是Rust的核心,也是其更有趣和独特的功能之一。“所有权”是指允许哪部分的代码修改内存。让我们从查看一些C++代码开始:
\begin{lstlisting}[language={Rust}, label={code:forktest},
	caption={forktest.rs}]
	int *dangling(void)
	{
		int i = 1234;
		return &i;
	}
	
	int add_one(void)
	{
		int *num = dangling();
		return *num + 1;
	}
\end{lstlisting}

dangling函数在栈上分配了一个整型,然后保存给一个变量i,最后返回了这个变量i的引用。这里有一个问题:当函数返回时栈内存变成失效。意味着在函数add\_one第二行,指针num指向了垃圾值,我们将无法得到想要的结果。虽然这个一个简单的例子,但是在C++的代码里会经常发生。当堆上的内存使用malloc(或new)分配,然后使用free(或delete)释放时,会出现类似的问题,但是您的代码会尝试使用指向该内存的指针执行某些操作。 更现代的C++使用RAII和构造函数/析构函数,但它们无法完全避免“悬空指针”。 这个问题被称为“悬空指针”,并且不可能编写出现“悬空指针”的Rust代码。 我们试试吧:
\begin{lstlisting}[language={Rust}, label={code:forktest},
	caption={forktest.rs}]
	fn dangling() -> &int {
		let i = 1234;
		return &i;
	}
	
	fn add_one() -> int {
		let num = dangling();
		return *num + 1;
	}
\end{lstlisting}

当你尝试编译这个程序时,你会得到一个有趣和非常长的错误信息:
\begin{lstlisting}[language={Rust}, label={code:forktest},
	caption={forktest.rs}]
	temp.rs:3:11: 3:13 error: borrowed value does not live long enough
	temp.rs:3     return &i;
	
	temp.rs:1:22: 4:1 note: borrowed pointer must be valid for the anonymous lifetime #1 defined on the block at 1:22...
	temp.rs:1 fn dangling() -> &int {
		temp.rs:2     let i = 1234;
		temp.rs:3     return &i;
		temp.rs:4 }
	
	temp.rs:1:22: 4:1 note: ...but borrowed value is only valid for the block at 1:22
	temp.rs:1 fn dangling() -> &int {      
		temp.rs:2     let i = 1234;            
		temp.rs:3     return &i;               
		temp.rs:4  }                            
	error: aborting due to previous error
\end{lstlisting}

为了完全理解这个错误信息,我们需要谈谈“拥有”某些东西意味着什么。 所以现在,让我们接受Rust不允许我们用悬空指针编写代码,一旦我们理解了所有权,我们就会回来看这块代码。

让我们先放下编程一会儿,先聊聊书籍。 我喜欢读实体书,有时候我真的很喜欢一本书,并告诉我的朋友他们应该阅读它。 当我读我的书时,我拥有它:这本书是我所拥有的。 当我把书借给别人一段时间,他们向我“借用”这本书。 当你借用一本书时,在特定的一段时间它是属于你的,然后你把它还给我,我又拥有它了。 对吗?

这个概念也直接应用于Rust代码:一些代码“拥有”一个指向内存的特定指针。 它是该指针的唯一所有者。 它还可以暂时将该内存借给其他代码:代码“借用”它。 借用它一段时间,称为“生命周期”。

这是关于所有权的所有。 那似乎并不那么难,对吧? 让我们回到那条错误信息:error: borrowed value does not live long enough。 我们试图使用Rust的借用指针&,借出一个特定的变量i。 但Rust知道函数返回后该变量无效,因此它告诉我们:
\begin{lstlisting}[language={Rust}, label={code:forktest},
	caption={forktest.rs}]
	borrowed pointer must be valid for the anonymous lifetime #1
	
	... but borrowed value is only valid for the block。
\end{lstlisting}

这是栈内存的一个很好的例子,但堆内存呢? Rust有第二种指针,一个'唯一'指针,你可以用$\sim$创建。 看看这个:
\begin{lstlisting}[language={Rust}, label={code:forktest},
	caption={forktest.rs}]
	fn dangling() -> ~int {
		let i = ~1234;
		return i;
	}
	
	fn add_one() -> int {
		let num = dangling();
		return *num + 1;
	}
\end{lstlisting}

此代码将成功编译。 请注意,我们使用指针指向该值而不是将1234分配给栈:$\sim$1234。 你可以大致比较这两行:
\begin{lstlisting}[language={Rust}, label={code:forktest},
	caption={forktest.rs}]
	// rust
	let i = ~1234;
	// C++
	int *i = new int;
	*i = 1234;
\end{lstlisting}

Rust能够推断出类型的大小,然后分配正确的内存大小并将其设置为您要求的值。 这意味着无法分配未初始化的内存:Rust没有null的概念。万岁! Rust和C++之间还有另外一个区别:Rust编译器还计算了i的生命周期,然后在它无效后插入相应的free调用,就像C++中的析构函数一样。 您可以获得手动分配堆内存的所有好处,而无需自己完成所有工作。 此外,所有这些检查都是在编译时完成的,因此没有运行时开销。 如果你编写了正确的C++代码,你将编写出与C++代码基本上相同的Rust代码。而且由于编译器的帮忙,编写错误的代码版本是不可能的。
你已经看到了一种情况,所有权和生命周期有利于防止在不太严格的语言中通常会出现的危险代码。现在让我们谈谈另一种情况:并发。

\textbf{2.并发:}

并发是当前软件世界中一个令人难以置信的热门话题。 对于计算机科学家来说,它一直是一个有趣的研究领域,但随着互联网的使用爆炸式增长,人们正在寻求改善给定的服务可以处理的用户数量。 并发是实现这一目标的一种方式。 但并发代码有一个很大的缺点:它很难推理,因为它是非确定性的。 编写好的并发代码有几种不同的方法,但让我们来谈谈Rust的所有权和生命周期的概念如何帮助实现正确并且并发的代码。

首先,让我们回顾一下Rust中的简单并发示例。 Rust允许你启动task,这是轻量级的“绿色”线程。 这些任务没有任何共享内存,因此,我们使用“通道”在task之间进行通信。 像这样:
\begin{lstlisting}[language={Rust}, label={code:forktest},
	caption={forktest.rs}]
	fn main() {
		let numbers = [1,2,3];
		
		let (port, chan)  = Chan::new();
		chan.send(numbers);
		
		do spawn {
			let numbers = port.recv();
			println!("{:d}", numbers[0]);
		}
	}
\end{lstlisting}

在这个例子中,我们创建了一个数字的vector。 然后我们创建一个新的Chan,这是Rust实现通道的包名。 这将返回通道的两个不同端:通道(channel)和端口(port)。 您将数据发送到通道端(channel),它从端口端(port)读出。 spawn函数可以启动一个task。 正如你在代码中看到的那样,我们在task中调用port.recv(),我们在外面调用chan.send(),传入vector。 然后打印vector的第一个元素。

这样做是因为Rust在通过channel发送时copy了vector。 这样,如果它是可变的,就不会有竞争条件。 但是,如果我们正在启动很多task,或者我们的数据非常庞大,那么为每个任务都copy副本会使我们的内存使用量膨胀而没有任何实际好处。

引入Arc。 Arc代表“原子引用计数”,它是一种在多个task之间共享不可变数据的方法。 这是一些代码:
\begin{lstlisting}[language={Rust}, label={code:forktest},
	caption={forktest.rs}]
	extern mod extra;
	use extra::arc::Arc;
	
	fn main() {
		let numbers = [1,2,3];
		
		let numbers_arc = Arc::new(numbers);
		
		for num in range(0, 3) {
			let (port, chan)  = Chan::new();
			chan.send(numbers_arc.clone());
			
			do spawn {
				let local_arc = port.recv();
				let task_numbers = local_arc.get();
				println!("{:d}", task_numbers[num]);
			}
		}
	}
\end{lstlisting}

这与我们之前的代码非常相似,除了现在我们循环三次,启动三个task,并在它们之间发送一个Arc。 Arc :: new创建一个新的Arc,.clone()返回Arc的新的引用,而.get()从Arc中获取该值。 因此,我们为每个task创建一个新的引用,将该引用发送到通道,然后使用引用打印出一个数字。 现在我们不copy vector。

Arcs非常适合不可变数据,但可变数据呢? 共享可变状态是并发程序的祸根。 您可以使用互斥锁(mutex)来保护共享的可变状态,但是如果您忘记获取互斥锁(mutex),则可能会发生错误。

Rust为共享可变状态提供了一个工具:RWArc。 Arc的这个变种允许Arc的内容发生变异。 看看这个:
\begin{lstlisting}[language={Rust}, label={code:forktest},
	caption={forktest.rs}]
	extern mod extra;
	use extra::arc::RWArc;
	
	fn main() {
		let numbers = [1,2,3];
		
		let numbers_arc = RWArc::new(numbers);
		
		for num in range(0, 3) {
			let (port, chan)  = Chan::new();
			chan.send(numbers_arc.clone());
			
			do spawn {
				let local_arc = port.recv();
				
				local_arc.write(|nums| {
					nums[num] += 1
				});
				
				local_arc.read(|nums| {
					println!("{:d}", nums[num]);
				})
			}
		}
	}
\end{lstlisting}

我们现在使用RWArc包来获取读/写Arc。 RWArc的API与Arc略有不同:读和写允许您读取和写入数据。 它们都将闭包作为参数,并且在写入的情况下,RWArc将获取互斥锁,然后将数据传递给此闭包。 闭包完成后,互斥锁被释放。

你可以看到在不记得获取锁的情况下是不可能改变状态的。 我们获得了共享可变状态的便利,同时保持不允许共享可变状态的安全性。

但是我们不能同时允许和禁止可变状态。 是什么赋予了这种能力的?

\textbf{3.unsafe:}

因此,Rust语言不允许共享可变状态,但我刚刚向您展示了一些允许共享可变状态的代码。 这怎么可能? 答案:unsafe。

你看,虽然Rust编译器非常聪明,并且可以避免你通常犯的错误,但它不是人工智能。 因为我们比编译器更聪明,有时候,我们需要克服这种安全行为。 为此,Rust有一个unsafe关键字。 在一个unsafe的代码块里,Rust关闭了许多安全检查。 如果您的程序出现问题,您只需要审核您在不安全范围内所做的事情,而不是整个程序。

如果Rust的主要目标之一是安全,为什么要关闭安全?
嗯,实际上只有三个主要原因:与外部代码连接,例如将FFI写入C库,性能(在某些情况下),以及围绕通常不安全的操作提供安全抽象。 我们的Arcs是最后一个目的的一个例子。 我们可以安全地分发对Arc的多个引用,因为我们确信数据是不可变的,因此可以安全地共享。 我们可以分发对RWArc的多个引用,因为我们知道我们已经将数据包装在互斥锁中,因此可以安全地共享。 但Rust编译器无法知道我们已经做出了这些选择,所以在Arcs的实现中,我们使用不安全的块来做(通常)危险的事情。 但是我们暴露了一个安全的接口,这意味着Arcs不可能被错误地使用。

这就是Rust的类型系统如何让你不会犯一些使并发编程变得困难的错误,同时也能获得像C++等语言一样的效率。

我希望这个对Rust的尝试能让您了解Rust是否适合您。 如果这是真的,我建议您查看完整的教程,以便对Rust的语法和概念进行全面,深入的探索。
\subsection{如何查资料}
\textbf{1、查手册}

\textbf{程序自带的文档}
(1) README 和 INSTALL

很多程序在编译或者安装过程中都会自带一个README和INSTALL文件, 不要漏掉, 否则可能会有重要的信息遗漏并导致某些严重问题。

其中, 如果INSTALL文件被单独呈现, 则其往往是解释软件的安装方式的, 有的软件有很特殊的安装要求, 如执行脚本的位置必须是在文件夹内或者文件夹外, 如果漏掉可能导致软件完全运行不起来。

README一般介绍软件的使用方式, 文档获取位置和帮助信息, 有时候也介绍安装方法。

例如, 你在NPUcore的源代码文件夹中可以找到README:
\begin{figure}[htb]
	\centering
	\includegraphics[width=\textwidth]{figures/02-01-readme.png}
	\caption{
		readme
	}
	\label{fig:readme}
\end{figure}
(2) help参数
绝大多数程序会自带一个help选项, 甚至不加任何参数。 例如, man命令的help参数:

\begin{lstlisting}[language={Rust}, label={code:forktest},
	caption={forktest.rs}]
	whatis --help
\end{lstlisting}

会打印出:
\begin{figure}[htb]
	\centering
	\includegraphics[width=\textwidth]{figures/02-01-help.png}
	\caption{
		help
	}
	\label{fig:help}
\end{figure}
那么, 如果你直接在终端中敲入help并回车, 会发生什么呢(假设你使用的是bash)?请自己试一下.

此外, 帮助文档的提供软件自身往往也会有自己的帮助文档, 显然自产自销是最合适的。

所以, 你不妨试一下man man, 或者进入man后有没有能看到的某些man自己的帮助文档(仔细找, 我这么说就一定有)。

之后的任何帮助套件也可以“自己帮助自己”, 所以文献中不再赘述。

(3) TAB补全

在Bash中,TAB补全是一种非常有用的功能,它可以让用户更快捷、更准确地输入命令和文件路径。在终端输入命令或文件路径时,如果按下TAB键,Bash会尝试自动补全输入的内容。

下面是关于Bash中TAB补全常见类型(如果没有特别说明, 则Bash会列出这些选项供选择):
1)命令补全:输入一个命令的前几个字母时,补全该命令的名称。如果有多个以该字符串开头的命令,
2)文件路径补全:输入一个文件或目录的路径时,补全路径中的文件或目录名称。如果有多个符合条件的文件或目录,
3)变量名补全:输入一个变量名时,补全该变量的名称。如果有多个符合条件的变量名,
4)命令参数补全:输入命令的参数时,补全该命令所支持的参数选项。如果有多个符合条件的参数选项,
5)目录补全:在输入路径时,如果您只知道路径中的某些部分,可以使用通配符进行补全。例如,输入"/u/lo*",按下TAB键可以自动补全为"/usr/local"。

总之,bash中的TAB补全是一种非常方便的功能,可以让用户更快速地输入命令和路径,并且减少输入错误的可能性。

此外, TAB补全需要程序自身和终端的支持, 有时候甚至需要单独配置, (例如rust的工具链就需要自行配置对应的shell补全选项)

\textbf{man}

在Linux操作系统中,man命令是一个非常重要的命令,它可以帮助用户查看Linux系统中各种命令的手册。

使用man只需要在终端中输入"man"加上要查看的命令名称,然后按下回车键即可。例如:

\begin{lstlisting}[language={Rust}, label={code:forktest},
	caption={forktest.rs}]
	man help
\end{lstlisting}

man命令将会显示出该命令的手册页,可以使用键盘上的箭头键进行滚动,并且可以使用“/”加上关键字进行查找。

在手册页中,可以查看该命令的使用方法、参数选项、示例以及其他相关信息。 man软件的本身的帮助信息可以在软件中按“h”查看。

当不再需要查看手册页时,可以按下“q”键退出man命令。

\textbf{完整手册}
多数成体系的大型软件系统会有自己对应的文档, 一般称为手册. 具体来说, 这种文档会出现在官方网站的Documentation环节, 且往往有在线或者线下PDF两种版本。

我们以GNU GCC为例, 在https://gcc.gnu.org/中, 浏览器搜索(一般快捷键是Ctrl-F)Documentation, 下方的Manual就是手册。点进去会有各种格式的手册。

有的成熟的软件或者语言会提供Tutorial 和 Reference Manual, 后者倾向于列举所有的性质, 前者则是为入门初学者提供的简单的教程。

一般而言, 绝大多数的软件是自身具有自己的手册的, 但部分软件的手册是集合型的, 或者本身就是其manpage的集合。

一个典型的例子是coreutils, 其中包括了cut, head, tail等简单工具;另一个是binutils, 包括各种GCC的常用工具。这时候需要自行查询其手册的所在之处。

\textbf{教材}

很多的软件都有自己的教材, 且层次从入门到精通都有, 如果你有需要, 可以找买一本合适的书从中学习。 一般教材会比官方手册更详细, 并提供作者自己的思考。

\textbf{TLDR}

TLDR是“Too long, don't read.”的缩写,
如果要最快获得某个命令的简单使用方法, tldr是一个不错的来源。例如我们输入

\begin{lstlisting}[language={Rust}, label={code:forktest},
	caption={forktest.rs}]
	$ tldr man
	
	Format and display manual pages.More information: https://www.man7.org/linux/man-pages/man1/man.1.html.
	
	- Display the man page for a command:
	man {{command}}
	
	- Display the man page for a command from section 7:
	man {{7}} {{command}}
	
	- List all available sections for a command:
	man -f {{command}}
	
	- Display the path searched for manpages:
	man --path
	
	- Display the location of a manpage rather than the manpage itself:
	man -w {{command}}
	
	- Display the man page using a specific locale:
	man {{command}} --locale={{locale}}
	
	- Search for manpages containing a search string:
	man -k "{{search_string}}"
\end{lstlisting}

\textbf{info}

注意, info是用某个目录作为中心数据库的, 所以完全可能存在在一个软件中可以阅读但在另一个软件中读不了的情况。

info中有大量长篇的完整文档, 一般就是上述完整手册。 一般各种IDE本身也会自带Info的阅读器。 只是info有自己的搜索, 历史记录等功能(有的功能是配合IDE使用的), 这里不再赘述。

此外, GNU套件几乎所有的工具都有info文档。所谓的GNU套件PDF文档就是用info相关的一个软件texinfo写的。

我们以gdb为例展示其内容。 终端输入“info gdb”可以得到:

\begin{figure}[htb]
	\centering
	\includegraphics[width=\textwidth]{figures/02-01-info.png}
	\caption{
		info
	}
	\label{fig:info}
\end{figure}

另外, info文档的安装方法如下:

对于软件自带文档,一般可以:

\begin{lstlisting}[language={Rust}, label={code:forktest},
	caption={info}]
	sudo apt-get install gdb-doc
	有时候需要去网站上搜索并下载, 就会需要:
	whereis info # 获得info的安装目录
	# 注意:有时候下载到的是一个info.tar.gz, 这时候需要自行解压, 
	# 但是如果是info.gz,则可以直接跳过这一步, 因为gz一般是自动解压的。
	# 另外,有时候文档是以texi后缀名出现的
	tar xf <infofilepath> <tmp_path>
	sudo install-info bison.info <one_of_the_info_paths>/dir
\end{lstlisting}

\textbf{自动补全与文档显示插件}

多数的IDE有自己的文档现实和自动补全插件, 可以在光标悬停在某个符号一段时间后自动显示特定的文档。

很多IDE还会集成之前的所说的这些文档查询方式, 从而在内部查询所有的文献。

请自行搜索自己的编辑器和IDE的文档寻找配置方式。

\textbf{搜索引擎}

如果你遇到什么工具你无法使用或者不会用, 可以尝试通过搜索引擎寻找替代品, 在线版或者其他能让你用上的方法(比如能够代理你请求的某些接口,软件和网站)。

\textbf{论坛}

论坛往往是最后一步, 一般来说很少出现别人没有发现过而自己发现的问题, 毕竟计算机工业确实过于发达了. 但是, 在stackoverflow和其他论坛上提问仍然有可能可以获得比较好的效果。

如果人家恰好碰到过或者有兴趣帮你解决问题的话是再幸运不过的事, 不过在此之前, 请先不要往下看, 尝试通过之前几步查找可能技术论坛以便解决问题。

然后这是一份作者根据自己回忆写的常见的论坛, 你可以试着在上面发帖. 此外, 一般使用量大的项目都有自己的论坛, 你也可以自行查找。









\chapter{相关工作(Related Work)}


\section{基于GDB内核调试}
在软件开发过程中,调试是不可避免的。一个程序往往不会一开始就按照程序员预期的方式运行, 对操作系统这样的复杂巨系统而言尤其如此。本节主要讲解调试软件GDB的使用方法。
\subsection{认识GDB}
GNU调试器(英语:GNU Debugger,缩写:GDB),是GNU软件系统中的标准调试器,此外GDB也是个具有移携性的调试器,经过移携需求的调修与重新编译,如今许多的类UNIX操作系统上都可以使用GDB,而现有GDB所能支持调试的编程语言有C、C++、Pascal以及FORTRAN。

\textbf{为什么需要GDB}

当程序出现错误时,开发者需要快速地找出错误的原因,并修复它们。很多人会倾向于使用"人工静态分析"(也就是目测法和冥想法)解决Bug。

然而,程序的运行时状态往往非常复杂,有时很难在代码中准确地定位错误。 具体来说, 目测的以下缺点导致开发者往往会百思不得其解进而无功而返:

\textbf{1.无法检测运行时问题:}代码静态分析只能检测静态代码问题,例如语法错误、类型错误等,它无法检测代码的运行时问题。例如,它无法检测到由于代码在特定环境下执行而引起的问题,如内存泄漏、死锁等。

\textbf{2.误报和漏报:}静态分析可能会误会遗漏问题。例如,可能会将某些无害的代码标记为错误,或者忽略某些实际上是错误的代码。这可能会导致开发人员浪费时间和精力来调查错误的根本原因。

\textbf{3.对高质量代码的依赖:}代码静态分析需要高质量的代码才能进行准确的分析。如果代码质量不好,例如缺乏注释、变量名不规范、代码冗余等,那么静态分析可能会产生误报或漏报。

\textbf{4.难以发现复杂的问题:}静态分析工具通常使用各种分析技术来分析代码,但这些技术很难发现复杂的问题,例如多线程问题、分布式系统问题等。这些问题通常需要动态调试或其他更高级的技术来解决。

同样, 也有人会尝试插入LOG打印部分状态, 但是这种方法除了费时费力, 且暴露状态不够精确的问题之外, 在OS中, 某些LOG会产生系统状态的改变, 进而影响结果, 导致debug失败。

这时候,调试工具就显得非常重要了。调试工具可以帮助开发者在运行时监视程序的状态,跟踪代码的执行流程,查看变量的值,以及定位错误的位置。 这正是GDB的用途。

GDB 最初由Richard Stallman在他的GNU Emacs 系统稳定后于1986年编写,并设计作为他的GNU系统的一部分。GDB是根据GNU通用公共许可证(GPL)发布的免费软件。它是在Berkeley Unix发行版附带的DBX调试器之后建模的。从1990 年到 1993 年,它由John Gilmore维护。现在由自由软件基金会任命的 GDB 指导委员会维护。

GDB允许用户查看一个程序在执行时“内部”的执行过程—,或者查看程序在崩溃时的内部状态。这些被调试的程序可以与 GDB 在同一台机器上(本地)、另一台机器(远程)或模拟器上执行。GDB 可以在大多数流行的 UNIX 和 Microsoft Windows 变体以及 Mac OS X 上运行。具体而言,目前 GDB 支持以下程序:Ada、Assembly、C、C++、D、Fortran、Go、Objective-C、OpenCL、Modula-2、Pascal、Rust 等。

GDB 默认只有命令行接口(CLI)可用,而不具备较能亲合上手、直觉操作的图形用户界面(GUI),不过此一弱处也已经有几个前端程序为其补强,例如DDD、GDBtk/Insight (页面存档备份,存于互联网档案馆)以及Emacs中的“GUD 模式”等,有了这些补强后,GDB在功效使用的便利性上就能够与“集成发展环境中的调试功效使用”相接近。

\textbf{GDB的启动}

显然, 启动gdb有不同的方法, 在终端中输入gdb是最简单的, 但NPUcore在RISC-V上构建, 因此不能直接使用本机的gdb(除非你使用一台RISC-V64计算机),因此我们推荐安装并使用gdb-multiarch(这里需要Ubuntu环境):
\begin{lstlisting}[language={Rust}, label={code:forktest},
	caption={forktest.rs}]
	$ sudo apt-get install gdb-multiarch
	# 然后启动:
	$ gdb-multiarch
\end{lstlisting}

注意, 这里实际上有"工作目录"的概念, 也就是你的当前目录实际上最好在项目或者源代码的路径上, 否则会需要手工加载源代码路径(方法见下文)。
\subsection{基于GDB的内核调试}
\textbf{QEMU虚拟机的相关命令}

介绍GDB为什么要先介绍虚拟机呢?因为正是QEMU与GDB合作, 才给了我们方便地进行大部分系统软件调试的机会。

作为一款全虚拟化虚拟机, QEMU能彻底模拟CPU的内部状态, 包括寄存器和其他部分, 因此很适合进行调试。

具体来说, QEMU配合GDB提供了单步执行、断点调试、内存监视、寄存器查看等。

用户可以使用调试功能逐步执行代码,查看每一步的运行结果和寄存器状态,同时还可以设置断点,方便定位问题所在。

利用QEMU提供的远程调试功能,允许用户在另一台计算机上通过网络连接到QEMU的调试接口进行调试。这个功能可以方便地在不同的计算机之间进行协作开发和调试。

这里我们只介绍本地的远程调试。

为了方便, 在os文件夹中, 使用下列make命令直接进行gdb调试:
\begin{lstlisting}[language={Rust}, label={code:forktest},
	caption={forktest.rs}]
	make gdb
\end{lstlisting}
其后端执行实际命令是:
\begin{lstlisting}[language={Rust}, label={code:forktest},
	caption={forktest.rs}]
	gdb:
	@qemu-system-riscv64 -machine virt -nographic -bios $(BOOTLOADER) -device loader,\
	file=target/riscv64gc-unknown-none-elf/debug/os,addr=0x80200000 -drive \
	file=$(U_FAT32),if=none,format=raw,id=x0 \
	-device virtio-blk-device,drive=x0,bus=virtio-mmio-bus.0 -smp threads=$(CORE_NUM) -S -s
\end{lstlisting}  
和do-run的内容相比,
\begin{lstlisting}[language={Rust}, label={code:forktest},
	caption={forktest.rs}]
	do-run:
	@qemu-system-riscv64 \
	-machine virt \
	-nographic \
	-bios $(BOOTLOADER) \
	-device loader,file=$(KERNEL_BIN),addr=$(KERNEL_ENTRY_PA) \
	-drive file=$(U_FAT32),if=none,format=raw,id=x0 \
	-device virtio-blk-device,drive=x0,bus=virtio-mmio-bus.0\
	-smp threads=$(CORE_NUM)
\end{lstlisting}  
不难发现最主要的差别在于后面多出的"-S -s"。 前面的S代表STOP, 意思是设置完虚拟机直接挂起, 停止一切执行, 直到接收到外部的continue信息为止.

第二个小写s等价于"-gdb tcp::1234",指的是开启远程调试,其在localhost(本机)的1234端口侦听GDB的信号, 等待连接。 然后, 就是我们的下一个工具GDB的任务和工作范畴了。

\begin{figure}[htb]
\centering
\includegraphics[width=\textwidth]{figures/02-02-GDB联调QEMU的逻辑流程.png}
\caption{
	GDB联调QEMU的逻辑流程
}
\label{fig:GDB联调QEMU的逻辑流程}
\end{figure}

\textbf{历史}
在GDB的命令行(和各种IDE的GDB控制台)中, 使用上下左右(或者Alt-P之类的自定义按键)可以直接显示之前的命令, 这和Bash是一致的。

但和Bash不同的是, 其重复执行上一条命令可以通过直接在不进行任何输入的时候敲回车实现:

\begin{lstlisting}[language={Rust}, label={code:forktest},
	caption={forktest.rs}]
	(gdb) stepi
	(gdb) (然后这里敲回车)
\end{lstlisting} 

这时候就前进了两条指令。但这个功能也会导致有时候多按了一下回车结果重复执行了某些只应当被执行一次的指令, 因此也要注意使用的场景。
\textbf{补全}

GDB命令的确数量庞大且内容复杂, 但是但作为一款老牌软件, GDB自然也有解决方案。一方面, GDB自己提供了强大的命令补全功能, 能像在Bash中一样TAB补全,例如如下所示的键盘输入:输入b然后按下TAB键, 会补全为break, 其他的, 如寄存器名称等往往也可以在有了部分提示前缀后进行补全。 因此不需要每次都键入完整的命令。

\textbf{辅助}

另一方面, 多数的IDE和编辑器都有辅助GDB的功能, 例如在某一行代码旁边点击行号附近的位置, 会出现一个圆点, 表示加入断点。

另外, 如果你需要重复某个命令多遍, 并不需要一直按着鼠标或者键盘回车键, 只需要在命令后面插入重复次数即可:

\begin{lstlisting}[language={Rust}, label={code:forktest},
	caption={forktest.rs}]
	(gdb) stepi 4
\end{lstlisting}

这里就前进了4条指令。

\textbf{常见命令}

进入gdb, 会看到"(gdb)"提示可以输入命令(有时候无法输入)。

如果在VSCode中使用, 还需要在之前加上exec

\begin{lstlisting}[language={Rust}, label={code:forktest},
	caption={forktest.rs}]
	exec gdb <...>
\end{lstlisting}

\textbf{设定命令}
(1) 连接

按照之前的方法启动QEMU后, GDB要通过下列命令连接本地的QEMU。

\begin{lstlisting}[language={Rust}, label={code:forktest},
	caption={forktest.rs}]
	target remote :1234
	
	<div align=center><img src="./pic/1.2/remote.png" style="zoom:100%"></div> 
	
	如果你之前指定了自定义的端口, 需要将1234换成其他的端口号。同时, 冒号之前实际上省略了localhost(也就是本机的“网址”), 如果你将来有自定义的地址或者网址, 也可以在前面补上。
\end{lstlisting}

(2) 加载调试信息

\begin{lstlisting}[language={Rust}, label={code:forktest},
	caption={forktest.rs}]
	(1)file
	在开始调试之前, 你首先需要加载调试信息。
	file target/riscv64gc-unknown-none-elf/debug/os
	从而加载os文件作为符号文件。请注意, 使用release版本的文件(在make命令中加入“MODE=release”得到)往往不带有任何的调试信息, 不适合用于debug, 但也不尽然: 你可以对着汇编语言调试。 当然, 就算使用了带有符号文件的版本,这种体验你总会遇到的, 因为操作系统总是要涉及某些底层。
	这里加载的是操作系统的符号, 那如果某些过程经过用户程序(作为一个操作系统, 你总会遇到这种问题), 如何添加用户程序的代码?这就要用到一下一个命令了。
	如果先连接QEMU后加载二进制文件, 就会出现上面的"A program is being debugged already."提示,当然这并不影响使用。
	
	$add-symbol-file
	
	$ add-symbol-file bash
	
\end{lstlisting}

如果你的当前工作文件夹中具有bash文件, 就可以直接添加, 否则需要自行前往特定的。 显然, 上面两个命令的顺序可以修改, 但注意, file只能有一个, symbol-file却可以有很多, 且符号文件指代的不一定是带有符号部分的可执行文件, 也可能是纯粹的符号文件(考虑到其和主题无关,这里的内容我们不拓展,读者可以进一步查找资料)。 这时候可以info files显示所有已经添加的符号和二进制文件

\begin{figure}[htb]
\centering
\includegraphics[width=\textwidth]{figures/02-02-info files.png}
\caption{
	info files
}
\label{fig:info files}
\end{figure}

\begin{lstlisting}[language={Rust}, label={code:forktest},
	caption={forktest.rs}]
	(1)directory
	设置完符号文件, 接着需要设置源代码目录, 这样在IDE/GDB中可以自动跳转到函数代码所在处。
	$dir ~/Downloads/SW/Bash/bash-5.1.16/
	$dir ~/Downloads/SW/Bash/bash-5.1.16/lib/sh/
	注意, 这里需要你将bash的源代码先提前下载好到某个目录, 并将上面的这个路径替换成正确的目录。
	最终得到:
	Source directories searched: /home/dragon/Downloads/SW/Bash/bash-5.1.16/lib/sh:/home/dragon/Downloads/SW/Bash/bash-5.1.16:$cdir:$cwd
	另外,部分的软件目录结构复杂, 这时候需要手动用上述命令添加。
	(2)break
	break用于设置断点。可以通过断点中断程序的执行并让你进入调试模式。
	一般常见的断点设定方式有:文件:行号格式和函数(方法)格式
	例如:(基于特定版本, 你的具体地址与行号可能不同)
	$(gdb) b src/main.rs:50
	Breakpoint 1 at 0x900000000004f158: file src/main.rs, line 59.
	$(gdb) b rust_main
	Breakpoint 2 at 0x900000000004f198: file src/main.rs, line 66.
	
	注意! 函数名方法有时候要指定域, 格式类似os::rust_main;
	如果要删除断点, 则可以
	$delete 1
	跟上断点号即可。
	(3)set
	set可以是多种的, 最典型的是让pc强制移动到某个位置, 例如:
	$set $pc=0x0
	回到最开始的执行点。 当然, 你也可以用它对别的地址/变量进行强行赋值。
	
\end{lstlisting}

\textbf{执行流}

(1)continue
很显然, GDB有两种状态, 停止和执行, 只有在停止的时候, 我们才能对其中的数据进行查看和修改, 对自己的命令进行调整,而continue正是从停止到执行的切换工具。

continue会继续执行程序直到遇到下一个断点或程序结束。 这条命令往往用于target remote :1234后继续执行。

例如, 我们一开始执行make gdb只有:

\begin{lstlisting}[language={Rust}, label={code:forktest},
	caption={forktest1.rs}]
	$ make gdb
\end{lstlisting}

一个光标停在原地。

但是, 如果你在gdb中输入continue并回车, 虚拟机马上就会停止冻结,开始执行指令(直到撞到某个停止条件为止):

\begin{figure}[htb]
\centering
\includegraphics[width=\textwidth]{figures/02-02-设置断点.png}
\caption{
	设置断点
}
\label{fig:设置断点}
\end{figure}

此时, 命令就会停在之前设定的断点上
(2)暂停或终止运行

在终端和多数IDE中, 暂停执行流是通过gdb控制台(注意不是虚拟机的终端, 而是gdb的控制台)ctrl-C(同时按下Ctrl和C)实现的, 这可以让程序暂停在当前执行到的位置。

如果你之前在执行QEMU时没有加入“-S”选项, 那么你可以用这条命令立即暂停执行流(当然其具体停止位置难以保证。)

完成后debug后, 可以用quit退出。

(3)next,step和finish

stepi前进一条指令.

\begin{figure}[htb]
\centering
\includegraphics[width=\textwidth]{figures/02-02-stepi.png}
\caption{
	stepi
}
\label{fig:stepi}
\end{figure}

next可以单步执行程序,跳过函数调用,例如(中间省略几步continue和break):

\begin{figure}[htb]
\centering
\includegraphics[width=\textwidth]{figures/02-02-next.png}
\caption{
	next
}
\label{fig:next}
\end{figure}

可以看到这里的几条函数都被跳过了。

step单步执行程序,进入函数调用(这里的执行流进入了函数调用):

\begin{figure}[htb]
\centering
\includegraphics[width=\textwidth]{figures/02-02-step.png}
\caption{
	step
}
\label{fig:step}
\end{figure}

finish执行完当前函数并返回到调用函数, 然后又回到了之前的函数:

\begin{figure}[htb]
\centering
\includegraphics[width=\textwidth]{figures/02-02-finish.png}
\caption{
	finish
}
\label{fig:finish}
\end{figure}

\textbf{查看命令}

在停止状态下, 我们可以backtrace显示函数调用栈:

\begin{figure}[htb]
\centering
\includegraphics[width=\textwidth]{figures/02-02-backtrace.png}
\caption{
	backtrace
}
\label{fig:backtrace}
\end{figure}

或者print打印寄存器/变量/内存地址的值:

\begin{figure}[htb]
	\centering
	\includegraphics[width=\textwidth]{figures/02-02-print.png}
	\caption{
		print
	}
	\label{fig:print}
\end{figure}

如果觉得每次都打印很麻烦,可以用display每次停在断点处时自动打印某个变量的值。一旦不需要, 可以undisplay该号码取消(类似delete语法),如下列这段话打印附近前后各6条汇编代码(其他的变量也可以打印,语法类似)

\begin{lstlisting}[language={Rust}, label={code:forktest},
	caption={forktest1.rs}]
	display/12i $pc-6*4
\end{lstlisting}

\begin{figure}[htb]
\centering
\includegraphics[width=\textwidth]{figures/02-02-display.png}
\caption{
	display
}
\label{fig:display}
\end{figure}

也可以info registers打印所有的寄存器

\begin{figure}[htb]
\centering
\includegraphics[width=\textwidth]{figures/02-02-info registers.png}
\caption{
	info registers
}
\label{fig:info registers}
\end{figure}

\textbf{自定义命令与脚本}

在使用GDB进行调试时,我们需要多次执行一些常见的指令。为了提高调试效率,我们可以使用define命令来定义自己的命令,简化重复操作。

下面我们来介绍如何定义一个自定义命令。首先,使用文本编辑器创建一个gdb脚本文件,例如mycommands.gdb。然后,在该文件中添加以下代码:

\begin{lstlisting}[language={Rust}, label={code:forktest},
	caption={forktest1.rs}]
	# 加载符号文件
	file target/riscv64gc-unknown-none-elf/debug/os
	add-symbol-file bash
	# 井号加入注释
	define mynext
	stepi
	info registers
	end
	# 添加bash的源代码
	dir 
\end{lstlisting}

这个自定义命令名为mynext,执行的操作包括执行下一条指令和显示所有寄存器的值。

接下来,启动GDB并加载定义的自定义命令。我们可以通过以下命令将mycommands.gdb文件加载到GDB中:

\begin{lstlisting}[language={Rust}, label={code:forktest},
	caption={forktest1.rs}]
	source mycommands.gdb
\end{lstlisting}

现在,我们可以在GDB中使用mynext命令来执行下一条指令并显示所有寄存器的值。只需在GDB提示符下输入“mynext”即可。

使用define自定义命令可以帮助我们快速地执行常见的调试操作,提高调试效率。 另外,其他的断点添加, 远程调试连接等命令也可以很方便地加入其中从而加速Debug过程。

在命令行下启动 GDB 并加载脚本时,可以使用 -x 或 –command 选项来指定要执行的脚本文件。该选项后跟要执行的脚本文件路径,如下所示:

\begin{lstlisting}[language={Rust}, label={code:forktest},
	caption={forktest1.rs}]
	# -x 选项指定了要执行的脚本文件路径,file_to_debug 则是要调试的目标程序的路径。
	gdb -x /path/to/script file_to_debug
	# 除了 -x 选项外,还可以使用 --init-command 选项指定要执行的初始化命令,该选项可以多次使用,每次指定一条命令,如下所示:
	gdb --init-command="set print pretty on" --init-command="set pagination off" file_to_debug
	# 上述命令中,--init-command 选项指定了要执行的初始化命令,可以多次使用,每次指定一条命令。
\end{lstlisting}


\subsection{git bisect——快速问题定位}
大家一定听过二分查找的算法, 如果我们发现某个Bug出现, 其实也可以通过二分查找定位到出错的版本。
git也自带了这个功能。使用git bisect, 我们可以在Git版本控制系统中进行二分查找,在版本历史中快速定位错误引入的位置。
\textbf{一般步骤}
我们以这些图文为例给出一个错误的处理
\begin{lstlisting}[language={Rust}, label={code:forktest},]
	$ git bisect start
	//运行 git bisect start 命令来启动一个二分查找会话。
	$ git bisect bad
	//用 git bisect bad 命令告诉 Git 当前版本存在问题。
	$ git bisect good HEAD~10
	//用 git bisect good 命令告诉 Git 一个知道没有问题的提交, 是这个提交的哈希值或分支名。git会给出估计的剩余步骤数
	Bisecting: 4 revisions left to test after this (roughly 2 steps)
	[f55d253527e3a72f730b06e6fbfe5e64f8594a27] fix: Change ELF related AuxV data alignment to repr(C), fixing the LPF.
	//Git 会自动切换到一个介于上述两个提交之间的提交,你需要在该提交上运行你的程序,检查问题是否存在。
	如果有问题,使用 git bisect bad 命令告诉 Git,否则使用 git bisect good 命令告诉 Git。
	你可以使用Git bisect run 切换提交后自动执行的命令(注意, git bisect run后仍然在原地, 这时候需要git bisect next才能进入下一个, 否则会冲突)
	Git 会根据你的反馈自动切换到下一个介于两个提交之间的提交,重复上述步骤,直到找到引入问题的提交。
	最后,使用 git bisect reset 命令退出二分查找会话。
\end{lstlisting}

\begin{figure}[htb]
	\centering
	\includegraphics[width=\textwidth]{figures/02-02-运行实例.png}
	\caption{
		运行实例
	}
	\label{fig:运行实例}
\end{figure}
\textbf{log与冲突回撤}
如果发现某个提交被错误标记(比如bad被错误标记为good), 尝试
\begin{lstlisting}[language={Rust}, label={code:forktest},]
	git bisect bad
	你会得到以下错误:
	ba246b7c5294eeabcdca705fafe8ea1015e6fd6e was both good and bad
	
	$ git bisect log//导出日志:
	
	git bisect start
	# good: [ba246b7c5294eeabcdca705fafe8ea1015e6fd6e] add: Expect script and clauses in Makefile for automation
	git bisect good ba246b7c5294eeabcdca705fafe8ea1015e6fd6e
	# bad: [ba246b7c5294eeabcdca705fafe8ea1015e6fd6e] add: Expect script and clauses in Makefile for automation
	git bisect bad ba246b7c5294eeabcdca705fafe8ea1015e6fd6e
	//重定向到文件
	$ git bisect log > bis.log
	//考虑修改错误
	$ git bisect start
	会得到以下信息:
	# bad: [ba246b7c5294eeabcdca705fafe8ea1015e6fd6e] add: Expect script and clauses in Makefile for automation
	$ git bisect bad ba246b7c5294eeabcdca705fafe8ea1015e6fd6e
	
	$ git bisect replay bis.log
	//回复到之前的状态
\end{lstlisting}  

\section{NPUcore内核代码结构及内核构建目标}
\subsection{NPUcore内核代码树}
下面是NPUcore的内核代码树:
\begin{figure}[htb]
	\centering
	\includegraphics[width=\textwidth]{figures/02-03-内核代码树1.png}
	\caption{
		内核代码树1
	}
	\label{fig:内核代码树1}
\end{figure}

\begin{figure}[htb]
	\centering
	\includegraphics[width=\textwidth]{figures/02-03-内核代码树2.png}
	\caption{
		内核代码树2
	}
	\label{fig:内核代码树2}
\end{figure}

\begin{figure}[htb]
	\centering
	\includegraphics[width=\textwidth]{figures/02-03-内核代码树3.png}
	\caption{
		内核代码树3
	}
	\label{fig:内核代码树3}
\end{figure}

\begin{figure}[htb]
	\centering
	\includegraphics[width=\textwidth]{figures/02-03-内核代码树4.png}
	\caption{
		内核代码树4
	}
	\label{fig:内核代码树4}
\end{figure}


\subsection{NPUcore学习路线}
下面是我们为你准备的NPUcore的学习路线
\begin{figure}[htb]
	\centering
	\includegraphics[width=\textwidth]{figures/02-03-NPUcore学习路线.png}
	\caption{
		NPUcore学习路线
	}
	\label{fig:NPUcore学习路线}
\end{figure}

希望你能渐渐的喜爱上操作系统~

\section{用户地址空间}
\begin{figure}[htb]
    \centering
    \includegraphics[width=\textwidth]{figures/04-04-用户地址空间示意图.png}
    \caption{
        用户虚拟地址空间示意
    }
    \label{fig:user virtual process}
\end{figure}
每一个进程都有一个独立的页表,当npucore实现进程切换的时候,对应的页表也会切换。

当一个用户进程通过系统调用向操作系统请求更多的用户空间时,npucore首先在os/src/frame_allocator.rs中的StackFrameAllocator的基于栈的数据结构实现空闲物理页的分配,然后将物理页的映射加入到用户的页表当中。
npucore将设置PTEflags::R,PTEflags::W,PTEflags::U,PTEflags::X以及PTEflags::V标志位到对应的页表项目,使得用户可以对分配的页面进行读写操作。
大多数的用户进程并不能完全利用所有的虚拟空间,对于没有用到的空间,它对应的页表项PTEflags::V标志位始终为0。

页表的设计有很多的好处。首先,首先不同的进程使用不同的页表,相同的虚拟地址映射到不同的物理地址,因此每一个进程可以拥有自己独立的内存空间。其次,用户的虚拟地址是连续的,对应的物理地址不一定是连续的,这样可以有效的避免内存碎片。最后,内核将所有的用户的跳板代码都映射到了同一段虚地址,可以有效的实现上下文切换。

如图\ref{fig:user virtual process}所示,用户的虚拟地址空间被分为了三个部分,分别是用户代码段,用户数据段以及用户堆栈段。用户代码段用于存放用户的代码,用户数据段用于存放用户的数据,用户堆栈段用于存放用户的堆栈。
其中,用户栈的初始内容如图中所示,由execve函数完成初始化,其中包含了用户的命令行参数和返回地址,紧接着就是main函数使用的栈空间。
堆是动态内存分配的区域,用于存储在运行时分配的数据。在npucore中,堆的起始地址通常在数据段结束后,并根据需要进行扩展。通过系统调用(如mmap和sbrk),用户程序可以动态地管理堆内存。
栈是用于存储函数调用和局部变量的区域。在npucore中,栈从高地址向低地址生长,通常位于用户地址空间的顶部。栈的大小可以通过操作系统配置或在运行时动态调整。。

代码段是存储进程执行指令的区域。在npucore中,代码段通常从虚拟地址0开始,并包含可执行程序的指令。这个区域对应于用户程序的文本部分。
数据段包含了全局变量和静态变量等数据。npucore将数据段安排在代码段之后,从一个虚拟地址开始,并在运行时动态地分配和使用。

用户在代码段执行用户程序,将elf指定的数据段内容存放在相应的数据段位置。因为我们对与内核的映射是直接映射,所以操作系统对于用户的地址空间是直接可见的。
同时,堆栈段的大小是动态变化的,当用户程序需要更多的堆栈空间时,会通过系统调用向操作系统请求更多的堆栈空间,操作系统会在用户的地址空间中分配更多的堆栈空间,并将堆栈空间映射到用户的地址空间中

npucore实现了execve来将elf文件加载到内存的进程地址空间中,实现了sbrk来动态的分配用户空间,实现了mmap来将文件映射到用户空间,实现了munmap来取消文件的映射。
execve将elf文件中程序所规定的代码和数据加载到内存中,然后操作系统需要建议对应的映射,从而实现了用户地址空间的建立。


\subsection{sbrk系统调用}
sbrk系统调用是早期的Unix系统中的一个系统调用,用于动态的分配用户空间。sbrk系统调用的原型如下:
\begin{lstlisting}[language=c]
    void *sbrk(intptr_t increment);
\end{lstlisting}
sbrk系统调用将堆的大小增加increment字节,并返回堆的起始地址。
如果increment为负数,则堆的大小减少increment字节。如果堆的大小超过了进程的地址空间,则sbrk系统调用返回-1,并设置errno为ENOMEM。
sbrk系统调用可以为一个进程扩大或者缩小堆的大小,主要的实现是由os/src/memory_set.rs中的sbrk函数完成。
sbrk函数调用Memoryset::mmap或者Memoryset::munmap来实现堆的扩大或者缩小。
mmap函数不仅用于sbrk系统调用,还用于mmap系统调用,用于将文件映射到用户空间和开辟匿名内存映射。
\begin{lstlisting}[language=rust,caption={sbrk}]
    pub fn sbrk(&mut self, heap_pt: usize, heap_bottom: usize, increment: isize) -> usize {
        let old_pt: usize = heap_pt;
        let new_pt: usize = old_pt + increment as usize;
        // 判断扩大堆还是缩小堆
        if increment > 0 {
            let limit = heap_bottom + USER_HEAP_SIZE;
            if new_pt > limit {
                return old_pt;
            } else {
                self.mmap(
                    old_pt,
                    increment as usize,
                    MapPermission::R | MapPermission::W | MapPermission::U,
                    MapFlags::MAP_ANONYMOUS | MapFlags::MAP_FIXED | MapFlags::MAP_PRIVATE,
                    1usize.wrapping_neg(),
                    0,
                );
                trace!("[sbrk] heap area expanded to {:X}", new_pt);
            }
        } else if increment < 0 {
            // 如果缩小后的堆地址小于堆底地址,则不进行缩小
            if new_pt <= heap_bottom {
                return old_pt;
            } else {
                self.munmap(old_pt, increment as usize).unwrap();
            }
        }
        new_pt
    }
\end{lstlisting}
sbrk的实现依赖于函数mmap的实现,将在接下来的章节介绍,mmap是比sbrk更加灵活的系统调用,能狗处理更复杂的内存分配和使用的情况。


\subsection{mmap}
\begin{lstlisting}[language=rust,caption={sys_mmap}]
    pub fn sys_mmap(
        start: usize,
        len: usize,
        prot: usize,
        flags: usize,
        fd: usize,
        offset: usize,
    ) -> isize
\end{lstlisting}
参数start:指向欲映射的内存起始地址,通常设为 NULL,代表让系统自动选定地址,映射成功后返回
该地址。
\begin{itemize}
    \item 参数len:代表将文件中多大的部分映射到内存。
    \item 参数prot:映射区域的保护方式。
    \item 参数flags:影响映射区域的各种特性。在调用mmap()时必须要指定MAP_SHARED 或MAP_PRIVATE。MAP_FIXED 如果参数start所指的地址无法成功建立映射时,则放弃映射,不对地址做修正。通常不鼓励用此旗标。
    MAP_SHARED对映射区域的写入数据会复制回文件内,而且允许其他映射该文件的进程共享。MAP_PRIVATE 对映射区域的写入操作会产生一个映射文件的复制,即私人的“写入时复制”(copy on
    write)对此区域作的任何修改都不会写回原来的文件内容。MAP_ANONYMOUS建立匿名映射。此时会忽略参数fd,不涉及文件,而且映射区域无法和其他进程共
    享。MAP_DENYWRITE只允许对映射区域的写入操作,其他对文件直接写入的操作将会被拒绝。MAP_LOCKED 将映射区域锁定住,这表示该区域不会被置换(swap)。
    \item 参数fd:要映射到内存中的文件描述符。如果使用匿名内存映射时,即flags中设置了MAP_ANONYMOUS,fd设为-1。
    \item 参数offset:文件映射的偏移量,通常设置为0,代表从文件最前方开始对应,offset必须是分页大小的整数倍。
    \item 返回值:若映射成功则返回映射区的内存起始地址,否则返回-1。
\end{itemize}

\begin{figure}[htb]
    \centering
    \includegraphics[width=\textwidth]{figures/04-04-内存地址空间.png}
    \caption{
        内存地址空间
    }
    \label{fig:内存地址空间}
\end{figure}

mmap用于把文件映射到用户空间中,简单说mmap就是把一个文件的内容在内存里面做一个映像。映
射成功后,用户对这段内存区域的修改可以直接反映到内核空间,同样,内核空间对这段区域的修改也
直接反映用户空间。那么对于内核空间<---->用户空间两者之间需要大量数据传输等操作的话效率是非
常高的。进程可以像读写内存一样对普通文件的操作。mmap系统调用使得进程之间通过映射同一个普
通文件实现共享内存。
UNIX网络编程第二卷进程间通信对mmap函数进行了说明。该函数主要用途有三个:
1、将一个普通文件映射到内存中,通常在需要对文件进行频繁读写时使用,这样用内存读写取代I/O读
写,以获得较高的性能;
2、将特殊文件进行匿名内存映射,可以为关联进程提供共享内存空间;
3、为无关联的进程提供共享内存空间,一般也是将一个普通文件映射到内存中。
第一步:找到最后一次mmap映射区域
\begin{lstlisting}[language=rust]
    let idx = self.last_mmap_area_idx();
\end{lstlisting}
每个进程所申请的用户地址空间是用vector存储的vm_area_struct的结构体,其中start和end就是代表
这一个area虚拟地址的开始和结束,而我们的结构体在vector中是按照虚拟地址升序排序的,也就是说
我们会找到最后一个符合要求的vector下标


第二步:mmap获取映射区域的起始地址,如果设置了MAP_FIXED,则会直接使用用户的参数start,
不对地址做修正。
\begin{lstlisting}[language=rust,caption={mmap}]
    let start_va: VirtAddr = if flags.contains(MapFlags::MAP_FIXED) {
        // unmap if exists
        self.munmap(start, len);
        start.into()
    }
\end{lstlisting}
如果没有设置,mmap就会拿到先前得到的最后一次mmap映射区域。如果设置了MAP_PRIVATE 或者
MAP_ANONYMOUS,就直接将要新映射的区域new_area与最后一次mmap映射区域合并并返回,没
有设置就会将最后一次mmap映射区域的末尾作为新映射的区域的起始。而新映射的区域的末尾则根据
用户参数len设置。
\begin{lstlisting}[language=rust,caption={mmap}]
    else {
        if let Some(idx) = idx {
        let area = &mut self.areas[idx];
        if flags.contains(MapFlags::MAP_PRIVATE | MapFlags::MAP_ANONYMOUS)
        && prot == area.map_perm
        && area.map_file.is_none()
        {
        let end_va: VirtAddr = area.inner.vpn_range.get_end().into();
        area.expand_to(VirtAddr::from(end_va.0 + len)).unwrap();
        return end_va.0 as isize;
        }
        area.inner.vpn_range.get_end().into()
        } else {
        MMAP_BASE.into()
        }
        };
        let mut new_area = MapArea::new(
        start_va,
        VirtAddr::from(start_va.0 + len),
        MapType::Framed,
        prot,
        None,
    );
        
\end{lstlisting}
第三步:mmap会判断是否是文件映射,即是否包含MAP_ANONYMOUS。如果没有包含,就会先获取
fd_table,并根据用户参数fd从fd_table找到对应的文件描述符,这里文件相关内容不作赘述。然后设
置文件偏移量offset并置入new_area中
\begin{lstlisting}[language=rust,caption={mmap}]
    if !flags.contains(MapFlags::MAP_ANONYMOUS) {
        let fd_table = task.files.lock();
        match fd_table.get_ref(fd) {
        Ok(file_descriptor) => {
        if !file_descriptor.readable() {
        return EACCES;
        }
        let file = file_descriptor.file.deep_clone();
        file.lseek(offset as isize, SeekWhence::SEEK_SET).unwrap();
        new_area.map_file = Some(file);
        }
        Err(errno) => return errno,
        }
    }
\end{lstlisting}
第四步:mmap将new_area加入到先前的vector中,而且需要保持原vector的有序性,这里使用了rust
语言的特性。
\begin{lstlisting}[language=rust,caption={mmap}]
    let (idx, _) = self
    .areas
    .iter()
    .enumerate()
    .skip_while(|(_, area)| {
    area.inner.vpn_range.get_start() >= VirtAddr::from(MMAP_END).into()
    })
    .find(|(_, area)| area.inner.vpn_range.get_start() >= start_va.into())
    .unwrap();
    self.areas.insert(idx, new_area);
\end{lstlisting}

\subsection{execve}
应用程序自身的角度来看,进程 (Process) 的一个经典定义是一个正在运行的程序实例。当程序运行在操作系统中的时候,从程序的视角来看,它会产生一种“幻觉”:即该程序是整个计算机系统中当前运行的唯一的程序,能够独占使用处理器、内存和外设,而且程序中的代码和数据是系统内存中唯一的对象。体表现为“进程”这个抽象概念。站在计算机系统和操作系统的角度来看,并不存在这种“幻觉”。事实上,在一段时间之内,往往会有多个程序同时或交替在操作系统上运行,因此程序并不能独占整个计算机系统。具体而言,进程是应用程序的一次执行过程。并且在这个执行过程中,由“操作系统”执行环境来管理程序执行过程中的 **进程上下文** – 一种控制流上下文。这里的进程上下文是指程序在运行中的各种物理/虚拟资源(寄存器、可访问的内存区域、打开的文件、信号等)的内容,特别是与程序执行相关的具体内容:内存中的代码和数据,栈、堆、当前执行的指令位置(程序计数器的内容)、当前执行时刻的各个通用寄存器中的值等。
\begin{figure}[htb]
    \centering
    \includegraphics[width=\textwidth]{figures/04-04-进程的地址空间.png}
    \caption{
        进程的地址空间
    }
    \label{fig:进程的地址空间}
\end{figure}

exec 用一个新的程序来代替当前进程的内存和寄存器,但是其文件描述符、进程 id 和父进程都是不变的。
它根据文件系统中保存的某个文件来初始化用户部分。`exec`通过 `open()`打开二进制文件。然后,它读取 ELF 头。应用程序以通行的 ELF 格式来描述。一个 ELF 二进制文件包括了一个 ELF 头,然后是连续几个程序段的头。
exec 会检查文件是否包含 ELF 二进制代码。一个 ELF 二进制文件是以4个“魔法数字”开头的,即 0x7F,“E”,“L”,“F”。如果 ELF 头中包含正确的魔法数字,`exec` 就会认为该二进制文件的结构是正确的。

\begin{lstlisting}[language=rust,caption={execve}]
    pub fn sys_execve(
        pathname: *const u8,
        mut argv: *const *const u8,
        mut envp: *const *const u8,
    ) -> isize
\end{lstlisting}
execve()的前置条件
fork()复制某一个进程,为execve()的执行,准备条件。
execve()的核心工作:
参数转换,将argv,envp转为Vec<String>。
open()打开path路径下的可执行文件,对其进行检查。
load_elf()将其加载到当前的空间中。
load_elf的核心工作:
将elf文件映射到内核空间的`MMAP_BASE`地址处。
在map_elf()中,把elf文件的内容拷贝到用户的MemorySet中


\chapter{编写第一个系统调用}

什么是系统调用?在我们使用C语言编程时,使用过库函数提供的一些基本的函数,例如:控制台输出、文件读写。
我们使用库函数完成基本的操作,库函数是对操作系统提供的系统调用的进一步封装,并隐藏掉了一些操作。
系统调用工作在最底层,使用POSIX提供的接口,直接操作硬件,完成最基本的操作。

为什么需要借助于操作系统,用户不能直接操作硬件呢?
凡是与资源有关的操作、会影响到其它进程的操作,为了方便管理资源(防止恶意操作)、使进程间隔离,
操作系统必须介入,实现统一管理调度。操作系统为上层编程语言提供了一套接口,这套接口就是系统调用。
用户库封装系统调用为库函数还有以下优点:

\textbf{1. 简化用户程序的编写:}通过封装系统调用,用户程序可以使用更为简单和直观的接口来完成复杂的系统操作,
无需了解系统调用的底层细节,减少程序员的开发难度。

\textbf{2. 提高代码的可维护性:}通过库函数的封装,程序员可以对库函数进行多层封装,使程序的可读性、可维护性更高。
当需要进行修改时,只需修改库函数的实现,无需修改应用程序的代码,降低了代码的耦合度,减少了代码维护的成本。

\textbf{3. 提高程序的移植性:}不同的操作系统和硬件平台实现系统调用的方式可能略有不同。
使用库函数来封装系统调用可以提高程序的移植性。如果需要在不同操作系统下运行程序,只需更改库函数的实现即可。

\textbf{4. 方便进行错误处理:}库函数可以对系统调用返回值进行处理,根据返回值不同的情况进行错误处理。
在使用系统调用时,程序员需要手动进行错误处理,使用库函数可以减轻程序员的工作量。

总之,封装系统调用为库函数可以使得程序更加简单、稳定、易维护、易移植,并且可以提高程序员的开发效率。

为了区分一个操作是用户完成的,还是依赖于操作系统完成的,每种指令集体系结构都对此做出了区分。
以RISC-V架构为例,CPU的工作状态分为用户态、内核态等。执行用户程序指令时的状态为用户态,需要发起系统调用时,
库函数中ecall指令会使CPU发生陷入提高特权级,到达内核态。内核态完成操作后,使用ret指令降低特权级,回到用户态。

本章将首先以三个系统调用为例子,讲解在用户态程序中如何使用系统调用,之后使用调试工具跟踪观察系统调用的实现,
最后将对系统调用以及用户态、内核态等展开详细介绍。

\section{使用系统调用}

本节将以三个常见系统调用为例,简要介绍在用户态用户进程是如何使用系统调用的。

\subsection{fork}

fork是一种用于克隆进程的全部内存空间的系统调用,是Linux系统中创建一个新进程的重要方法,除了第一个进程,
所有的进程都是由fork创建的。

这是一个 Rust 语言编写的程序,主要目的是展示如何使用 fork 函数创建子进程,如\autoref{code:forktest}所示。

\begin{lstlisting}[language={Rust}, label={code:forktest},
    caption={forktest.rs}]
#![no_std]
#![no_main]

#[macro_use]
extern crate user_lib;

use user_lib::{exit, fork, wait};

const MAX_CHILD: usize = 30;

#[no_mangle]
pub fn main() -> i32 {
    for i in 0..MAX_CHILD {
        let pid = fork();
        if pid == 0 {
            println!("I am child {}", i);
            exit(0);
        } else {
            println!("forked child pid = {}", pid);
        }
        assert!(pid > 0);
    }
    let mut exit_code: i32 = 0;
    for _ in 0..MAX_CHILD {
        if wait(&mut exit_code) <= 0 {
            panic!("wait stopped early");
        }
    }
    if wait(&mut exit_code) > 0 {
        panic!("wait got too many");
    }
    println!("forktest pass.");
    0
}
\end{lstlisting}

\subsection{exec}
\subsection{sbrk}
\section{利用GDB跟踪getpid系统调用}

NPUcore的系统调用是基于中断来实现的,大致会经历以下步骤,如\autoref{table:系统调用通用过程}:
(用户态,内核态由CPU特定寄存器中的几位来表示)

\begin{table}[h]
    \centering
    \caption{系统调用通用过程}
    \label{table:系统调用通用过程}
    \begin{tabular}{|c|c|}
        \hline
        \textbf{用户态}  & \textbf{内核态}     \\\hline
                        & hello.c(执行ecall) \\\hline
        硬件断点保存     &                    \\\hline
        OS手动断点保存   &                    \\\hline
        中断处理         &                    \\\hline
        中断返回,OS手动断点恢复 &                \\\hline
        ret 硬件断点恢复 &                    \\\hline
                        & hello.c继续执行     \\\hline
    \end{tabular}
\end{table}

下面请你自己动手,使用调试软件跟踪一遍系统调用。
\section{系统调用机制与中断}
\section{用户地址空间}
\begin{figure}[htb]
    \centering
    \includegraphics[width=\textwidth]{figures/04-04-用户地址空间示意图.png}
    \caption{
        用户虚拟地址空间示意
    }
    \label{fig:user virtual process}
\end{figure}
每一个进程都有一个独立的页表,当npucore实现进程切换的时候,对应的页表也会切换。

当一个用户进程通过系统调用向操作系统请求更多的用户空间时,npucore首先在os/src/frame_allocator.rs中的StackFrameAllocator的基于栈的数据结构实现空闲物理页的分配,然后将物理页的映射加入到用户的页表当中。
npucore将设置PTEflags::R,PTEflags::W,PTEflags::U,PTEflags::X以及PTEflags::V标志位到对应的页表项目,使得用户可以对分配的页面进行读写操作。
大多数的用户进程并不能完全利用所有的虚拟空间,对于没有用到的空间,它对应的页表项PTEflags::V标志位始终为0。

页表的设计有很多的好处。首先,首先不同的进程使用不同的页表,相同的虚拟地址映射到不同的物理地址,因此每一个进程可以拥有自己独立的内存空间。其次,用户的虚拟地址是连续的,对应的物理地址不一定是连续的,这样可以有效的避免内存碎片。最后,内核将所有的用户的跳板代码都映射到了同一段虚地址,可以有效的实现上下文切换。

如图\ref{fig:user virtual process}所示,用户的虚拟地址空间被分为了三个部分,分别是用户代码段,用户数据段以及用户堆栈段。用户代码段用于存放用户的代码,用户数据段用于存放用户的数据,用户堆栈段用于存放用户的堆栈。
其中,用户栈的初始内容如图中所示,由execve函数完成初始化,其中包含了用户的命令行参数和返回地址,紧接着就是main函数使用的栈空间。
堆是动态内存分配的区域,用于存储在运行时分配的数据。在npucore中,堆的起始地址通常在数据段结束后,并根据需要进行扩展。通过系统调用(如mmap和sbrk),用户程序可以动态地管理堆内存。
栈是用于存储函数调用和局部变量的区域。在npucore中,栈从高地址向低地址生长,通常位于用户地址空间的顶部。栈的大小可以通过操作系统配置或在运行时动态调整。。

代码段是存储进程执行指令的区域。在npucore中,代码段通常从虚拟地址0开始,并包含可执行程序的指令。这个区域对应于用户程序的文本部分。
数据段包含了全局变量和静态变量等数据。npucore将数据段安排在代码段之后,从一个虚拟地址开始,并在运行时动态地分配和使用。

用户在代码段执行用户程序,将elf指定的数据段内容存放在相应的数据段位置。因为我们对与内核的映射是直接映射,所以操作系统对于用户的地址空间是直接可见的。
同时,堆栈段的大小是动态变化的,当用户程序需要更多的堆栈空间时,会通过系统调用向操作系统请求更多的堆栈空间,操作系统会在用户的地址空间中分配更多的堆栈空间,并将堆栈空间映射到用户的地址空间中

npucore实现了execve来将elf文件加载到内存的进程地址空间中,实现了sbrk来动态的分配用户空间,实现了mmap来将文件映射到用户空间,实现了munmap来取消文件的映射。
execve将elf文件中程序所规定的代码和数据加载到内存中,然后操作系统需要建议对应的映射,从而实现了用户地址空间的建立。


\subsection{sbrk系统调用}
sbrk系统调用是早期的Unix系统中的一个系统调用,用于动态的分配用户空间。sbrk系统调用的原型如下:
\begin{lstlisting}[language=c]
    void *sbrk(intptr_t increment);
\end{lstlisting}
sbrk系统调用将堆的大小增加increment字节,并返回堆的起始地址。
如果increment为负数,则堆的大小减少increment字节。如果堆的大小超过了进程的地址空间,则sbrk系统调用返回-1,并设置errno为ENOMEM。
sbrk系统调用可以为一个进程扩大或者缩小堆的大小,主要的实现是由os/src/memory_set.rs中的sbrk函数完成。
sbrk函数调用Memoryset::mmap或者Memoryset::munmap来实现堆的扩大或者缩小。
mmap函数不仅用于sbrk系统调用,还用于mmap系统调用,用于将文件映射到用户空间和开辟匿名内存映射。
\begin{lstlisting}[language=rust,caption={sbrk}]
    pub fn sbrk(&mut self, heap_pt: usize, heap_bottom: usize, increment: isize) -> usize {
        let old_pt: usize = heap_pt;
        let new_pt: usize = old_pt + increment as usize;
        // 判断扩大堆还是缩小堆
        if increment > 0 {
            let limit = heap_bottom + USER_HEAP_SIZE;
            if new_pt > limit {
                return old_pt;
            } else {
                self.mmap(
                    old_pt,
                    increment as usize,
                    MapPermission::R | MapPermission::W | MapPermission::U,
                    MapFlags::MAP_ANONYMOUS | MapFlags::MAP_FIXED | MapFlags::MAP_PRIVATE,
                    1usize.wrapping_neg(),
                    0,
                );
                trace!("[sbrk] heap area expanded to {:X}", new_pt);
            }
        } else if increment < 0 {
            // 如果缩小后的堆地址小于堆底地址,则不进行缩小
            if new_pt <= heap_bottom {
                return old_pt;
            } else {
                self.munmap(old_pt, increment as usize).unwrap();
            }
        }
        new_pt
    }
\end{lstlisting}
sbrk的实现依赖于函数mmap的实现,将在接下来的章节介绍,mmap是比sbrk更加灵活的系统调用,能狗处理更复杂的内存分配和使用的情况。


\subsection{mmap}
\begin{lstlisting}[language=rust,caption={sys_mmap}]
    pub fn sys_mmap(
        start: usize,
        len: usize,
        prot: usize,
        flags: usize,
        fd: usize,
        offset: usize,
    ) -> isize
\end{lstlisting}
参数start:指向欲映射的内存起始地址,通常设为 NULL,代表让系统自动选定地址,映射成功后返回
该地址。
\begin{itemize}
    \item 参数len:代表将文件中多大的部分映射到内存。
    \item 参数prot:映射区域的保护方式。
    \item 参数flags:影响映射区域的各种特性。在调用mmap()时必须要指定MAP_SHARED 或MAP_PRIVATE。MAP_FIXED 如果参数start所指的地址无法成功建立映射时,则放弃映射,不对地址做修正。通常不鼓励用此旗标。
    MAP_SHARED对映射区域的写入数据会复制回文件内,而且允许其他映射该文件的进程共享。MAP_PRIVATE 对映射区域的写入操作会产生一个映射文件的复制,即私人的“写入时复制”(copy on
    write)对此区域作的任何修改都不会写回原来的文件内容。MAP_ANONYMOUS建立匿名映射。此时会忽略参数fd,不涉及文件,而且映射区域无法和其他进程共
    享。MAP_DENYWRITE只允许对映射区域的写入操作,其他对文件直接写入的操作将会被拒绝。MAP_LOCKED 将映射区域锁定住,这表示该区域不会被置换(swap)。
    \item 参数fd:要映射到内存中的文件描述符。如果使用匿名内存映射时,即flags中设置了MAP_ANONYMOUS,fd设为-1。
    \item 参数offset:文件映射的偏移量,通常设置为0,代表从文件最前方开始对应,offset必须是分页大小的整数倍。
    \item 返回值:若映射成功则返回映射区的内存起始地址,否则返回-1。
\end{itemize}

\begin{figure}[htb]
    \centering
    \includegraphics[width=\textwidth]{figures/04-04-内存地址空间.png}
    \caption{
        内存地址空间
    }
    \label{fig:内存地址空间}
\end{figure}

mmap用于把文件映射到用户空间中,简单说mmap就是把一个文件的内容在内存里面做一个映像。映
射成功后,用户对这段内存区域的修改可以直接反映到内核空间,同样,内核空间对这段区域的修改也
直接反映用户空间。那么对于内核空间<---->用户空间两者之间需要大量数据传输等操作的话效率是非
常高的。进程可以像读写内存一样对普通文件的操作。mmap系统调用使得进程之间通过映射同一个普
通文件实现共享内存。
UNIX网络编程第二卷进程间通信对mmap函数进行了说明。该函数主要用途有三个:
1、将一个普通文件映射到内存中,通常在需要对文件进行频繁读写时使用,这样用内存读写取代I/O读
写,以获得较高的性能;
2、将特殊文件进行匿名内存映射,可以为关联进程提供共享内存空间;
3、为无关联的进程提供共享内存空间,一般也是将一个普通文件映射到内存中。
第一步:找到最后一次mmap映射区域
\begin{lstlisting}[language=rust]
    let idx = self.last_mmap_area_idx();
\end{lstlisting}
每个进程所申请的用户地址空间是用vector存储的vm_area_struct的结构体,其中start和end就是代表
这一个area虚拟地址的开始和结束,而我们的结构体在vector中是按照虚拟地址升序排序的,也就是说
我们会找到最后一个符合要求的vector下标


第二步:mmap获取映射区域的起始地址,如果设置了MAP_FIXED,则会直接使用用户的参数start,
不对地址做修正。
\begin{lstlisting}[language=rust,caption={mmap}]
    let start_va: VirtAddr = if flags.contains(MapFlags::MAP_FIXED) {
        // unmap if exists
        self.munmap(start, len);
        start.into()
    }
\end{lstlisting}
如果没有设置,mmap就会拿到先前得到的最后一次mmap映射区域。如果设置了MAP_PRIVATE 或者
MAP_ANONYMOUS,就直接将要新映射的区域new_area与最后一次mmap映射区域合并并返回,没
有设置就会将最后一次mmap映射区域的末尾作为新映射的区域的起始。而新映射的区域的末尾则根据
用户参数len设置。
\begin{lstlisting}[language=rust,caption={mmap}]
    else {
        if let Some(idx) = idx {
        let area = &mut self.areas[idx];
        if flags.contains(MapFlags::MAP_PRIVATE | MapFlags::MAP_ANONYMOUS)
        && prot == area.map_perm
        && area.map_file.is_none()
        {
        let end_va: VirtAddr = area.inner.vpn_range.get_end().into();
        area.expand_to(VirtAddr::from(end_va.0 + len)).unwrap();
        return end_va.0 as isize;
        }
        area.inner.vpn_range.get_end().into()
        } else {
        MMAP_BASE.into()
        }
        };
        let mut new_area = MapArea::new(
        start_va,
        VirtAddr::from(start_va.0 + len),
        MapType::Framed,
        prot,
        None,
    );
        
\end{lstlisting}
第三步:mmap会判断是否是文件映射,即是否包含MAP_ANONYMOUS。如果没有包含,就会先获取
fd_table,并根据用户参数fd从fd_table找到对应的文件描述符,这里文件相关内容不作赘述。然后设
置文件偏移量offset并置入new_area中
\begin{lstlisting}[language=rust,caption={mmap}]
    if !flags.contains(MapFlags::MAP_ANONYMOUS) {
        let fd_table = task.files.lock();
        match fd_table.get_ref(fd) {
        Ok(file_descriptor) => {
        if !file_descriptor.readable() {
        return EACCES;
        }
        let file = file_descriptor.file.deep_clone();
        file.lseek(offset as isize, SeekWhence::SEEK_SET).unwrap();
        new_area.map_file = Some(file);
        }
        Err(errno) => return errno,
        }
    }
\end{lstlisting}
第四步:mmap将new_area加入到先前的vector中,而且需要保持原vector的有序性,这里使用了rust
语言的特性。
\begin{lstlisting}[language=rust,caption={mmap}]
    let (idx, _) = self
    .areas
    .iter()
    .enumerate()
    .skip_while(|(_, area)| {
    area.inner.vpn_range.get_start() >= VirtAddr::from(MMAP_END).into()
    })
    .find(|(_, area)| area.inner.vpn_range.get_start() >= start_va.into())
    .unwrap();
    self.areas.insert(idx, new_area);
\end{lstlisting}

\subsection{execve}
应用程序自身的角度来看,进程 (Process) 的一个经典定义是一个正在运行的程序实例。当程序运行在操作系统中的时候,从程序的视角来看,它会产生一种“幻觉”:即该程序是整个计算机系统中当前运行的唯一的程序,能够独占使用处理器、内存和外设,而且程序中的代码和数据是系统内存中唯一的对象。体表现为“进程”这个抽象概念。站在计算机系统和操作系统的角度来看,并不存在这种“幻觉”。事实上,在一段时间之内,往往会有多个程序同时或交替在操作系统上运行,因此程序并不能独占整个计算机系统。具体而言,进程是应用程序的一次执行过程。并且在这个执行过程中,由“操作系统”执行环境来管理程序执行过程中的 **进程上下文** – 一种控制流上下文。这里的进程上下文是指程序在运行中的各种物理/虚拟资源(寄存器、可访问的内存区域、打开的文件、信号等)的内容,特别是与程序执行相关的具体内容:内存中的代码和数据,栈、堆、当前执行的指令位置(程序计数器的内容)、当前执行时刻的各个通用寄存器中的值等。
\begin{figure}[htb]
    \centering
    \includegraphics[width=\textwidth]{figures/04-04-进程的地址空间.png}
    \caption{
        进程的地址空间
    }
    \label{fig:进程的地址空间}
\end{figure}

exec 用一个新的程序来代替当前进程的内存和寄存器,但是其文件描述符、进程 id 和父进程都是不变的。
它根据文件系统中保存的某个文件来初始化用户部分。`exec`通过 `open()`打开二进制文件。然后,它读取 ELF 头。应用程序以通行的 ELF 格式来描述。一个 ELF 二进制文件包括了一个 ELF 头,然后是连续几个程序段的头。
exec 会检查文件是否包含 ELF 二进制代码。一个 ELF 二进制文件是以4个“魔法数字”开头的,即 0x7F,“E”,“L”,“F”。如果 ELF 头中包含正确的魔法数字,`exec` 就会认为该二进制文件的结构是正确的。

\begin{lstlisting}[language=rust,caption={execve}]
    pub fn sys_execve(
        pathname: *const u8,
        mut argv: *const *const u8,
        mut envp: *const *const u8,
    ) -> isize
\end{lstlisting}
execve()的前置条件
fork()复制某一个进程,为execve()的执行,准备条件。
execve()的核心工作:
参数转换,将argv,envp转为Vec<String>。
open()打开path路径下的可执行文件,对其进行检查。
load_elf()将其加载到当前的空间中。
load_elf的核心工作:
将elf文件映射到内核空间的`MMAP_BASE`地址处。
在map_elf()中,把elf文件的内容拷贝到用户的MemorySet中


\chapter{NPUcore-IMPACT 增量}

在上述基础上,我们继续做了许多努力,让 NPUcore-IMPACT 通过了初赛的所有测试用例,以及实验性地初步支持了 EXT4 文件系统。

后文我们会分别详细地介绍这两部分的内容。

\section{初赛期间的增量}

\subsection{初赛测试用例}

我们针对性地对初赛的测试用例进行了调试,将问题归类定位到了如下两点。

然后我们分别对每个问题进行了细致的分析,最终逐个击破,通过了初赛的所有测试用例。

\textbf{1. statx 系统调用}

LoongArch 赛道的初赛测试用例中,mmap 与 munmap 这两个测例涉及到了一个新的系统调用 statx。

\begin{lstlisting}[label={man:statx}, caption={statx 手册}]
NAME
       statx - get file status (extended)

LIBRARY
       Standard C library (libc, -lc)

SYNOPSIS
       #define _GNU_SOURCE          /* See feature_test_macros(7) */
       #include <fcntl.h>           /* Definition of AT_* constants */
       #include <sys/stat.h>

       int statx(int dirfd, const char *restrict pathname, int flags,
                 unsigned int mask, struct statx *restrict statxbuf);

STANDARDS
       Linux.

HISTORY
       Linux 4.11, glibc 2.28.
\end{lstlisting}

如手册 \ref{man:statx} 中所示,这个系统调用涉及了文件信息的获取。

我们为 NPUcore-IMPACT 实现了这个新的系统调用,并与文件系统进行了整合。

\begin{lstlisting}[language={Rust}, caption={statx 系统调用入口}]
let ret = match syscall_id {
    // ...
    SYSCALL_STATX => sys_statx(
        args[0],
        args[1] as *const u8,
        args[2] as u32,
        args[3] as u32,
        args[4] as *mut u8,
    ),
    // ...
};
\end{lstlisting}

\begin{lstlisting}[language={Rust}, caption={statx 系统调用实现}]
pub trait File: DowncastSync {
    // ...
    fn get_statx(&self) -> Statx;
    // ...
}
\end{lstlisting}

随后我们进行了测试,成功通过了 mmap 与 munmap 测试用例。

\textbf{2. 文件描述符分配}

通过对 openat 测试用例进行调试,我们最终发现问题出在操作系统对文件描述符的分配上。

Unix 标准要求操作系统分配文件描述符时,总是分配该进程还未使用的最小的文件描述符;而 NPUcore 回收进程关闭的文件描述符时,使用了一个线性表;操作系统重新分配之前回收的文件描述符时,没有使用表中最小的文件描述符,最终导致出现了问题。

\begin{lstlisting}[language={Rust}, caption={回收文件描述符}]
match self.inner[fd].take() {
    Some(file_descriptor) => {
        self.recycled.push(fd as u8);
        // TODO: maybe replace this with balanced binary tree?
        self.recycled.sort_by(|a, b| b.cmp(a));
        Ok(file_descriptor)
    }
    None => Err(EBADF),
}
\end{lstlisting}

我们选择了在回收文件描述符后进行一次排序来解决这个问题。

这个方案不一定是性能最佳的方案,我们还有以下方案可选:

\begin{enumerate}
    \item 回收时不进行排序,重新分配时使用 $O(n)$ 时间寻找最小的文件描述符;
    \item 使用二叉平衡树替换线性表,从而在 $O(\log n)$ 时间进行回收与重新分配,但也许会带来内存分配的额外开销。
\end{enumerate}

未来此处成为性能瓶颈时,根据性能测试结果选用最优方案会是更好的选择。

\subsection{EXT4 文件系统}

EXT4(fourth extended filesystem)是 Linux 内核的一个日志文件系统,是 EXT3 文件系统的继任者。EXT4 文件系统具有许多改进和新特性,使其在性能、可靠性和可扩展性方面优于前代文件系统。

与 NPUcore 先前使用的 FAT32 文件系统相比,EXT4 文件系统不仅允许了更大的文件大小与卷大小,更有着显著的性能和效率提升。EXT4 文件系统使用了延迟分配和多块分配策略,显著减少了碎片并提高了写入性能;同时它支持 Extents 和更高效的分配策略,提高了文件操作的速度和效率。此外,EXT4 文件系统还支持日志记录,通过记录元数据变化确保系统崩溃时的数据一致性和完整性,检查速度快且更可靠。

ext4文件系统原理如下:
\begin{enumerate}
    \item 日志功能:ext4采用了日志功能(journaling),即在对文件系统进行操作时,会先将操作记录在日志中,然后再执行操作。这样可以在系统异常关机或崩溃时恢复文件系统的一致性。
    \item 内存缓存:ext4使用了内存缓存,将磁盘上的数据加载到内存中进行读写操作,以提高文件的访问速度。
    \item 数据块分配:ext4使用了多级索引结构来分配存储空间。它将文件系统的空间分为固定大小的块,每个块可以存储一定大小的数据。通过索引结构,可以快速定位文件的数据块。
    \item 空闲块管理:ext4使用了位图和B树来管理空闲块。位图记录着每个块的使用情况,而B树则用于索引和定位空闲块。
    \item 快照功能:ext4支持快照功能,可以在不影响原有数据的情况下对文件系统进行备份和恢复。
    \item 后日志预分配:ext4采用了一种称为"delayed allocation"的技术,即将数据块的分配推迟到真正需要写入数据时再进行。这样可以提高写入性能,减少磁盘的碎片化。
    \item 逐项更新:ext4在写入数据时,不会一次性更新整个文件,而是根据需要只更新部分数据。这种逐项更新的方式可以减少磁盘的I/O次数,提高性能。
\end{enumerate}

我们为 NPUcore-IMPACT 实验性地加入了 EXT4 文件系统支持,使得其可以从 EXT4 文件系统启动,并读写其中的文件。



\subsection{EXT4 文件系统的实现过程}

在我们提交的最终版中,我们使用了 lwext4 作为 EXT4 文件系统驱动,同时我们也会在后面介绍NPUcore对于其它EXT4-like文件系统适配的可能性。

lwext4 是一个针对嵌入式系统设计的轻量级 EXT4 文件系统实现,它旨在提供 EXT4 文件系统的关键特性,同时保持低资源消耗和高性能,以适应嵌入式系统的限制。
为了让 lwext4 能与 NPUcore-IMPACT 一起工作,我们对 NPUcore-IMPACT 的文件系统设计做出了一定调整。

我们借助 Rust 的 trait 语言特性,设计了一个 File trait,用于表示一个抽象的文件,或者说一个可以对其进行读写的对象。

\begin{lstlisting}[language={Rust}, caption={File trait}]
pub trait File: DowncastSync {
    fn deep_clone(&self) -> Arc<dyn File>;
    fn readable(&self) -> bool;
    fn writable(&self) -> bool;
    fn read(&self, offset: Option<&mut usize>, buf: &mut [u8]) -> usize;
    fn write(&self, offset: Option<&mut usize>, buf: &[u8]) -> usize;
    fn r_ready(&self) -> bool;
    fn w_ready(&self) -> bool;
    fn read_user(&self, offset: Option<usize>, buf: UserBuffer) -> usize;
    fn write_user(&self, offset: Option<usize>, buf: UserBuffer) -> usize;
    fn get_size(&self) -> usize;
    fn get_stat(&self) -> Stat;
    fn get_statx(&self) -> Statx;
    fn get_file_type(&self) -> DiskInodeType;
    fn is_dir(&self) -> bool {
        self.get_file_type().is_dir()
        // self.get_file_type() == DiskInodeType::Directory
    }
    fn is_file(&self) -> bool {
        self.get_file_type().is_file()
        // self.get_file_type() == DiskInodeType::File
    }
    fn info_dirtree_node(&self, dirnode_ptr: Weak<DirectoryTreeNode>);
    fn get_dirtree_node(&self) -> Option<Arc<DirectoryTreeNode>>;
    /// open
    fn open(&self, flags: OpenFlags, special_use: bool) -> Arc<dyn File>;
    fn open_subfile(&self) -> Result<Vec<(String, Arc<dyn File>)>, isize>;
    /// create
    fn create(&self, name: &str, file_type: DiskInodeType) -> Result<Arc<dyn File>, isize>;
    fn link_child(&self, name: &str, child: &Self) -> Result<(), isize>
    where
        Self: Sized;
    /// delete(unlink)
    fn unlink(&self, delete: bool) -> Result<(), isize>;
    /// dirent
    fn get_dirent(&self, count: usize) -> Vec<Dirent>;
    /// offset
    fn get_offset(&self) -> usize {
        self.lseek(0, SeekWhence::SEEK_CUR).unwrap()
    }
    fn lseek(&self, offset: isize, whence: SeekWhence) -> Result<usize, isize>;
    /// size
    fn modify_size(&self, diff: isize) -> Result<(), isize>;
    fn truncate_size(&self, new_size: usize) -> Result<(), isize>;
    // time
    fn set_timestamp(&self, ctime: Option<usize>, atime: Option<usize>, mtime: Option<usize>);
    /// cache
    fn get_single_cache(&self, offset: usize) -> Result<Arc<Mutex<PageCache>>, ()>;
    fn get_all_caches(&self) -> Result<Vec<Arc<Mutex<PageCache>>>, ()>;
    /// memory related
    fn oom(&self) -> usize;
    /// poll, select related
    fn hang_up(&self) -> bool;
    /// iotcl
    fn ioctl(&self, _cmd: u32, _argp: usize) -> isize {
        ENOTTY
    }
    /// fcntl
    fn fcntl(&self, cmd: u32, arg: u32) -> isize;
}
\end{lstlisting}

在此基础上,我们为 lwext4 提供的 ext4_file 类型实现我们的 File trait,让 NPUcore-IMPACT 可以对其进行读写,从而实现 EXT4 文件系统的支持。

由于 lwext4 依赖 libc 进行内存分配,为了让它能工作在没有 libc 的环境下,我们还需要对其做出一定修改。

\begin{lstlisting}[language={C}, caption={管理 lwext4 内存}]
#if CONFIG_USE_USER_MALLOC

#define ext4_malloc  ext4_user_malloc
#define ext4_calloc  ext4_user_calloc
#define ext4_realloc ext4_user_realloc
#define ext4_free    ext4_user_free

#else

#define ext4_malloc  malloc
#define ext4_calloc  calloc
#define ext4_realloc realloc
#define ext4_free    free

#endif
\end{lstlisting}

我们希望让 NPUcore-IMPACT 为 lwext4 管理内存,为此我们实现 ext4_user_malloc、ext4_user_calloc、ext4_user_realloc、ext4_user_free 这四个内存管理函数,并将其与 lwext4 链接,从而让 lwext4 可以使用我们为它分配的内存,并在合适的时候回收这些内存。

\begin{lstlisting}[language={Rust}, caption={NPUcore-IMPACT 为 lwext4 分配内存}]
#[no_mangle]
pub extern "C" fn ext4_user_malloc(size: ::core::ffi::c_size_t) -> *mut ::core::ffi::c_void {
    HEAP_ALLOCATOR
        .lock()
        .alloc(Layout::array::<u8>(size).unwrap())
        .unwrap()
        .as_ptr() as *mut ::core::ffi::c_void
}
\end{lstlisting}

为了便于调试,我们需要在 lwext4 执行时打印日志,得益于 Rust 与 C 跨语言互操作十分方便,我们直接在 Rust 侧编写了打印日志的工具函数。

\begin{lstlisting}[language={Rust}, caption={在 lwext4 的 C 语言代码中打印日志}]
#[no_mangle]
pub extern "C" fn os_log(str: *const ::core::ffi::c_char) {
    let str = unsafe { CStr::from_ptr(str) };
    log::info!("{str:?}");
}

#[no_mangle]
pub extern "C" fn os_var_log(name: *const ::core::ffi::c_char, value: ::core::ffi::c_int) {
    let name = unsafe { CStr::from_ptr(name) };
    log::info!("{name:?}: {value}");
}
\end{lstlisting}

使用 \#[no_mangle] 可以让编译器不对函数名字进行混淆,使得我们可以在 C 语言侧直接调用 os_log 与 os_var_log 日志函数。


\subsubsection{LA 体系下 FAT32 与 EXT4 的区别}

\begin{enumerate}
    \item \textit{与指令集相关}
    % INPROCESS
    \item \textit{与 NPUcore 相关}
    % INPROCESS
\end{enumerate}

\subsubsection{敲定实现方式}

我们参考了历年不同赛道的优秀作品,最后给出了如下的适配方式:

\textit{我们采用第三方包将稳定 C 库作为外部库调入 NPUcore 中,如\autoref{ext4-complexe}所示:}

\begin{table}[htbp]
    \centering
    \begin{tabular}{|c|c|}
        \hline
        选用技术栈 & 作用 \\
        \hline
        lwext4 & 稳定的 ext4 文件系统外部库 \\
        bindgen & rust-lang 官方开发的FFI生成工具 \\
        \hline
    \end{tabular}
    \caption{选用技术栈}
\end{table}


\begin{figure}
    \centering
    \includegraphics[width=0.6\linewidth]{figs/plan-ext.png}
    \caption{ext4 实现结构图}
    \label{ext4-complexe}
\end{figure}

\begin{enumerate}
    \item \textit{根据 lwext4 或者类似的库理清楚他的函数调用,必要的话给出一个 .h 文件用于包装函数入口:} \\ \textit{The wrapper.h file will include all the various headers containing declarations of structs and functions we would like bindings for. In the particular case of bzip2, this is pretty easy since the entire public API is contained in a single header. For a project like SpiderMonkey, where the public API is split across multiple header files and grouped by functionality, we'd want to include all those headers we want to bind to in this single wrapper.h entry point for bindgen.}\footnote{参考 bindgen 手册https://rust-lang.github.io/rust-bindgen/tutorial-2.html},这意味着,\textbf{对于一个比较复杂而分散的项目,我们最好给出一个包装文件}.
    \item \textit{对于转换完成的rs库,视情况给出rust调用}
    \item \textit{转换我们的fs适配新的rs库} \\ 这部分很简单,我们相当于已经拿来一个ext文件系统了,剩下的就是直接使用调用就行了。在makefile里和rust代码里加入feature就可以做到针对不同文件系统的编译与运行
\end{enumerate}

\vspace{1em}

对于其中可能出现的问题,可见如下列表:

\begin{enumerate}
    \item \textbf{移植的时候会不会出现不适配龙芯情况:}99\%不会,目前查出来 Bindgen 使用 Clang 对 C 文件进行编译,之后反编译(\textit{仅使用 Clang ,不使用 LLVM 编译为机器码})回 Rust ,所以生成的代码最后编译时间还是走的 make 中的 loongarch-gcc .具体编译环节的参考如下:https://blog.csdn.net/xhhjin/article/details/81164076
    \item \textbf{lwext4 的水平如何,是否会存在包本身的问题:} C 语言库,方便阅读;稳定性比较强,多平台测试过,支持小端序,测试过的架构有 x86/AMD64 , ARM 系列以及其的各种嵌入式架构
\end{enumerate}

\subsubsection{第一次适配(LWEXT4-C + Bindgen)}

在第一次适配中,我们试图通过上述方式完成 EXT4 文件系统对于 LA 的适配,然而,我们遇到了许多问题
\begin{enumerate} 
    \item \textit{no_std 环境问题:}我们发现,离开了标准 C 环境的 lwext4 的适配情况并没有我们想象的顺利。在一步步 debug 的过程中,我们经历了如下问题:
    \begin{figure}[htbp] 
        \centering 
        \includegraphics[width=0.5\linewidth]{figs/ext4c.png} 
        \caption{Debug 流程图} 
        \label{debug-ext4c} 
    \end{figure} 
    %\item \textit{工作量问题:}我们试图向内核中添加 C_std 环境,但是由于较大的工作量失败了
\end{enumerate}

\subsubsection{第二次适配(LWEXT4-RUST)}

经过一定时间的查找资料,我们发现了下一个 lwext4 库,其 Supported Features 具体如下:\footnote{github网址:https://github.com/elliott10/lwext4_rust},然而这个包的适配过程仍然十分艰难:

\begin{itemize}
    \item lwext4_rust for x86_64, riscv64 and aarch64 on Rust OS is supported
    \item File system mount and unmount operations
    \item Filetypes: regular, directories, softlinks
    \item Journal recovery \& transactions
    \item memory as Block Cache
\end{itemize}

由于在其 Dependences 中发现了如下信息:

\begin{center}
    \textbf{C musl-based cross compile toolchains}
    \begin{itemize}
        \centering
        \item x86_64-linux-musl-gcc
        \item riscv64-linux-musl-gcc
        \item aarch64-linux-musl-gcc
    \end{itemize}
\end{center}

\textit{我们认为,其在我们拥有 LA 相关工具链的情况下是可以适配至我们的 LA 指令集操作系统上的}

经过一段时间的分析,我们认为 lwext4 系列的库\textbf{由于一定原因与 LA 指令集并不适配}

\subsubsection{第三次适配(EXT4-View)}

由于前两次适配都设计到lwext4相关,并且其存在于mkfs不相干的特性(但这并不是使用lwext4-mkfs没有成功的根本原因)。于是我们在github上自行检索并找到了这个EXT4-View\footnote{https://github.com/nicholasbishop/ext4-view-rs}仓库。我们试图将这个版本的EXT4与我们的NPUcore进行适配。

EXT4-View由一个谷歌研究员开发并持续维护中。该库提供了一个 Rust crate,允许对 ext4 文件系统进行\textbf{只读访问}。该 crate是no_std,因此可在嵌入式上下文中使用。不过,它需要 alloc。
这个仓库的基本属性可以总结为如下的部分:
\begin{enumerate}
    \item 所有有效的 ext4 文件系统都应该是可读的。
    \item 无效数据绝不会导致崩溃、panic或无限循环。
    \item 主软件包中没有不安全代码(允许在依赖包中出现)。
\end{enumerate}
使用方法为:
\begin{lstlisting}[language={Rust}, caption={ext4-view在kernel中的基本使用方法示例}]
use ext4_view::{Ext4, Metadata};

let fs_data: Vec<u8> = get_fs_data_from_somewhere();
let fs = Ext4::load(Box::new(data_source))?;

// If the std feature is enabled, you can load a filesystem by path:
let fs = Ext4::load_from_path(std::path::Path::new("some-fs.bin"))?;

// The Ext4 type has methods very similar to std::fs:
let path = "/some/file/path";
let file_data: Vec<u8> = fs.read(path)?;
let file_str: String = fs.read_to_string(path)?;
let exists: bool = fs.exists(path)?;
let metadata: Metadata = fs.metadata(path)?;
for entry in fs.read_dir("/some/dir")? {
    let entry = entry?;
    println!("{}", entry.path().display());
}
    \end{lstlisting}

而载入这个文件系统的方法只有两步,首先需要将img转换为.bin文件,并将.bin引入到kernel中,用一个指针指向它作为基本目录。实现代码如下:
\begin{lstlisting}[language={Rust}, caption={将测例加载进入kernel}]
    fn load_test_disk1() -> Ext4 {
        const DATA: &[u8] = include_bytes!("../../test_data/test_disk1.bin");
        Ext4::load(Box::new(DATA.to_vec())).unwrap()
    }

\end{lstlisting}

在适配中我们发现了两个明显的缺点:第一,由于加载kernel的地址为0x9000000090000000,计算后发现,只有64M的空间。所以我们的uImage大小不能超过64M,而本次全部测例有120M左右,因此没有办法将全部测例封装并测试。
第二,也是最致命的缺点。经过三天的适配后,我们发现内核中存在了很多奇怪的bug,包括但不限于找不到根目录,块设备加载出错等。刚开始我们认为是我们的kernel适配没有完全成功,而经过检查后发现是仓库本身存在问题,目前的版本不是完全完善的ext4版本。
而该仓库也仅有一个只读文件系统,没有办法完成针对本次比赛“完整的”EXT4适配,因此我们最终也放弃了这个仓库。

这个仓库值得后续的高度关注,因为它代码风格统一,接口完善,适配简单,应该能成为后续适配者的一个优质选择。

\subsubsection{第四次适配(Alien-rust)}

最后抱着试一试的态度,我们找到了往年的特等奖得主Alien\footnote{https://gitlab.eduxiji.net/202310007101563/Alien}仓库(该仓库仍然在持续开源并推进中。它一个用 rust 实现的简单操作系统。目的是探索如何使用模块来构建一个完整的操作系统,因此系统由一系列独立的模块组成。

他们的仓库中有提到对于lwext4的修复与推进工作:有了c实现的支持,我们只需要在rust中生成相关的头文件以及静态库。在做这部分之前,我们首先查看了一下crates.io中是否已有相关的实现,幸运的是,2年前已经有一个实现lwext4, 在简单阅读了其实现之后,我们打算参考其实现重新编写,因为其已经缺乏维护,并且不包含对no_std环境的支持。这个已有的实现给予我们很好的想法。
最终我们根据Alien的提示,适配了一个仍然针对LA2K1000开发板存在bug的NPUcore版本。该版本仅支持执行部分测例,并没有实现完整的“文件系统”应有的部分。但是我们仍将我们针对这个仓库的适配过程做一个小总结。

首先我们需要将文件系统从原先的FAT32转换为lwext4中的对应的Inode和FileSystem。这里的EasyFileSystem是一层针对文件系统的抽象接口。
\begin{lstlisting}[language={Rust}, caption={FILE_SYSTEM修改}]
pub type EasyFileSystem = lwext4_rs::FileSystem<crate::arch::BlockDeviceImpl>;
type DiskInodeType = lwext4_rs::FileType;
    lazy_static! {
        pub static ref FILE_SYSTEM: EasyFileSystem = EasyFileSystem::new(
            MountHandle::mount(
                RegisterHandle::register(BlockDevice::new(BlockDeviceImpl::new()), "shit".to_string())
                    .unwrap(),
                "/".to_string(),
                false,
                false,
            )
            .unwrap()
        )
        .unwrap();
    }
\end{lstlisting}

我们针对这层抽象继续修改对应的根目录:
\begin{lstlisting}[language={Rust}, caption={ROOT修改}]
    lazy_static! {
        pub static ref ROOT: Arc<DirectoryTreeNode> = {
            FILE_SYSTEM.readdir("/").unwrap();
    
            let inode = DirectoryTreeNode::new(
                "".to_string(),
                Arc::new(FileSystem::new(FS::Fat32)),
                Arc::new(OpenOptions::new().read(true).write(true).open("/").unwrap()),
                // OSInode::new(Arc::new()),
                Weak::new(),
            );
            inode.add_special_use();
            inode
        };
        static ref DIRECTORY_VEC: Mutex<(Vec<Weak<DirectoryTreeNode>>, usize)> =
            Mutex::new((Vec::new(), 0));
        static ref PATH_CACHE: Mutex<(String, Weak<DirectoryTreeNode>)> =
            Mutex::new(("".to_string(), Weak::new()));
    }
    \end{lstlisting}

针对这层块设备,我们也适配了对应的PCI和SATA块的读写部分,可以识别到测例。这里的lock和unlock方法为开发中,因为诸多测例都不需要这个方法,close则为默认关闭成功。
\begin{lstlisting}[language={Rust}, caption={ROOT修改}]
    impl lwext4_rs::BlockDeviceInterface for SataBlock{
        fn read_block(&mut self, buf: &mut [u8], mut block_id: u64, block_count: u32) -> lwext4_rs::Result<usize> {
            // kernel BLOCK_SZ=2048, SATA BLOCK_SIZE=512,four times
            block_id = block_id * (BLOCK_SZ as u64 / BLOCK_SIZE as u64);
            for buf in buf.chunks_mut(BLOCK_SIZE) {
                self.0
                    .lock()
                    .read_block(block_id, buf);
                block_id += 1;
            }
            Ok(0)
        }
    
        fn write_block(&mut self, buf: &[u8], mut block_id: u64, block_count: u32) -> lwext4_rs::Result<usize> {
            block_id = block_id * (BLOCK_SZ as u64 / BLOCK_SIZE as u64);
            for buf in buf.chunks(BLOCK_SIZE) {
                self.0
                    .lock()
                    .write_block(block_id, buf);
                block_id += 1;
            }
            Ok(0)
        }
        
        fn close(&mut self) -> lwext4_rs::Result<()> {
            Ok(())
        }
    
        fn open(&mut self) -> lwext4_rs::Result<lwext4_rs::BlockDeviceConfig> {
            Ok(lwext4_rs::BlockDeviceConfig{
                block_size: BLOCK_SIZE as u32,
                block_count: 999,
                part_size: BLOCK_SIZE as u64 * 2,
                part_offset: 0
            })
        }
    
        fn lock(&mut self) -> lwext4_rs::Result<()> {
            Ok(())
        }
    
        fn unlock(&mut self) -> lwext4_rs::Result<()> {
            Ok(())
        }
    }
    \end{lstlisting}
    
我们最终在决赛提交的也是这个版本,虽然它仍然有各种问题,但是我们将测例直接放入kernel中是可以跑出对应分数的。这个文件系统适配仍然非常不完善,甚至在文件系统初始化时都会报panic(我们跳过文件系统这一层,直接执行测例跑出的分数),因此我们后续仍然会持续推进并开发。
\section{用户地址空间}
\begin{figure}[htb]
    \centering
    \includegraphics[width=\textwidth]{figures/04-04-用户地址空间示意图.png}
    \caption{
        用户虚拟地址空间示意
    }
    \label{fig:user virtual process}
\end{figure}
每一个进程都有一个独立的页表,当npucore实现进程切换的时候,对应的页表也会切换。

当一个用户进程通过系统调用向操作系统请求更多的用户空间时,npucore首先在os/src/frame_allocator.rs中的StackFrameAllocator的基于栈的数据结构实现空闲物理页的分配,然后将物理页的映射加入到用户的页表当中。
npucore将设置PTEflags::R,PTEflags::W,PTEflags::U,PTEflags::X以及PTEflags::V标志位到对应的页表项目,使得用户可以对分配的页面进行读写操作。
大多数的用户进程并不能完全利用所有的虚拟空间,对于没有用到的空间,它对应的页表项PTEflags::V标志位始终为0。

页表的设计有很多的好处。首先,首先不同的进程使用不同的页表,相同的虚拟地址映射到不同的物理地址,因此每一个进程可以拥有自己独立的内存空间。其次,用户的虚拟地址是连续的,对应的物理地址不一定是连续的,这样可以有效的避免内存碎片。最后,内核将所有的用户的跳板代码都映射到了同一段虚地址,可以有效的实现上下文切换。

如图\ref{fig:user virtual process}所示,用户的虚拟地址空间被分为了三个部分,分别是用户代码段,用户数据段以及用户堆栈段。用户代码段用于存放用户的代码,用户数据段用于存放用户的数据,用户堆栈段用于存放用户的堆栈。
其中,用户栈的初始内容如图中所示,由execve函数完成初始化,其中包含了用户的命令行参数和返回地址,紧接着就是main函数使用的栈空间。
堆是动态内存分配的区域,用于存储在运行时分配的数据。在npucore中,堆的起始地址通常在数据段结束后,并根据需要进行扩展。通过系统调用(如mmap和sbrk),用户程序可以动态地管理堆内存。
栈是用于存储函数调用和局部变量的区域。在npucore中,栈从高地址向低地址生长,通常位于用户地址空间的顶部。栈的大小可以通过操作系统配置或在运行时动态调整。。

代码段是存储进程执行指令的区域。在npucore中,代码段通常从虚拟地址0开始,并包含可执行程序的指令。这个区域对应于用户程序的文本部分。
数据段包含了全局变量和静态变量等数据。npucore将数据段安排在代码段之后,从一个虚拟地址开始,并在运行时动态地分配和使用。

用户在代码段执行用户程序,将elf指定的数据段内容存放在相应的数据段位置。因为我们对与内核的映射是直接映射,所以操作系统对于用户的地址空间是直接可见的。
同时,堆栈段的大小是动态变化的,当用户程序需要更多的堆栈空间时,会通过系统调用向操作系统请求更多的堆栈空间,操作系统会在用户的地址空间中分配更多的堆栈空间,并将堆栈空间映射到用户的地址空间中

npucore实现了execve来将elf文件加载到内存的进程地址空间中,实现了sbrk来动态的分配用户空间,实现了mmap来将文件映射到用户空间,实现了munmap来取消文件的映射。
execve将elf文件中程序所规定的代码和数据加载到内存中,然后操作系统需要建议对应的映射,从而实现了用户地址空间的建立。


\subsection{sbrk系统调用}
sbrk系统调用是早期的Unix系统中的一个系统调用,用于动态的分配用户空间。sbrk系统调用的原型如下:
\begin{lstlisting}[language=c]
    void *sbrk(intptr_t increment);
\end{lstlisting}
sbrk系统调用将堆的大小增加increment字节,并返回堆的起始地址。
如果increment为负数,则堆的大小减少increment字节。如果堆的大小超过了进程的地址空间,则sbrk系统调用返回-1,并设置errno为ENOMEM。
sbrk系统调用可以为一个进程扩大或者缩小堆的大小,主要的实现是由os/src/memory_set.rs中的sbrk函数完成。
sbrk函数调用Memoryset::mmap或者Memoryset::munmap来实现堆的扩大或者缩小。
mmap函数不仅用于sbrk系统调用,还用于mmap系统调用,用于将文件映射到用户空间和开辟匿名内存映射。
\begin{lstlisting}[language=rust,caption={sbrk}]
    pub fn sbrk(&mut self, heap_pt: usize, heap_bottom: usize, increment: isize) -> usize {
        let old_pt: usize = heap_pt;
        let new_pt: usize = old_pt + increment as usize;
        // 判断扩大堆还是缩小堆
        if increment > 0 {
            let limit = heap_bottom + USER_HEAP_SIZE;
            if new_pt > limit {
                return old_pt;
            } else {
                self.mmap(
                    old_pt,
                    increment as usize,
                    MapPermission::R | MapPermission::W | MapPermission::U,
                    MapFlags::MAP_ANONYMOUS | MapFlags::MAP_FIXED | MapFlags::MAP_PRIVATE,
                    1usize.wrapping_neg(),
                    0,
                );
                trace!("[sbrk] heap area expanded to {:X}", new_pt);
            }
        } else if increment < 0 {
            // 如果缩小后的堆地址小于堆底地址,则不进行缩小
            if new_pt <= heap_bottom {
                return old_pt;
            } else {
                self.munmap(old_pt, increment as usize).unwrap();
            }
        }
        new_pt
    }
\end{lstlisting}
sbrk的实现依赖于函数mmap的实现,将在接下来的章节介绍,mmap是比sbrk更加灵活的系统调用,能狗处理更复杂的内存分配和使用的情况。


\subsection{mmap}
\begin{lstlisting}[language=rust,caption={sys_mmap}]
    pub fn sys_mmap(
        start: usize,
        len: usize,
        prot: usize,
        flags: usize,
        fd: usize,
        offset: usize,
    ) -> isize
\end{lstlisting}
参数start:指向欲映射的内存起始地址,通常设为 NULL,代表让系统自动选定地址,映射成功后返回
该地址。
\begin{itemize}
    \item 参数len:代表将文件中多大的部分映射到内存。
    \item 参数prot:映射区域的保护方式。
    \item 参数flags:影响映射区域的各种特性。在调用mmap()时必须要指定MAP_SHARED 或MAP_PRIVATE。MAP_FIXED 如果参数start所指的地址无法成功建立映射时,则放弃映射,不对地址做修正。通常不鼓励用此旗标。
    MAP_SHARED对映射区域的写入数据会复制回文件内,而且允许其他映射该文件的进程共享。MAP_PRIVATE 对映射区域的写入操作会产生一个映射文件的复制,即私人的“写入时复制”(copy on
    write)对此区域作的任何修改都不会写回原来的文件内容。MAP_ANONYMOUS建立匿名映射。此时会忽略参数fd,不涉及文件,而且映射区域无法和其他进程共
    享。MAP_DENYWRITE只允许对映射区域的写入操作,其他对文件直接写入的操作将会被拒绝。MAP_LOCKED 将映射区域锁定住,这表示该区域不会被置换(swap)。
    \item 参数fd:要映射到内存中的文件描述符。如果使用匿名内存映射时,即flags中设置了MAP_ANONYMOUS,fd设为-1。
    \item 参数offset:文件映射的偏移量,通常设置为0,代表从文件最前方开始对应,offset必须是分页大小的整数倍。
    \item 返回值:若映射成功则返回映射区的内存起始地址,否则返回-1。
\end{itemize}

\begin{figure}[htb]
    \centering
    \includegraphics[width=\textwidth]{figures/04-04-内存地址空间.png}
    \caption{
        内存地址空间
    }
    \label{fig:内存地址空间}
\end{figure}

mmap用于把文件映射到用户空间中,简单说mmap就是把一个文件的内容在内存里面做一个映像。映
射成功后,用户对这段内存区域的修改可以直接反映到内核空间,同样,内核空间对这段区域的修改也
直接反映用户空间。那么对于内核空间<---->用户空间两者之间需要大量数据传输等操作的话效率是非
常高的。进程可以像读写内存一样对普通文件的操作。mmap系统调用使得进程之间通过映射同一个普
通文件实现共享内存。
UNIX网络编程第二卷进程间通信对mmap函数进行了说明。该函数主要用途有三个:
1、将一个普通文件映射到内存中,通常在需要对文件进行频繁读写时使用,这样用内存读写取代I/O读
写,以获得较高的性能;
2、将特殊文件进行匿名内存映射,可以为关联进程提供共享内存空间;
3、为无关联的进程提供共享内存空间,一般也是将一个普通文件映射到内存中。
第一步:找到最后一次mmap映射区域
\begin{lstlisting}[language=rust]
    let idx = self.last_mmap_area_idx();
\end{lstlisting}
每个进程所申请的用户地址空间是用vector存储的vm_area_struct的结构体,其中start和end就是代表
这一个area虚拟地址的开始和结束,而我们的结构体在vector中是按照虚拟地址升序排序的,也就是说
我们会找到最后一个符合要求的vector下标


第二步:mmap获取映射区域的起始地址,如果设置了MAP_FIXED,则会直接使用用户的参数start,
不对地址做修正。
\begin{lstlisting}[language=rust,caption={mmap}]
    let start_va: VirtAddr = if flags.contains(MapFlags::MAP_FIXED) {
        // unmap if exists
        self.munmap(start, len);
        start.into()
    }
\end{lstlisting}
如果没有设置,mmap就会拿到先前得到的最后一次mmap映射区域。如果设置了MAP_PRIVATE 或者
MAP_ANONYMOUS,就直接将要新映射的区域new_area与最后一次mmap映射区域合并并返回,没
有设置就会将最后一次mmap映射区域的末尾作为新映射的区域的起始。而新映射的区域的末尾则根据
用户参数len设置。
\begin{lstlisting}[language=rust,caption={mmap}]
    else {
        if let Some(idx) = idx {
        let area = &mut self.areas[idx];
        if flags.contains(MapFlags::MAP_PRIVATE | MapFlags::MAP_ANONYMOUS)
        && prot == area.map_perm
        && area.map_file.is_none()
        {
        let end_va: VirtAddr = area.inner.vpn_range.get_end().into();
        area.expand_to(VirtAddr::from(end_va.0 + len)).unwrap();
        return end_va.0 as isize;
        }
        area.inner.vpn_range.get_end().into()
        } else {
        MMAP_BASE.into()
        }
        };
        let mut new_area = MapArea::new(
        start_va,
        VirtAddr::from(start_va.0 + len),
        MapType::Framed,
        prot,
        None,
    );
        
\end{lstlisting}
第三步:mmap会判断是否是文件映射,即是否包含MAP_ANONYMOUS。如果没有包含,就会先获取
fd_table,并根据用户参数fd从fd_table找到对应的文件描述符,这里文件相关内容不作赘述。然后设
置文件偏移量offset并置入new_area中
\begin{lstlisting}[language=rust,caption={mmap}]
    if !flags.contains(MapFlags::MAP_ANONYMOUS) {
        let fd_table = task.files.lock();
        match fd_table.get_ref(fd) {
        Ok(file_descriptor) => {
        if !file_descriptor.readable() {
        return EACCES;
        }
        let file = file_descriptor.file.deep_clone();
        file.lseek(offset as isize, SeekWhence::SEEK_SET).unwrap();
        new_area.map_file = Some(file);
        }
        Err(errno) => return errno,
        }
    }
\end{lstlisting}
第四步:mmap将new_area加入到先前的vector中,而且需要保持原vector的有序性,这里使用了rust
语言的特性。
\begin{lstlisting}[language=rust,caption={mmap}]
    let (idx, _) = self
    .areas
    .iter()
    .enumerate()
    .skip_while(|(_, area)| {
    area.inner.vpn_range.get_start() >= VirtAddr::from(MMAP_END).into()
    })
    .find(|(_, area)| area.inner.vpn_range.get_start() >= start_va.into())
    .unwrap();
    self.areas.insert(idx, new_area);
\end{lstlisting}

\subsection{execve}
应用程序自身的角度来看,进程 (Process) 的一个经典定义是一个正在运行的程序实例。当程序运行在操作系统中的时候,从程序的视角来看,它会产生一种“幻觉”:即该程序是整个计算机系统中当前运行的唯一的程序,能够独占使用处理器、内存和外设,而且程序中的代码和数据是系统内存中唯一的对象。体表现为“进程”这个抽象概念。站在计算机系统和操作系统的角度来看,并不存在这种“幻觉”。事实上,在一段时间之内,往往会有多个程序同时或交替在操作系统上运行,因此程序并不能独占整个计算机系统。具体而言,进程是应用程序的一次执行过程。并且在这个执行过程中,由“操作系统”执行环境来管理程序执行过程中的 **进程上下文** – 一种控制流上下文。这里的进程上下文是指程序在运行中的各种物理/虚拟资源(寄存器、可访问的内存区域、打开的文件、信号等)的内容,特别是与程序执行相关的具体内容:内存中的代码和数据,栈、堆、当前执行的指令位置(程序计数器的内容)、当前执行时刻的各个通用寄存器中的值等。
\begin{figure}[htb]
    \centering
    \includegraphics[width=\textwidth]{figures/04-04-进程的地址空间.png}
    \caption{
        进程的地址空间
    }
    \label{fig:进程的地址空间}
\end{figure}

exec 用一个新的程序来代替当前进程的内存和寄存器,但是其文件描述符、进程 id 和父进程都是不变的。
它根据文件系统中保存的某个文件来初始化用户部分。`exec`通过 `open()`打开二进制文件。然后,它读取 ELF 头。应用程序以通行的 ELF 格式来描述。一个 ELF 二进制文件包括了一个 ELF 头,然后是连续几个程序段的头。
exec 会检查文件是否包含 ELF 二进制代码。一个 ELF 二进制文件是以4个“魔法数字”开头的,即 0x7F,“E”,“L”,“F”。如果 ELF 头中包含正确的魔法数字,`exec` 就会认为该二进制文件的结构是正确的。

\begin{lstlisting}[language=rust,caption={execve}]
    pub fn sys_execve(
        pathname: *const u8,
        mut argv: *const *const u8,
        mut envp: *const *const u8,
    ) -> isize
\end{lstlisting}
execve()的前置条件
fork()复制某一个进程,为execve()的执行,准备条件。
execve()的核心工作:
参数转换,将argv,envp转为Vec<String>。
open()打开path路径下的可执行文件,对其进行检查。
load_elf()将其加载到当前的空间中。
load_elf的核心工作:
将elf文件映射到内核空间的`MMAP_BASE`地址处。
在map_elf()中,把elf文件的内容拷贝到用户的MemorySet中


\chapter{进程管理}
\section{进程生命周期和资源复用}
\subsection{进程生命周期}
进程指的是在系统中运行的一个程序的实例。而进程的生命周期包括从创建,就绪,阻塞,运行中,退出。
\begin{figure}[htb]
    \centering
    \includegraphics[width=\textwidth]{figures/05-01-进程生命周期示意图.png}
    \caption{
        进程生命周期示意图
    }
    \label{fig:user virtual process}
\end{figure}
在一个进程被创建之后,他会进入npucore中的就绪队列,在被操作系统调度之后将会进入到运行状态。
在运行状态下时间片耗尽或者是主动让出CPU的时候,进程会进入到就绪队列中,等待下一次被调度。
在运行的时候调用诸如wait等系统调用,进程会进入到阻塞队列中,等待被唤醒。
当进程执行结束,他会退出,释放系统资源。

npucore团队在进行性能调优之时,发现在操作系统运行示例程序时,IO操作导致CPU挂起的性能损失非常之大,因此我们将调度器进行了大改,使其完全支持了阻塞式的进程调度模式。
阻塞和非阻塞IO是访问设备的两种模式,驱动程序可以灵活的支持者两种用户空间对设备的访问方式。
阻塞操作是指在执行设备操作时,若不能获得资源,则挂起进程直到满足可操作的条件后再进行操
作。被挂起的进程进入睡眠状态,被调度器的运行队列移走,直到等待的条件被满足。
非阻塞是指在不能进行设备操作时,并不挂起,他要么放弃,要么不停地查询,直到可以进行操作为止。
在阻塞访问时,不能获取资源的进程将进入休眠,它将会让出CPU,因为阻塞的进程会进入休眠状态,所
以必须要有一个动作能唤醒该进程,唤醒进程的地方最大的可能发生在中断里面,因为在硬件资源获得的
同时往往伴随着一个中断。而非阻塞的进程则不断的尝试,直到可以进行IO。
与 linux 操作系统的设计类似, npucore采用等待队列的方式实现阻塞式调度器,将在后续章节介绍。
为了可以在npucore上同时运行多个进程,npucore实现了进程的创建,在内核中加载进程到内存,同时为其分配系统资源。
npucore为每一个进程分配系统资源,包括内存、文件描述符、CPU等,而实现系统资源的分配的方式是通过系统调用fork。
npucore中,fork系统调用用于创建一个新的进程,新的进程称为子进程,原来的进程称为父进程。
而所有的其他进程都是initproc的子进程,他们通过fork得到。initproc是需要在内核启动过程中创建的第一个进程。
对应的代码如下:
\begin{lstlisting}[language={Rust}]
    lazy_static! {
    pub static ref INITPROC: Arc<TaskControlBlock> = Arc::new({
        let elf = ROOT_FD.open("initproc", OpenFlags::O_RDONLY, true).unwrap();
        TaskControlBlock::new(elf)
    });
}
\end{lstlisting}
当创建一个新的进程时,用户进程通过fork得到一个原本进程的副本,为其分配系统资源,然后调用execve来讲elf文件加载到内存以创建一个新的进程。
每个进程都有自己的内存空间、代码和数据,它们是系统中资源的分配单位。他们的创建是由elf文件指定的。

为了保证所有的进程都能够被调度,从而避免进程饥饿的发生,npucore实现了进程的调度机制,从而实现阻塞和唤醒。
当一个进程主动放弃CPU或者被动的被剥夺CPU的使用权时,它会让出CPU,变成等待状态,这个过程称为阻塞。
在进程的视角看来,他会有一个自己独占CPU的“幻觉”,因为每一个阻塞和唤醒的时候进程的状态总是保持不变的。这样可以保证进程执行的正确性。
而npucore让一个被阻塞的进程重新开始执行的行为叫做唤醒。唤醒的同时会恢复进程的现场,包括阻塞时的寄存器状态。

而当一个进程执行结束,它就会退出,将它所占有的系统资源释放。进程的退出保证了npucore避免出现资源的永久占用的情况。
上述过程就是一个进程从“生”到“死”,保证了npucore可以正确且高效的执行对应的程序。

\subsection{资源复用}
为了实现多进程同时运行,操作系统需要对CPU,内存,外设等资源进行复用。
复用在资源有限的情况下是一个常用且实际的思想。围绕着复用的思想,我们可以提出以下几个问题:

如何实现上下文切换?

虽然保存现场思想是简单的,但是实际的实现却不是那么显然。在npucore中我们使用了一段所有进程共享的跳板代码和一个进程的私有的保存现场的帧来实现。

如何让进程如何实现透明调度,也就是用户进程不知道自己被调度了?

npucore实现了内置的计时器,当计时器中断发生时,内核会调用schedule函数,从而实现进程的调度。

进程的资源回收不能由进程自己来完成,否则进程退出时会出现资源泄露的情况,如何实现进程的资源回收?

npucore在exit之后,会释放一部分资源,但是不会释放所有的资源,从而进入僵尸状态,父进程来完成剩下的进程资源的回收。

如何在并发的情况下不会错过对进程的唤醒?

npucore中是一个单核的操作系统,在进入关键代码的时候会保证CPU不会调度其他进程,从而保证了进程的唤醒不会被错过。

此外,在内存方面,npucore采用了虚拟内存的方式,将物理内存映射到虚拟内存,从而实现了内存的复用。
对于用户的elf程序,程序的入口总是相同的,从某种程度上讲,程序将会“共享”同一个地址。
但是实际上,物理地址并不能被共享,所以虚拟内存就派上了用场。在上一章我们已经介绍了虚拟内存的实现,这里不再赘述。
I/O设备的复用将在I/O章节详细介绍。在这一章中,我们将介绍进程所拥有的各种内核的资源,诸如文件描述符等。

管理进程时,npucore使用了TCB的数据结构,不同于传统的PCB数据结构,npucore将线程视为共享栈的进程。
TCB的数据结构如下:
\begin{lstlisting}[language={Rust}]
pub struct TaskControlBlock {
    // immutable
    pub pid: PidHandle,
    pub tid: usize,
    pub tgid: usize,
    pub kstack: KernelStack,
    pub ustack_base: usize,
    pub exit_signal: Signals,
    // mutable
    inner: Mutex<TaskControlBlockInner>,
    // shareable and mutable
    pub exe: Arc<Mutex<FileDescriptor>>,
    pub tid_allocator: Arc<Mutex<RecycleAllocator>>,
    pub files: Arc<Mutex<FdTable>>,
    pub fs: Arc<Mutex<FsStatus>>,
    pub vm: Arc<Mutex<MemorySet>>,
    pub sighand: Arc<Mutex<Vec<Option<Box<SigAction>>>>>,
    pub futex: Arc<Mutex<Futex>>,
}
\end{lstlisting}
在上面的定义中,我们可以看到TCB中包含了进程的进程号,线程号,线程组号,内核栈,用户栈,退出信号,以及一些共享的资源。
后面将具体讲解如何管理这些资源以及进程之间的调度。
\section{进程状态控制}
\subsection{进程管理重要数据结构}
\subsection{进程创建}
execve 被用于替换当前进程的地址空间和上下文为新程序的,新程序将取代原程序执行。如果 execve 执行成功,原程序的代码和数据将被替换为新程序的代码和数据,并开始执行新程序。
filename:要执行的新程序的文件路径。
argv:参数数组,用于传递给新程序的命令行参数。
envp:环境变量数组,用于设置新程序的环境变量。
\begin{lstlisting}[language={Rust}, label={code:forktest},
    caption={forktest.rs}]
    pub fn sys_execve(
        pathname: *const u8,
        mut argv: *const *const u8,
        mut envp: *const *const u8,
    ) -> isize {}
\end{lstlisting}
在NPUcore中,若需要创建一个新的进程,总体上的过程和方法如下所示:

(1)从系统文件中找到内核加载第一个初始进程的elf文件,获取代码的数据和内容。然后调用TCB中的new()方法创建内核的第一个进程initproc。

(2)其余所有的进程均由初始进程initproc进行fork而来,初始进程initproc是所有进程的父进程。因此创建新进程的第二步便是调用fork系统调用

(3)在调用fork后,新的进程还需要加载独立的程序代码和数据文件,这时便需要用到exec系统调用。
##### sys_exec()

exec()的作用:

fork通常只创建现有进程的拷贝
exec可以在现有fork的基础上加载一个新应用的 ELF 可执行文件中的代码和数据替换原有的应用地址空间中的内容,并开始执行。

1.首先看函数的输入参数与返回值
\begin{lstlisting}[language={Rust}, label={code:forktest},
    caption={forktest.rs}]
    pub fn sys_execve(
        pathname: *const u8,
        mut argv: *const *const u8,
        mut envp: *const *const u8,
    ) -> isize {
        //load_elf 成功
        SUCCESS
        //失败
        Err
    }
\end{lstlisting}
2.将argv和envp变量字符串向量,通过循环,将*const *const u8转为Vec<String>数据类型。为后面的执行做准备。
\begin{lstlisting}[language={Rust}, label={code:forktest},
    caption={forktest.rs}]
    let mut argv_vec: Vec<String> = Vec::with_capacity(16);
    let mut envp_vec: Vec<String> = Vec::with_capacity(16);
    if !argv.is_null() {
        loop {
            let arg_ptr = match translated_ref(token, argv) {
                Ok(argv) => *argv,
                Err(errno) => return errno,
            };
            if arg_ptr.is_null() {
                break;
            }
            argv_vec.push(match translated_str(token, arg_ptr) {
                Ok(arg) => arg,
                Err(errno) => return errno,
            });
            unsafe {
                argv = argv.add(1);
            }
        }
    }
    if !envp.is_null() {
        loop {
            let env_ptr = match translated_ref(token, envp) {
                Ok(envp) => *envp,
                Err(errno) => return errno,
            };
            if env_ptr.is_null() {
                break;
            }
            envp_vec.push(match translated_str(token, env_ptr) {
                Ok(env) => env,
                Err(errno) => return errno,
            });
            unsafe {
                envp = envp.add(1);
            }
        }
    }


\end{lstlisting}
3.debug 层信息输出
\begin{lstlisting}[language={Rust}, label={code:forktest},
    caption={forktest.rs}]
    debug!(
        "[exec] argv: {:?} /* {} vars */, envp: {:?} /* {} vars */",
        argv_vec,
        argv_vec.len(),
        envp_vec,
        envp_vec.len()
    );
  \end{lstlisting}
4.准备将 ELF 文件的内容载入当前进程
要对读取的文件做以下检查
    1、文件大小大于等于 4
    2、同时,ELF文件以 4 位魔数"\x7fELF"开头,因此应检查文件首是否有此魔数。
\begin{lstlisting}[language={Rust}, label={code:forktest},
        caption={forktest.rs}]
        match working_inode.open(&path, OpenFlags::O_RDONLY, false) {
            Ok(file) => {
                if file.get_size() < 4 {
                    return ENOEXEC;
                }
                let mut magic_number = Box::<[u8; 4]>::new([0; 4]);
                // this operation may be expensive... I'm not sure
                file.read(Some(&mut 0usize), magic_number.as_mut_slice());
                let elf = match magic_number.as_slice() {
                    b"\x7fELF" => file,
                    b"#!" => {
                        let shell_file = working_inode
                            .open(DEFAULT_SHELL, OpenFlags::O_RDONLY, false)
                            .unwrap();
                        argv_vec.insert(0, DEFAULT_SHELL.to_string());
                        shell_file
                    }
                    _ => return ENOEXEC,
                };
    
                
\end{lstlisting}
5.真正的载入过程。loaf_elf将当前的elf文件内容覆盖当前的进程。
show_frame_consumption!宏输出对应的信息

\begin{lstlisting}[language={Rust}, label={code:forktest},
    caption={forktest.rs}]
    let task = current_task().unwrap();
    show_frame_consumption! {
        "load_elf";
        if let Err(errno) = task.load_elf(elf, &argv_vec, &envp_vec) {
            return errno;
        };
    }
    // should return 0 in success
    SUCCESS
}
Err(errno) => errno,
}
\end{lstlisting}

clone()
从linux 2.3.3开始,glibc的`fork()`封装作为NPTL(Native POSIX Threads Library)线程实现的一部分。直接调用`fork()`等效于调用[clone(2)](https://man7.org/linux/man-pages/man2/clone.2.html)时仅指定`flags`为`SIGCHLD`(共享信号句柄表)。

创建线程的函数`pthread_create`内部使用的也是clone函数。在glibc的`/sysdeps/unix/sysv/linux/createthread.c`源码中可以看到,创建线程的函数`create_thread`中使用了clone函数,并指定了相关的`flags`:
\begin{lstlisting}[language={Rust}, label={code:forktest},
    caption={forktest.rs}]
    const int clone_flags = (CLONE_VM | CLONE_FS 
    | CLONE_FILES | CLONE_SYSVSEM 
    | CLONE_SIGHAND | CLONE_THREAD 
    | CLONE_SETTLS | CLONE_PARENT_SETTID 
    | CLONE_CHILD_CLEARTID | 0);
  \end{lstlisting}
同样,npucore中的`clone`也使用了这种模式,以这两种方式来创建进程和线程

clone的使用
\begin{lstlisting}[language={Rust}, label={code:forktest},
    caption={forktest.rs}]
    pub fn sys_clone(
        flags: u32,
        stack: *const u8,
        ptid: *mut u32,
        tls: usize,
        ctid: *mut u32,
    ) -> isize
  \end{lstlisting}
描述
该系统调用用于创建一个新的子进程,类似fork(2)。与fork(2)相比,它可以更精确地控制调用进程和子
进程之间的执行上下文细节。例如,使用这些系统调用,调用者可以控制两个进程之间是否共享虚拟地
址空间,文件描述符表以及信号句柄表等。也可以通过这些系统调用将子进程放到不同的命名空间中。
参数
flags:
包括 CloneFlags 和 exit_signal 两部分 :
子进程结束信号 exit_signal
当子进程退出时,会像父进程发送一个信号。退出信号在 clone() 的 flags 的低字节中指定如果该信
号不是 SIGCHLD ,那么父进程在使用wait(2)等待子进程退出时必须指定 __WALL 或WCLONE选项。如
果没有指定任何信号(即,0),则在子进程退出后不会向父进程发送任何信号。
以下是npucore中如何从 flags 中获取子进程结束信号的代码块
\begin{lstlisting}[language={Rust}, label={code:forktest},
    caption={forktest.rs}]
    let exit_signal = match Signals::from_signum((flags & 0xff) as usize) {
        Ok(signal) => signal,
        Err(_) => {
            warn!(
                "[sys_clone] signum of exit_signal is unspecified or invalid: {}",
                (flags & 0xff) as usize
            );
            // This is permitted by standard, but we only support 64 signals
            Signals::empty()
        }
    };
  \end{lstlisting}
CloneFlags
CloneFlags 用来指定 clone 系统调用的行为,npucore的 CloneFlags 中有如下几种值
\begin{lstlisting}[language={Rust}, label={code:forktest},
    caption={forktest.rs}]
    bitflags! {
        pub struct CloneFlags: u32 {
            //const CLONE_NEWTIME         =   0x00000080;
            const CLONE_VM              =   0x00000100;
            const CLONE_FS              =   0x00000200;
            const CLONE_FILES           =   0x00000400;
            const CLONE_SIGHAND         =   0x00000800;
            const CLONE_PIDFD           =   0x00001000;
            const CLONE_PTRACE          =   0x00002000;
            const CLONE_VFORK           =   0x00004000;
            const CLONE_PARENT          =   0x00008000;
            const CLONE_THREAD          =   0x00010000;
            const CLONE_NEWNS           =   0x00020000;
            const CLONE_SYSVSEM         =   0x00040000;
            const CLONE_SETTLS          =   0x00080000;
            const CLONE_PARENT_SETTID   =   0x00100000;
            const CLONE_CHILD_CLEARTID  =   0x00200000;
            const CLONE_DETACHED        =   0x00400000;
            const CLONE_UNTRACED        =   0x00800000;
            const CLONE_CHILD_SETTID    =   0x01000000;
            const CLONE_NEWCGROUP       =   0x02000000;
            const CLONE_NEWUTS          =   0x04000000;
            const CLONE_NEWIPC          =   0x08000000;
            const CLONE_NEWUSER         =   0x10000000;
            const CLONE_NEWPID          =   0x20000000;
            const CLONE_NEWNET          =   0x40000000;
            const CLONE_IO              =   0x80000000;
        }
    }
  \end{lstlisting}
下文将介绍几个和传入参数有关的 cloneflags
stack:
stack 参数指定了子进程使用的栈的位置。由于子进程和调用进程可能会共享内存,因此不能在调用进
程的栈中运行子进程。调用进程必须为子进程的栈配置内存空间,并向 clone() 传入一个执行该空间的
指针。运行的所有处理器的栈都是向下生长的,因此 stack 通常指向为子进程栈设置的内存空间的最顶
端地址。注意, clone() 没有为调用者提供一种可以将堆栈区域的大小通知内核的方法。

ptid,ctid:
与 CloneFlags 中的 CLONE_CHILD_SETTID 和 CLONE_PARENT_SETTID 有关
CLONE_CHILD_SETTID
在ctid的位置上保存线程ID。保存操作会在 clone 调用返回控制到子进程的用户空间前完成。(注意,在
clone调用返回父进程前,保存操作可能是未完成的,它与是否引入 CLONE_VM 标志相关)
CLONE_PARENT_SETTID
在父进程的ptid中保存子线程ID。在Linux 2.5.32-2.5.48版本中,有一个标志 CLONE_SETTID 做了同样
的事情。保存操作会在clone调用将控制返回给用户空间前完成。
tid:
与 CloneFlags 中的 CLONE_SETTLS 有关
CLONE_SETTLS
将TLS(Thread Local Storage)保存到 tls 字段中。
npucore中clone的具体实现
\begin{lstlisting}[language={Rust}, label={code:forktest},
    caption={forktest.rs}]
    let child = parent.sys_clone(flags, stack, tls, exit_signal);
  \end{lstlisting}
npucore中clone的具体实现实际是实现在TCB结构体的方法 sys_clone 里,这里就是调用了
parent(也就是clone的调用者)的sys_clone,它返回新创建的进程的TCB结构体
首先, clone 会检查 CloneFlags 中是否有 CLONE_VM 和 CLONE_THREAD
如果有 CLONE_VM ,则调用进程和子进程会运行在系统的同一个内存空间中。调用进程或子进程对内存
的写操作都可以被对方看到。如果没有设置 CLONE_VM ,则子进程会运行在执行clone时的调用进程的
一份内存空间的拷贝中。
如果有 CLONE_THREAD ,子线程会放到与调用进程相同的线程组中。当一个 clone 调用没有指定
CLONE_THREAD 时,生成的线程会放到一个新的线程组中,其TGID等于该线程的TID
\begin{lstlisting}[language={Rust}, label={code:forktest},
    caption={forktest.rs}]
    let memory_set = if flags.contains(CloneFlags::CLONE_VM) {
        self.vm.clone()
    } else {
        crate::mm::frame_reserve(16);
        Arc::new(Mutex::new(MemorySet::from_existing_user(
            &mut self.vm.lock(),
        )))
    };
    
    let tid_allocator = if flags.contains(CloneFlags::CLONE_THREAD) {
        self.tid_allocator.clone()
    } else {
        Arc::new(Mutex::new(RecycleAllocator::new()))
    };
    
    // 这里改变了源代码的顺序,此处的tid实为下文声明的
    if flags.contains(CloneFlags::CLONE_THREAD) {
        memory_set.lock().alloc_user_res(tid, stack.is_null());
    }
  \end{lstlisting}
接着clone会像操作系统申请pid,tid,tgid和内核栈,因为npucore模仿了linux的“以轻量级进程代替
线程”,所以这里的pid与tid相同,为线程号,而tgid代表该进程的进程号
\begin{lstlisting}[language={Rust}, label={code:forktest},
    caption={forktest.rs}]
    // alloc a pid and a kernel stack in kernel space
    let pid_handle = pid_alloc();
    let tid = tid_allocator.lock().alloc();
    let tgid = if flags.contains(CloneFlags::CLONE_THREAD) {
        self.tgid
    } else {
        pid_handle.0
    };
    let kstack = kstack_alloc();
    let kstack_top = kstack.get_top();
    
  \end{lstlisting}
然后clone会进行新进程TCB的构造
\begin{lstlisting}[language={Rust}, label={code:forktest},
    caption={forktest.rs}]
    let task_control_block = Arc::new(TaskControlBlock {
        pid: pid_handle,
        tid,
        tgid,
        kstack,
        ustack_base: if !stack.is_null() {
            stack as usize
        } else {
            ustack_bottom_from_tid(tid)
        },
        exit_signal,
        exe: self.exe.clone(),
        tid_allocator,
        files: if flags.contains(CloneFlags::CLONE_FILES) {
            self.files.clone()
        } else {
            Arc::new(Mutex::new(self.files.lock().clone()))
        },
        fs: if flags.contains(CloneFlags::CLONE_FS) {
            self.fs.clone()
        } else {
            Arc::new(Mutex::new(self.fs.lock().clone()))
        },
        vm: memory_set,
        sighand: if flags.contains(CloneFlags::CLONE_SIGHAND) {
            self.sighand.clone()
        } else {
            Arc::new(Mutex::new(self.sighand.lock().clone()))
        },
        futex: if flags.contains(CloneFlags::CLONE_SYSVSEM) {
            self.futex.clone()
        } else {
            // maybe should do clone here?
            Arc::new(Mutex::new(Futex::new()))
        },
        inner: Mutex::new(TaskControlBlockInner {
            // inherited
            pgid: parent_inner.pgid,
            heap_bottom: parent_inner.heap_bottom,
            heap_pt: parent_inner.heap_pt,
            // clone
            sigpending: parent_inner.sigpending.clone(),
            // new
            children: Vec::new(),
            rusage: Rusage::new(),
            clock: ProcClock::new(),
            clear_child_tid: 0,
            robust_list: RobustList::default(),
            timer: [ITimerVal::new(); 3],
            sigmask: Signals::empty(),
            // compute
            trap_cx_ppn,
            task_cx: TaskContext::goto_trap_return(kstack_top),
            parent: if flags.contains(CloneFlags::CLONE_PARENT)
                | flags.contains(CloneFlags::CLONE_THREAD)
            {
                parent_inner.parent.clone()
            } else {
                Some(Arc::downgrade(self))
            },
            // constants
            task_status: TaskStatus::Ready,
            exit_code: 0,
        }),
    });
  \end{lstlisting}
这里对相关 CloneFlags 进行说明
CLONE_FILES 子进程与父进程共享相同的文件描述符(file descriptor)表
CLONE_FS 子进程与父进程共享相同的文件系统,包括root、当前目录、umask
CLONE_SIGHAND 子进程与父进程共享相同的信号处理(signal handler)表
CLONE_SYSVSEM 如果设置了该标志,则子进程和调用进程会共享一组System V semaphore
adjustment (semadj) 值(参见semop(2))。这种情况下,共享列表会在共享该列表的所有进程之间累加
semadj 值,并且仅当共享列表的最后一个进程终止(或使用unshare(2)停止共享列表)时才会执行
semaphore adjustments。如果没有设置该标志,则子进程会有一个独立的 semadj 列表,且初始为
空。
与信号量操作有关。
\begin{lstlisting}[language={Rust}, label={code:forktest},
    caption={forktest.rs}]
    files: if flags.contains(CloneFlags::CLONE_FILES) {
        self.files.clone()
    } else {
        Arc::new(Mutex::new(self.files.lock().clone()))
    },
    
    
    fs: if flags.contains(CloneFlags::CLONE_FS) {
        self.fs.clone()
    } else {
        Arc::new(Mutex::new(self.fs.lock().clone()))
    },
    
    
    sighand: if flags.contains(CloneFlags::CLONE_SIGHAND) {
            self.sighand.clone()
        } else {
            Arc::new(Mutex::new(self.sighand.lock().clone()))
        },
    
    
    futex: if flags.contains(CloneFlags::CLONE_SYSVSEM) {
        self.futex.clone()
    } else {
        // maybe should do clone here?
        Arc::new(Mutex::new(Futex::new()))
    },
  \end{lstlisting}
这里会修改父进程的子进程表,如果设置了 CLONE_PARENT 或 CLONE_THREAD 则代表新建进程与调用进
程是兄弟关系,反之则调用进程是新建进程的父进程
\begin{lstlisting}[language={Rust}, label={code:forktest},
    caption={forktest.rs}]
    // add child
    if flags.contains(CloneFlags::CLONE_PARENT) || flags.contains(CloneFlags::CLONE_THREAD) {
        if let Some(grandparent) = &parent_inner.parent {
            grandparent
                .upgrade()
                .unwrap()
                .acquire_inner_lock()
                .children
                .push(task_control_block.clone());
        }
    } else {
        parent_inner.children.push(task_control_block.clone());
    }
  \end{lstlisting}
最后, clone 会设置新进程的上下文以及tls并返回
\begin{lstlisting}[language={Rust}, label={code:forktest},
    caption={forktest.rs}]
    let trap_cx = task_control_block.acquire_inner_lock().get_trap_cx();
    if flags.contains(CloneFlags::CLONE_THREAD) {
        *trap_cx = *parent_inner.get_trap_cx();
    }
    // we also do not need to prepare parameters on stack, musl has done it for us
    if !stack.is_null() {
        trap_cx.gp.sp = stack as usize;
    }
    // set tp
    if flags.contains(CloneFlags::CLONE_SETTLS) {
        trap_cx.gp.tp = tls;
    }
    // for child process, fork returns 0
    trap_cx.gp.a0 = 0;
    // modify kernel_sp in trap_cx
    trap_cx.kernel_sp = kstack_top;
    // return
    task_control_block
\end{lstlisting}
\subsection{进程切换}
进程切换是操作系统中的一项核心功能,其作用是在多任务操作系统中实现进程间的切换,使得每个进程都可以独立地运行并且感觉到自己在独占CPU。 \\
实现进程的切换具有以下几个重要作用:
\begin{enumerate}
	\item 提高系统的并发性:进程切换可以使得多个进程在同一时刻并发运行,从而提高系统的效率和资源利用率。
	\item 实现多任务操作系统的核心功能:进程切换是多任务操作系统的核心功能之一,它可以使得每个进程都独立运行,并且感觉到自己在独占CPU。
\end{enumerate}
总之,进程切换是操作系统中一项重要的功能,它可以提高系统的并发性和效率,支持多任务和多用户系统,并且是操作系统实现任务调度、资源分配等核心功能的基础。\\

\subsubsection{进程切换的基本流程}
NPUcore进程的切换大致可以分为以下几个步骤:
\begin{enumerate}
	\item 用户态陷入内核态(系统调用或中断)
	\item 切换到调度器进程
	\item 切换到新进程的内核线程
	\item 从内核态返回
\end{enumerate}

\begin{figure}[ht]
	\centering
	\includegraphics[width=0.8\textwidth]{figures/05-02-03-进程切换.png}
	\caption{进程切换}
\end{figure}



\newpage

为了在进程之间进行切换,NPUcore需要在内核态执行两次上下文切换:从旧进程的内核线程切换到CPU的调度器线程,以及从调度器线程切换到新进程的内核线程。
进程切换的核心\textbf{\_\_switch}并不了解线程,它只是简单地保存和恢复寄存器集合,即上下文。上下文是CPU保存的当前进程执行的状态。
对于RISC-V,进程上下文包括:\textbf{ra}进程的返回地址、\textbf{sp}进程内核栈指针、\textbf{s0-s11}被调用者保存寄存器。\\

\begin{table}[h]
	\centering
	\begin{tabular}{|c|c|c|c|}
		\hline
		\textbf{Register} & \textbf{ABI Name} & \textbf{Description} & \textbf{Saver} \\
		\hline
		x0 & zero & Hard-wired zero & --- \\
		x1 & ra & Return address & Caller \\
		x2 & sp & Stack pointer & Callee \\
		x3 & gp & Global pointer & --- \\
		x4 & tp & Thread pointer & --- \\
		x5-7 & t0-2 & Temporaries & Caller \\
		x8 & s0/fp & Saved register/frame pointer & Callee \\
		x9 & s1 & Saved register & Callee \\
		x10-11 & a0-1 & Function arguments/return values & Caller \\
		x12-17 & a2-7 & Function arguments & Caller \\
		x18-27 & s2-11 & Saved registers & Callee \\
		x28-31 & t3-6 & Temporaries & Caller \\
		f0-7 & ft0-7 & FP temporaries & Caller \\
		f8-9 & fs0-1 & FP saved registers & Callee \\
		f10-11 & fa0-1 & FP arguments/return values & Caller \\
		f12-17 & fa2-7 & FP arguments & Caller \\
		f18-27 & fs2-11 & FP saved registers & Callee \\
		f28-31 & ft8-11 & FP temporaries & Caller \\
		\hline
	\end{tabular}
	\caption{RISC-V calling convention register usage.}
\end{table}

\newpage

\subsubsection{NPUcore实现进程切换的方法}
进程的切换依靠 os/src/task/mod.rs 中的 suspend\_current\_and\_run\_next 函数实现。
该函数的应用主要有如下两个场景:
\begin{itemize}
	\item{应用程序手动使用 sys\_yield 系统调用让出当前进程的占用权。}
	\begin{lstlisting}[language={Rust}, label={code:syscall yield},caption={syscall yield}]
		pub fn sys_yield() -> isize {
			suspend_current_and_run_next();
			SUCCESS
		}
	\end{lstlisting}
	\item{内核触发定时中断}
	\begin{lstlisting}[language={Rust}, label={code:Timer Interrupt},caption={Timer Interrupt}]
		Trap::Interrupt(Interrupt::SupervisorTimer) => {
			do_wake_expired();
			set_next_trigger();
			suspend_current_and_run_next();
		}
	\end{lstlisting}
\end{itemize}

NPUcore中,suspend\_current\_and\_run\_next 函数实现如下:
\begin{lstlisting}[language={Rust}, label={code:suspend current and run next}, caption={suspend current and run next}]
	pub fn suspend_current_and_run_next() {
		// There must be an application running.
		let task = take_current_task().unwrap();
		
		// ---- hold current PCB lock
		let mut task_inner = task.acquire_inner_lock();
		let task_cx_ptr = &mut task_inner.task_cx as *mut TaskContext;
		// Change status to Ready
		task_inner.task_status = TaskStatus::Ready;
		drop(task_inner);
		// ---- release current PCB lock
		
		// push back to ready queue.
		add_task(task);
		// jump to scheduling cycle
		schedule(task_cx_ptr);
	}
\end{lstlisting}

函数首先获取了一个当前正在执行进程的 PCB ,命名为 task 。然后从刚才得到的 PCB 里面的可变部分(类型为使用 MutexGuard 锁保护住的一个泛型结构体
MutexGuard<TaskControlBlockInner> ),获取当前进程的 task\_cx (类型是一个结构体 TaskContext ,保存着当前进程的上下文信息),然后把他转化成一个可变的裸指针。
再当前进程的状态从“正在运行”改为“就绪”,也就是停止当前进程。接着手动调用 drop 让 task\_inner 的引用计数值减一,并调用了 add\_task 方法将进程加入就绪队列。
最后调用 schedule 函数完成进程的切换。\\

TaskContext 的定义如下。ra 为调用后需要返回位置的pc值,它记录了 \_\_switch 函数返回之后应该跳转到哪里继续执行,从而在任务切换完成并 ret 之后能到正确的位置;
sp 为当前程序用户栈的栈指针;s0~s11 是需要保存的12个寄存器。
\begin{lstlisting}[language={Rust}, label={code:TaskContext}, caption={TaskContext}]
	pub struct TaskContext {
		ra: usize,
		sp: usize,
		s: [usize; 12],
	}
\end{lstlisting}

add\_task方法是将当前的线程加入到懒分配的全局变量TASK\_MANAGER的就绪队列中。
\begin{lstlisting}[language={Rust}, label={code:add task}, caption={add task}]
	lazy_static! {
		pub static ref TASK_MANAGER: Mutex<TaskManager> = Mutex::new(TaskManager::new());
	}
	
	pub fn add_task(task: Arc<TaskControlBlock>) {
		TASK_MANAGER.lock().add(task);
	}
	impl TaskManager {
		pub fn add(&mut self, task: Arc<TaskControlBlock>) {
			self.ready_queue.push_back(task);
		}
	}
\end{lstlisting}

schedule 函数负责将当前的进程切换到处理器调度进程(idle task)。
处理器调度进程由懒分配的全局变量PROCESSOR管理,切换过程通过汇编代码\_\_switch实现。
\begin{lstlisting}[language={Rust}, label={code:TaskContext}, caption={TaskContext}]
	pub struct Processor {
		current: Option<Arc<TaskControlBlock>>,
		idle_task_cx: TaskContext,
	}
	lazy_static! {
		pub static ref PROCESSOR: Mutex<Processor> = Mutex::new(Processor::new());
	}
	pub fn schedule(switched_task_cx_ptr: *mut TaskContext) {
		let idle_task_cx_ptr = PROCESSOR.lock().get_idle_task_cx_ptr();
		unsafe {
			__switch(switched_task_cx_ptr, idle_task_cx_ptr);
		}
	}
\end{lstlisting}

\_\_switch 函数接受两个参数,旧进程的内核栈指针和新进程的内核栈指针。
它将旧进程的上下文(sp,ra,s0-s11)保存到内存中,然后从内存从取出新进程的上下文到CPU的寄存器中。
\begin{lstlisting}[language={Rust}, label={code:switch}, caption={__switch}]
	.altmacro
	.macro SAVE_SN n
	sd s\n, (\n+2)*8(a0)
	.endm
	.macro LOAD_SN n
	ld s\n, (\n+2)*8(a1)
	.endm
	.section .text
	.globl __switch
	__switch:
	# __switch(
	#     current_task_cx_ptr: *mut TaskContext,
	#     next_task_cx_ptr: *const TaskContext
	# )
	# save kernel stack of current task
	sd sp, 8(a0)
	# save ra & s0~s11 of current execution
	sd ra, 0(a0)
	.set n, 0
	.rept 12
	SAVE_SN %n
	.set n, n + 1
	.endr
	# restore ra & s0~s11 of next execution
	ld ra, 0(a1)
	.set n, 0
	.rept 12
	LOAD_SN %n
	.set n, n + 1
	.endr
	# restore kernel stack of next task
	ld sp, 8(a1)
	ret
\end{lstlisting}
\subsection{进程调度}
导言:
计算机在运行的过程中,经常会出现这样一种情况,那就是内存中的可执行程序(进程)个数大于处理器的个数,为了使得这些程序都能完成自己的任务,处理器可以共享给这些程序使用。而操作系统在这其中起到的作用就是让这些进程高效合理的利用处理器资源,而这个过程我们称之为进程调度(scheduling)。进程调度是进程管理的重要组成部分。

我们的生活中处处存在调度,比如说食堂打饭,一个窗口的阿姨如何满足众多的食客,又比如一家工厂里的一台机床如何处理众多等待加工的零件。调度就是在一定的约束条件下,将有限的资源合理的分配给若干个任务,使得这些任务满足一些指标。回到我们要将的进程调度上,也就是在有限的cpu资源下,操作系统通过某种调度策略,使得各进程在时间上能够得到处理机完成自己的任务,并且满足一定的性能指标。

然而,NPUcore是如何完成这个看似简单的操作呢?接下来我们将阐述npucore的进程调度这个核心的问题。
\\[10pt]
5.2.4.1 调度策略

我们先大体介绍一下几种常用的调度策略,并详细说明npucore采用的调度策略。

(1)先来先服务:
先来先服务(first-come first-severed,FIFO,先进先出)调度策略的基本思想是按照进程请求处理器的先后顺序使用处理器。操作系统会创建两个队列,一个称之为就绪队列,一个称之为阻塞队列。

(2)最短作业优先(SJF):
在作业调度中,该算法每次从后备作业队列中挑选估计服务时间最短的一个或几个作业,将他们调入内存,分配必要的资源,创建进程并放入就绪队列。

(3)基于时间片的轮转:
如果操作系统给每个运行的进程的运行时间是一个足够小的时间片(time slice),时间片一到,就抢占当前进程并切换到另外一个进程进行执行。这样进程以时间片为单位轮流占用处理器执行。对于交互式进程而言,就有比较大的机会在较短时间内执行,从而有助于减少响应时间。这种调度策略称为时间片轮转(Round-Robin,RR)调度,基本思路即从就绪队列头取出一个进程,让他运行一个时间片,然后把它放回队列尾,再从队列头取下一个进程执行,周而复始。

(4)多级反馈队列调度:
操作系统根据进程过去一段的执行特征来预测进程未来一段时间里的执行情况,并以此假设为依据来动态的设置进程的优先级,调度选择优先级最高的进程执行。在该调度中,为进程添加了一个名为优先级的属性,并且优先级是可以根据过去的行为反馈来动态的调整。这其中可以细分为“固定优先级的多级无反馈队列”,“可降低优先级的多级反馈队列”,“可提升/降低优先级的多级反馈队列”

除了以上的几种调度策略外,根据计算机系统的不同,还有许多种的进程调度策略,读者可以自行在网上查询资料了解。接下来,我们将详细介绍npucore所使用的进程调度策略。
\\[10pt]
5.2.4.2 npucore的调度策略

(1)npucore所使用的调度策略为时间片轮转(RR),时间片轮转调度的基本思想是让每个线程在就绪队列中的等待时间与占用cpu的执行时间成正比例。其大致实现是:

\quad 1.将所有的就绪线程按照FCFS原则,排成一个就绪队列。

\quad 2.每次调度时将cpu分派(dispatch)给队首进程,让其执行一个时间片。

\quad 3.在时钟中断时,统计比较当前线程时间片是否已经用完。

\textbullet 若用完,则调度器暂停当前进程的执行,将其送到就绪队列的队尾,并通过切换执行就绪队列的队首进程。

\textbullet 若没有用完,则线程继续使用。
\begin{figure}
    \centering
    \includegraphics{figures/05-02-04时间片轮转.png}
\end{figure}
(2)npucore进程调度的创新:
npucore团队在进行性能调优的过程中,发现操作系统运行示例程序时,IO操作导致cpu挂起的性能损失很大,因此团队对调度器进行修改,使其支持阻塞式的进程调度模式。

阻塞式和非阻塞式IO是访问设备的两种模式,驱动程序可以灵活的支持两种IO模式。

\textbullet 阻塞操作指的是在执行设备操作时,如果得不到资源,那么进程就会挂起一直到满足可以操作的条件后再进行操作,被挂起的进程会进入睡眠状态,进入阻塞队列中,直到被唤醒。

\textbullet 非阻塞指的是不能进行设备操作时不进行挂起,要么一直等待,要么放弃处理机。

采用阻塞式进程调度模式的好处是显而易见的,不能获取资源的进程将会被休眠,让出CPU供给其他进程,直到得到资源被唤醒,唤醒进程的代码于中断之中,因为在硬件获得资源的同时往往伴随着一个中断。

最后还需要讲的是,npucore中进程的状态,其分为就绪,正在执行,阻塞(interruptible)状态:

\begin{figure}[H]
    \centering
    \includegraphics{figures/05-02-04进程状态.png}
\end{figure}


5.2.4.3 进程调度相关代码

前文中我们简要的介绍了何为进程调度以及常见的几种进程调度,还有npucore使用了什么调度策略,接下来我们将根据具体的代码来讲解进程调度。
\\[5pt]

1.进程调度的容器:任务管理器
任务管理器TaskManager是进程调度的容器,其包含一个就绪队列和一个可中断的睡眠状态队列。我们知道,进程由数据结构TCB来描述,而目前正在运行的TCB处于processor(处理器)之中管理,而一台计算机不只有一个进程在工作 ,那些没有在processor上运行的进程TCB则处于任务管理器中的两个队列之中,一部分是可以准备运行的就绪TCB,一部分是处于休眠状态的TCB,该休眠状态可以被中断程序唤醒,因此是可中断的队列。

根据有没有oom_handler特性对于TaskManger有两种数据结构,当有时候,会包含ActiveTracker这个数据结构,这个的作用主要是维护一个位图来跟踪进程的活动状态,可以检查和设置活动状态
\begin{lstlisting}[language=rust]
    pub struct ActiveTracker {
            bitmap: Vec<u64>,
        }
        
        #[cfg(feature = "oom_handler")]
        #[allow(unused)]
        impl ActiveTracker {
            pub const DEFAULT_SIZE: usize = SYSTEM_TASK_LIMIT;
            pub fn new() -> Self {
                let len = (Self::DEFAULT_SIZE + 63) / 64;
                let mut bitmap = Vec::with_capacity(len);
                bitmap.resize(len, 0);
                Self { bitmap }
            }
            pub fn check_active(&self, pid: usize) -> bool {
                (self.bitmap[pid / 64] & (1 << (pid % 64))) != 0
            }
            pub fn check_inactive(&self, pid: usize) -> bool {
                (self.bitmap[pid / 64] & (1 << (pid % 64))) == 0
            }
            pub fn mark_active(&mut self, pid: usize) {
                self.bitmap[pid / 64] |= 1 << (pid % 64)
            }
            pub fn mark_inactive(&mut self, pid: usize) {
                self.bitmap[pid / 64] &= !(1 << (pid % 64))
            }
        } 
\end{lstlisting}

当没有oom_handler特性时TaskManger的数据结构如下:
\begin{lstlisting}[language=rust]
    pub struct TaskManager {
    pub ready_queue: VecDeque<Arc<TaskControlBlock>>,
    pub interruptible_queue: VecDeque<Arc<TaskControlBlock>>,
}
\end{lstlisting}
包含了就绪队列和一个可中断的等待队列构成,都是存储进程控制块的队列,其中进程控制块使用原子引用计数智能指针进行引用计数。这有助于有效地管理进程的生命周期和共享。
其中就绪队列ready_queue中存储已准备好运行的进程。

进程在这个队列中等待被调度器选中并执行。当一个进程处于等待 CPU 资源的状态,但是已经准备好执行时,它会被添加到这个队列。

可中断等待队列interruptible_queue专门用于存放处于可中断的睡眠状态的进程,这些进程可以被中断唤醒以继续执行。

关于TaskManger的具体实现如下:
\begin{lstlisting}[language=rust]
    impl TaskManager {
    #[cfg(feature = "oom_handler")]
    pub fn new() -> Self {
        Self {
            ready_queue: VecDeque::new(),
            interruptible_queue: VecDeque::new(),
            active_tracker: ActiveTracker::new(),
        }
    }
    #[cfg(not(feature = "oom_handler"))]
    pub fn new() -> Self {
        Self {
            ready_queue: VecDeque::new(),
            interruptible_queue: VecDeque::new(),
        }
    }
    pub fn add(&mut self, task: Arc<TaskControlBlock>) {
        self.ready_queue.push_back(task);
    }
    #[cfg(feature = "oom_handler")]
    pub fn fetch(&mut self) -> Option<Arc<TaskControlBlock>> {
        match self.ready_queue.pop_front() {
            Some(task) => {
                self.active_tracker.mark_active(task.pid.0);
                Some(task)
            }
            None => None,
        }
    }
    #[cfg(not(feature = "oom_handler"))]
    pub fn fetch(&mut self) -> Option<Arc<TaskControlBlock>> {
        self.ready_queue.pop_front()
    }
    pub fn add_interruptible(&mut self, task: Arc<TaskControlBlock>) {
        self.interruptible_queue.push_back(task);
    }
    pub fn drop_interruptible(&mut self, task: &Arc<TaskControlBlock>) {
        self.interruptible_queue
            .retain(|task_in_queue| Arc::as_ptr(task_in_queue) != Arc::as_ptr(task));
    }
    pub fn find_by_pid(&self, pid: usize) -> Option<Arc<TaskControlBlock>> {
        self.ready_queue
            .iter()
            .chain(self.interruptible_queue.iter())
            .find(|task| task.pid.0 == pid)
            .cloned()
    }
    pub fn find_by_tgid(&self, tgid: usize) -> Option<Arc<TaskControlBlock>> {
        self.ready_queue
            .iter()
            .chain(self.interruptible_queue.iter())
            .find(|task| task.tgid == tgid)
            .cloned()
    }
    pub fn ready_count(&self) -> u16 {
        self.ready_queue.len() as u16
    }
    pub fn interruptible_count(&self) -> u16 {
        self.interruptible_queue.len() as u16
    }
    pub fn wake_interruptible(&mut self, task: Arc<TaskControlBlock>) {
        match self.try_wake_interruptible(task) {
            Ok(_) => {}
            Err(_) => {
                log::trace!("[wake_interruptible] already waken");
            }
        }
    }
    pub fn try_wake_interruptible(
        &mut self,
        task: Arc<TaskControlBlock>,
    ) -> Result<(), WaitQueueError> {
        self.drop_interruptible(&task);
        if self.find_by_pid(task.pid.0).is_none() {
            self.add(task);
            Ok(())
        } else {
            Err(WaitQueueError::AlreadyWaken)
        }
    }
    #[allow(unused)]
    // debug use only
    pub fn show_ready(\&self) {
        self.ready_queue.iter().for_each(|task| {
            log::error!("[show_ready] pid: {}", task.pid.0);
        })
    }
    #[allow(unused)]
    // debug use only
    pub fn show_interruptible(\&self) {
        self.interruptible_queue.iter().for_each(|task| {
            log::error!("[show_interruptible] pid: {}", task.pid.0);
        })
    }
}
\end{lstlisting}
\begin{table}[H]
    \centering
    \begin{tabularx}{17cm}{|X|X|X|X|X|}
        \hline
        方法名 & 目的 & 参数 & 操作 & 返回值 \\
        \hline
        new & 初始化一个TaskManager实例。& 无 & 无 & 返回一个包含两个空队列的TaskManager实例。 \\
        \hline
        add & 将进程添加到就绪队列。 & task: Child(代表一个进程)。 & 无 & 无 \\
        \hline
        fetch & 从就绪队列中获取一个进程。 & 无 & ready_queue头部弹出一个进程,并返回它,如果有oom_handler特性,则将弹出进程标记为活跃状态。 & 返回Option<Child>表示可能获取到的进程。 \\
        \hline
        add_interruptible & 将可中断的进程添加到可中断队列。 & task: Child(代表一个进程)。 & 将 task 添加到 interruptible_queue 尾部。 & 无 \\
        \hline
        drop_interruptible & 从可中断队列中移除指定的可中断进程。 & task:\&Child(代表一个进程)。 & 保留不等于 task 的进程,从而移除了匹配的进程。 & 无 \\
        \hline
        find_by_pid & 根据进程 ID 在队列中查找进程 & pid: usize(进程 ID)。 & 无 & 返回 Option<Child>,表示找到的进程。 \\
        \hline
        find_by_tgid & 根据线程组 ID 在队列中查找进程。 & tgid: usize(线程组 ID)。 & 无 & 返回 Option<Child>,表示找到的进程。 \\
        \hline
        ready_count & 获取就绪队列中进程的数量 & 无 & 无 & 返回 u16 类型的进程数量 \\
        \hline
        interruptible_count	& 获取可中断队列中进程的数量 & 无 & 无 & 返回 u16 类型的进程数量 \\
        \hline
        wake_interruptible & 将已唤醒的可中断进程移动到就绪队列中 & task: Child(代表一个已唤醒的进程) & 调用 try_wake_interruptible 方法,忽略可能的错误 & 无 \\
        \hline
        try_wake_interruptible & 尝试将已唤醒的可中断进程移动到就绪队列中 & task: Child(代表一个已唤醒的进程) & 如果进程不存在于就绪队列中,将其添加到就绪队列中,否则返回错误 & 返回 Result<(), WaitQueueError>,表示操作成功或已经唤醒 \\
        \hline
        show_ready & 仅用于调试,打印就绪队列的进程 PID。这些方法通过入队出队,查找等方法实现了进程的调度。 & 无 & 无 & 无 \\
        \hline
        show_interruptible & 仅用于调试,打印可中断队列的进程 PID。这些方法通过入队出队,查找等方法实现了进程的调度 & 无 & 无 & 无 \\
        \hline
    \end{tabularx}
\end{table}
这是最主要的进程调度的数据结构TaskManger的主要内容,Npucore除了TaskManger提供的两个队列之外还建立了Waitqueue,TimeoutWaitQueue两个队列分别为等待队列和超时等待队列,这两个队列的结构体为:
\begin{lstlisting}[language=rust]
    pub struct WaitQueue {
    inner: VecDeque<Weak<TaskControlBlock>>,
}
pub struct TimeoutWaitQueue {
    inner: BinaryHeap<TimeoutWaiter>,
}
\end{lstlisting}
都接收了TCB的弱引用作为队列元素,在Npucore 中,TaskManger包含的方法实现了最主要的进程调度,WaitQueue和TimeoutWaitQueue则是在进程/线程同步方面Futex的视线中发挥作用

以下是关于WaitQueue的实现:
\begin{lstlisting}[language=rust]
    impl WaitQueue {
    pub fn new() -> Self {
        Self {
            inner: VecDeque::new(),
        }
    }
    pub fn add_task(&mut self, task: Weak<TaskControlBlock>) {
        self.inner.push_back(task);
    }
    pub fn pop_task(&mut self) -> Option<Weak<TaskControlBlock>> {
        self.inner.pop_front()
    }
    pub fn contains(&self, task: &Weak<TaskControlBlock>) -> bool {
        self.inner
            .iter()
            .any(|task_in_queue| Weak::as_ptr(task_in_queue) == Weak::as_ptr(task))
    }
    pub fn is_empty(&self) -> bool {
        self.inner.is_empty()
    }
    pub fn wake_all(&mut self) -> usize {
        self.wake_at_most(usize::MAX)
    }
    pub fn wake_at_most(&mut self, limit: usize) -> usize {
        if limit == 0 {
            return 0;
        }
        let mut manager = TASK_MANAGER.lock();
        let mut cnt = 0;
        while let Some(task) = self.inner.pop_front() {
            match task.upgrade() {
                Some(task) => {
                    let mut inner = task.acquire_inner_lock();
                    match inner.task_status {
                        super::TaskStatus::Interruptible => {
                            inner.task_status = super::task::TaskStatus::Ready
                        }
                        _ => continue,
                    }
                    drop(inner);
                    if manager.try_wake_interruptible(task).is_ok() {
                        cnt += 1;
                    }
                    if cnt == limit {
                        break;
                    }
                }
            
                None => continue,
            }
        }
        cnt
    }
}
\end{lstlisting}
\begin{table}[H]
    \centering
    \begin{tabularx}{17cm}{|X|X|X|X|X|}
        \hline
        方法名 & 目的 & 参数 & 操作 & 返回值 \\
        \hline
        new & 创建一个新的 WaitQueue 实例 & 无 & 初始化内部使用 VecDeque 存储的任务队列 & 新创建的 WaitQueue 实例 \\
        \hline
        add_task & 将一个任务添加到 WaitQueue 中 & 自身引用,task - 用 Weak 包装的任务 & 将任务添加到队列的末尾 & 无 \\
        \hline
        pop_task & 从 WaitQueue 中弹出一个任务 & 无 & 从队列的前端弹出一个任务 & 返回一个 Option,可能是弹出的任务,如果队列为空则为 None \\
        \hline
        contains & 判断 WaitQueue 中是否包含与给定任务相等的元素 & 自身引用,task-要判断的任务 & 比较任务的指针是否相等 & 如果队列中包含相等的任务则返回 true,否则返回 false \\
        \hline
        is_empty & 判断 WaitQueue 是否为空 & 自身引用 & 检查内部队列是否为空 & 如果队列为空则返回 true,否则返回 false \\
        \hline
        wake_all & 唤醒 WaitQueue 中的所有任务 & 自身可变引用 & 将所有任务的状态更改为 Ready & 返回实际唤醒的任务数量 \\
        \hline
        wake_at_most & 唤醒 WaitQueue 中不超过 limit 数量的任务 & 自身可变引用,limit - 最大唤醒数量 & 将不超过 limit 数量的任务状态更改为 Ready & 返回实际唤醒的任务数量 \\
        \hline
    \end{tabularx}
\end{table}

以下是关于TimeoutWaitQueue的实现:

\begin{lstlisting}[language=rust]
    impl TimeoutWaitQueue {
    pub fn new() -> Self {
        Self {
            inner: BinaryHeap::new(),
        }
    }
    pub fn add_task(&mut self, task: Weak<TaskControlBlock>, timeout: TimeSpec) {
        self.inner.push(TimeoutWaiter { task, timeout });
    }
    pub fn wake_expired(&mut self, now: TimeSpec) {
        let mut manager = TASK_MANAGER.lock();
        while let Some(waiter) = self.inner.pop() {
            // the remaining tasks in heap haven't reach their timeout
            if waiter.timeout > now {
                log::trace!(
                    "[wake_expired] no more expired, next pending task timeout: {:?}, now: {:?}",
                    waiter.timeout,
                    now
                );
                self.inner.push(waiter);
                break;
            // wake one task
            } else {
                match waiter.task.upgrade() {
                    Some(task) => {
                        let mut inner = task.acquire_inner_lock();
                        match inner.task_status {
                            super::TaskStatus::Interruptible => {
                                inner.task_status = super::task::TaskStatus::Ready
                            }
                            _ => continue,
                        }
                        drop(inner);
                        log::trace!(
                            "[wake_expired] pid: {}, timeout: {:?}",
                            task.pid.0,
                            waiter.timeout
                        );
                        manager.wake_interruptible(task);
                    }
                    None => continue,
                }
            }
        }
    }
    #[allow(unused)]
    // debug use only
    pub fn show_waiter(&self) {
        for waiter in self.inner.iter() {
            log::error!("[show_waiter] timeout: {:?}", waiter.timeout);
        }
    }
}
\end{lstlisting}
\begin{table}[H]
    \begin{tabularx}{17cm}{|X|X|X|X|X|}
        \hline
        方法名 & 目的 & 参数 & 操作 & 返回值 \\
        \hline
        new & 创建一个新的 TimeoutWaitQueue 实例 & 无 & 初始化内部使用 BinaryHeap 存储的超时等待队列 & 新创建的 TimeoutWaitQueue 实例 \\
        \hline
        add_task & 将一个带有超时时间的任务添加到 TimeoutWaitQueue 中 & 自身可变引用,task - 用 Weak 包装的任务。timeout - 任务的超时时间。 &  将任务包装成 TimeoutWaiter 结构,并将其按照超时时间顺序插入二进制堆中 & 无 \\
        \hline
        wake_expired & 唤醒 TimeoutWaitQueue 中已经超时的任务 & 自身可变引用,now-当前时间 & 弹出二进制堆顶部的任务,检查其超时时间是否已经到达。如果任务还未超时,将其重新插入堆中并结束。如果任务已经超时,将任务的状态更改为 Ready,并唤醒对应的任务。 & 无 \\
        \hline
        show_waiter & 用于调试,打印当前 TimeoutWaitQueue 中的等待者(任务)的超时时间 & 自身引用 & 打印每个等待者的超时时间 & 无 \\
        \hline
    \end{tabularx}
\end{table}

2.阻塞式进程调度的实现:

多路复用:

多路复用的实现如下:当一个进程等待磁盘请求时,OS 使之进入睡眠状态,然后调度执行另一个进程。另外,当一个进程耗尽了它在处理器上运行的时间片后,OS 使用时钟中断强制它停止运行,这样调度器才能调度运行其他进程。这样的多路复用机制为进程提供了独占处理器的假象,类似于OS 使用内存分配器和页表硬件为进程提供了独占内存的假象。

实现多路复用有几个难点。首先,应该如何从运行中的一个进程切换到另一个进程?OS 采用了普通的上下文切换机制;虽然这里的思想是非常简洁明了的,但是其代码实现是操作系统中最晦涩难懂的一部分。第二,如何让上下文切换透明化?OS 只是简单地使用时钟中断处理程序来驱动上下文切换。第三,可能出现多个 CPU 同时切换进程的情况,那么我们必须使用一个带锁的方案来避免竞争。第四,进程退出时必须释放其占用内存与资源,但由于它本身在使用自己的资源(譬如其内核栈),所以不能由该进程本身释放其占有的所有资源。

OS 必须为进程提供互相协作的方法。譬如,父进程需要等待子进程结束,以及读取管道数据的进程需要等待其他进程向管道中写入数据。与其让这些等待中的进程消耗 CPU 资源,不如让它们暂时放弃 CPU,进入睡眠状态来等待其他进程发出事件来唤醒它们。但我们必须要小心设计以防睡眠进程遗漏事件通知。

上下文切换:

如下图所示,OS 在低层次中实现了两种上下文切换:从进程的内核线程切换到当前 CPU 的调度器线程,从调度器线程到进程的内核线程。OS 永远不会直接从用户态进程切换到另一个用户态进程;这种切换是通过用户态-内核态切换(系统调用或中断)、切换到调度器、切换到新进程的内核线程、最后这个陷入返回实现的。我们将以一个简单的例子,详细介绍了这一过程。

\begin{figure}[H]
    \centering
    \includegraphics{figures/05-02-04上下文切换.png}
\end{figure}

设有两个用户态进程 A 和 B,它们在操作系统内核的管理下运行。CPU 当前正在执行进程 A。

首先是用户态到内核态的切换。当进程 A 需要执行一个系统调用(例如,读写文件、申请内存等)或者发生一个中断事件(例如,时钟中断、硬件中断等),CPU 会触发一个异常或中断,将控制权转移到操作系统内核。

触发异常或中断: 比如,进程 A执行了一个系统调用指令或硬件设备触发了中断。

保存用户态上下文: 操作系统内核保存进程 A 的用户态上下文,包括通用寄存器、程序计数器、堆栈指针等。

切换到内核态: CPU 转到内核态执行,此时操作系统内核可以访问更多的特权指令和数据结构。

其次是在内核态的处理过程。在内核态,操作系统会执行一些必要的操作,比如:

\quad  \textbullet 系统调用处理: 根据系统调用的类型,执行相应的内核代码完成用户请求的操作。

\quad  \textbullet 调度器的工作: 决定下一个要执行的进程。这可能涉及到选择一个就绪队列中的新进程。

内核线程切换: 如果需要切换到一个新的用户态进程(假设是进程 B),操作系统可能会将控制权切换到该进程的内核线程。

之后为内核态到用户态的切换:

\quad \textbullet 恢复用户态上下文: 如果切换到了新的用户态进程 B,操作系统会从进程 B 的内核线程中恢复用户态上下文。

\quad \textbullet 切换到用户态: CPU 会从内核态切换回用户态,开始执行进程 B 的用户态代码。

最后,进程 B 开始执行:

\quad \textbullet 加载用户态上下文: 进程 B 的用户态上下文被加载到 CPU 寄存器中。
	
\quad \textbullet 继续执行: CPU 开始执行进程 B 的用户态代码,从上次中断或系统调用的位置继续执行。

整个过程中,关键的步骤包括从用户态到内核态的切换,内核态的处理过程,以及从内核态回到用户态的切换。这个过程确保了不同进程之间的无缝切换,使得操作系统能够有效地进行多任务调度。

任务切换:

当一台计算机开始运行时,操作系统首先会通过懒分配创建一个初始的进程,该进程是一个全局的进程,接下来的进程都由它创建。初始进程创建以后,根据计算机的需要或者用户的操作,操作系统开始由clone函数来创建一个个新的进程并加入到就绪队列中。按照时间片
轮转的想法,操作系统给正在处理器上运行的(也就是processor所拥有的TCB)进程分配了一个时间片,在自然运行到时间耗尽后,操作系统就会使用sys_yield()函数让出当前进程对处理器的占有权,又或者内核出现错误造成trap,程序也会使当前进程让出处理器,其中的核心函数便是suspend_current_and_run_next,该函数将当前进程的上下文进行保存,并将进程的状态改成ready,表示进程准备执行,然后将他放回到ready队列中的队尾进行排队,等待下一次轮转到它。

suspend_current_and_run_next函数会使得当前进程从current_task切换到中转进程idle,而调度器如何将idle切换为next_task呢?

它将使用run_tasks()函数完成从idle到next_task的切换。run_tasks()函数位于os/src/task/processor.rs下。让我们看 os/src/task/processor.rs 下的 run_task 函数:
\begin{lstlisting}[language=rust]
    pub fn run_tasks() {
    loop {
        let mut processor = PROCESSOR.lock();
        if let Some(task) = fetch_task() {
            let idle_task_cx_ptr = processor.get_idle_task_cx_ptr();
            // access coming task TCB exclusively
            let next_task_cx_ptr = {
                let mut task_inner = task.acquire_inner_lock();
                task_inner.task_status = TaskStatus::Running;
                &task_inner.task_cx as *const TaskContext
            };
            processor.current = Some(task);
            // release processor manually
            drop(processor);
            unsafe {
                __switch(idle_task_cx_ptr, next_task_cx_ptr);
            }
        } else {
            drop(processor);
            // we have no ready tasks, try to wake some...
            do_wake_expired();
        }
    }
}
\end{lstlisting}

可以看到,操作系统采⽤"轮询机制",在操作系统运⾏的任⼀时刻都在尝试从idle流切换到下一个进程(采用loop死循环),接着我们分析⼀下具体的过程。

fetch_task() 的作用是选取⼀个将要执⾏的进程 

\begin{lstlisting}[language=rust]
    pub fn fetch_task() -> Option<Arc<TaskControlBlock>> { 
TASK_MANAGER.lock().fetch() 
}
impl TaskManager { 
//... 
pub fn fetch(&mut self) -> Option<Arc<TaskControlBlock>> { 
self.ready_queue.pop_front() 
} 
//... 
} 
\end{lstlisting}

可以看到,调度器的选择是将“就绪任务”的队列进⾏出队操作。
因此我们可以知道, NPUcore 中对进程的调度实际上就是对 ready_queue 进行管理。 

等待队列:

进程的调度过程中,有一些进程是需要等待某种资源,例如IO输入这样的资源,因此才陷入阻塞过程中,他们不得到自己所需的资源前,便无法被唤醒加入就绪队列中(因此它不同于interruptible_queue中的进程,在可中断睡眠队列中的进程,需要等待中断操作就可以被唤醒),因此这涉及到了waitqueue这个重要的数据结构。

等待队列(wait_queue)是用于管理等待特定资源或时间的进程或线程,它是一种先进先出的数据结构。该队列通常用于解决并发编程中的同步和互斥问题。当多个进程或者线程需要访问共享资源时,如果资源已经被占用,那么需要将正在等待的进程放入等待队列,以便在资源可用时依次获得访问权限。

当某资源得到释放,是的等待队列中的进程可被唤醒时,操作系统会调用wake_at_most这个函数。

\begin{lstlisting}[language=rust]
            if limit == 0 {
            return 0;
        }
        let mut manager = TASK_MANAGER.lock();
        let mut cnt = 0;
        while let Some(task) = self.inner.pop_front() {
            match task.upgrade() {
                Some(task) => {
                    let mut inner = task.acquire_inner_lock();
                    match inner.task_status {
                        super::TaskStatus::Interruptible => {
                            inner.task_status = super::task::TaskStatus::Ready
                        }
                        _ => continue,
                    }
                    drop(inner);
                    if manager.try_wake_interruptible(task).is_ok() {
                        cnt += 1;
                    }
                    if cnt == limit {
                        break;
                    }
                }
                None => continue,
            }
        }
        cnt
    }
\end{lstlisting}
该函数的功能为:从等待队列中唤醒最多limit个任务,并返回实际唤醒的任务数量。\\
首先,函数检查limit是否为0,如果为0,则表示没有唤醒任何任务。\\
然后,函数获取全局的任务管理器的锁,确保任务管理器的线程安全性。\\
接下来,函数使用cnt记录唤醒的任务的数量,初始为0.\\
然后,函数进入循环,从等待队列(self.inner)的头部弹出任务,并进行处理。\\
在处理吃钱,函数会尝试将任务引用升级为ARC类型的对象。如果该对象有效,则执行以下操作:\\
1)获取任务对象的内部锁(acquire_inner_lock)\\
2)根据任务的状态进行不同处理,interruptible的任务,将其状态设置为ready,如果为其他状态,则不需要唤醒,继续处理下一个任务。\\
3)释放任务对象的内部锁(drop(inner))\\
4)尝试使用try_wake_interruptible方法唤醒任务\\
5)检查cnt计数器是否达到limit限制,达到则跳出循环。\\
如果任务对象引用升级为None,表示任务已经被销毁了,函数会处理下一个任务。\\
最后返回唤醒的任务数量cnt。\\

经过该函数的处理,那些得到了资源的进程,可以被唤醒进入ready队列中,等待时间片的轮转。而该函数中又涉及到了一个关键的函数try_wake_interruptible:

\begin{lstlisting}[language=rust]
        pub fn try_wake_interruptible(
        &mut self,
        task: Arc<TaskControlBlock>,
    ) -> Result<(), WaitQueueError> {
        self.drop_interruptible(&task);
        if self.find_by_pid(task.pid.0).is_none() {
            self.add(task);
            Ok(())
        } else {
            Err(WaitQueueError::AlreadyWaken)
        }
    }
\end{lstlisting}
该函数接受一个任务对象task作为参数,并尝试唤醒该任务,返回一个result类型,表示唤醒操作的结果。\\
首先,函数调用self.drop_interruptible方法,该方法会从等待队列中移除具有相同pid的任务。该步骤确保在唤醒任务之前先将其从等待队列中移除,避免重复唤醒。\\
接下来,函数使用self.find_by_pid方法来检查ready队列中是否具有相同pid的任务,如果返回None,表示不存在。\\
在这种情况下,函数调用self.add方法将任务添加到准备队列中,并返回ok结果。\\
如果ready队列中具有相同pid的任务,那么函数表示唤醒失败,已经唤醒。\\

除了等待队列以外,还有一个也很重要的数据结构被用在进程的调度中,那就是TimeoutWaitQueue(超时等待池)。

超时等待池用于处理等待超时的情况。在某些场景下,等待某个资源或者事件的进程或线程可能需要在一定的时间内得到相应,如果超过指定时间仍然没有得到响应,那么就需要采取相应的措施。超时等待池允许进程或者线程设置一个超时时间,如果超过时间却未能唤醒,那么就会触发超时逻辑。

超时等待池的存在是为了处理等待时间的限制,以避免进程或者线程在等待资源或时间上长时间阻塞。它提供了一种在等待超时后继续执行的机制,从而增加了系统的可靠性和响应性。

npucore依靠定时器在一定的时间间隔下进入trap,自动唤醒超时等待池中的进程。

在os/src/trap/mod.rs源文件中的trap_handle函数中有对SupervisorTimer中断的处理方式:

\begin{lstlisting}[language=rust]
            Trap::Interrupt(Interrupt::SupervisorTimer) => {
            do_wake_expired();
            set_next_trigger();
            suspend_current_and_run_next();
        }
\end{lstlisting}
该函数首先调用了do_wake_expired方法:
\begin{lstlisting}[language=rust]
    pub fn do_wake_expired() {
    TIMEOUT_WAITQUEUE
        .lock()
        .wake_expired(crate::timer::TimeSpec::now());
}
\end{lstlisting}
其中又调用了TimeoutWaitQueue的wake_expired方法:
\begin{lstlisting}[language=rust]
        pub fn wake_expired(&mut self, now: TimeSpec) {
        let mut manager = TASK_MANAGER.lock();
        while let Some(waiter) = self.inner.pop() {
            // the remaining tasks in heap haven't reach their timeout
            if waiter.timeout > now {
                log::trace!(
                    "[wake_expired] no more expired, next pending task timeout: {:?}, now: {:?}",
                    waiter.timeout,
                    now
                );
                self.inner.push(waiter);
                break;
            // wake one task
            } else {
                match waiter.task.upgrade() {
                    Some(task) => {
                        let mut inner = task.acquire_inner_lock();
                        match inner.task_status {
                            super::TaskStatus::Interruptible => {
                                inner.task_status = super::task::TaskStatus::Ready
                            }
                            _ => continue,
                        }
                        drop(inner);
                        log::trace!(
                            "[wake_expired] pid: {}, timeout: {:?}",
                            task.pid.0,
                            waiter.timeout
                        );
                        manager.wake_interruptible(task);
                    }
                    None => continue,
                }
            }
        }
    }
\end{lstlisting}
该方法获取一个队self的可变引用,以及一个类型为timespec的now参数,没有指定返回来类型。该方法的作用是从任务列表中唤醒已经过期的任务,并将其状态改为ready。
\\[10pt]
首先代码通过获取task_manager对象上的锁来创建一个manager变量,并获取一个互斥锁的保护。\\
接着,代码进入循环,不断的从self.inner弹出一个元素来遍历任务。\\
在循环的每一次迭代中,检查当前任务(waiter)是否已经过期,如果没有过期,代码会将任务放回self.inner,并中断循环。\\
如果任务已经过期,代码会尝试将任务唤醒。它通过调用waiter.task.upgrade()获取任务的强引用。如果任务仍然存在,那么获取内部锁,检查任务状态。\\
如果状态为interruptible,则改为 ready,表示就绪。\\
最后,代码释放内部锁,调用wake_interruptible(task)来唤醒任务。\\
如果任务的强引用已经无效,则说明任务已经被销毁,代码忽略此任务,继续处理下一个任务。
\\[10pt]

由此,我们就讲完了npucore中进程调度相关的内容,本节我们只对重要的地方做了比较详细的介绍,实际上进程的调度是个复杂的问题,光靠这些短短的文字是讲不清楚的,读者可以在闲暇之余,查询网上资料加深自己对进程调度的了解





\subsection{进程退出}

当一个进程完成自己的工作后,就需要调用exit进行“退出”以结束自己的生命周期。在xv6中,当一个子进程退出时它并不是直接死掉,而是将状态转变为Zombie。此后,当父进程调用wait时,将发现子进程可以退出,并由父进程负责释放子进程相关的内存空间。倘若父进程在子进程之前退出了,则由初始进程initproc接收子进程并负责它们退出。下面我们结合代码来阐述这一过程:

首先来看sys\_exit系统调用:

\begin{lstlisting}[language={Rust}, label={code:exit},
	caption={os/src/syscall/process.rs}]
pub fn sys_exit(exit_code: i32) -> ! {
	exit_current_and_run_next(exit_code);
	panic!("Unreachable in sys_exit!");
}
\end{lstlisting}

事实上,当应用调用sys\_exit系统调用主动退出,或是执行出错由内核终止之后,内核中都将调用exit\_current\_and\_run\_next函数退出当前进程并切换到下一个进程。

exit\_current\_and\_run\_next函数以一个退出码作为参数。当在sys\_exit系统调用中正常退出时,退出码由应用传到内核中;而对于出错退出的情况(例如访存错误、非法指令异常等),则是由内核指定一个特定的退出码(具体可在trap\_handler函数中查看)。最终,这个退出码会写入当前进程的TCB中,具体如下:

\begin{lstlisting}[language={Rust}, label={code:exit},
	caption={os/src/task/mod.rs}]
pub fn exit_current_and_run_next(exit_code: i32) {
	let task = take_current_task().unwrap();
	let mut inner = task.inner_exclusive_access();
	inner.task_status = TaskStatus::Zombie;
	inner.exit_code = exit_code;
	// ++++++ access initproc TCB exclusively
	{
		let mut initproc_inner = INITPROC.inner_exclusive_access();
		for child in inner.children.iter() {
			child.inner_exclusive_access().parent = Some(Arc::downgrade(&INITPROC));
			initproc_inner.children.push(child.clone());
		}
	}
	// ++++++ stop exclusively accessing parent PCB
	inner.children.clear();
	inner.memory_set.recycle_data_pages();
	drop(inner);
	// **** stop exclusively accessing current PCB
	drop(task);
	let mut _unused = TaskContext::zero_init();
	schedule(&mut _unused as *mut _);
}
\end{lstlisting}

第2行,我们从处理器监控PROCESSOR中取出而不是获取一个拷贝,这是为了正确维护TCB的引用计数。

第4行,我们将TCB中的状态改为为TaskStatus::Zombie即僵尸进程,令其后续被父进程在waitpid系统调用时进行回收。

第5行,我们将传入的退出码exit\_code写入TCB中,使后续父进程可以收集该退出码。

第7$\sim$13行,我们将当前进程的所有子进程挂在初始进程 initproc下面。做法是遍历当前进程的每个子进程,修改其父进程为初始进程,并把它们加入初始进程的子进程向量中。最后在第15行,我们将当前进程的子进程向量清空。

第16行,我们对当前进程占用的资源进行早期回收。我们调用了recycle\_data\_pages函数:

\begin{lstlisting}[language={Rust}, label={code:exit},
	caption={os/src/mm/memory\_set.rs}]
impl MemorySet {
	pub fn recycle_data_pages(&mut self) {
		self.areas.clear();
	}
}
\end{lstlisting}

注意该函数只是将地址空间中的逻辑段列表areas清空,从而使应用的地址空间被回收(即进程的数据段和代码段对应的物理页帧被回收)。然而用来存放页表的那些物理页帧此时还不会被回收,需要由父进程最后来回收子进程剩余的占用资源。

第21行,我们最后调用了schedule触发调度与任务切换,由于我们再也不会回到该进程的执行过程中,因此无需进行任务上下文的保存。

以上是子进程进行退出的过程,下面介绍父进程通过sys\_wait4系统调用来回收子进程资源的实现:

\begin{lstlisting}[language={Rust}, label={code:wait4},
	caption={os/src/syscall/process.rs}]
pub fn sys_wait4(pid: isize, status: *mut u32, option: u32, ru: *mut Rusage) -> isize {
	let option = WaitOption::from_bits(option).unwrap();
	info!("[sys_wait4] pid: {}, option: {:?}", pid, option);
	let task = current_task().unwrap();
	let token = task.get_user_token();
	loop {
		// find a child process
		
		// ---- hold current PCB lock
		let mut inner = task.acquire_inner_lock();
		if inner
			.children
			.iter()
			.find(|p| pid == -1 || pid as usize == p.getpid())
			.is_none()
		{
			return ECHILD;
			// ---- release current PCB lock
		}
		inner
			.children
			.iter()
			.filter(|p| pid == -1 || pid as usize == p.getpid())
			.for_each(|p| {
				trace!(
				"[sys_wait4] found child pid: {}, status: {:?}",
				p.pid.0,
				p.acquire_inner_lock().task_status
				)
			});
		let pair = inner.children.iter().enumerate().find(|(_, p)| {
			// ++++ temporarily hold child PCB lock
			p.acquire_inner_lock().is_zombie() && (pid == -1 || pid as usize == p.getpid())
			// ++++ release child PCB lock
		});
		if let Some((idx, _)) = pair {
			// drop last TCB of child
			let child = inner.children.remove(idx);
			trace!("[wait4] release zombie task, pid: {}", child.pid.0);
			// confirm that child will be deallocated after being removed from children list
			assert_eq!(Arc::strong_count(&child), 1);
			// if main thread exit
			if child.pid.0 == child.tgid {
				let found_pid = child.getpid();
				// ++++ temporarily hold child lock
				let exit_code = child.acquire_inner_lock().exit_code;
				// ++++ release child PCB lock
				if !status.is_null() {
					// this may NULL!!!
					match translated_refmut(token, status) {
						Ok(word) => *word = exit_code,
						Err(errno) => return errno,
					};
				}
				return found_pid as isize;
			}
		} else {
			drop(inner);
			if option.contains(WaitOption::WNOHANG) {
				return SUCCESS;
			} else {
				block_current_and_run_next();
				debug!("[sys_wait4] --resumed--");
			}
		}
	}
}
\end{lstlisting}

sys\_wait4函数的参数为指定进程的pid、可存储退出码的用户空间区域的指针status、函数执行方式的选项option、以及可存储进程资源占用信息的ru指针。

第11$\sim$19行,我们判断当前进程是否具有符合要求的子进程。当传入的pid为-1时,任何一个子进程都算是符合要求;但传入的pid不为-1的时候,则只有子进程的PID恰好与传入的pid相同时,才算符合条件。若没有符合条件的子进程,则函数直接返回ECHILD。

第20$\sim$36行,我们判断符合要求的子进程中是否有僵尸进程,若有,记录它在当前TCB子进程向量中的下标;若无,则进入第57~65行的处理.

第38行,我们将子进程从向量中移除并置于当前上下文中。

第41行,我们确认这是对于该子进程控制块的唯一一次强引用,即它不会出现在某个进程的子进程向量中,更不会出现在处理器监控器或者任务管理器中。当它所在的代码块结束,这次引用变量的生命周期结束,将导致该子进程进程控制块的引用计数变为0,彻底回收掉它占用的所有资源,包括其内核栈、PID还有应用地址空间存放页表的那些物理页帧等。

第43$\sim$56行,我们将收集的子进程信息进行返回,包括退出码等,最后以回收的子进程PID作为函数返回值返回。

第57$\sim$65行,是对于没有可回收的子进程的情况的处理。倘若函数选项参数option具有WNOHANG标志,则说明调用该函数的进程不会被挂起,返回SUCCESS继续执行;若无该标志,则会切换至下一个进程执行。

至此,父进程对子进程的回收机制也介绍完毕。下面我们最后再介绍一个进程的退出机制:kill。

如果说exit机制是一个进程的“自杀”,则kill机制使得一个进程可以“杀死”其他进程。我们来看具体实现:

\begin{lstlisting}[language={Rust}, label={code:kill},
	caption={os/src/syscall/process.rs}]
pub fn sys_kill(pid: usize, sig: usize) -> isize {
	let signal = match Signals::from_signum(sig) {
		Ok(signal) => signal,
		Err(_) => return EINVAL,
	};
	if pid > 0 {
		if let Some(task) = find_task_by_tgid(pid) {
			if !signal.is_empty() {
				let mut inner = task.acquire_inner_lock();
				inner.add_signal(signal);
				// wake up target process if it is sleeping
				if inner.task_status == TaskStatus::Interruptible {
					inner.task_status = TaskStatus::Ready;
					drop(inner);
					wake_interruptible(task);
				}
			}
			SUCCESS
		} else {
			ESRCH
		}
	} else if pid == 0 {
		todo!()
	} else if (pid as isize) == -1 {
		todo!()
	} else { // (pid as isize) < -1
		todo!()
	}
}
\end{lstlisting}

事实上,sys\_kill系统调用的实现非常简单。该函数有两个参数,一个是目的进程的pid,一个是将给目的进程赋予的信号sig。kill机制的实现实际上就是找出要“杀死”的进程,并给它赋予一个信号量,该信号可以为“自杀”的信号,然后再将该进程唤醒,让其自我终结。

第6$\sim$21行,就是先使用find\_task\_by\_tgid函数来找出目的进程,若有,则为其赋予一个指定的信号,然后再将该进程从阻塞态唤醒,然后返回SUCCESS;若无,则返回ESRCH代表未找到目的进程。

如你所见,本系统调用仅是kill机制的初步实现,还未最终完善。

\chapter{初识块设备}
\section{驱动程序}

驱动程序是一种软件组件,是操作系统与机外设之间的接口,可让操作系统和设备彼此通信。从操作系统架构上看,驱动程序与I/O设备靠的更近,离应用程序更远,这使得驱动程序需要站在协助所有进程的全局角度来处理各种I/O操作。这也就意味着在驱动程序的设计实现中,尽量不要与单个进程建立直接的联系,而是在全局角度对I/O设备进行统一处理。
	
上面只是介绍了CPU和I/O设备之间的交互手段。如果从操作系统角度来看,我们还需要对特定设备编写驱动程序。它一般需包括如下一些操作:
	
\begin{itemize}
	\item 定义设备相关的数据结构,包括设备信息、设备状态、设备操作标识等
	\item 设备初始化,即完成对设备的初始配置,分配I/O操作所需的内存,设置好中断处理例程
	\item 如果设备会产生中断,需要有处理这个设备中断的中断处理例程(Interrupt Handler)
	\item 根据操作系统上层模块(如文件系统)的要求(如读磁盘数据),给I/O设备发出命令,检测和处理设备出现的错误
	\item 与操作系统上层模块或应用进行交互,完成上层模块或应用的要求(如上传读出的磁盘数据)
\end{itemize}
	
从驱动程序I/O操作的执行模式上看,主要有两种模式的I/O操作:异步和同步。同步模式下的处理逻辑类似函数调用,从应用程序发出I/O请求,通过同步的系统调用传递到操作系统内核中,操作系统内核的各个层级进行相应处理,并最终把相关的I/O操作命令转给了驱动程序。一般情况下,驱动程序完成相应的I/O操作会比较慢(相对于CPU而言),所以操作系统会让代表应用程序的进程进入等待状态,进行进程切换。但相应的I/O操作执行完毕后(操作系统通过轮询或中断方式感知),操作系统会在合适的时机唤醒等待的进程,从而进程能够继续执行。
		
异步I/O操作是一个效率更高的执行模式,即应用程序发出I/O请求后,并不会等待此I/O操作完成,而是继续处理应用程序的其它任务(这个任务切换会通过运行时库或操作系统来完成)。调用异步I/O操作的应用程序需要通过某种方式(比如某种异步通知机制)来确定I/O操作何时完成。注:这部分可以通过协程技术来实现,但目前我们不会就此展开讨论。
	
编写驱动程序代码其实需要的知识储备还是比较多的,需要注意如下的一些内容:
	
\begin{itemize}
	\item 了解硬件规范:从而能够正确地与硬件交互,并能处理访问硬件出错的情况;
	\item 了解操作系统,由于驱动程序与它所管理的设备会同时执行,也可能与操作系统其他模块并行/并发访问相关共享资源,所以需要考虑同步互斥的问题(后续会深入讲解操作系统同步互斥机制),并考虑到申请资源失败后的处理;
	\item 理解驱动程序执行中所在的可能的上下文环境:如果是在进行中断处理(如在执行 trap\_handler 函数),那是在中断上下文中执行;如果是在代表进程的内核线程中执行后续的I/O操作(如收发TCP包),那是在内核线程上下文执行。这样才能写出正确的驱动程序。
\end{itemize}


\section{SD卡的驱动框架}
6.2.1 SD/MMC卡介绍

1.1.什么是MMC卡

MMC:MMC就是MultiMediaCard的缩写,即多媒体卡。它是一种非易失性存储器件,体积小巧(24mm*32mm*1.4mm),容量大,耗电量低,传输速度快,广泛应用于消费类电子产品中。

1.2.什么是SD卡

SD:SD卡为Secure Digital Memory Card, 即安全数码卡。它在MMC的基础上发展而来,增加了两个主要特色:SD卡强调数据的安全安全,可以设定所储存的使用权限,防止数据被他人复制;另外一个特色就是传输速度比2.11版的MMC卡快。在数据传输和物理规范上,SD卡(24mm*32mm*2.1mm,比MMC卡更厚一点),向前兼容了MMC卡.所有支持SD卡的设备也支持MMC卡。SD卡和2.11版的MMC卡完全兼容。
\begin{figure}[H]
    \centering
    \includegraphics{figures/06-02-SD.png}
\end{figure}

SD卡内部结构如上图所示

1.3.什么是SDIO

SDIO:SDIO是在SD标准上定义了一种外设接口,它和SD卡规范间的一个重要区别是增加了低速标准。在SDIO卡只需要SPI和1位SD传输模式。低速卡的目标应用是以最小的硬件开销支持低速IO能力。

1.4.什么是MCI

MCI:MCI是Multimedia Card Interface的简称,即多媒体卡接口。上述的MMC,SD,SDI卡定义的接口都属于MCI接口。MCI这个术语在驱动程序中经常使用,很多文件,函数名字都包括”mci”.

1.5.MMC/SD/SDIO卡的区别
\begin{figure}[H]
    \centering
    \includegraphics{figures/06-02-区别.png}
\end{figure}

SD卡内部有7个寄存器,如下表1所示.其中OCR,CID,CSD和SCR寄存器保存卡的配置信息;RCA寄存器保存着通信过程中卡当前暂时分配的地址(只适合SD模式);卡状态(Card Status)和SD状态(SD Status)寄存器保存着卡的状态(例如,是否写成功,通信的CRC校验是否正确等),这两个寄存器的内容与通信模式(SD模式或SPI模式)相关.MMC卡没有SCR和SD Status寄存器.
\begin{figure}[H]
    \centering
    \includegraphics{figures/06-02-寄存器表.png}
\end{figure}

OCR寄存器保存着SD/MMC卡的供电电允许范围.如下表2所示:如果OCR寄存器的某位为1,表示卡支持该位对应的电压。最后一位表示卡上电后的状态(是否处于”忙状态”),如果该位为0,表示忙,如果为1,表示处于空闲状态(MMC/SD协议P60)。
\begin{figure}[H]
    \centering
    \includegraphics{figures/06-02-供电点允.png}
\end{figure}

CID为一个16个字节的寄存器,该寄存器包含一个独特的卡标识号。如下表3所示:
\begin{figure}[H]
    \centering
    \includegraphics{figures/06-02-卡标标识.png}
\end{figure}

SD/SDIO有以下几种传输模式:
(1)SPI mode,独立序列输入和独立序列输出,使⽤CS、DI、SCLK与DO(SD卡的片选、数据输入、
时钟与数据输出)四根信号线进行数据传输。

(2)1-bit mode,独立指令和数据通道,只支持1位宽的数据传输。

(3)4-bit mode,使用额外的针脚以及某些重新设置的针脚。支持四位宽的并行传输。

6.2.2 k210联动SD卡

K210 裸机使用SD卡,下图是SD卡对应接口
\begin{figure}[H]
    \centering
    \includegraphics{figures/06-02-接口标.png}
\end{figure}

为了协同软硬接口,我们需要在代码中定义相关常量,使软硬接口保持一致,这即为遵守SD卡协议。
\begin{lstlisting}[language={Rust}, label={code:inode},
	caption={SD卡协议}]
    const MMIO =[
        (0x0C00_0000, 0x3000), /* PLIC */
        (0x0C20_0000, 0x1000), /* PLIC */
        (0x3800_0000, 0x1000), /* UARTHS */
        (0x3800_1000, 0x1000), /* GPIOHS */
        (0x5020_0000, 0x1000), /* GPIO */
        (0x5024_0000, 0x1000), /* SPI_SLAVE */
        (0x502B_0000, 0x1000), /* FPIOA */
        (0x502D_0000, 0x1000), /* TIMER0 */
        (0x502E_0000, 0x1000), /* TIMER1 */
        (0x502F_0000, 0x1000), /* TIMER2 */
        (0x5044_0000, 0x1000), /* SYSCTL */
        (0x5200_0000, 0x1000), /* SPI0 */
        (0x5300_0000, 0x1000), /* SPI1 */
        (0x5400_0000, 0x1000), /* SPI2 */
        ];
\end{lstlisting}


应用程序通过文件系统接口如open()、read()、write()、close()等访问文件系统,根据文件系统inode节点,接着找到文件在SD卡驱动上的块号。文件系统通过块设备驱动层与SD卡协议层对接,块设备驱动层定义了抽象块设备的接口,主要包括对块设备的读写接口。SD卡协议层主要负责按照SD卡标准规范向SD卡发送指令或者接收响应数据;硬件接口层则负责按照硬件板卡的引脚定义操作GPIO或SPI引脚实现与SD卡的数据交互。

6.2.3 SD卡的命令
 
SD卡命令共分为12类,分别为class0到Class11.
 
3.2.1. Class0 :(卡的识别、初始化等基本命令集)
CMD0:复位SD 卡。

CMD1:读OCR寄存器。

CMD9:读CSD寄存器。

CMD10:读CID寄存器。

CMD12:停止读多块时的数据传输。

CMD13:读 Card_Status 寄存器。

3.2.2.Class2 (读卡命令集):

CMD16:设置块的长度。

CMD17:读单块。

CMD18:读多块,直至主机发送CMD12为止 。
 
3.2.3.Class4(写卡命令集) :

CMD24:写单块。

CMD25:写多块。

CMD27:写CSD寄存器 。

3.2.4.Class5 (擦除卡命令集):

CMD32:设置擦除块的起始地址。

CMD33:设置擦除块的终止地址。

CMD38: 擦除所选择的块。


3.2.5.Class6(写保护命令集):

CMD28:设置写保护块的地址。

CMD29:擦除写保护块的地址。

CMD30: Ask the card for the status of the write protection bits

 class7:卡的锁定,解锁功能命令集。

 class8:申请特定命令集 。

 class10 -11 :保留。

(注:完整细节请自行参考SD卡协议官方文档)

6.2.4 SD卡工作流程

SD卡工作流程大致可以分为3个大的步骤:初始化sd卡、写sd卡、读sd卡。在SPI模式下,SD卡工作模式分为卡识别模式和数据传输模式。如下图为卡在识别模式下的命令流程。

\begin{figure}[H]
    \centering
    \includegraphics{figures/06-02-命令流程.png}
\end{figure}

在复位后,查找总线上的新卡的时候,主机会处于“卡识别模式”。卡在复位后会处于识别模式。不同的SD卡可能支持不同版本的协议或者不同的⼯作电压。因此,作为主机,在与SD卡进行交互之初,主机需要获取卡的工作电压范围和卡的类型。在卡识别期间,时钟频率应该保持在100~400kHZ之间。

1_在主机和SD卡进行任何通信之前,主机不知道SD卡支持的工作电压范围,卡也不知道是否支持主机当前提供的电压。因此主机首先使用默认的电压发送一条reset指令(CMD0)。

2_为了验证SD卡的接口操作状态,主机发送SEND_IF_COND(CMD8),用于取得SD卡支持工作的电压范围数据。SD卡通过检测CMD8的参数部分来检查主机使用的工作电压,主机通过分析回传的CMD8参数部分来校验SD卡是否可以在所给电压下工作,如果SD卡可以在指定电压下工作,则它回送CMD8的命令响应字 。如果不支持所给电压,则SD卡不会给出任何响应信息,并继续处于IDLE状态。

3_在发送ACMD41命令初始化高容量的SD卡前,需要强制发送CMD8命令。强制低电压主机在发送CMD8前发送ACMD41,万一双重电压SD卡没有收到CMD8命令且工作在高电压状态,在这种情况下,低电压主机不能不发送CMD8命令给卡,则收到ACMD41后进入无活动状态。

4_SD_SEND_OP_COND(ACMD)命令是为SD卡主机识别卡或者电压不匹配时拒绝卡的机制设计的。主机发送命令操作数代表要求的电压窗口大小。如果SD卡在所给的范围内不能实现数据传输,将放弃下一步的总线操作而进入无活动。操作状态寄存器也将被定义。

5_在主机发出复位命令(CMD0)后,主机将先发送CMD8再发送ACMD41命令重新初始化SD卡。

当总线被激合后,主机就开始卡的初始化和识别3处理。初始化处理设置它的操作状态和是设置OCR中的HCS比特命令SD_SEND_OP_COND(ACMD41)开始。HCS比特位被设置为1表示主机支持高容量SD卡。HCS被设置为0表示主机不支持高容量SD卡。卡的初始化和识别流程见下图。
\begin{figure}[H]
    \centering
    \includegraphics{figures/06-02-初始化.png}
\end{figure}
(注:这里不推荐使用NPUCore或rCore作为代码范式来讲解,应直接看SD卡协议文档)

卡在识别模式结束后,主机时钟fpp(数据传输时钟频率)将保存为fod(卡识别模式下的时钟),由于有些卡对操作时钟有限制。主机必须发送SEND_CSD(CMD9)来获得卡规格数据积存器内容,如块大小,卡容量。广播命令SET_DSR(CMD4)配置所有识别卡的驱动阶段。它对DSR积存器进行编程以适应应用总线布局,总线上的卡数目和数据传输频率。SD卡数据传输模式的流程图如下图。
\begin{figure}[H]
    \centering
    \includegraphics{figures/06-02-数据传输.png}
\end{figure}

1CMD7命令用来选择某个SD卡,使其进入Transfer状态,在指定时间段内,只有一个卡能处于Transfer状态。当某个先前被选中的处于Transfer状态的SD卡接收到CMD7之后,会释放与控制器的连接,并进入Stand-by态。当CMD7使用保留地址0x0000时,所有的SD卡都会进入Stand-by状态 。


2所有的数据读命令都可以被停止命令(CMD12)在任意时刻终止。数据传输会终止,SD卡返回Transfer状态。读命令有:块读操作(CMD17)、多块读操作(CMD18)、发送写保护(CMD30)、发送scr(ACMD51)以及读模式下的普通命令(CMD56)。

3所有的数据写命令都可以被停止命令(CMD12)在任意时刻终止。写命令也会在取消选择命令(CMD7)之前停止。写命令有:块写操作(CMD24,CMD25)、编程命令(CMD27)、锁定/解锁命令(CMD42)以及写模式下的普通命令(CMD56)。

4数据传输一旦完成,SD卡会退出数据写状态,进入Programming状态(传输成功)或者Transfer状态(传输失败)。

\section{QEMU下SD卡的驱动理解}

\chapter{理解文件系统}




\section{文件系统概述}
文件最早来自于计算机用户需要把数据持久保存在 持久存储设备上的需求。由于放在内存中的数据在计算机关机或掉电后就会消失,所以应用程序要把内存中需要保存的数据放到 持久存储设备的数据块(比如磁盘的扇区等)中存起来。随着操作系统功能的增强,在操作系统的管理下,应用程序不用理解持久存储设备的硬件细节,而只需对文件这种持久存储数据的抽象进行读写就可以了,由操作系统中的文件系统和存储设备驱动程序一起来完成繁琐的持久存储设备的管理与读写。所以本章要完成的操作系统的第一个核心目标是: 让应用能够方便地把数据持久保存起来 。

对于应用程序访问持久存储设备的需求,内核需要新增两种文件:常规文件和目录文件,它们均以文件系统所维护的磁盘文件形式被组织并保存在持久存储设备上。

这里简要介绍一下在内核中添加文件系统的大致开发过程。

\textbf{第一步:是能够写出与文件访问相关的应用}

这里是参考了Linux的创建/打开/读写/关闭文件的系统调用接口,力图实现一个简化版的文件系统模型 。在用户态我们只需要遵从相关系统调用的接口约定,在用户库里完成对应的封装即可。

\textbf{第二步:实现 easyfs 文件系统}

由于 Rust 语言的特点,我们可以在用户态实现文件系统,并在用户态完成文件系统功能的基本测试并基本验证其实现正确性之后,就可以放心的将该模块嵌入到操作系统内核中。当然,有了文件系统的具体实现,还需要对上一章的操作系统内核进行扩展,实现文件系统对接的接口,这样才可以让操作系统拥有一个简单可用的文件系统。这样内核就可以支持具有文件读写功能的复杂应用。当内核进一步支持应用的命令行参数后,就可以进一步提升应用程序的灵活性,让应用的开发和调试变得更为轻松。
文件系统的整体架构自下而上可分为五层:
\begin{itemize}
    \item 磁盘块设备接口层:读写磁盘块设备的trait接口
    \item 块缓存层:位于内存的磁盘块数据缓存
    \item 磁盘数据结构层:表示磁盘文件系统的数据结构
    \item 磁盘块管理器层:实现对磁盘文件系统的管理
    \item 索引节点层:实现文件创建/文件打开/文件读写等操作
\end{itemize}
它的最底层就是对块设备的访问操作接口。两个函数 read\_block 和 write\_block ,分别代表将数据从块设备读到内存缓冲区中,或者将数据从内存缓冲区写回到块设备中,数据需要以块为单位进行读写。

尽管在操作系统的最底层(即块设备驱动程序)已经有了对块设备的读写能力,但从编程方便/正确性和读写性能的角度来看,仅有块读写这么基础的底层接口是不足以实现高效的文件系统。比如,某应用将一个块的内容读到内存缓冲区,对缓冲区进行修改,并尚未写回块设备时,如果另外一个应用再次将该块的内容读到另一个缓冲区,而不是使用已有的缓冲区,这将会造成数据不一致问题。此外还有可能增加很多不必要的块读写次数,大幅降低文件系统的性能。因此,通过程序自动而非程序员手动地对块缓冲区进行统一管理也就很必要了,该机制被我们抽象为第二层,即块缓存层。

有了块缓存,我们就可以在内存中方便地处理文件系统在磁盘上的各种数据了,这就是第三层文件系统的磁盘数据结构。

文件系统中的所有需要持久保存的数据都会放到磁盘上,这包括了管理这个文件系统的 超级块 (Super Block),管理空闲磁盘块的索引节点位图区和数据块位图区 ,以及管理文件的索引节点区和放置文件数据的数据块区组成。文件系统中管理这些磁盘数据的控制逻辑主要集中在磁盘块管理器中,这是文件系统的第四层。

对于单个文件的管理和读写的控制逻辑主要是 索引节点(文件控制块) 来完成,这是文件系统的第五层。

\textbf{第三步:把easyfs文件系统加入内核中}

这还需要做两件事情,第一件是在Qemu模拟的 virtio 块设备上实现块设备驱动程序 。由于我们可以直接使用 virtio-drivers crate中的块设备驱动,所以只要提供这个块设备驱动所需要的内存申请与释放以及虚实地址转换的4个函数就可以了。而我们之前操作系统中的虚存管理实现中,已经有这些函数,这使得块设备驱动程序很简单,且具体实现细节都被 virtio-drivers crate封装好了。

第二件事情是把文件访问相关的系统调用与easyfs文件系统连接起来。在easfs文件系统中是没有进程的概念的。而进程是程序运行过程中访问资源的管理实体,而之前的进程没有管理文件这种资源。 为此我们需要扩展进程的管理范围,把文件也纳入到进程的管理之中。 由于我们希望多个进程都能访问文件,这意味着文件有着共享的天然属性,这样自然就有了open/close/read/write这样的系统调用,便于进程通过互斥或共享方式访问文件。

内核中的进程看到的文件应该是一个便于访问的Inode,这就要对Inode 结构进一步封装,形成 OSInode 结构,以表示进程中一个打开的常规文件。而进程为了进一步管理多个文件,需要扩展文件描述符表。这样进程通过系统调用打开一个文件后,会将文件加入到自身的文件描述符表中,并进一步通过文件描述符(也就是某个特定文件在自身文件描述符表中的下标)来读写该文件。对于应用程序而言,它理解的磁盘数据是常规的文件和目录,不是 OSInode 这样相对复杂的结构。其实常规文件对应的 OSInode 是操作系统内核中的文件控制块数据结构的实例,它实现了 File Trait 定义的函数接口。这些 OSInode 实例会放入到进程文件描述符表中,并通过 sys\_read/write 系统调用来完成读写文件的服务。这样就建立了文件与 OSInode 的对应关系,通过上面描述的三个开发步骤将形成包含文件系统的操作系统内核,可给应用提供基于文件的系统调用服务。

\section{块设备接口层}

块设备接口层为文件系统提供了对块设备进行读写的操作接口,其代码在easy-fs/src/block\_dev.rs中。作为easy-fs库的最底层,其声明了一个块设备的抽象接口BlockDevice,该trait定义了设备驱动所需要实现的读写接口,如下:

\begin{lstlisting}[language={Rust}, label={code:blockdevice},
	caption={easy-fs/src/block\_dev.rs}]
pub trait BlockDevice: Send + Sync + Any {
	/// Read block from BlockDevice
	/// The function panics when the size of 'buf' is not a multiple of BLOCK_SZ
	fn read_block(\&self, block_id: usize, buf: \&mut [u8]);
	
	/// Write block into the file system.
	/// The function panics when the size of 'buf' is not a multiple of BLOCK_SZ
	fn write_block(\&self, block_id: usize, buf: \&[u8]);
	
	/// We should rewrite the API for K210 since it supports NATIVE multi-block clearing
	fn clear_block(\&self, block_id: usize, num: u8) {
		self.write_block(block_id, \&[num; BLOCK_SZ]);
	}
	
	/// We should rewrite the API for K210 if it supports NATIVE multi-block clearing
	fn clear_mult_block(\&self, block_id: usize, cnt: usize, num: u8) {
		for i in block_id..block_id + cnt {
			self.write_block(i, \&[num; BLOCK_SZ]);
		}
	}
}
\end{lstlisting}

该trait主要实现两个抽象方法:

\begin{itemize}
	\item [$\bullet$]
	read\_block:将编号为block\_id的块从磁盘读入内存中的缓冲区buf;
	\item [$\bullet$]
	write\_block:将内存中的缓冲区buf中的数据写入磁盘编号为block\_id的块。
\end{itemize}

在easy-fs中并没有一个实现了BlockDevice trait的具体类型,因为块设备仅支持以块为单位进行随机读写,而这两个方法需要由具体的块设备驱动来实现。实际上,这是需要由文件系统的使用者(比如操作系统内核或直接测试easy-fs文件系统的easy-fs-fuse应用程序)提供并接入到easy-fs库的。

块设备接口层及其以下的设备驱动层实现后,向上为接下来要介绍的easy-fs库的块缓存层提供这两个方法调用,进行块缓存的管理。换句话说,easy-fs可以访问实现了BlockDevice trait的块设备驱动程序。

同时,Npucore为了实现K210提供的关于用给定字符清空单个或若干个连续磁盘块的API,还对该trait的write\_block方法进行了进一步封装,增加了clear\_block与clear\_mult\_block方法,实现了按照fat32文件系统的磁盘块大小,写入给定字符的功能。至于更复杂的功能,则需要由上层层次进行实现。

Npucore目前实现了对SD卡的驱动实现。在os/src/drivers/block/sdcard.rs中对上述trait进行了实现,通过K210依赖中提供的GPIO端口SPI串口底层支持,可以对SD卡直接进行操作。具体代码如下:

\begin{lstlisting}[language={Rust}, label={code:blockdevice},
	caption={os/src/drivers/block/sdcard.rs}]
impl BlockDevice for SDCardWrapper {
	fn read_block(&self, block_id: usize, buf: &mut [u8]) {
		let lock = self.0.lock();
		let mut result = lock.read_sector(buf, block_id as u32);
		let mut cont_cnt = 0;
		while result.is_err() {
			if cont_cnt >= 0 {
				log::error!("[sdcard] read_sector(buf, {}) error. Retrying...", block_id);
				result = lock.read_sector(buf, block_id as u32);
			}
			cont_cnt += 1;
			if cont_cnt >= 5 {
				log::error!(
				"[sdcard] read_sector(buf[{}], {}) error exceeded contineous retry count, waiting...",
				buf.len(),
				block_id
				);
				Self::wait_for_one_sec();
				if lock.read_sector(buf, block_id as u32).is_err() {
					lock.init();
					Self::wait_for_one_sec();
				} else {
					break;
				}
				cont_cnt = 0;
			}
		}
	}
	fn write_block(&self, block_id: usize, buf: &[u8]) {
		let lock = self.0.lock();
		let mut result = lock.write_sector(buf, block_id as u32);
		let mut cont_cnt = 0;
		while result.is_err() {
			if cont_cnt >= 0 {
				log::error!(
				"[sdcard] write_sector(buf, {}) error. Retrying...",
				block_id
				);
				result = lock.write_sector(buf, block_id as u32);
			}
			cont_cnt += 1;
			if cont_cnt >= 5 {
				log::error!(
				"[sdcard] write_sector(buf[{}], {}) error exceeded contineous retry count, waiting...",
				buf.len(),
				block_id
				);
				Self::wait_for_one_sec();
				if lock.write_sector(buf, block_id as u32).is_err() {
					lock.init();
					Self::wait_for_one_sec();
				} else {
					break;
				}
				cont_cnt = 0;
			}
		}
	}
}
\end{lstlisting}

这两个方法实现了对块号为block\_id的SD存储区进行直接读写。注意,还设置了在I/O故障的情况的读写最大重试次数为5。
\section{buffer cache层}
文件系统中的块缓存层(Buffer Cache)扮演着非常重要的角色,主要目的是提高文件系统性能和效率。以下是块缓存层的几个关键作用:

1.减少磁盘I/O操作:块缓存层通过在内存中缓存最近或频繁访问的磁盘块来减少对磁盘的直接访问。这是因为内存访问速度远快于磁盘。

2.加速数据访问:当应用程序请求数据时,操作系统首先在块缓存中查找。如果找到所需的数据块,就可以直接从内存中读取,而无需等待磁盘的慢速读取。

3.合并写操作:写操作经常被先写入到块缓存中,并在稍后的某个时刻一起写入磁盘。这种方式可以合并多个小的写操作为一个大的磁盘I/O操作,提高写入效率。

4.减少重复数据读取:如果多个应用或进程读取相同的数据块,这些数据块只需要从磁盘读取一次,之后可以从块缓存中获取,减少了重复的磁盘访问。

5.提供一致性和同步机制:块缓存可以保证文件系统中数据的一致性。例如,在崩溃恢复期间,它可以帮助恢复未完成的写操作,确保文件系统的完整性。

6.支持异步写操作:数据可以先缓存在块缓存中,然后异步地写入磁盘。这允许程序继续执行而不必等待磁盘I/O操作完成。

7.减轻磁盘的负载:通过减少对磁盘的访问次数,块缓存有助于降低磁盘的工作负载,延长其使用寿命。
\\[10pt]

Npucore中对于文件系统同样实现了buff cache层,其主要定义于easy-fs/src/block_cache.rs文件中。

首先是对于整个块缓存的定义:
\begin{lstlisting}[language={Rust},caption={Cache缓存对象基本操作}]
    pub trait Cache {

    fn read<T, V>(&self, offset: usize, f: impl FnOnce(&T) -> V) -> V;

    fn modify<T, V>(&mut self, offset: usize, f: impl FnOnce(\&mut T) -> V) -> V;

    fn sync(&self, _block_ids: Vec<usize>, _block_device: &Arc<dyn BlockDevice>) {}
}
\end{lstlisting}
这个 trait 定义了一个缓存对象的基本操作。

read<T, V>(\&self, offset: usize, f: impl FnOnce(\&T) -> V) -> V: 允许以只读方式访问缓存。它接受一个偏移量和一个闭包 f,闭包作用于缓存中的数据,并返回一个结果。这是一个泛型函数,支持不同类型的数据。

modify<T, V>(\&mut self, offset: usize, f: impl FnOnce(\&mut T) -> V) -> V: 类似于 read,但用于可变地修改缓存。它也接受一个偏移量和一个闭包,但闭包可以修改数据。


sync(\&self, _block_ids: Vec<usize>, _block_device: \&Arc<dyn BlockDevice>): 将缓存中的数据写回到磁盘。接受一个块ID列表和一个指向块设备的引用。
\\[10pt]

为了更好的管理缓存空间,我们需要实现一个CacheManager对于缓存进行更好的管理,第一点是可以统一管理不同类型的缓存,如块缓存和页表缓存,其次可以实现资源管理和性能优化,npucore中对oom的具体实现的是综合了引用计数和优先级两种策略的方法,这有助于提高系统的整体性能,减少磁盘I/O操作。

\begin{lstlisting}[language={Rust},caption={CacheManager的实现}]
    pub trait CacheManager {
    /// The constant to mark the cache size.
    const CACHE_SZ: usize;

    type CacheType: Cache;

    /// Constructor to the struct.
    fn new() -> Self
    where
        Self: Sized;
    /// Tell cache manager to write back cache and release memory
    /// Argument:
    /// 'neighbor': A closure to get block ids when cache miss.
    /// 'block_device': The pointer to the block_device.
    /// Return Value:
    /// Number of caches freed
    fn oom<FUNC>(
        &self,
        _neighbor: FUNC,
        _block_device: &Arc<dyn BlockDevice>
    ) -> usize
    where
        FUNC: Fn(usize) -> Vec<usize>
    {
        unreachable!()
    }
    /// When file size changed, we should notify cache manager to drop some cache
    /// Argument:
    /// 'new_size': File's new size
    fn notify_new_size(
        &self,
        _new_size: usize
    ) {
        unreachable!()
    }
}
\end{lstlisting}
这个 trait 定义了缓存管理器的基本功能。

new() -> Self: 静态方法,用于创建一个新的缓存管理器实例。

oom<FUNC>(\&self, _neighbor: FUNC, _block_device: \&Arc<dyn BlockDevice>) -> usize: 当系统内存不足时,释放一些缓存。接受一个邻居闭包来获取块ID,以及一个指向块设备的引用。

notify_new_size(\&self, _new_size: usize): 当文件大小变化时,通知缓存管理器释放⼀些缓存块
\\[10pt]

然后就是对于缓存管理的分别具体实现,因为我们有块缓存和页缓存两种缓存块,所以我们这里分别对CacheManager进一步封装实现了BlockCacheManager和PageManager:

\begin{lstlisting}[language={Rust},caption={BCM和CM}]
    pub trait BlockCacheManager: CacheManager {
    /// Try to get the block cache and return `None` if not found.
    /// Argument:
    /// 'block_id': The demanded block id(for block cache).
    /// 'inner_cache_id': The ordinal number of the cache inside the file(for page cache).
    /// Return Value:
    /// If found, return Some(pointer to cache)
    /// otherwise, return None
    fn try_get_block_cache(
        &self,
        block_id: usize,
    ) -> Option<Arc<Mutex<Self::CacheType>>>;

    /// Attempt to get block cache from the cache.
    /// If failed, the manager should try to copy the block from sdcard.
    /// Argument:
    /// 'block_id': The demanded block id(for block cache).
    /// 'inner_cache_id': The ordinal number of the cache inside the file(for page cache).
    /// 'neighbor': A closure to get block ids when cache miss.
    /// 'block_device': The pointer to the block_device.
    /// Return Value:
    /// Pointer to cache
    fn get_block_cache<FUNC>(
        &self,
        block_id: usize,
        block_device: &Arc<dyn BlockDevice>,
    ) -> Arc<Mutex<Self::CacheType>>;
}
\end{lstlisting}

try_get_block_cache 该函数尝试获取传⼊参数 block_id (磁盘编号)和 inner_cache_id (管理器内部编号)对应的 Cache。  

get_block_cache 该函数获取传⼊参数 block_id (磁盘编号)和 inner_cache_id (管理器内部编号)对应的 Cache 。如果内存中没有,则会通过 block_device 从块设备中读⼊对应的内容:

\begin{lstlisting}[language={Rust},caption={PageCacheManager}]
    pub trait PageCacheManager: CacheManager {
    /// Try to get the block cache and return `None` if not found.
    /// Argument:
    /// 'block_id': The demanded block id(for block cache).
    /// 'inner_cache_id': The ordinal number of the cache inside the file(for page cache).
    /// Return Value:
    /// If found, return Some(pointer to cache)
    /// otherwise, return None
    fn try_get_page_cache(
        &self,
        inner_cache_id: usize,
    ) -> Option<Arc<Mutex<Self::CacheType>>>;

    /// Attempt to get block cache from the cache.
    /// If failed, the manager should try to copy the block from sdcard.
    /// Argument:
    /// 'block_id': The demanded block id(for block cache).
    /// 'inner_cache_id': The ordinal number of the cache inside the file(for page cache).
    /// 'neighbor': A closure to get block ids when cache miss.
    /// 'block_device': The pointer to the block_device.
    /// Return Value:
    /// Pointer to cache
    fn get_page_cache<FUNC>(
        &self,
        inner_id: usize,
        neighbor: FUNC,
        block_device: &Arc<dyn BlockDevice>,
    ) -> Arc<Mutex<Self::CacheType>>
    where
        FUNC: Fn() -> Vec<usize>;
}
\end{lstlisting}
其中两个方法和BlockCacheManager中两个方法的实现是一样的,唯一区别在于两个管理的cache(大小)不同。



\section{索引节点层}
\section{目录和路径名}
\section{文件描述符层}
\section{文件相关系统调用}
npucore实现了许多POSIX规定的系统调用,本章主要介绍几个与文件相关的系统调用,包括文件的创建、打开、关闭、读写、删除等。
\subsection{write}
对于文件的读是最常使用的系统调用之一,包括输入输出都是基于终端来实现的(终端本身也可以当作一个文件)。
下面我们来介绍npucore中write的实现。

write系统调用的函数签名如下:
\begin{lstlisting}[language={C}]
int write(int fd, const void *buf, size_t count);
\end{lstlisting}
write系统调用将buf中的count个字节写入文件描述符fd所指向的文件中。成功时返回写入的字节数,失败时返回-1。

NPUCore中的代码实现如下:
\begin{lstlisting}[language={Rust}, label={lst:write}]
pub fn sys_write(fd: usize, buf: usize, count: usize) -> isize {
    let task = current_task().unwrap();
    let fd_table = task.files.lock();
    let file_descriptor = match fd_table.get_ref(fd) {
        Ok(file_descriptor) => file_descriptor,
        Err(errno) => return errno,
    };
    if !file_descriptor.writable() {
        return EBADF;
    }
    let token = task.get_user_token();
    file_descriptor.write_user(
        None,
        UserBuffer::new({
            match translated_byte_buffer(token, buf as *const u8, count) {
                Ok(buffer) => buffer,
                Err(errno) => return errno,
            }
        }),
    ) as isize
}
\end{lstlisting}
其逐代码块的解释如下:
current_task().unwrap() 获取当前任务(线程)的引用。
task.files.lock() 获取当前任务的文件表的锁,以确保并发访问的同步。
通过文件描述符 fd 从文件表中获取文件描述符的引用,使用 fd_table.get_ref(fd)。如果获取失败,则返回相应的错误号。
检查文件是否可写,如果不可写则返回 EBADF 错误号。
获取当前任务的用户令牌,通常用于在用户空间和内核空间之间传递数据。
将用户空间缓冲区的数据写入文件描述符。这里使用了 translated_byte_buffer 函数来将用户空间的字节缓冲区翻译成内核地址空间,确保正确的内存访问。
最后,将写入的字节数作为 isize 类型返回。

如果从代码的逻辑层次来分析,其调用过程如下:

sys_writet。获取当前进程的引用,然后获取文件描述符表,然后根据该表与参数fd获取对相应文件描述符的引用。这之后检查文件是否可写并获取user_token,这些操作都完成之后,以buffer(缓冲区)与count(缓冲区长度)两个参数转换为buffer,传入FileDescriptor::write_user中。

FileDescriptor::write_user。直接调用<dyn File>的write_user函数

OSInode::write_user。如果参数中offset为none(比如由write系统调用时就为none),则先获取文件偏移的锁、inode(OSInode中inner)的锁,然后看是否设置了追加标志,如果是,则将文件偏移放到最后(即从文件末尾开始写入)。然后调用OSInode.inner的write_at_block_cache_lock(即Inode::write_at_block_cache_lock)。完成之后,修改记录的文件偏移,返回写入的字节数。

Inode::write_at_block_cache_lock。内部有一个loop,计算块的末尾、写入并更新大小、移动到下一个块。到这里,写入过程就结束了。然而此时数据仍然留在内存中,还没有被持久化。被持久化的过程是另一个独立的过程,将在IO设备章节详细介绍。在这里做一个简要讲解:当满足某个条件时(比如缓冲区接近满溢),应当调用OSInode的oom函数,将满足条件的缓冲块落盘。


\subsection{read}
read函数用于从文件描述符中读取数据,并将数据存储到缓冲区中。这个系统调用在应用程序与操作系统内核之间搭建了一座桥梁,允许应用程序从指定的文件描述符中读取数据。以下是read系统调用的工作流程。

调用请求:应用程序通过执行read系统调用,并传入相应的文件描述符、缓冲区地址和要读取的字节数,发起读取请求。

参数验证:内核首先验证参数的有效性,包括文件描述符的合法性和缓冲区地址的可访问性。

读取操作:内核根据文件描述符找到相应的文件或数据流,并从中读取指定数量的字节。这个过程可能涉及文件系统的访问、网络操作或设备交互。

数据传输:读取的数据被存储在内核缓冲区中,然后被复制到用户提供的缓冲区。这个过程涉及从内核空间到用户空间的数据传输。

返回结果:read调用最终返回读取的字节数。如果到达文件末尾,则返回0。如果发生错误,则返回一个负值,并设置相应的错误码。

阻塞与非阻塞模式:read操作可以在阻塞或非阻塞模式下进行。在阻塞模式下,如果没有可用数据,read调用会阻塞调用进程,直到有数据可读。在非阻塞模式下,如果没有数据可读,read会立即返回,通常是一个错误码。

接下来是在NPUCore中read的具体实现:

\begin{lstlisting}[language=rust]
pub fn sys_read(fd: usize, buf: usize, count: usize) -> isize {
    let task = current_task().unwrap();
    let fd_table = task.files.lock();
    let file_descriptor = match fd_table.get_ref(fd) {
        Ok(file_descriptor) => file_descriptor,
        Err(errno) => return errno,
    };
    // fd is not open for reading
    if !file_descriptor.readable() {
        return EBADF;
    }
    let token = task.get_user_token();
    file_descriptor.read_user(
        None,
        UserBuffer::new({
            match translated_byte_buffer(token, buf as *const u8, count) {
                Ok(buffer) => buffer,
                Err(errno) => return errno,
            }
        }),
    ) as isize
}
\end{lstlisting}

获取当前任务:函数首先获取当前正在执行的任务或进程的上下文。

获取文件描述符表:锁定当前任务的文件描述符表,并尝试从中获取与传入的文件描述符fd相对应的文件描述符对象。

错误处理:如果指定的文件描述符不存在或其他错误发生,函数将返回相应的错误码。

检查读取权限:确认获取到的文件描述符是否具有读取权限。如果没有,返回EBADF错误码,表示文件描述符不合法或不支持读取操作。

获取用户令牌:获取与当前任务关联的用户令牌,用于后续的权限验证和内存安全检查。

处理用户缓冲区:使用$translated\_byte\_buffer$函数将用户空间的缓冲区地址和长度转换为内核可以操作的缓冲区对象UserBuffer。这一步骤涉及内存地址转换和错误检查。

执行读取操作:调用文件描述符的$read\_user$方法,从文件或数据流中读取数据到用户提供的缓冲区中。

返回结果:返回从文件描述符中实际读取的字节数。如果读取过程中发生错误,返回相应的错误码。

\subsection{open}
首先,我们分析一下NPUcore文件系统提供给应用的接口,即用户态的sys_open 系统调用。

在读写一个常规文件之前,应用首先需要通过内核提供的 sys_open 系统调用让该文件在进程的文件描述符表中占一项,并得到操作系统的返回值——文件描述符,即文件关联的表项在文件描述表中的索引值:
\begin{lstlisting}[language=rust]
// user/src/syscall.rs
/// syscall ID:56
pub fn sys_open(path: &str, flags: u32) -> isize {
	syscall(SYSCALL_OPEN, [path.as_ptr() as usize, flags as usize, 0])
}
\end{lstlisting}

如上所示 sys_open 的功能为打开一个常规文件,并返回可以访问它的文件描述符。参数path描述要打开的文件的文件名(简单起见,文件系统不需要支持目录,所有的文件都放在根目录 / 下),flags 描述打开文件的标志,具体含义下面给出。至于返回值,如果出现了错误则返回 -1,否则返回打开常规文件的文件描述符。可能的错误原因是:文件不存在。

然后我们讲解一下flags,目前我们的内核支持以下几种标志(多种不同标志可能共存):
\begin{itemize}
\item [$\bullet$]
如果 flags 为 0,则表示以只读模式 RDONLY 打开;
\item [$\bullet$]
如果 flags 第 0 位被设置(0x001),表示以只写模式 WRONLY 打开;
\item [$\bullet$]
如果 flags 第 1 位被设置(0x002),表示既可读又可写 RDWR ;
\item [$\bullet$]
如果 flags 第 9 位被设置(0x200),表示允许创建文件 CREATE ,在找不到该文件的时候应创建文件;如果该文件已经存在则应该将该文件的大小归零;
\item [$\bullet$]
如果 flags 第 10 位被设置(0x400),则在打开文件的时候应该清空文件的内容并将该文件的大小归零,也即 TRUNC 。
\end{itemize}

注意 flags 里面的权限设置只能控制进程对本次打开的文件的访问。一般情况下,在打开文件的时候首先需要经过文件系统的权限检查,比如一个文件自身不允许写入,那么进程自然也就不能以 WRONLY 或 RDWR 标志打开文件。但在我们简化版的文件系统中文件不进行权限设置,这一步就可以绕过。

在用户库 user_lib 中,我们将该系统调用封装为 open 接口:
\begin{lstlisting}[language=rust]
	// user/src/lib.rs
	
	bitflags! {
		pub struct OpenFlags: u32 {
			const RDONLY = 0;
			const WRONLY = 1 << 0;
			const RDWR = 1 << 1;
			const CREATE = 1 << 9;
			const TRUNC = 1 << 10;
		}
	}
	
	pub fn open(path: &str, flags: OpenFlags) -> isize {
		sys_open(path, flags.bits)
	}
\end{lstlisting}

如上,借助 bitflags! 宏我们将一个 u32 的 flags 包装为一个 OpenFlags 结构体更易使用,它的 bits 字段可以将自身转回 u32 ,它也会被传给 sys_open。
\begin{lstlisting}[language=rust]
	// user/src/syscall.rs
	/// syscall ID:56
	pub fn sys_open(path: &str, flags: u32) -> isize {
		syscall(SYSCALL_OPEN, [path.as_ptr() as usize, flags as usize, 0])
	}
\end{lstlisting}

如上,sys_open 传给内核的参数只有待打开文件的文件名字符串的起始地址(和之前一样,我们需要保证该字符串以 $\backslash$0 结尾)还有标志位。由于每个通用寄存器为 64 位,我们需要先将 u32 的 flags 转换为 usize 。

接下来,我们分析一下 sys_open 在内核中的实现。


\subsection{close}
首先,我们分析一下NPUcore文件系统提供给应用的接口,即用户态的 sys_close 系统调用。

在打开文件,对文件完成了读写操作后,还需要关闭文件,这样才让进程释放被这个文件占用的内核资源。 close 的调用参数是文件描述符,当文件被关闭后,该文件在内核中的资源会被释放,文件描述符会被回收。这样,进程就不能继续使用该文件描述符进行文件读写了。
\begin{lstlisting}[language=rust]
	// usr/src/lib.rs
	pub fn close(fd: usize) -> isize { sys_close(fd) }
	
	// user/src/syscall.rs
	const SYSCALL_CLOSE: usize = 57;
	
	pub fn sys_close(fd: usize) -> isize {
		syscall(SYSCALL_CLOSE, [fd, 0, 0])
	}
\end{lstlisting}

如上,sys_close 的功能是当前进程关闭一个文件,参数 fd 表示要关闭的文件的文件描述符,如果成功关闭则返回 0 ,否则返回 -1 。可能的出错原因:传入的文件描述符并不对应一个打开的文件。

接下来,我们分析一下 sys_open 在内核中的实现。

关闭文件的系统调用 sys_close 实现非常简单,我们只需将进程控制块中的文件描述符表对应的一项改为 None 代表它已经空闲即可,同时这也会导致内层的引用计数类型 Arc 被销毁,会减少一个文件的引用计数,当引用计数减少到 0 之后文件所占用的资源就会被自动回收。
\begin{lstlisting}[language=rust]
	pub fn sys_close(fd: usize) -> isize {
		info!("[sys_close] fd: {}", fd);
		let task = current_task().unwrap();
		let mut fd_table = task.files.lock();
		match fd_table.remove(fd) {
			Ok(_) => SUCCESS,
			Err(errno) => errno,
		}
	}
\end{lstlisting}

\subsection{fstat,fstatat}
NPUcore为用户提供了fstat和fstatat两个系统调用用于获取文件信息。相比之下,fstatat的功能更加完善和丰富,实现也更加复杂,这里先从fstatat开始分析。

fstatat和fstat的函数原型如下:

\begin{lstlisting}[language={Rust}, label={code:fstat,fstatat},
	caption={fstat,fstatat}]
	pub fn sys_fstatat(dirfd: usize, path: *const u8, buf: *mut u8, flags: u32) -> isize {}
	pub fn sys_fstat(fd: usize, statbuf: *mut u8) -> isize {}
		
\end{lstlisting}

从参数可以看出,statat可以按照路劲访问文件信息,而fstat则只能通过进程的文件描述符表访问。这是两者的主要区别。
对于fstat,首先进行的是获取当前进程信息以及解析路径:

\begin{lstlisting}[language={Rust}, label={code:fstatat part1},
	caption={fstatat_part1}]
    let token = current_user_token();
let path = match translated_str(token, path) {
	Ok(path) => path,
	Err(errno) => return errno,
};
let flags = match FstatatFlags::from_bits(flags) {
	Some(flags) => flags,
	None => {
		warn!("[sys_fstatat] unknown flags");
		return EINVAL;
	}
};

info!(
"[sys_fstatat] dirfd: {}, path: {:?}, flags: {:?}",
dirfd as isize, path, flags,
);

let task = current_task().unwrap();
	
\end{lstlisting}

flags这里只是进行了解析,但是在之后实现中没有涉及,这里也不过多叙述。在这里获取了当前运行的进程,并且将切片类型的path映射到当前地址空间,并且转换成string类型。

\begin{lstlisting}[language={Rust}, label={code:fstatat part2},
	caption={fstatat_part2}]
let file_descriptor = match dirfd {
	AT_FDCWD => task.fs.lock().working_inode.as_ref().clone(),
	fd => {
		let fd_table = task.files.lock();
		match fd_table.get_ref(fd) {
			Ok(file_descriptor) => file_descriptor.clone(),
			Err(errno) => return errno,
		}
	}
};

match file_descriptor.open(&path, OpenFlags::O_RDONLY, false) {
	Ok(file_descriptor) => {
		copy_to_user(token, &file_descriptor.get_stat(), buf as *mut Stat);
		SUCCESS
	}
	Err(errno) => errno,
}
\end{lstlisting}

这里先获取当前目录的文件描述符,之后调用open函数通过路径打开文件。如果文件打开成功,则调用copy_to_user函数将文件信息内容拷贝到buf中,之后函数返回。

对于文件信息的内容,NPUcore中使用结构体Stat表示,具体代码如下:
\begin{lstlisting}[language={Rust}, label={code:Stat},
	caption={Stat}]
#[derive(Clone, Copy, Debug)]
#[repr(C)]
/// Store the file attributes from a supported file.
pub struct Stat {
	/// ID of device containing file
	st_dev: u64,
	/// Inode number
	st_ino: u64,
	/// File type and mode   
	st_mode: u32,
	/// Number of hard links
	st_nlink: u32,
	/// User ID of the file's owner.
	st_uid: u32,
	/// Group ID of the file's group.
	st_gid: u32,
	/// Device ID (if special file)
	st_rdev: u64,
	__pad: u64,
	/// Size of file, in bytes.
	st_size: i64,
	/// Optimal block size for I/O.
	st_blksize: u32,
	__pad2: i32,
	/// Number 512-byte blocks allocated.
	st_blocks: u64,
	/// Backward compatibility. Used for time of last access.
	st_atime: TimeSpec,
	/// Time of last modification.
	st_mtime: TimeSpec,
	/// Time of last status change.
	st_ctime: TimeSpec,
	__unused: u64,
}
\end{lstlisting}

结构体每一个字段的含义已经给出。

对于系统调用fstat,其功能相较于fstatat更为简单。仅仅只有通过进程的文件描述符表获取文件描述符和相关信息的拷贝。具体代码如下:

\begin{lstlisting}[language={Rust}, label={code:fstat},
	caption={fstat}]
pub fn sys_fstat(fd: usize, statbuf: *mut u8) -> isize {
	let task = current_task().unwrap();
	let token = task.get_user_token();
	
	info!("[sys_fstat] fd: {}", fd);
	let file_descriptor = match fd {
		AT_FDCWD => task.fs.lock().working_inode.as_ref().clone(),
		fd => {
			let fd_table = task.files.lock();
			match fd_table.get_ref(fd) {
				Ok(file_descriptor) => file_descriptor.clone(),
				Err(errno) => return errno,
			}
		}
	};
	copy_to_user(token, &file_descriptor.get_stat(), statbuf as *mut Stat);
	SUCCESS
}
\end{lstlisting}

由于其本身传入参数的限制,fstat函数只能获取当前进程的文件信息。
\section{虚拟文件系统及接口}
\section{文件共享与PIPE}
\subsection{软链接与硬链接}
\subsection{管道机制}

\chapter{嵌入式硬件平台简介及内核运行}
\section{基于嵌入式硬件的操作系统开发流程}
\section{基于RISC-V64的NPUcore内核运行}
\subsection{星光二代开发板简介}
\textbf{昉·星光 2 简介}

\begin{figure}[ht]
	\centering
	\includegraphics[width=0.5\linewidth]{figures/08-02-昉·星光 2.jpg}
	\caption{昉·星光 2}
\end{figure}

昉·星光 2 是一款集成了GPU的高性能RISC-V单板计算机。它搭载四核64位RV64GC ISA的芯片平台(SoC),工作频率最高可达1.5 GHz,开源的昉·星光 2具有强大的软件适配性,官方适配Debian操作系统,同时通过社区合作适配各种Linux发行版,包括Ubuntu、OpenSUSE、OpenKylin、OpenEuler、Deepin等,及在这些操作系统上运行的各类软件。接下来我们的上板实验就在昉·星光 2 上进行。

\textbf{硬件准备}

在进行上板实验前,首先需要准备以下硬件设备:

{
	\ding{172} 昉·星光 2 开发板
	
	\centering
	\includegraphics[width=0.5\linewidth]{figures/08-02-昉·星光 2.jpg}
	
	\raggedright
	\setlength{\parindent}{2em}
	\ding{173} 开发板充电线(type-c 充电线)
	
	\centering
	\includegraphics[width=0.5\linewidth]{figures/08-02-开发板充电线.jpg}
	
	\raggedright
	\setlength{\parindent}{2em}
	\ding{174} USB TO TTL串口通信转化器
	
	\centering
	\includegraphics[width=0.5\linewidth]{figures/08-02-USB TO TTL 串口通信转化器.jpg}
	
	\raggedright
	\setlength{\parindent}{2em}
	\ding{175} 母对母杜邦线
	
	\centering
	\includegraphics[width=0.5\linewidth]{figures/08-02-杜邦线.jpg}
	
	\raggedright
	\setlength{\parindent}{2em}
	\ding{176} RJ45 网线
	
	\centering
	\includegraphics[width=0.5\linewidth]{figures/08-03-RJ45网线.jpg}
	
	\raggedright
	\setlength{\parindent}{2em}
	\ding{177} 拓展坞
	
	\centering
	\includegraphics[width=0.5\linewidth]{figures/08-02-拓展坞.jpg}
	
	\raggedright
	\setlength{\parindent}{2em}
	拓展坞需要带至少一个网线接口,两个USB接口。如果笔记本电脑自带网线接口,并且电脑上的两个USB接口都空闲则不需要该拓展坞。
}

\textbf{昉·星光 2 引脚介绍}

查阅星光二代官网手册,得到星光二代引脚分布如图所示,
\begin{figure}[ht]
	\centering
	\includegraphics[width=0.8\textwidth]{figures/08-02-星光二代pin图.jpg}
	\caption{星光二代pin图}
\end{figure}

我们只需要进行串口连接,因此只需要关注红框内的 GND,UART TX,UART RX 串口即可,即图中的6、8、10号接口。

\textbf{硬件连接}

{
	\ding{172} 串口连接
	
	\raggedright
	\setlength{\parindent}{2em}
	首先,使用杜邦线连接串口通信转换器的TXD,RXD,GND 接口,仅连接这三个接口即可,剩余悬空,如下图所示。
	
	\centering
	\includegraphics[width=0.58\linewidth]{figures/08-02-杜邦线连接串口转换器.jpg}
	
	\raggedright
	\setlength{\parindent}{2em}
	
	然后,使用杜邦线连接昉·星光 2 单板计算机,连接到图中序号\ding{175}的引脚处。
	
	
	\centering
	\includegraphics[width=0.58\linewidth]{figures/08-02-赛昉结构.jpg}
	
	\raggedright
	\setlength{\parindent}{2em}
	
	连接好后如下图所示
	
	\centering
	\includegraphics[width=0.58\linewidth]{figures/08-02-杜邦线连接赛昉.jpg}
	
	\raggedright
	\setlength{\parindent}{2em}
	
	\ding{173} 电源连接
	
	将电源线的 type-c 接口接入到图中插口\ding{179}处即可
	
	\centering
	\includegraphics[width=0.58\linewidth]{figures/08-02-赛昉结构.jpg}
	
	\raggedright
	\setlength{\parindent}{2em}
	
	连接好后如下图所示
	
	\centering
	\includegraphics[width=0.58\linewidth]{figures/08-02-连接电源线.jpg}
	
	\raggedright
	\setlength{\parindent}{2em}
	\ding{172} 网线连接
	
	将电源线的 type-c 接口接入到图中插口\ding{185}处的任一插口即可
	
	\centering
	\includegraphics[width=0.58\linewidth]{figures/08-02-赛昉结构.jpg}
	
	
	\raggedright
	\setlength{\parindent}{2em}
	
	连接好后如下图所示
	
	\centering
	\includegraphics[width=0.58\linewidth]{figures/08-02-连接网线.jpg}
	
}

至此连接完毕,接下来只需将拓展坞连接到电脑即可将开发板连接到电脑,实现插拔方便快捷。

\begin{figure}[htb]
	\centering
	\includegraphics[width=0.58\textwidth]{figures/08-02-连接完毕总览.jpg}
	\caption{
		连接完毕全貌
	}
	\label{fig:pipe}
\end{figure}


\subsection{面向RISC-V64的开发环境搭建与内核运行}

\begin{enumerate}
	\item \textbf{编译操作系统内核}
	
	从github上获取NPUcore源码后,使用make命令可以编译内核。编译完成后,会生成os.bin文件,这个便是烧入开发板的内核镜像。由于NPUcore直接使用的是risc-v64的交叉编译工具,该操作可以直接生成risc-v64内核。
	
	\item \textbf{串口连接开发板}
	
	目前使用的星光二号开发板需要使用串口实现交互。可使用的串口工具种类较多,这里以windows操作系统为例,使用
	MobaXterm 工具连接串口。
	
	首先将连接好的开发板插入电脑,开发板指示灯亮起(红色),串口转化器指示灯亮起(红色常亮,有蓝色灯闪烁),说明连接正常。
	~~
	
	\centering
	\includegraphics[width=0.58\linewidth]{figures/08-02-星光二号线路.jpg}
	\raggedright
	
	
	打开MobaXterm软件,显示界面如下图:
	
	\centering
	\includegraphics[width=0.58\linewidth]{figures/08-02-MobaXterm界面.jpg}
	\raggedright
	
	点击左上角session:
	
	\centering
	\includegraphics[width=0.58\linewidth]{figures/08-02-session.jpg}
	\raggedright
	
	弹出的窗口中选择“serial”,表示使用新建一个串口会话:
	
	\centering
	\includegraphics[width=0.58\linewidth]{figures/08-02-串口设置.jpg}
	\raggedright
	
	
	需要设置串口端口和波特率。这里波特率选择115200bps,端口号在开发板正确连接后会自动识别,选择之后点击OK,进入一个命令行:
	
	\centering
	\includegraphics[width=0.58\linewidth]{figures/08-02-串口命令行.jpg}
	\raggedright
	
	切换为英文输入法,输入任意字符,回车,观察到开发板终端的输出:
	
	\centering
	\includegraphics[width=0.58\linewidth]{figures/08-02-串口输出.jpg}
	\raggedright
	
	此时开发板和终端已通过串口连接,终端可以使用命令与开发板进行交互。
	
	\item \textbf{使用tftp协议传输内核镜像}
	
	由于串口用于交互,所以需要通过tftp网络传输内核到开发板。同样tftp网络可以使用多种软件搭建,这里由于
	MobaXterm支持tftp功能,故用MobaXterm举例如何搭建tftp服务器以及如何传输内核文件。
	
	保持串口界面,点击界面左上角的“Server”选项,弹出如下窗口:
	
	\centering
	\includegraphics[width=0.58\linewidth]{figures/08-02-tftp服务器搭建.jpg}
	\raggedright
	
	在右方“Root directory”可以设置tftp服务器根地址,也就是开发板可以访问到的目录路径。左上角点击运行按钮可以运行tftp服务:
	
	这时tftp服务器搭建成功,但是需要设置IP地址,将开发板和服务器设置为相同子网,才可以使用tftp服务。
	
	如果终端使用的是自动获取IP地址,则可以通过以下命令查看IP地址:
	
	ifconfig
	
	如果使用的是手动设置的IP地址,以上方法也可以查看。如果需要检查连通性,可使用ping命令,例如如果查到终端IP地址为168.254.242.234,则可以使用如下命令:
	
	ping 168.254.242.234
	
	在“StarFive \#”终端提示符下,输入以下命令设置开发板IP地址:
	
	setenv ipaddr 168.254.242.233
	
	此处需要和终端IP地址不同但是需要和终端IP在同一个局域网中。
	
	执行以下命令设置服务器IP地址:
	
	setenv serverip 168.254.242.234
	
	确认os.bin已经在tftp服务器根目录下,并且tftp服务是打开状态,输入下面命令:
	
	tftpboot 0x80200000 os.bin
	
	便可以将内核镜像下载到开发板上,并设置起始地址为0x80200000。正常执行结果如下:
	
	\item \textbf{运行内核}
	
	上述过程已经将内核镜像下载到开发板上,而且设置了起始地址为0x80200000。接下来输入以下命令便可以开始运行内核:
	
	go 0x80200000
	
	运行结果如下:
	
	
	\centering
	\includegraphics[width=0.58\linewidth]{figures/08-02-内核运行.jpg}
	\raggedright
	
	自此内核启动完成。
	
	
	
	
\end{enumerate}
\section{基于LoongArch的NPUcore内核运行}

\chapter{编写面向POSIX标准的系统调用}
\section{POSIX标准简介}
%\chapter{操作系统内核构建概述}
%\section{什么是操作系统}
%\subsection{体系结构}

\section{Busybox简介}
\subsection{简单介绍}

\begin{table}[h!]
	\begin{center}
		\caption{Busybox源码目录结构}
		\begin{tabular}{c|c|c} % <-- Alignments: 1st column left, 2nd middle and 3rd right, with vertical lines in between
			\textbf{序号} & \textbf{目录名称} & \textbf{功能说明}\\
			\hline
			1 & applets & 实现applets框架的文件,目录中包含了几个main0的文件。\\
			2 & applets sh & 此目录包含了几个作为shel脚本实现的applet示例。\\
			3 & arch & 包含用于不同体系架构的makefile文件。\\
			4 & archival & 与压缩相关命令的实现源文件。\\
			5 & configs & Busybox自带的默认配置文件。\\
			6 & console-tools & 与控制台相关的一些命令。\\
			7 & coreutils & 常用的一些核心命令。例如chgrp、m等。\\
			8 & debianutils & 针对Debian的套件。\\
			9 & e2fsprogs & 针对Linux Ext2 FS prog的命令,例chattr、 Isattr。\\
			10 & editors & 常用的编辑命令,例如diff、vi等。\\
			11 & findutils sh & 用于查找的命令。\\
			12 & include & Busybox项目的头文件。\\
			13 & init & init进程的实现源码目录。\\
			14 & klibc-utils & klibc命令套件。\\
			15 & libbb & 与Busybox实现相关的库文件。\\
			16 & libpwdgrp & libpwdgrp相关的命令。\\
			17 & loginutils & 与用户管理相关的命令。\\
			18 & mailutils & 与mail相关的命令套件。\\
			19 & miscutils & 该文件下是一些杂项命令,针对特定应用场景。\\
			20 & modutils & 与模块相关的命令。\\
			21 & networking & 与网络相关的命令,例arp。\\
			22 & printutils & Print相关的命令。\\
			23 & procps & 与内存、进程相关的命令。\\
			24 & runit & 与Runit实现相关的命令。\\
			25 & shell & 与shell相关的命令。\\
			26 & sysklogd & 系统日志记录工具相关的命令。\\
			27 & util-inux & Linux下常用的命令,主要与文件系统操作相关的命令。\\
		\end{tabular}
	\end{center}
\end{table}


Busybox是一个针对嵌入式系统的轻量级工具集合,旨在通过整合常用的
Linux命令和服务程序,将它们合并为一个单一的可执行文件。这个项目
最初于1996年诞生,当时嵌入式系统并不像今天这般普及。Busybox的
最初目的是为软盘系统设计的,因为在那个时候,可移动存储介质的容
量十分有限,软盘是主要的存储媒介之一。

Busybox的设计理念非常巧妙。相较于单独存放每个命令所需的存储空间
,Busybox通过将不同命令的共享部分整合到一起,极大地减小了可执行
文件的体积。举例来说,诸如grep和find这样的命令,尽管功能有所差
异,但它们都需要从文件系统中搜索文件,Busybox将这部分代码进行
共享,从而节省了空间。

Busybox的核心特点在于其高度紧凑的特性。它可以包含最基本的系统
命令,例如文件列表显示命令ls和文件内容查看命令cat,同时也能整
合更复杂的程序,如文本搜索命令grep和文件查找命令find,甚至还
能将HTTP服务器整合进同一个软件包中。

对于嵌入式系统来说,存储空间十分宝贵,而Busybox的存在则为这些
系统提供了解决方案。通常情况下,Busybox的可执行文件大小仅约1MB
左右,相比于分散存放各个命令所需的存储空间,这是一个相当节省空间
的选择。用户可以通过建立链接的方式,与传统的命令一样使用Busybox
,只不过它将多个功能整合到一个文件中,从而在嵌入式系统中占用更少
的存储空间。

Busybox源码目录结构图如上,方便以后对Busybox做裁剪的时候参考。

\subsection{工作原理}
Busybox利用了shell传递给C语言main()函数的参数,回想一下C语言
main()函数的定义:int main(int argc,char *argv[])

在main()函数的定义中argc是传递进来的参数个数,argv是一个字符串
数组,数据的每一项都是一个参数内容。其中,argv[0]是从命令行调用
的程序名。下面是一个简单的程序,使用argv[0]确定调用来自哪个程序
。

\begin{lstlisting}[language=Rust]
	//test.c
	#include <stdio.h>
	/*定义主函数*/
	int main(int argc,char *argv[])
	int i;
	for(i=0;i<argc ;i++){
		//for循环语句
		printf("argv[%d]=%s\n",i,argv[i]);//打印程序参数内容
	}
	return 0;
}
\end{lstlisting}

调用这个程序会显示所调用的第一个参数是该程序的名字。可以对这个可执行程序重新进行命名,此时再调用就会得到该程序的新名字。另外,可以创建一个到可执行程序的符号链接,在执行这个符号链接时,就可以看到这个符号链接的名字。

\begin{lstlisting}[language=Rust]
$  gcc -Wall -o test test.c
$  ./test argl arg2
argv[0]=./test
argv[1]=arg1
argv[2]=arg2

$  mv test newtest
$  ./newtest argl
argv[0]=./newtest
argv[1]=arg1

$  ln -s newtest linktest
$  ./linktest arg
argv[0]=./linktest
argv[1]=arg
\end{lstlisting}

Busybox使用符号链接屏蔽了程序调用细节。从用户的角度看,使用Busybox与使用传统的命令效果是相同的。Busybox为其包含的每个系统程序都建立了类似的符号链接。当用户使用符号链接调用Busybox的时候,Busybox通过argv[0]参数调用对应的功能函数。

\subsection{使用方法}
Busybox 的编译过程与Linux内核的编译类似。

Busybox的使用有三种方式:

Busybox后直接跟命令,如 Busybox ls。

直接将Busybox重命名,如 cp Busybox tar。

创建符号链接,如 ln -s Busybox rm。

以上方法中,第三种方法最方便,但为Busybox中每个命令都创建一个软链接,相当费事,Busybox提供自动方法:Busybox编译成功后,执行make install,则会产生一个\_install目录,其中包含了Busybox及每个命令的软链接
Busybox的使用方法与传统的Unix工具类似,通常的语法格式为:
Busybox [选项] [命令] [参数]。

Busybox的命令和参数根据具体的工具而定,可以通过以下方式获取帮助信息:
Busybox --help

\subsubsection{Busybox安装}
首先进入Busybox官网https://www.busybox.net/,选择所需版本下载。

\begin{figure}[H]
\centering
\includegraphics[width=17cm,height=10cm]{figures/09-02-Busybox安装1.png}
\caption{09-02-Busybox官网界面}
\end{figure}  

接下来将右击解压也可以使用命令行解压。

\begin{figure}[H]
\centering
\includegraphics[width=10cm,height=6cm]{figures/09-02-Busybox安装2.png}
\caption{Busybox安装包解压}
\end{figure}  

进入解压后的目录

\begin{lstlisting}[language=Rust]
make defconfig  //使用默认配置,让Busybox包含常用命令和工具
make menuconfig  //在上述基础上,自己更改配置
\end{lstlisting}

\begin{figure}[H]
\centering
\includegraphics[width=14cm,height=8cm]{figures/09-02-Busybox安装3.png}
\caption{Busybox配置界面}
\end{figure}  

BusyBox Setting->Build Options->[ 选]Build Busybox as a static binary (no shared libs)

Shells->chose your default shell(ash):

BusyBox Setting->[*]Don’t use/usr(否则Busybox会安装到ubuntu的/usr下,会覆盖原系统原有的命令)

Coreutils—>sync

Linux System Utilities—>nsenter

Linux System Utilities—>Support mounting NFS filesystems(网络文件系统)

Networking Utilities—>inetd(超级服务器)

Busybox settings ->build options ->build with large file support

编译和安装Busybox

\begin{lstlisting}[language=Rust]
make
make install
\end{lstlisting}

当出现下图所示时就表明已经安装完成了。

\begin{figure}[H]
\centering
\includegraphics[width=14cm,height=8cm]{figures/09-02-Busybox安装4.png}
\caption{Busybox安装成功}
\end{figure}  

执行完成后会发现多了一个\_install目录。

\subsection{常用的命令}
\subsubsection{安装和登录命令}
\textbf{reboot:}

作用:reboot命令的作用是重新启动计算机,它的使用权限是系统管理者。

格式:reboot [-n] [-w] [-d] [-f] [-i]

主要参数:

-n: 在重开机前不做将记忆体资料写回硬盘的动作。

-w: 并不会真的重开机,只是把记录写到/var/log/wtmp文件里。

-d: 不把记录写到/var/log/wtmp文件里(-n这个参数包含了-d)。

-i: 在重开机之前先把所有与网络相关的装置停止。

\textbf{mount:}

作用:mount命令的作用是加载文件系统,它的用权限是超级用户或/etc/fstab中允许的使用者。

格式:mount -a [-fv] [-t vfstype] [-n] [-rw][-F] device dir

主要参数:

-v:显示信息,通常和-f用来除错。

-a:将/etc/fstab中定义的所有文件系统挂上。

-F:这个命令通常和-a一起使用,它会为每一个mount的动作产生一个行程负责执行。在系统需要挂上大量NFS文件系统时可以加快加载的速度。

\textbf{exit:}

作用:exit命令的作用是退出系统,它的使用权限是所有用户。

格式:exit

\subsubsection{文件处理命令}

\textbf{mkdir:}

作用:mkdir命令的作用是建立名称为dirname的子目录,与MS DOS下的md命令类似,它的使用权限是所有用户。

格式:mkdir [options] 目录名

主要参数:

-m, --mode=模式:设定权限,与chmod类似。

-p, --parents:需要时创建上层目录;如果目录早已存在,则不当作错误。

-v, --verbose:每次创建新目录都显示信息。

\textbf{grep:}

作用:grep命令可以指定文件中搜索特定的内容,并将含有这些内容的行标准输出。grep全称是Global Regular ExpressionPrint,表示全局正则表达式版本,它的使用权限是所有用户。

格式:grep [options]

主要参数:

-c:只输出匹配行的计数。

-I:不区分大小写(只适用于单字符)。

-h:查询多文件时不显示文件名。

\textbf{find:}

作用:find命令的作用是在目录中搜索文件,它的使用权限是所有用户。

格式:find [path][options][expression]

path指定目录路径,系统从这里开始沿着目录树向下查找文件。它是一个路径列表,相互用空格分离,如果不写path,那么默认为当前目录。

主要参数:

-depth:使用深度级别的查找过程方式,在某层指定目录中优先查找文件内容。

-maxdepth levels:
表示至多查找到开始目录的第level层子目录。level是一个非负数,如果level是0的话表示仅在当前目录中查找。

-mindepth levels:表示至少查找到开始目录的第level层子目录。

\subsubsection{系统管理命令}

\textbf{reboot:}

作用:df命令用来检查文件系统的磁盘空间占用情况,使用权限是所有用户。

格式:df [options]

主要参数:

-s:对每个Names参数只给出占用的数据块总数。

-a:递归地显示指定目录中各文件及子目录中各文件占用的数据块数。若既不指定-s,也不指定-a,则只显示Names中的每一个目录及其中的各子目录所占的磁盘块数。

\textbf{top:}

作用:top命令用来显示执行中的程序进程,使用权限是所有用户。

格式:top [-] [d delay] [q] [c] [S] [n]

主要参数:

d:指定更新的间隔,以秒计算。

q:没有任何延迟的更新。如果使用者有超级用户,则top命令将会以最高的优先序执行。

c:显示进程完整的路径与名称。

\textbf{free:}

作用:free命令用来显示内存的使用情况,使用权限是所有用户。

格式:free [-b|-k|-m] [-o] [-s delay] [-t] [-V]

主要参数:

-b -k -m:分别以字节(KB、MB)为单位显示内存使用情况。

\subsubsection{网络操作命令}

\textbf{ifconfig:}

作用:ifconfig用于查看和更改网络接口的地址和参数,包括IP地址、网络掩码、广播地址,使用权限是超级用户。

格式:ifconfig -interface [options] address

主要参数:

-interface:指定的网络接口名,如eth0和eth1。

up:激活指定的网络接口卡。

down:关闭指定的网络接口。

\textbf{ip:}

作用:ip是iproute2软件包里面的一个强大的网络配置工具,它能够替代一些传统的网络管理工具,例如ifconfig、route等,使用权限为超级用户。几乎所有的Linux发行版本都支持该命令。

格式:ip [OPTIONS] OBJECT [COMMAND [ARGUMENTS]]

主要参数:

-V,-Version 打印ip的版本并退出。

-s,-stats,-statistics 输出更为详尽的信息。如果这个选项出现两次或多次,则输出的信息将更为详尽。

-f,-family 这个选项后面接协议种类,包括inet、inet6或link,强调使用的协议种类。如果没有足够的信息告诉ip使用的协议种类,ip就会使用默认值inet或any。link比较特殊,它表示不涉及任何网络协议。

\subsubsection{系统安全相关命令}

\textbf{su:}

作用:su的作用是变更为其它使用者的身份,超级用户除外,需要键入该使用者的密码。

格式:su [选项]... [-] [USER [ARG]...]

主要参数:

-f , --fast:不必读启动文件(如 csh.cshrc 等),仅用于csh或tcsh两种Shell。

-l ,--login:加了这个参数之后,就好像是重新登陆为该使用者一样,大部分环境变量(例如HOME、SHELL和USER等)都是以该使用者(USER)为主,并且工作目录也会改变。如果没有指定USER,缺省情况是root。

-m, -p ,--preserve-environment:执行su时不改变环境变数。


\textbf{umask:}

作用:umask设置用户文件和目录的文件创建缺省屏蔽值,若将此命令放入profile文件,就可控制该用户后续所建文件的存取许可。它告诉系统在创建文件时不给谁存取许可。使用权限是所有用户。

格式:umask [-p] [-S] [mode]

主要参数:

-S:确定当前的umask设置。

-p:修改umask 设置。

\textbf{chmod:}

作用:chmod命令是非常重要的,用于改变文件或目录的访问权限,用户可以用它控制文件或目录的访问权限,使用权限是超级用户。

格式:chmod命令有两种用法。一种是包含字母和操作符表达式的字符设定法(相对权限设定)chmod [who] [+ | - | =] [mode] 文件名;另一种是包含数字的数字设定法(绝对权限设定)chmod [mode] 文件名。

主要参数:

对字符设定法而言:

操作对象who可以是下述字母中的任一个或它们的组合

u:表示用户,即文件或目录的所有者。

g:表示同组用户,即与文件属主有相同组ID的所有用户。

\subsubsection{其他命令}

\textbf{tar:}

作用:tar命令是Unix/Linux系统中备份文件的可靠方法,几乎可以工作于任何环境中,它的使用权限是所有用户。

格式:tar [主选项+辅选项] 文件或目录

主要参数:

-c 创建新的档案文件。如果用户想备份一个目录或是一些文件,就要选择这个选项。

-r 把要存档的文件追加到档案文件的未尾。例如用户已经做好备份文件,又发现还有一个目录或是一些文件忘记备份了,这时可以使用该选项,将忘记的目录或文件追加到备份文件中。

-t 列出档案文件的内容,查看已经备份了哪些文件。

\section{支持BusyBox所需系统调用}
\section{重要系统调用实现}



\subsection{mmap}
\subsubsection{内存映射}
内存映射,简而言之就是将用户空间的一段内存区域映射到内核空间,映射成功后,用户对这段内存区域的修改可以直接反映到内核空间,同样,内核空间对这段区域的修改也直接反映用户空间。
以下是一个把普遍文件映射到用户空间的内存区域的示意图。

\begin{figure}[H]
    \centering
    \caption[short]{内存映射}
    \includegraphics[width=0.8\linewidth]{figures/09-04-mmap-内存映射.png}
\end{figure}

\subsubsection{mmap主要用途}
\textbf{传统的读写文件}

一般来说,修改一个文件的内容需要如下3个步骤:把文件内容读入到内存中。修改内存中的内容。把内存的数据写入到文件中。

从传统读写文件的过程中,我们可以发现有个地方可以优化:如果可以直接在用户空间读写页缓存,那么就可以免去将页缓存的数据复制到用户空间缓冲区的过程。
那么,有没有这样的技术能实现上面所说的方式呢?答案是肯定的,就是mmap。

\textbf{使用mmap读写文件}

mmap用于把文件映射到用户空间中,简单说mmap就是把一个文件的内容在内存里面做一个映像。那么对于内核空间与用户空间两者之间需要大量数据传输等操作的效率是非常高的。进程可以像读写内存一样对普通文件的操作。
mmap系统调用也可以使得进程之间通过映射同一个普通文件实现共享内存。

该函数主要用途有三个:
\begin{enumerate}
    \item 将一个普通文件映射到内存中,通常在需要对文件进行频繁读写时使用,这样用内存读写取代I/O读写,以获得较高的性能;
    \item 将特殊文件进行匿名内存映射,可以为关联进程提供共享内存空间;
    \item 为无关联的进程提供共享内存空间,一般也是将一个普通文件映射到内存中。
\end{enumerate}

\subsubsection{mmap实现}
下面将结合NPUcore的代码来介绍mmap的实现

从上文可知,要实现mmap,就要实现用户空间到内核空间的内存映射。所以mmap会分别确定要映射的用户空间和内存空间。

\textbf{用户空间}

mmap会申请一块适合的虚拟内存作为待映射的用户空间
\begin{lstlisting}[language={Rust},
	caption={os/src/mm/memory_set.rs}]
let area: &mut MapArea = &mut self.areas[idx];
let start_va = area.inner.vpn_range.get_end()
let end_va = start_va + len;

let mut new_area: MapArea = MapArea::new(
    start_va,
    end_va,
    MapType::Framed,
    map_perm,
    map_file,
);
\end{lstlisting}
MapArea是一个描述虚拟地址的结构体,它指向某块用户空间的首尾,mmap在获取该结构体后会将其链入其进程管理的所有虚拟地址中。

\begin{figure}[H]
    \centering
    \caption[short]{虚拟地址结构体}
    \includegraphics[width=0.8\linewidth]{figures/09-04-mmap-虚拟地址结构体.png}
\end{figure}

\textbf{内核空间}

mmap获取待映射的内核空间是通过进程打开的文件描述符获取的
\begin{lstlisting}[language={Rust},
	caption={os/src/mm/memory_set.rs}]
let fd_table = task.files.lock();
match fd_table.get_ref(fd) {
    Ok(file_descriptor) => {
        if !file_descriptor.readable() {
            return EACCES;
        }
        let file = file_descriptor.file.deep_clone();
        file.lseek(offset as isize, SeekWhence::SEEK_SET).unwrap();
        new_area.map_file = Some(file);
    }
    Err(errno) => return errno,
}
\end{lstlisting}
NPUcore中将文件对应的内核空间file存入了new\_area.map\_file中
这样既确定了要映射的用户空间和内存空间,也完成了它们之间的映射。

\textbf{其他}

mmap支持文件映射和匿名映射,如果是文件映射,则会通过上述的确定内核空间的过程,如果是匿名映射,那么该进程只是申请了一块空间,这片空间不与任何文件关联。


\subsection{munmap}

\subsubsection{munmap用途与实现}
munmap执行与mmap相反的操作,也就是删除用户空间与内核空间的映射。

munmap会通过用户传入的地址信息删除页表中的有关映射,并将在mmap中新插入的Maparea删除。

因大部分内容已在mmap介绍过,这里不做赘述。
\input{chapters/09/04/execve.tex}
\subsection{openat}

\subsubsection{openat函数整体介绍}
openat函数用于打开文件。是 Linux 系统中用于打开文件的系统调用之一。它与 open 函数类似,但提供了更灵活和安全的文件路径处理方式。
openat() 在很多方面类似于 open(),但它提供了更加灵活和安全的文件路径处理方式:
\begin{itemize}
    \item 指定目录文件描述符: openat() 可以使用一个目录文件描述符作为起始位置,使得可以在指定目录下打开文件,而不受当前工作目录的影响。
    \item 相对路径打开: 可以使用相对于指定目录文件描述符的相对路径打开文件,避免了对当前工作目录的依赖。
\end{itemize}
openat的参数与返回值展示如下:
\begin{lstlisting}[language={Rust}, 
    caption={openat的参数与返回值}]
pub fn sys_openat(dirfd: usize, path: *const u8, flags: u32, mode: u32) -> isize
\end{lstlisting}
解释如下:
\begin{itemize}
    \item dirfd 是一个打开的目录文件描述符,用于指定相对路径的起始位置。可以是当前工作目录的文件描述符,或者是先前通过 open() 或 openat() 打开的目录文件描述符。
    \item path 是要打开的文件的路径名,可以是绝对路径或相对于 dirfd 的相对路径。
    \item flags 包含文件打开的标志,比如 O_RDONLY、O_WRONLY、O_CREAT、O_TRUNC 等,控制文件的打开方式。
    \item mode 是文件的权限设置,仅当 O_CREAT 被设置时才会生效,用于新建文件的权限设置。
\end{itemize}

\subsubsection{整体流程}
流程图展示如下:
\begin{figure}[H]
    \centering
    \scalebox{0.5}{\includegraphics{figures/09-04-openat-函数流程.png}}
    \caption{execve流程图}
\end{figure}

\subsubsection{代码详析}
1、函数输入参数展示如下
\begin{lstlisting}[language={Rust}, 
    caption={openat的参数与返回值}]
pub fn sys_openat(dirfd: usize, path: *const u8, flags: u32, mode: u32) -> isize
\end{lstlisting}
2、常规操作,获取tcb信息,token,path、mode参数形式转换,获取fd_table
\begin{lstlisting}[language={Rust}, 
    caption={openat 常规操作,获取tcb信息、token、path参数形式转换}]
let task = current_task().unwrap();
let token = task.get_user_token();
let path = match translated_str(token, path) {
    Ok(path) => path,
    Err(errno) => return errno,
};
let flags = match OpenFlags::from_bits(flags) {
    Some(flags) => flags,
    None => {
        warn!("[sys_openat] unknown flags");
        return EINVAL;
    }
};
let mode = StatMode::from_bits(mode);
info!(
    "[sys_openat] dirfd: {}, path: {}, flags: {:?}, mode: {:?}",
    dirfd as isize, path, flags, mode
);
let mut fd_table = task.files.lock();
\end{lstlisting}
3、接着,函数尝试从当前任务的文件表(fd_table)中获取文件描述符。根据传入的 dirfd 参数,它会判断是否为特殊值 AT_FDCWD(表示当前工作目录),若不是则尝试从文件表中获取对应的文件描述符。
\begin{lstlisting}[language={Rust}, 
    caption={openat获取目录文件描述符}]
let file_descriptor = match dirfd {
    AT_FDCWD => task.fs.lock().working_inode.as_ref().clone(),
    fd => {
        let fd_table = task.files.lock();
        match fd_table.get_ref(fd) {
            Ok(file_descriptor) => file_descriptor.clone(),
            Err(errno) => return errno,
        }
    }
};
\end{lstlisting}
4、根据获取到的文件描述符,调用其 open 方法打开传入的 path 路径的文件,使用传入的 flags 和 mode。
如果打开成功,则将新的文件描述符加入到任务的文件表中,并返回该文件描述符作为结果。
\begin{lstlisting}[language={Rust}, 
    caption={openat-打开文件,返回值设置为文件描述符的}]
let new_file_descriptor = match file_descriptor.open(&path, flags, false) {
    Ok(file_descriptor) => file_descriptor,
    Err(errno) => return errno,
};

let new_fd = match fd_table.insert(new_file_descriptor) {
    Ok(fd) => fd,
    Err(errno) => return errno,
};
new_fd as isize
\end{lstlisting}
5、深入文件描述符的open函数。
总体来说,这段代码是文件描述符对象的 open 方法实现,用于处理文件系统中文件的打开操作。它检查路径是否有效,根据文件的类型进行不同的处理,最终尝试打开指定路径的文件或目录,并返回一个新的文件描述符对象或相应的错误信息。
首先,该方法接受三个参数:path(文件路径)、flags(打开文件的标志)、special_use(特殊用途标志)。
如果传入的路径 path 为空字符串 "",则直接返回当前文件描述符的克隆,即 Ok(self.clone())。
接着,如果当前文件描述符对应的文件是一个文件(不是目录),且传入的 path 不是以斜杠 / 开头(即相对路径),则返回错误 ENOTDIR(表示不是一个目录)。
之后,方法尝试从当前文件描述符对应的节点(即文件或目录)获取节点 inode。
如果成功获取到 inode,则调用 inode 的 open 方法,尝试以给定的 flags 和 special_use 打开传入的路径 path。如果打开成功,则获取到一个新的文件描述符 file。
根据传入的 flags 是否包含 O_CLOEXEC 标志,设置 cloexec 变量,用于指示是否在 file 上设置了 O_CLOEXEC(close-on-exec)标志。
最后,通过 Self::new() 创建一个新的文件描述符对象,返回一个包含新文件描述符的 Result。

\begin{lstlisting}[language={Rust}, 
    caption={文件描述符的open函数}]
    pub fn open(&self, path: &str, flags: OpenFlags, special_use: bool) -> Result<Self, isize> {
        if path == "" {
            return Ok(self.clone());
        }
        if self.file.is_file() && !path.starts_with('/') {
            return Err(ENOTDIR);
        }
        let inode = self.file.get_dirtree_node();
        let inode = match inode {
            Some(inode) => inode,
            None => return Err(ENOENT),
        };
        let file = match inode.open(path, flags, special_use) {
            Ok(file) => file,
            Err(errno) => return Err(errno),
        };
        let cloexec = flags.contains(OpenFlags::O_CLOEXEC);
        Ok(Self::new(cloexec, false, file))
    }
\end{lstlisting}

补充:O_CLOEXEC是在打开文件描述符时设置的一个标志,它表示在执行  execve()  调用时,该文件描述符将被关闭,避免子进程继承该文件描述符。

具体来说,当一个进程(父进程)打开一个文件描述符并设置了  O_CLOEXEC  标志后,如果在这个进程中调用  execve()  或  exec()  等系统调用来执行另一个程序,那么在新程序执行时,该文件描述符将会被自动关闭。这样可以确保新程序不会继承或意外使用父进程中打开的文件描述符,从而提高了安全性和可预测性。

这种行为通常用于父进程打开的文件描述符并不需要在子进程中保持打开的情况下。通过设置  O_CLOEXEC  标志,可以避免不必要的文件描述符传递到子进程中,减少了资源泄漏和意外的文件操作可能性。

6、文件目录树节点的open方法详析
这段 Rust 代码实现了 inode 对象的 open 方法,用于在文件系统中打开文件或目录。让我们逐步解析这段代码:

\textbf{日志记录和路径重定向:}

代码开始处记录了调试日志,包括当前工作目录和传入的路径。
接着定义了一些特殊路径和相应的重定向规则,例如将特定路径重定向到 BUSYBOX_PATH 或 LIBC_PATH 等。
经过一系列的条件判断,根据特定规则将传入的 path 重新赋值,以便后续处理。

\textbf{定打开的起始节点(inode):}

根据传入的 path 是否以 / 开头来确定打开文件的起始节点 inode,如果以 / 开头则从根节点(ROOT)开始,否则从当前节点(self)开始。

\textbf{路径缓存和文件系统操作:}

\begin{itemize}
    \item 使用互斥锁 PATH_CACHE.lock() 获取路径缓存,如果存在缓存且路径匹配,则直接使用缓存的 inode。\
    \item 否则,解析路径并逐级切换目录,获取到最终的目标 inode。
    \item 如果打开的文件不存在,根据标志位创建新的文件。
    \item 在文件存在的情况下,根据传入的标志位进行相应的操作,比如截断文件或检查文件状态。
\end{itemize}

\textbf{文件打开及特殊处理:}

\begin{itemize}
    \item 在文件打开之前,进行了一系列的检查,例如判断是否是目录、文件是否正在被使用、是否需要截断文件等。
    \item 根据 special_use 参数进行特殊处理,如果需要特殊处理,则增加 spe_usage 的计数值。
    \item 如果路径以 / 开头且与路径缓存不同,则更新路径缓存。
\end{itemize}

\textbf{返回结果:}

最后调用文件对象的 open 方法打开文件,根据操作结果返回成功打开的文件或相应的错误码。

7、深入OSInode的open方法

\textbf{创建文件对象:}

方法接受两个参数:flags(打开文件的标志)和 special_use(特殊用途标志)。
根据传入的标志位,设置文件对象的一些属性,比如 readable 表示文件是否可读、writable 表示文件是否可写、append 表示是否在末尾追加写入数据等。

\textbf{构建新的文件对象:}

使用 Arc::new() 创建一个新的文件对象 Arc<dyn File>。
在构建文件对象时,将文件对象的各种属性设置为对应的标志位状态或传入的参数状态,比如可读性、可写性、追加写入等。
将原始文件对象的内部状态 inner 进行克隆,offset 采用 Mutex 进行多线程安全的偏移量管理。

最后将创建的对象返回

\begin{lstlisting}[language={Rust}, 
    caption={OSInode的open方法}]
fn open(&self, flags: OpenFlags, special_use: bool) -> Arc<dyn File> {
    Arc::new(Self {
        readable: flags.contains(OpenFlags::O_RDONLY) || flags.contains(OpenFlags::O_RDWR),
        writable: flags.contains(OpenFlags::O_WRONLY) || flags.contains(OpenFlags::O_RDWR),
        special_use,
        append: flags.contains(OpenFlags::O_APPEND),
        inner: self.inner.clone(),
        offset: Mutex::new(0),
        dirnode_ptr: self.dirnode_ptr.clone(),
    })
}
\end{lstlisting}

\subsection{fstat}


\textbf{fstat}主要功能在文件系统章节已经介绍,用于由文件描述符,获取文件当前状态。
所以本小节将从fstat所用到的\textbf{stat结构体},与\textbf{利用fd查找文件}两个角度出发,来解释fstat函数在操作系统文件部分中的实现和应用。\\

\subsubsection{系统调用原型}
\begin{lstlisting}[language={Rust}, label={code:new_area},
	caption={os/src/syscall/fs.rs}]
    pub fn sys_fstat(fd: usize, statbuf: *mut u8);
\end{lstlisting}
\\
参数解释:\\
fd:文件描述符(file descriptor),是一个非负整数,用于唯一标识打开的文件。\\
statbuf:一个指向stat结构的指针,用于接收文件详细信息(指向内容如代码片段9.20)。\\
返回值: 执行成功则返回SUCCESS,失败返回errno并跳入。成功和失败的判定是在于有没有通过fd找到文件,且返回相关的stat信息。\\

\subsubsection{fstat流程}
根据fstat系统调用的功能,我们可以写出fstat函数的流程与伪代码。需要注意的是,传入的fd可能本身不存在,所以为了提升鲁棒性,我们还需要判断fd是否为正,否则直接跳入errno。
例外:fd为AT_FDCWD(-100),表明当filename为相对路径的情况下,将当前进程的工作目录设置为起始路径。此时我们则需要对当前inode进行克隆,防止死锁。
\begin{figure}[H]
    \centering
    \scalebox{0.18}{\includegraphics{figures/09-04-fstat-流程图.png}}
    \caption{fstat流程框图,黄色框代表输入,红色与绿色框分别是两种不同的输出,蓝色框为流程}
\end{figure}

\begin{algorithm}[tb]
    \caption{fstat}
    \label{alg:algorithm}
    \textbf{Input}: fd, statbuf\\
    \textbf{Output}: SUCCESS~or~ERRNO
    \begin{algorithmic}[1] %[1] enables line numbers

        \State Let $task=current~task$.
        \State Let $token=user~token$.
        \If{$fd<0$~or~$fd~unmatched$}
        \State \textbf{return} $ERRNO$
        \Else \If{$fd$~\textbf{equal}~$AT\underline{~~}FDCWD$}
        \State lock and clone, copy stat to buf
        \Else
        \State copy stat to buf
        \EndIf
        \State \textbf{return} $SUCCESS$
        \EndIf
    \end{algorithmic}
\end{algorithm}



\floatname{algorithm}{Code}
\begin{algorithm}[tb]
\centering
\begin{lstlisting}[language={Rust}, label={code:new_area},
	caption={stat示意结构体}]
    struct stat {
    dev_t     st_dev;         // 文件所在设备的设备号
    ino_t     st_ino;         // 文件的i节点号
    mode_t    st_mode;        // 文件的访问权限
    nlink_t   st_nlink;       // 文件的硬链接数
    uid_t     st_uid;         // 文件所有者的用户ID
    gid_t     st_gid;         // 文件所有者的组ID
    dev_t     st_rdev;        // 如果文件是特殊字符设备或块设备,保存设备号
    off_t     st_size;        // 文件的大小(常以字节为单位)
    blksize_t st_blksize;     // 文件I/O缓冲区大小
    blkcnt_t  st_blocks;      // 分配给文件的块数
    time_t    st_atime;       // 文件的最近访问时间
    time_t    st_mtime;       // 文件的最近修改时间
    time_t    st_ctime;       // 文件的最近更改时间(包括权限和归属人等)
};
\end{lstlisting}
\end{algorithm}

\subsubsection{fstat代码详解}

第一步,获取当前线程,老生常谈,不多介绍。
\begin{lstlisting}[language={Rust}, label={code:new_area},
	caption={获取线程}]
    let task = current_task().unwrap();
    let token = task.get_user_token();
\end{lstlisting}
第二步,搜索fd是否存在。本处省略了对fd大于0的判断。
1.首先,通过match语句对文件描述符进行匹配。
2.如果匹配到的文件描述符是AT_FDCWD,代表当前工作目录的文件描述符,那么将通过task.fs.lock().working_inode.as_ref().clone()获取到当前任务(task)的文件系统(fs)的锁,然后获取到工作inode(working inode)的引用,并进行克隆(clone)得到一个新的file_descriptor。
3.如果匹配到的文件描述符不是AT_FDCWD,则会执行下面的代码块。
4.在下面的代码块中,首先获取到当前任务的文件表(fd_table)的锁,并使用get_ref(fd)方法获取到指定文件描述符的引用。
5.如果获取成功(Ok),则克隆(clone)该文件描述符,并将克隆得到的新的file_descriptor赋值给file_descriptor变量。
如果获取失败(Err),则会返回一个表示错误码(errno)的值。
\begin{lstlisting}[language={Rust}, label={code:new_area},
	caption={fd查找}]
    let file_descriptor = match fd {
        AT_FDCWD => task.fs.lock().working_inode.as_ref().clone(),
        fd => {
            let fd_table = task.files.lock();
            match fd_table.get_ref(fd) {
                Ok(file_descriptor) => file_descriptor.clone(),
                Err(errno) => return errno,
            }
        }
    };
\end{lstlisting}
第三步,将fd搜索到的stat信息,写入用户定义的buffer中。我们直接使用copy_to_user函数即可。函数结束,顺利返回SUCCESS。
\begin{lstlisting}[language={Rust}, label={code:new_area},
	caption={赋值stat buffer}]
    copy_to_user(token, &file_descriptor.get_stat(), statbuf as *mut Stat);
    return SUCCESS;
\end{lstlisting}
这里我们也给出stat的结构体(代码片段9.20),供大家参考。
\subsection{fstatat}
前面的stat是通过fd去获得stat状态,而相比于fstat,fstatat可以通过路径名来定位文件,而不仅仅依赖文件描述符来获取文件信息。
本小节将重点介绍\textbf{fstatat系统调用相较于fstat不同的地方},因此会忽略部分与fstat相同或类似的算法部分。

\subsubsection{系统调用原型}
\begin{lstlisting}[language={Rust},
	caption={os/src/syscall/fs.rs}]
    pub fn sys_fstatat(dirfd: usize, path: *const u8, buf: *mut u8, flags: u32);
\end{lstlisting}
\\
参数解释:\\
dirfd:用于定位文件的目录文件描述符(file descriptor)。它可以是当前工作目录的文件描述符,也可以是其它目录的文件描述符。\\
path:文件的路径名。它可以是绝对路径或相对路径。\\
buf:一个指向stat结构的指针,用于接收文件详细信息。\\
flags:一些标记选项,控制fstatat的行为。\\
在NPUcore中,flags有如下几种类型:
\begin{lstlisting}[language={Rust},
	caption={NPUcore中的fstatat~flags选项}]
pub struct FstatatFlags: u32 {
        const AT_EMPTY_PATH = 0x1000; \\允许空路径名(仅用于目录文件)
        const AT_NO_AUTOMOUNT = 0x800; \\禁止系统自动挂载文件系统
        const AT_SYMLINK_NOFOLLOW = 0x100; \\不跟随符号链接
    }
\end{lstlisting}
\subsubsection{fstatat流程}
fstatat函数相较于fstat多了两个阶段。第一是根据Flags查找并获取当前的fstatat标记选项,第二是以只读模式打开dirfd所指向的file_descriptor,如果可以正常打开,再拷贝stat状态至buffer。
为此我们需要在fstat的流程图中添加部分流程框。
\begin{figure}[H]
    \centering
    \scalebox{0.13}{\includegraphics{figures/09-04-fstatat-流程图.png}}
    \caption{fstatat流程框图,灰色框代表与fstat重复部分,黄色框代表输入,红色与绿色框分别是两种不同的输出,靛蓝色框代表与fstat不同的流程。}
\end{figure}
\subsubsection{fstatat代码详解}
由于fstatat与fstat耦合代码较多,这里只涉及fstatat独有的部分代码。
第一个不同是获取Flag部分。
1.首先,通过match语句将flags值转换为FstatatFlags类型的枚举值。
2.将FstatatFlags的flags值转换为FstatatFlags枚举类型的标志位。如果转换成功(Some(flags)),则将转换后的枚举值赋给flags变量。如果转换失败(None),则执行下面的代码块。
3.在代码块中,打印一个警告信息(warn!("[sys_fstatat] unknown flags"))表示传入的flags值无法识别。最后返回一个表示无效参数(EINVAL)的错误码。
\begin{lstlisting}[language={Rust},
	caption={FstatatFlag判断}]
    let flags = match FstatatFlags::from_bits(flags) {
        Some(flags) => flags,
        None => {
            warn!("[sys_fstatat] unknown flags");
            return EINVAL;
        }
    };
\end{lstlisting}
第二个不同是文件存在性验证部分。
1.首先,通过match语句对file_descriptor对象的open方法进行匹配。该方法使用给定的path路径、打开标志(OpenFlags::O_RDONLY表示只读打开),且不需要创建新文件。如果文件打开成功(Ok),则执行下面的代码块。
2.在代码块中,首先通过file_descriptor.get_stat()方法获取到文件描述符对应文件的统计信息(Stat类型),然后调用copy_to_user函数将stat状态复制到用户提供的buf指针指向的内存中,最后返回一个表示成功的值(SUCCESS)。
3.如果文件打开失败(Err),则执行errno代码块,并返回一个表示错误码(errno)的值。
\begin{lstlisting}[language={Rust},
	caption={存在性验证与stat拷贝}]
    match file_descriptor.open(&path, OpenFlags::O_RDONLY, false) {
        Ok(file_descriptor) => {
            copy_to_user(token, &file_descriptor.get_stat(), buf as *mut Stat);
            SUCCESS
        }
        Err(errno) => errno,
    }
\end{lstlisting}
\subsection{write}
\subsubsection{write函数整体介绍}
write函数通常用于将数据从用户空间写入文件或其他的输出设备,如果文件不可写,则返回相应的错误类型。
\noindent
write的参数与返回值展示如下:
\begin{lstlisting}[language={Rust}, 
    caption={write的参数与返回值}]
pub fn sys_openat(dirfd: usize, path: *const u8, flags: u32, mode: u32) -> isize
\end{lstlisting}
解释如下:
\begin{itemize}
    \item fd 表示文件描述符,一般通过open函数的返回值获得,对当前任务队列的活跃文件进行唯一标识
    \item buf 表示写入数据的缓冲区的指针
    \item count 表示要写入的字节数,和buf参数一起可以表示写入数据的全部内容
\end{itemize}
\subsubsection{整体流程}
流程图展示如下:
\begin{figure}[H]
    \centering
    \scalebox{0.5}{\includegraphics{figures/09-04-write-流程图.png}}
    \caption{wtire流程图}
\end{figure}

\subsubsection{代码详析}
1、函数输入参数展示如下
\begin{lstlisting}[language={Rust},
    caption={write的参数与返回值}]
    pub fn sys_write(fd: usize, buf: usize, count: usize) -> isize
\end{lstlisting}

2、获取TCB信息并通过get_ref函数和得到的文件描述符表fd_table,将参数fd转化成包含文件信息的文件描述符结构体指针file_descriptor
\begin{lstlisting}[language={Rust},
    caption={含有文件信息文件描述符指针的获取}]
    let task = current_task().unwrap();
    let fd_table = task.files.lock();
    let file_descriptor = match fd_table.get_ref(fd) {
        Ok(file_descriptor) => file_descriptor,
        Err(errno) => return errno,
    };
\end{lstlisting}

3、判断文件是否可写,如果不可写则返回EBADF(Bad file number,值为-9)
\begin{lstlisting}[language={Rust},
    caption={文件权限判断}]
    if !file_descriptor.writable() {
        return EBADF;
    }
\end{lstlisting}

4、获取用户空间令牌,用于下文缓冲区数据的获取

\begin{lstlisting}[
    language={Rust},
    caption={获取用户空间令牌}
]
let token = task.get_user_token();
    
\end{lstlisting}

5、通过调用write_user函数,将用户空间数据写入文件描述符对应的文件中
write_user有两个参数
\begin{lstlisting}[
    language={Rust},
    caption={write_user参数}
]
fn write_user(&self, offset: Option<usize>, buf: UserBuffer) -> usize
\end{lstlisting}
\begin{itemize}
    \item offset 表示用户指定的固定偏移量(相对于文件开始位置的偏移),不受文件指针偏移的影响.
    \item buf 表示一个UserBuffer结构体,其内部信息通过translated_byte_buffer函数获得
\end{itemize}
    UserBuffer结构体内容如下,包括一个表示数据具体内容的Vec<\&mut[u8]> 和表示长度的len
\begin{lstlisting}[
    language={Rust},
    caption={UserBuffer结构体}
]
pub struct UserBuffer {
    /// The segmented array, or, a "vector of vectors".
    /// # Design Information
    /// In Rust, reference lifetime is a must for this template.
    /// The lifetime of buffers is `static` because the buffer 'USES A' instead of 'HAS A'
    pub buffers: Vec<&'static mut [u8]>,
    /// The total size of the Userbuffer.
    pub len: usize,
}
    
\end{lstlisting}

translated_byte_buffer函数接受三个参数,用户空间令牌token,缓冲区指针ptr和缓冲区长度len,返回值为一个Result包裹的Vec<\&mut[u8]>变量

\begin{lstlisting}[
    language={Rust},
    caption={translated_byte_buffer函数的参数和返回值}
]
pub fn translated_byte_buffer(
    token: usize,
    ptr: *const u8,
    len: usize,
) -> Result<Vec<&'static mut [u8]>, isize> 
    
\end{lstlisting}

6、深入文件描述符的write_user函数,我们以OSInode的实现为例进行讲解。
首先初始化写入总字节数变量total_write_size并获取文件的写锁。接下来分两种情况进行处理,
如果偏移量不为None,则通过循环遍历用户缓冲区,调用self.inner.write_at_block_cache_lock函数
分次将数据写入文件,同时确保实际写入的字节数与数据片段长度相等,并维护偏移量offset(指向下一个位置)和写入总字节数total_write_size
\begin{lstlisting}[
    language={Rust},
    caption={偏移量不为None}
]
Some(mut offset) => {
    let mut offset = &mut offset;
    for slice in buf.buffers.iter() {
        let write_size =
            self.inner
                .write_at_block_cache_lock(&inode_lock, *offset, *slice);
        assert_eq!(write_size, slice.len());
        *offset += write_size;
        total_write_size += write_size;
    }
}
\end{lstlisting}

如果偏移量为None,则需额外对offset进行处理,如果不是追加模式则设置offset为文件描述符的偏移量,如果为追加模式则将offset设置为文件的当前大小,其余的操作与offset不为None的情况相同。
\begin{lstlisting}[
    language={Rust},
    caption={偏移量为None的情况}
]
None => {
    let mut offset = self.offset.lock();
    if self.append {
        *offset = self.inner.get_file_size_wlock(&inode_lock) as usize;
    }
    for slice in buf.buffers.iter() {
        let write_size =
            self.inner
                .write_at_block_cache_lock(&inode_lock, *offset, *slice);
        assert_eq!(write_size, slice.len());
        *offset += write_size;
        total_write_size += write_size;
    }
}
\end{lstlisting}
7、我们继续深入,来分析write的块级实现write_at_block_cache_lock,它接受三个参数

\begin{itemize}
    \item inode_lock 表示文件的写锁
    \item offset 表示写数据位置的偏移量,即写数据位置相对文件开始位置的偏移
    \item buf 表示具体要写入文件的数据,类型为\&[u8]
\end{itemize}

write_at_block_cache_lock函数首先通过获得的偏移量的信息来判断是否需要进行空间的扩展,如果diff_len大于0,则调用modify_size_lock扩展diff_len大小的空间(分配新簇)
\begin{lstlisting}[
    language={Rust},
    caption={判断是否进行空间扩展分配}
]
let mut start = offset;
        let old_size = self.get_file_size() as usize;
        let diff_len = buf.len() as isize + offset as isize - old_size as isize;
        if diff_len > 0 as isize {
            // allocate as many clusters as possible.
            self.modify_size_lock(inode_lock, diff_len, false);
        }
\end{lstlisting}

接着计算写入结束位置end并确保start小于end,同时初始化写入的缓存块索引 start_cache 和写入的总字节数 write_size。

\begin{lstlisting}[
    language={Rust},
    caption={其他变量初始化}
]
let end = (offset + buf.len()).min(self.get_file_size() as usize);

debug_assert!(start <= end);

let mut start_cache = start / PageCacheManager::CACHE_SZ;
let mut write_size = 0;
\end{lstlisting}

最后循环处理并写入数据块缓存,每次循环维护当前数据块的结束位置end_current_block,并通过get_cache方法获取数据块的缓存,调用modify将数据从用户提供的缓冲区buf复制到数据块缓存,
同时维护写入字节数和数据块的写入位置。

\begin{lstlisting}[
    language={Rust},
    caption={循环处理数据块并写入}
]
loop {
    // calculate end of current block
    let mut end_current_block =
        (start / PageCacheManager::CACHE_SZ + 1) * PageCacheManager::CACHE_SZ;
    end_current_block = end_current_block.min(end);
    // write and update write size
    let lock = self.file_content.read();
    let block_write_size = end_current_block - start;
    self.file_cache_mgr
        .get_cache(
            start_cache,
            || -> Vec<usize> { self.get_neighboring_sec(&lock.clus_list, start_cache) },
            &self.fs.block_device,
        )
        .lock()
        // I know hardcoding 4096 in is bad, but I can't get around Rust's syntax checking...
        .modify(0, |data_block: &mut [u8; 4096]| {
            let src = &buf[write_size..write_size + block_write_size];
            let dst = &mut data_block[start % PageCacheManager::CACHE_SZ
                ..start % PageCacheManager::CACHE_SZ + block_write_size];
            dst.copy_from_slice(src);
        });
    drop(lock);
    write_size += block_write_size;
    // move to next block
    if end_current_block == end {
        break;
    }
    start_cache += 1;
    start = end_current_block;
}
\end{lstlisting}
\subsection{read}
read函数的实现逻辑基本与write函数相同,只在对块缓存的处理上有不同之处,所以在这里对重复的内容不在赘述,只进行不同之处的讲解。

在read函数的块级操作中,与write的区别主要体现在read_at_block_cache_rlock方法的循环处理中。
write中循环处理是将用户缓存区写入块缓存中,但是在read中调用的是块级的.get_cache().lock.().read()方法,
是将块缓存的内容(src)copy到用户到的buf(dst)中,最后返回给用户以达到读文件的效果。除此之外,read与write的另一不同之处在于read并没有追加模式。(write有对于追加写的判断)

\begin{lstlisting}[
    language={Rust},
    caption={read与write实现的不同之处}
]
.get_cache(
    start_cache,
    || -> Vec<usize> { self.get_neighboring_sec(&lock.clus_list, start_cache) },
    &self.fs.block_device,
)
.lock()
// I know hardcoding 4096 in is bad, but I can't get around Rust's syntax checking...
.read(0, |data_block: &[u8; 4096]| {
    let dst = &mut buf[read_size..read_size + block_read_size];
    let src = &data_block[start % PageCacheManager::CACHE_SZ
        ..start % PageCacheManager::CACHE_SZ + block_read_size];
    dst.copy_from_slice(src);
});
\end{lstlisting}


\end{document}
